% Intro

При работе с графом зачастую требуется найти в нем множество путей от одной или нескольких вершин к другим. В случае, когда важно лишь знать о существовании этих путей, говорят, что требуется решить задачу достижимости в графе. При этом на множество искомых путей могут накладываться некоторые ограничения, заданные с помощью формальных языков. В итоге в помеченном графе достижимыми будут считаться только те вершины, путь к которым будет образовывать слова, принадлежащие заданному формальному языку.

Для формального описания свойств меток путей в графе используются регулярные и контекстно-свободные грамматики. Регулярные ограничения на пути в графе стали ключевым компонентом языков запросов к графовым базам данных~\cite{intro_rpq}. С помощью контекстно-свободных языков можно выразить более широкий класс ограничений, по этой причине они нашли широкое применение и в других областях таких, как биоинформатика~\cite{intro_bioinf}, статический анализ кода~\cite{intro_code_analysis} и др.

Несмотря на то, что регулярные и контекстно-свободные запросы к базам данных активно используются на практике, существующие алгоритмы показывают невысокую производительность на реальных входных данных, что было показано в работе \cite{intro_eval_cfpq} для контекстно-свободных грамматик и в работе \cite{intro_eval_rpq} для регулярных грамматик. В работе \cite{intro_eval_rpq} было исследовано два основных подхода к регулярным запросам и выявлено, что каждый из этих подходов эффективен только для конкретных видов графов.

В то же время, актуальным становится применение классических алгоритмов анализа графов, представленных с помощью операций линейной алгебры, исследования которых показывают их эффективность на реальных данных. Благодаря их популярности появляются стандарты~\cite{intro_standards_for_graph_algorithm_primitives} такие, как GraphBLAS~\cite{intro_graphblas}, определяющие примитивы для реализации алгоритмов в терминах линейной алгебры. Существует множество реализаций этих примитивов, которые удобно использовать для создания высокоэффективных алгоритмов. 

Как и классические алгоритмы поиска путей в графе, алгоритмы достижимости с формальными ограничениями разделяются на несколько видов. Алгоритмы для одной исходной вершины (single source), нескольких исходных вершин (multiple source) или всех вершин (all paths). Алгоритмы для нескольких вершин являются менее изученными, при этом имеют активно применяются в графовых базах данных.

Тем самым, на основе классических алгоритмов анализа графов, кажется возможным построить эффективную реализацию решения задачи достижимости с ограничениями в виде формальных языков. Разработка такого алгоритма для нескольких стартовых вершин позволит глубже исследовать производительность матричных алгоритмов для решения задачи достижимости с несколькими стартовыми вершинами, а также создать высокоэффективное решение, применимое к реальным графовым данным.

Таким образом, в рамках данной работы проводится разработка алгоритма для задачи регулярной достижимости для нескольких стартовых вершин. Главной особенностью алгоритма является то, что он основан на операциях линейной алгебры и за его основу взят один из классических алгоритмов анализа графов --- обход в ширину. Предлагается реализация алгоритма и проводится экспериментальное исследование эффективности разработанного решения.
