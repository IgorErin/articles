% Related

\label{sec:relatedworks}
В обзоре приводится основная терминология, используемая в работе. После чего рассматриваются существующие алгоритмы решающую задачу достижимости c формальными ограничениями. Подробнее разбираются алгоритмы, основанные на операциях линейной алгебры.

\subsection{Основные термины}

Определим основные объекты из теории формальных языков, которые используются в описании исследуемых и разрабатываемых алгоритмов, а также формально определим задачу достижимости с регулярными ограничениями.

\noindent\textbf{Определение}  \textit{Конечным автоматом} называется формальная система $\langle Q, \Sigma, P, Q_{src}, F \rangle$, где 
\begin{itemize}
    \item $Q$ --- конечное непустое множество состояний,
    \item $\Sigma$ --- конечный входной алфавит,
    \item $P$ --- отображение $Q \times \Sigma \rightarrow Q$,
    \item $Q_{src} \subset Q$ --- множество начальных состояний,
    \item $F$ --- множество конечных состояний
\end{itemize}

Теперь определим грамматики, используемые для задания ограничений в графе.

\noindent\textbf{Определение} \textit{Формальной грамматикой} называется четверка \\ $\langle V_N, V_T, P, S \rangle$, где
\begin{itemize}
    \item $V_N, V_T$ --- конечные и непересекающиеся алфавиты нетерминалов и терминалов соответственно, 
    \item $P$ --- конечное множество правил,
    \item $S$ --- стартовый нетерминал.
\end{itemize}

\noindent\textbf{Определение} \textit{Регулярной грамматикой} называется формальная грамматика, правила которой могут быть заданы как $A \rightarrow aB$, $A \rightarrow a$, либо как $A \rightarrow Ba$, $A \rightarrow a$ где $a \in V_T$, $A,B \in V_N$.

\subsection{Формулировка задачи достижимости}\label{sec:3.3}

Теперь сформулируем задачу регулярной достижимости (RPQ) и её частный случай для нескольких стартовых вершин.

\noindent\textbf{Определение} \textit{Задачей регулярной достижимости в графе} называется следующая задача: имея помеченный граф $D$ и регулярный язык $G$, требуется найти такое множество всех пар вершин, для которых существует хотя бы один путь от начальной вершины к конечной, что слово, полученной конкатенацией меток ребер графа $D$ будет принадлежать данному регулярному языку $G$.

Для нескольких стартовых вершин задачу можно определить разными способами. При этом в каждом из этих определений входными данными являются граф $D$ с метками на ребрах и регулярный язык, заданный грамматикой $G$. В графе выбирается некоторое множество начальных вершин $V_{src}$, из которого требуется найти достижимые вершины.

\noindent\textbf{Определение}
\label{related_task1}\textit{Постановка задачи 1}. Необходимо найти такое множество вершин графа, что для каждой вершины из этого множества существует хотя бы один путь, начало которого содержится в множестве начальных вершин. При этом метки на ребрах этого пути при конкатенации образуют слово, принадлежащее языку грамматики $G$.

Эту задачу можно переформулировать другим способом, желая найти конкретную исходную вершину для каждой достижимой вершины.

\noindent\textbf{Определение}
\label{related_task2}\textit{Постановка задачи 2}. Найти множество пар $(v, w)$ вершин, такое что $v \in V_{src}$, $w \not\in V_{src}$, существует хотя бы один путь из $v$ в $w$ такой, что метки на ребрах этого пути принадлежат языку грамматики $G$.


\subsection{Основные алгоритмы RPQ}

Можно выделить две основные группы алгоритмов RPQ: основанные на реляционной алгебре и конечных автоматах \cite{related_rpq_book}.

\subsubsection{Использование реляционной алгебры}

Этот подход анализирует регулярное выражение, содержащее информацию о запросе к графовой базе данных.

Для решения задачи достижимости используется специальный оператор реляционной алгебры, который вычисляет транзитивное замыкание на множестве вершин графа. Тогда, представив регулярное выражение в виде синтаксического дерева, можно использовать этот оператор для обхода дерева регулярного выражения и выявления достижимых вершин. При этом вычисление транзитивного замыкания и  построение дерева на его основе это трудоемкие операции, которые сказываются на эффективности алгоритма. 

\subsubsection{Использование автоматов}

Известно, что регулярное выражение можно представить в виде конечного автомата, принимающего тот же язык, что и регулярное выражение. Тогда, представив граф в виде автомата, обозначив его вершины за состояния, а ребра за переходы, можно получить два автомата, пересечение которых будет содержать информацию о достижимых вершинах. В автомате пересечения каждая пара, состоящая из начального состояния и соответствующего ему конечного состояния, будет образовывать искомое множество достижимых вершин в графе.

Для получение пар достижимых вершин не обязательно строить полный автомат пересечения, достаточно лишь устроить обход в ширину построенных изначально автоматов. Перед тем как совершить переходы в графе к новому фронту вершин путь до текущих вершин во фронте проверяется на автомате регулярного выражения. Тогда в новый фронт обхода графа попадают только те вершины, которые были приняты автоматом.

Данный подход можно оптимизировать, совершая обход одновременно по двум автоматам. Более того, известно как представить обход в ширину с помощью операций линейной алгебры на основе матричного умножения, что на больших графах может существенно ускорить описанный алгоритм.

\subsection{Алгоритмы, основанные на линейной алгебре}

К настоящему времени существует лишь несколько достаточно эффективных реализаций матричных алгоритмов поиска путей в графе с формальными ограничениями. Многие из них были разработаны для контекстно-свободных грамматик и будут разобраны в этом разделе. Для регулярных грамматик аналогичных матричных алгоритмов найдено не было.

\subsubsection{Алгоритмы для всех пар вершин}

В упомянутом~\cite{intro_eval_cfpq} сравнительном исследовании алгоритмов был исследован алгоритм Рустама Азимова~\cite{related_rustam_azimov}. За основу этого алгоритма взято вычисление транзитивного замыкания.

Алгоритм принимает на вход граф и контекстно-свободную грамматику, выраженную в ослабленной нормальной форме Хомского. Далее он оперирует над множеством булевых матриц, соответствующих каждому нетерминальному символу. Тем самым, при реализации алгоритма большое влияние на скорость алгоритма влияет использование эффективных библиотек примитивов линейной алгебры.

Помимо этого алгоритма в работе~\cite{related_kron} был разработан алгоритм основанный на тензорном произведении. Он также использует операции матричного умножения, но в отличии от предыдущего алгоритма не требует модификации изначальной контекстно-свободной грамматики. Этот алгоритм основывается на вычислении пересечения автоматов, каждый из которых выражает представление графа и грамматики соответственно. 

На вход алгоритму подается конечный автомат, описывающий сам граф, где вершины являются состояниями, а ребра описывают переходы в автомате. Вторым аргументом алгоритм получает рекурсивный автомат, описывающий ограничения на метки в графе. Известно, что благодаря тензорному произведению можно вычислить пересечение двух автоматов, именно с помощью этой операции строится новая матрица, описывающая переходы в новом автомате. После чего полученный автомат пересечения транзитивно замыкается, чтобы его переходы не содержали нетерминальных символов. Эти операции применяются в цикле для матриц смежности входных автоматов пока любая из матриц смежности автомата графа меняется.

Операции вычисления транзитивного замыкания и тензорного произведения находят пути в графе для всех пар вершин. По этой причине наивная реализация алгоритмов для нескольких стартовых вершин этими методами будет неэффективна, так как все вершины графа буду считаться начальными и для всех них посчитаются пути до других вершин. В связи с этим возникает идея отказаться от вычисления тензорного произведения и использовать классические алгоритмы обхода графа, такие как обход в ширину, для матричного представления автоматов. Для того, чтобы вычислять их пересечение, нужно синхронизировать обход в ширину для соответствующих матриц. 

\subsubsection{Алгоритмы для нескольких стартовых вершин}

Используя различные алгебраические структуры, можно модернизировать представленные алгоритмы для решения задачи достижимости с несколькими стартовыми вершинами. Так, например, можно ограничить вычисляемые пути, производя поиск только по нескольким стартовым вершинам, и не хранить информацию об остальных вершинах при обходе графа. В одной из таких работ~\cite{related_ars} была представлена модификация алгоритма Рустама Азимова, созданная для нескольких стартовых вершин и встроенная в графовую базу данных.

Из всего этого следует, что ведется активная работа в исследовании алгоритмов поиска путей в графе с формальными ограничениями. На основе уже известных алгоритмов, таких как пересечение двух автоматов графа и грамматики через их синхронный обход, кажется возможным создать эффективный алгоритм для нескольких стартовых вершин, основанный на классических алгоритмах обхода графа, используя при этом менее затратные алгебраические операции и оптимизируя его под конкретную задачу. Тем не менее, остается важным использовать специализированные матричные библиотеки для ускорения вычисления операций линейной алгебры.
