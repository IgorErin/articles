% !TEX TS-program = xelatex
% !BIB program = bibtex
% !TeX spellcheck = ru_RU
% About magic macroses see also  
% https://tex.stackexchange.com/questions/78101/ 

% По умолчанию используется шрифт 14 размера. Если нужен 12-й шрифт, уберите опцию [14pt]
\documentclass[14pt
  , russian
  %, xcolor={svgnames}
  ]{matmex-diploma-custom}
\usepackage[table]{xcolor}
\usepackage{graphicx}
\usepackage{tabularx}
\newcolumntype{Y}{>{\centering\arraybackslash}X}
\usepackage{amsmath}
\usepackage{amsthm}
\usepackage{amsfonts}
\usepackage{amssymb}
\usepackage{mathtools}
\usepackage{thmtools}
\usepackage{thm-restate}
\usepackage{tikz}
\usepackage{wrapfig}
% \usepackage[kpsewhich,newfloat]{minted}
% \usemintedstyle{vs}
\usepackage[inline]{enumitem}
\usepackage{subcaption}
\usepackage{caption}
\usepackage[nocompress]{cite}
\usepackage{makecell}
% \setitemize{noitemsep,topsep=0pt,parsep=0pt,partopsep=0pt}
% \setenumerate{noitemsep,topsep=0pt,parsep=0pt,partopsep=0pt}


\graphicspath{ {resources/} }

% 
% % \documentclass 
% %   [ a4paper        % (Predefined, but who knows...)
% %   , draft,         % Show bad things.
% %   , 12pt           % Font size.
% %   , pagesize,      % Writes the paper size at special areas in DVI or
% %                    % PDF file. Recommended for use.
% %   , parskip=half   % Paragraphs: noindent + gap.
% %   , numbers=enddot % Pointed numbers.
% %   , BCOR=5mm       % Binding size correction.
% %   , submission
% %   , copyright
% %   , creativecommons 
% %   ]{eptcs}
% % \providecommand{\event}{ML 2018}  % Name of the event you are submitting to
% % \usepackage{breakurl}             % Not needed if you use pdflatex only.
% 
% \usepackage{underscore}           % Only needed if you use pdflatex.
% 
% \usepackage{booktabs}
% \usepackage{amssymb}
% \usepackage{amsmath}
% \usepackage{mathrsfs}
% \usepackage{mathtools}
% \usepackage{multirow}
% \usepackage{indentfirst}
% \usepackage{verbatim}
% \usepackage{amsmath, amssymb}
% \usepackage{graphicx}
% \usepackage{xcolor}
% \usepackage{url}
% \usepackage{stmaryrd}
% \usepackage{xspace}
% \usepackage{comment}
% \usepackage{wrapfig}
% \usepackage[caption=false]{subfig}
% \usepackage{placeins}
% \usepackage{tabularx}
% \usepackage{ragged2e}
% \usepackage{soul}
\usepackage{csquotes}
% \usepackage{inconsolata}
% 
% \usepackage{polyglossia}   % Babel replacement for XeTeX
%   \setdefaultlanguage[spelling=modern]{russian}
%   \setotherlanguage{english}
% \usepackage{fontspec}    % Provides an automatic and unified interface 
%                          % for loading fonts.
% \usepackage{xunicode}    % Generate Unicode chars from accented glyphs.
% \usepackage{xltxtra}     % "Extras" for LaTeX users of XeTeX.
% \usepackage{xecyr}       % Help with Russian.
% 
% %% Fonts
% \defaultfontfeatures{Mapping=tex-text}
% \setmainfont{CMU Serif}
% \setsansfont{CMU Sans Serif}
% \setmonofont{CMU Typewriter Text}

\usepackage[final]{listings}

\lstdefinelanguage{ocaml}{
keywords={@type, function, fun, let, in, match, with, when, class, type,
nonrec, object, method, of, rec, repeat, until, while, not, do, done, as, val, inherit, and,
new, module, sig, deriving, datatype, struct, if, then, else, open, private, virtual, include, success, failure,
lazy, assert, true, false, end},
sensitive=true,
commentstyle=\small\itshape\ttfamily,
keywordstyle=\ttfamily\bfseries, %\underbar,
identifierstyle=\ttfamily,
basewidth={0.5em,0.5em},
columns=fixed,
fontadjust=true,
literate={->}{{$\to$}}3 {===}{{$\equiv$}}1 {=/=}{{$\not\equiv$}}1 {|>}{{$\triangleright$}}3 {\\/}{{$\vee$}}2 {/\\}{{$\wedge$}}2 {>=}{{$\ge$}}1 {<=}{{$\le$}} 1,
morecomment=[s]{(*}{*)}
}

\lstset{
mathescape=true,
%basicstyle=\small,
identifierstyle=\ttfamily,
keywordstyle=\bfseries,
commentstyle=\scriptsize\rmfamily,
basewidth={0.5em,0.5em},
fontadjust=true,
language=ocaml
}
 
\newcommand{\cd}[1]{\texttt{#1}}
\newcommand{\inbr}[1]{\left<#1\right>}


\newcolumntype{L}[1]{>{\raggedright\let\newline\\\arraybackslash\hspace{0pt}}m{#1}}
\newcolumntype{C}[1]{>{\centering\let\newline\\\arraybackslash\hspace{0pt}}m{#1}}
\newcolumntype{R}[1]{>{\raggedleft\let\newline\\\arraybackslash\hspace{0pt}}m{#1}}



\usepackage{soul}
\usepackage[normalem]{ulem}
%\sout{Hello World}

% перевод заголовков в листингах
\renewcommand\lstlistingname{Листинг}
\renewcommand\lstlistlistingname{Листинги}

\usepackage{afterpage}
\usepackage{pdflscape}
% TODO: Понять, почему я выделил то, что тут в отдельный файл
\usepackage{listings}
\usepackage{tikz}
\usetikzlibrary{decorations.pathreplacing,calc,shapes,positioning,tikzmark}

\newcounter{tmkcount}

\tikzset{
  use tikzmark/.style={
    remember picture,
    overlay,
    execute at end picture={
      \stepcounter{tmkcount}
    },
  },
  tikzmark suffix={-\thetmkcount}
}

\usepackage{totcount}

\usepackage{caption}
\usepackage{listings}

\DeclareCaptionFont{white}{ \color{white} }
\DeclareCaptionFormat{listing}{
    \parbox{\textwidth}{\hspace{15pt}#1#2#3}
}
\captionsetup[lstlisting]{ format=listing
  %, labelfont=white, textfont=white
  , singlelinecheck=false, margin=0pt, font={bf}
}

\begin{document}
%% Если что-то забыли, при компиляции будут ошибки Undefined control sequence \my@title@<что забыли>@ru
%% Если англоязычная титульная страница не нужна, то ее можно просто удалить.
\filltitle{ru}{
    %% Актуально только для курсовых/практик. ВКР защищаются не на кафедре а в ГЭК по направлению, 
    %%   и к моменту защиты вы будете уже не в группе.
    chair              = {Кафедра системного программирования},
    group              = {19Б.10-мм},
    %
    %% Макрос filltitle ненавидит пустые строки, поэтому обязателен хотя бы символ комментария на строке
    %% Актуально всем.
    title              = {Разработка алгоритма для задачи достижимости с регулярными ограничениями},
    % 
    %% Здесь указывается тип работы. Возможные значения:
    %%   coursework - отчёт по курсовой работе;
    %%   practice - отчёт по учебной практике;
    %%   prediploma - отчёт по преддипломной практике;
    %%   master - ВКР магистра;
    %%   bachelor - ВКР бакалавра.
    type               = {practice},
    %
    %% Здесь указывается вид работы. От вида работы зависят критерии оценивания.
    %%   solution - <<Решение>>. Обучающемуся поручили найти способ решения проблемы в области разработки программного обеспечения или теоретической информатики с учётом набора ограничений.
    %%   experiment - <<Эксперимент>>. Обучающемуся поручили изучить возможности, достоинства и недостатки новой технологии, платформы, языка и т. д. на примере какой-то задачи.
    %%   production - <<Производственное задание>>. Автору поручили реализовать потенциально полезное программное обеспечение.
    %%   comparison - <<Сравнение>>. Обучающемуся поручили сравнить несколько существующих продуктов и/или подходов.
    %%   theoretical - <<Теоретическое исследование>>. Автору поручили доказать какое-то утверждение, исследовать свойства алгоритма и т.п., при этом не требуя написания кода.
    kind               = {solution},
    %
    author             = {Порсев Денис Витальевич},
    % 
    %% Актуально только для ВКР. Указывается код и название направления подготовки. Типичные примеры:
    %%   02.03.03 <<Математическое обеспечение и администрирование информационных систем>>
    %%   02.04.03 <<Математическое обеспечение и администрирование информационных систем>>
    %%   09.03.04 <<Программная инженерия>>
    %%   09.04.04 <<Программная инженерия>>
    %% Те, что с 03 в середине --- бакалавриат, с 04 --- магистратура.
    specialty          = {02.03.03 <<Математическое обеспечение и администрирование информационных систем>>},
    % 
    %% Актуально только для ВКР. Указывается шифр и название образовательной программы. Типичные примеры:
    %%   СВ.5006.2017 <<Математическое обеспечение и администрирование информационных систем>>
    %%   СВ.5162.2020 <<Технологии программирования>>
    %%   СВ.5080.2017 <<Программная инженерия>>
    %%   ВМ.5665.2019 <<Математическое обеспечение и администрирование информационных систем>>
    %%   ВМ.5666.2019 <<Программная инженерия>>
    %% Шифр и название программы можно посмотреть в учебном плане, по которому вы учитесь. 
    %% СВ.* --- бакалавриат, ВМ.* --- магистратура. В конце --- год поступления (не обязательно ваш, если вы были в академе/вылетали).
    programme          = {СВ.5006.2017 <<Математическое обеспечение и администрирование информационных систем>>},
    % 
    %% Актуально только для ВКР, только для матобеса и только 2017-2018 годов поступления. Указывается профиль подготовки, на котором вы учитесь.
    %% Названия профилей можно найти в учебном плане в списке дисциплин по выбору. На каком именно вы, вам должны были сказать после второго курса (можно уточнить в студотделе).
    %% Вот возможные вариканты:
    %%   Математические основы информатики
    %%   Информационные системы и базы данных
    %%   Параллельное программирование
    %%   Системное программирование
    %%   Технология программирования
    %%   Администрирование информационных систем
    %%   Реинжиниринг программного обеспечения
    % profile            = {Системное программирование},
    % 
    %% Актуально всем.
    %supervisorPosition = {проф. каф. СП, д.ф.-м.н., проф.}, % Терехов А.Н.
    supervisorPosition = {доцент кафедры информатики, к.ф.-м.н.,}, % Григорьев С.В.
    supervisor         = {Григорьев С.В.}
    % 

    % consultantPosition = {должность ООО <<Место работы>> степень},
    % consultant         = {К.К. Консультант}
}

\maketitle
\setcounter{tocdepth}{2}
\tableofcontents

% \begin{abstract}
%   В курсаче не нужен
% \end{abstract}

\section*{Введение}
% Intro

При работе с графом зачастую требуется найти в нем множество путей от одной или нескольких вершин к другим. В случае, когда важно лишь знать о существовании этих путей, говорят, что требуется решить задачу достижимости в графе. При этом на множество искомых путей могут накладываться некоторые ограничения, заданные с помощью формальных языков. В итоге в помеченном графе достижимыми будут считаться только те вершины, путь к которым будет образовывать слова, принадлежащие заданному формальному языку.

Для формального описания свойств меток путей в графе используются регулярные и контекстно-свободные грамматики. Регулярные ограничения на пути в графе стали ключевым компонентом языков запросов к графовым базам данных~\cite{intro_rpq}. С помощью контекстно-свободных языков можно выразить более широкий класс ограничений, по этой причине они нашли широкое применение и в других областях таких, как биоинформатика~\cite{intro_bioinf}, статический анализ кода~\cite{intro_code_analysis} и др.

Несмотря на то, что регулярные и контекстно-свободные запросы к базам данных активно используются на практике, существующие алгоритмы показывают невысокую производительность на реальных входных данных, что было показано в работе \cite{intro_eval_cfpq} для контекстно-свободных грамматик и в работе \cite{intro_eval_rpq} для регулярных грамматик. В работе \cite{intro_eval_rpq} было исследовано два основных подхода к регулярным запросам и выявлено, что каждый из этих подходов эффективен только для конкретных видов графов.

В то же время, актуальным становится применение классических алгоритмов анализа графов, представленных с помощью операций линейной алгебры, исследования которых показывают их эффективность на реальных данных. Благодаря их популярности появляются стандарты~\cite{intro_standards_for_graph_algorithm_primitives} такие, как GraphBLAS~\cite{intro_graphblas}, определяющие примитивы для реализации алгоритмов в терминах линейной алгебры. Существует множество реализаций этих примитивов, которые удобно использовать для создания высокоэффективных алгоритмов. 

Как и классические алгоритмы поиска путей в графе, алгоритмы достижимости с формальными ограничениями разделяются на несколько видов. Алгоритмы для одной исходной вершины (single source), нескольких исходных вершин (multiple source) или всех вершин (all paths). Алгоритмы для нескольких вершин являются менее изученными, при этом имеют активно применяются в графовых базах данных.

Тем самым, на основе классических алгоритмов анализа графов, кажется возможным построить эффективную реализацию решения задачи достижимости с ограничениями в виде формальных языков. Разработка такого алгоритма для нескольких стартовых вершин позволит глубже исследовать производительность матричных алгоритмов для решения задачи достижимости с несколькими стартовыми вершинами, а также создать высокоэффективное решение, применимое к реальным графовым данным.

Таким образом, в рамках данной работы проводится разработка алгоритма для задачи регулярной достижимости для нескольких стартовых вершин. Главной особенностью алгоритма является то, что он основан на операциях линейной алгебры и за его основу взят один из классических алгоритмов анализа графов --- обход в ширину. Предлагается реализация алгоритма и проводится экспериментальное исследование эффективности разработанного решения.


\section{Постановка задачи}
% !TeX spellcheck = ru_RU
\label{sec:task}
Предметом данной работы является улучшение библиотеки \\ Brahma.FSharp c целью её применения в инструментах обобщённой разреженной линейной алгебры на GPGPU. Для этого были поставлены следующие задачи:
 \begin{enumerate}
 \item реализовать поддержку обобщенных атомарных операций;
 \item реализовать поддержку трансфера пользовательских типов данных;
 \item обеспечить возможность управления выделением памяти на OpenCL устройстве;
 \item обеспечить возможность параллельного исполнения OpenCL ядер;
 \item сравнить производительность полученного решения с аналогами.
 \end{enumerate}


\section{Обзор}
% Related

\label{sec:relatedworks}
В обзоре приводится основная терминология, используемая в работе. После чего рассматриваются существующие алгоритмы решающую задачу достижимости c формальными ограничениями. Подробнее разбираются алгоритмы, основанные на операциях линейной алгебры.

\subsection{Основные термины}

Определим основные объекты из теории формальных языков, которые используются в описании исследуемых и разрабатываемых алгоритмов, а также формально определим задачу достижимости с регулярными ограничениями.

\noindent\textbf{Определение}  \textit{Конечным автоматом} называется формальная система $\langle Q, \Sigma, P, Q_{src}, F \rangle$, где 
\begin{itemize}
    \item $Q$ --- конечное непустое множество состояний,
    \item $\Sigma$ --- конечный входной алфавит,
    \item $P$ --- отображение $Q \times \Sigma \rightarrow Q$,
    \item $Q_{src} \subset Q$ --- множество начальных состояний,
    \item $F$ --- множество конечных состояний
\end{itemize}

Теперь определим грамматики, используемые для задания ограничений в графе.

\noindent\textbf{Определение} \textit{Формальной грамматикой} называется четверка \\ $\langle V_N, V_T, P, S \rangle$, где
\begin{itemize}
    \item $V_N, V_T$ --- конечные и непересекающиеся алфавиты нетерминалов и терминалов соответственно, 
    \item $P$ --- конечное множество правил,
    \item $S$ --- стартовый нетерминал.
\end{itemize}

\noindent\textbf{Определение} \textit{Регулярной грамматикой} называется формальная грамматика, правила которой могут быть заданы как $A \rightarrow aB$, $A \rightarrow a$, либо как $A \rightarrow Ba$, $A \rightarrow a$ где $a \in V_T$, $A,B \in V_N$.

\subsection{Формулировка задачи достижимости}\label{sec:3.3}

Теперь сформулируем задачу регулярной достижимости (RPQ) и её частный случай для нескольких стартовых вершин.

\noindent\textbf{Определение} \textit{Задачей регулярной достижимости в графе} называется следующая задача: имея помеченный граф $D$ и регулярный язык $G$, требуется найти такое множество всех пар вершин, для которых существует хотя бы один путь от начальной вершины к конечной, что слово, полученной конкатенацией меток ребер графа $D$ будет принадлежать данному регулярному языку $G$.

Для нескольких стартовых вершин задачу можно определить разными способами. При этом в каждом из этих определений входными данными являются граф $D$ с метками на ребрах и регулярный язык, заданный грамматикой $G$. В графе выбирается некоторое множество начальных вершин $V_{src}$, из которого требуется найти достижимые вершины.

\noindent\textbf{Определение}
\label{related_task1}\textit{Постановка задачи 1}. Необходимо найти такое множество вершин графа, что для каждой вершины из этого множества существует хотя бы один путь, начало которого содержится в множестве начальных вершин. При этом метки на ребрах этого пути при конкатенации образуют слово, принадлежащее языку грамматики $G$.

Эту задачу можно переформулировать другим способом, желая найти конкретную исходную вершину для каждой достижимой вершины.

\noindent\textbf{Определение}
\label{related_task2}\textit{Постановка задачи 2}. Найти множество пар $(v, w)$ вершин, такое что $v \in V_{src}$, $w \not\in V_{src}$, существует хотя бы один путь из $v$ в $w$ такой, что метки на ребрах этого пути принадлежат языку грамматики $G$.


\subsection{Основные алгоритмы RPQ}

Можно выделить две основные группы алгоритмов RPQ: основанные на реляционной алгебре и конечных автоматах \cite{related_rpq_book}.

\subsubsection{Использование реляционной алгебры}

Этот подход анализирует регулярное выражение, содержащее информацию о запросе к графовой базе данных.

Для решения задачи достижимости используется специальный оператор реляционной алгебры, который вычисляет транзитивное замыкание на множестве вершин графа. Тогда, представив регулярное выражение в виде синтаксического дерева, можно использовать этот оператор для обхода дерева регулярного выражения и выявления достижимых вершин. При этом вычисление транзитивного замыкания и  построение дерева на его основе это трудоемкие операции, которые сказываются на эффективности алгоритма. 

\subsubsection{Использование автоматов}

Известно, что регулярное выражение можно представить в виде конечного автомата, принимающего тот же язык, что и регулярное выражение. Тогда, представив граф в виде автомата, обозначив его вершины за состояния, а ребра за переходы, можно получить два автомата, пересечение которых будет содержать информацию о достижимых вершинах. В автомате пересечения каждая пара, состоящая из начального состояния и соответствующего ему конечного состояния, будет образовывать искомое множество достижимых вершин в графе.

Для получение пар достижимых вершин не обязательно строить полный автомат пересечения, достаточно лишь устроить обход в ширину построенных изначально автоматов. Перед тем как совершить переходы в графе к новому фронту вершин путь до текущих вершин во фронте проверяется на автомате регулярного выражения. Тогда в новый фронт обхода графа попадают только те вершины, которые были приняты автоматом.

Данный подход можно оптимизировать, совершая обход одновременно по двум автоматам. Более того, известно как представить обход в ширину с помощью операций линейной алгебры на основе матричного умножения, что на больших графах может существенно ускорить описанный алгоритм.

\subsection{Алгоритмы, основанные на линейной алгебре}

К настоящему времени существует лишь несколько достаточно эффективных реализаций матричных алгоритмов поиска путей в графе с формальными ограничениями. Многие из них были разработаны для контекстно-свободных грамматик и будут разобраны в этом разделе. Для регулярных грамматик аналогичных матричных алгоритмов найдено не было.

\subsubsection{Алгоритмы для всех пар вершин}

В упомянутом~\cite{intro_eval_cfpq} сравнительном исследовании алгоритмов был исследован алгоритм Рустама Азимова~\cite{related_rustam_azimov}. За основу этого алгоритма взято вычисление транзитивного замыкания.

Алгоритм принимает на вход граф и контекстно-свободную грамматику, выраженную в ослабленной нормальной форме Хомского. Далее он оперирует над множеством булевых матриц, соответствующих каждому нетерминальному символу. Тем самым, при реализации алгоритма большое влияние на скорость алгоритма влияет использование эффективных библиотек примитивов линейной алгебры.

Помимо этого алгоритма в работе~\cite{related_kron} был разработан алгоритм основанный на тензорном произведении. Он также использует операции матричного умножения, но в отличии от предыдущего алгоритма не требует модификации изначальной контекстно-свободной грамматики. Этот алгоритм основывается на вычислении пересечения автоматов, каждый из которых выражает представление графа и грамматики соответственно. 

На вход алгоритму подается конечный автомат, описывающий сам граф, где вершины являются состояниями, а ребра описывают переходы в автомате. Вторым аргументом алгоритм получает рекурсивный автомат, описывающий ограничения на метки в графе. Известно, что благодаря тензорному произведению можно вычислить пересечение двух автоматов, именно с помощью этой операции строится новая матрица, описывающая переходы в новом автомате. После чего полученный автомат пересечения транзитивно замыкается, чтобы его переходы не содержали нетерминальных символов. Эти операции применяются в цикле для матриц смежности входных автоматов пока любая из матриц смежности автомата графа меняется.

Операции вычисления транзитивного замыкания и тензорного произведения находят пути в графе для всех пар вершин. По этой причине наивная реализация алгоритмов для нескольких стартовых вершин этими методами будет неэффективна, так как все вершины графа буду считаться начальными и для всех них посчитаются пути до других вершин. В связи с этим возникает идея отказаться от вычисления тензорного произведения и использовать классические алгоритмы обхода графа, такие как обход в ширину, для матричного представления автоматов. Для того, чтобы вычислять их пересечение, нужно синхронизировать обход в ширину для соответствующих матриц. 

\subsubsection{Алгоритмы для нескольких стартовых вершин}

Используя различные алгебраические структуры, можно модернизировать представленные алгоритмы для решения задачи достижимости с несколькими стартовыми вершинами. Так, например, можно ограничить вычисляемые пути, производя поиск только по нескольким стартовым вершинам, и не хранить информацию об остальных вершинах при обходе графа. В одной из таких работ~\cite{related_ars} была представлена модификация алгоритма Рустама Азимова, созданная для нескольких стартовых вершин и встроенная в графовую базу данных.

Из всего этого следует, что ведется активная работа в исследовании алгоритмов поиска путей в графе с формальными ограничениями. На основе уже известных алгоритмов, таких как пересечение двух автоматов графа и грамматики через их синхронный обход, кажется возможным создать эффективный алгоритм для нескольких стартовых вершин, основанный на классических алгоритмах обхода графа, используя при этом менее затратные алгебраические операции и оптимизируя его под конкретную задачу. Тем не менее, остается важным использовать специализированные матричные библиотеки для ускорения вычисления операций линейной алгебры.


% \section{Background (опционально)}
% Здесь пишется некоторая дополнительная информация о том, зачем делается то, что делается.

% Например, в работе придумывается какой-то новый метод решения формул в SMT в теориях с числами. Без каких-то дополнительных пояснений будет казаться, что работа состоит из жестокого "матана" и совсем не по теме кафедры. Поэтому, в данном разделе стоит рассказать, что все эти методы примеряются для верификации в проекте \vsharp{}, и поэтому непосредственно связаны с тематикой кафедры.


\section{Реализация}
% Method

В данном разделе представлено подробное описание разработанного алгоритма. Автором предлагается новый алгоритм, основанный на классическом матричном алгоритме обхода в ширину для нескольких стартовых вершин~\cite{method_msbfs}.
Он использует операции умножения матрицы на матрицу, применение масок к полученному результату для выделения текущего фронта обхода и другие классические матричные операций стандарта GraphBLAS. В конце рассматривается модификация этого алгоритма для решения второй поставленной задачи.

Далее представлено описание алгоритма, решающего первую поставленную задачу.

\subsection{Входные данные}
    
Алгоритм принимает на вход граф $\mathcal{G}$, детерминированный конечный автомат $\mathcal{R}$, описывающий регулярную грамматику, и множество начальных вершин $V_{src}$ графа.

Граф $\mathcal{G}$ и автомат $\mathcal{R}$ можно представить в виде булевых матриц смежности. Так, в виде словаря для каждой метки графа заводится булева матрица смежности, на месте $(i, j)$ ячейки которой стоит 1, если $i$ и $j$ вершины графа соединены ребром данной метки. Такая же операция проводится для автомата грамматики $\mathcal{R}$.

Далее, мы оперируем с двумя словарями, где ключом является символ метки ребра графа или символ алфавита автомата, а значением --- соответствующая им булевая матрица.

Для каждого символа из пересечения этих множеств строится матрица $\mathfrak{D}$, как прямая сумма булевых матриц. То есть, строится матрица $\mathfrak{D} = Bool_{\mathcal{R}_a} \bigoplus Bool_{\mathcal{G}_a}$, которая определяется как

\begin{equation}
\mathfrak{D} = 
  \left[
    \begin{matrix}
        Bool_{\mathcal{R}_a} & 0\\
        0 & Bool_{\mathcal{G}_a}
    \end{matrix}
  \right]
\end{equation}

Где $\mathcal{R}_{a}$ и $\mathcal{G}_{a}$ матрицы смежности соответствующих символов автомата грамматики $\mathcal{R}$ и графа $\mathcal{G}$ для символа $a \in A_\mathcal{R} \cap A_\mathcal{G}$, $A_\mathcal{R} \cap A_\mathcal{G}$ --- пересечение алфавитов. Такая конструкция позволяет синхронизировать алгоритм обхода в ширину одновременно для графа и грамматики.

Далее вводится матрица $M$, хранящая информацию о фронте обхода. Она нужна для выделения множества пройденных вершин и не допускает зацикливание алгоритма.
\begin{equation}
M^{k \times (k + n)} =
  \left[
    \begin{matrix}
        Id_k & Matrix_{k \times n }
    \end{matrix}
  \right]
\end{equation}

Где $Id_k$ --- единичная матрица размера $k$, $k$ --- количество вершин в автомате $\mathcal{R}$, $Matrix_{k \times n }$ --- матрица, хранящая в себе маску пройденных вершин в автомате графа, $n$ --- количество вершин в графе $\mathcal{G}$.

\subsection{Выходные данные}

На выходе строится множество $\mathcal{P}$ пар вершин $(v, w)$ графа $\mathcal{G}$ таких, что вершина $w$ достижима из множества начальных вершин, при этом $v \in V_{src}$, $w \not\in V_{src}$. Это множество представляется в виде матрицы размера $|V|\times|V|$, где $(i,j)$ ячейка содержит 1, если пара вершин с индексами $(i, j) \in \mathcal{P}$.

\subsection{Процесс обхода графа}

Алгоритм обхода заключается в последовательном умножении матрицы $M$ текущего фронта на матрицу $\mathfrak{D}$. В результате чего, находится матрица $M'$ содержащая информацию о вершинах, достижимых на следующем шаге. Далее, с помощью операций перестановки и сложения векторов $M'$ преобразуется к виду матрицы $M$ и присваивается ей. Итерации продолжаются пока $M'$ содержит новые вершины, не содержащиеся в $M$. На листинге~\ref{BFSRPQ1} представлен этот алгоритм.

\begin{algorithm}[t]
  \caption{Алгоритм достижимости в графе с регулярными ограничениями на основе поиска в ширину, выраженный с помощью операций матричного умножения}\label{BFSRPQ1}
  \begin{algorithmic}[1]
    \Procedure{BFSBasedRPQ}{$\mathcal{R}=\langle Q, \Sigma, P, F, q \rangle,\mathcal{G}=\langle V, E, L \rangle, V_{src}$}
    \State $\mathcal{P}\gets~${Матрица смежности графа}
    \State $\mathfrak{D}\gets Bool_\mathcal{R} \bigoplus Bool_\mathcal{G}$\Comment{Построение матриц $\mathfrak{D}$}
    \State $M\gets CreateMasks(|Q|,|V|)$ \Comment{Построение матрицы $M$}
    \State $M'\gets SetStartVerts(M, V_{src})$  \Comment{Заполнение нач. вершин}
    
    \While{Матрица~$M$~меняется}{}
      \State $M\gets M'\langle\neg M\rangle$\Comment{Применение комплементарной маски}
      \ForAll{$a\in (\Sigma \cap L)$}
        \State $M'\gets M~$any.pair$~\mathfrak{D}$
        \Comment{Матр. умножение в полукольце}
        \State $M'\gets TransformRows(M')$\label{TransformRows}
        \Comment{Приведение $M'$ к виду $M$}
      \EndFor
        \State {$Matrix\gets extractRightSubMatrix(M')$}
        \State $V\gets Matrix.reduceVector()$ \Comment{Сложение по столбцам}
        \For{$k \in 0\dots|V_{src}|-1$}
            \State $W\gets\mathcal{P}.getRow(k)$
            \State $\mathcal{P}.setRow(k, V+W)$
      \EndFor
    \EndWhile
    \State \textbf{return} $\mathcal{P}$
    \EndProcedure
  \end{algorithmic}
\end{algorithm}

В алгоритме~\ref{BFSRPQ1}, в~\ref{TransformRows} строке происходит трансформация строчек в матрице $M'$. Это делается для того, чтобы представить полученную во время обхода матрицу $M'$, содержащую новый фронт, в виде матрицы $M$. Для этого требуется так переставить строчки $M'$, чтобы она содержала корректные по своему определению значения. То есть, имела единицы на главной диагонали, а все остальные значения в первых $k$ столбцах были нулями. Подробнее эта процедура описана в листинге~\ref{AlgoTransformRows}.

\begin{algorithm}[H]
  \caption{Алгоритм трансформации строчек}\label{AlgoTransformRows}
  \begin{algorithmic}[1]
    \Procedure{TransformRows}{$M$}
        \State{$T \gets extractLeftSubMatrix(M)$}
        \State{$Ix, Iy \gets$ итераторы по индексам ненулевых элементов $T$}
        \For{$i \in 0\dots|Iy|$}
            \State{$R\gets M.getRow(Ix[i])$}
            \State{$M'.setRow(Iy[i], R + M'.getRow(Iy[i]))$}
        \EndFor
    \EndProcedure
  \end{algorithmic}
\end{algorithm}

\pagebreak

\subsection{Модификации алгоритма}

Рассмотрим $V_{src}$ --- множество начальных вершин, состоящее из $r$ элементов. Для каждой начальной вершины $v_{src}^i \in V_{src}$ отметим соответствующие индексы в матрице $M$ единицами, получив матрицу $M(v_{src}^i)$,  и построим матрицу $\mathfrak{M}$ следующим образом.

\begin{equation}
\mathfrak{M}^{(k*r) \times (k + n)} =
  \left[
    \begin{matrix}
        M(v_{src}^1) \\
        M(v_{src}^2) \\ 
        M(\dots) \\
        M(v_{src}^r) \\
    \end{matrix}
  \right]
\end{equation}

Матрица $\mathfrak{M}$ собирается из множества матриц $M(v_{src}^i)$ и позволяет хранить информацию о том, из какой начальной вершины достигаются новые вершины во время обхода. 

\begin{algorithm}[t]
  \caption{Модификация алгоритма для поиска конкретной исходной вершины}\label{BFSRPQ2}
  \begin{algorithmic}[1]
    \Procedure{BFSBasedRPQ}{$\mathcal{R}=\langle Q, \Sigma, P, F, q \rangle,\mathcal{G}=\langle V, E, L \rangle, V_{src}$}
    \State $\mathcal{P}\gets~${Матрица смежности графа}
    \State $\mathfrak{D}\gets Bool_\mathcal{R} \bigoplus Bool_\mathcal{G}$
    \State $\mathfrak{M}\gets CreateMasks(|Q|,|V|)$
    \State $\mathfrak{M}'\gets SetStartVerts(\mathfrak{M}, V_{src})$  
    
    \While{Матрица~$\mathfrak{M}$~меняется}{}
      \State $\mathfrak{M}\gets \mathfrak{M}'\langle\neg\mathfrak{M}\rangle$
      \ForAll{$a\in (\Sigma \cap L)$}
        \State $\mathfrak{M}'\gets \mathfrak{M}~$any.pair$~\mathfrak{D}$
        \ForAll{$M \in \mathfrak{M}'$}
            \State $M\gets TransformRows(M)$
        \EndFor
      \EndFor
      \ForAll{$M_k \in \mathfrak{M}'$}
        \State $Matrix\gets extractSubMatrix(M)$
        \State $V\gets Matrix.reduceVector()$
        \State $W\gets\mathcal{P}.getRow(k)$
        \State $\mathcal{P}.setRow(k, V+W)$
      \EndFor
    \EndWhile
    \State \textbf{return} $\mathcal{P}$
    \EndProcedure
  \end{algorithmic}
\end{algorithm}

В листинге~\ref{BFSRPQ2} представлен модифицированный алгоритм. Основное его отличие заключается в том, что для каждой достижимой вершины находится конкретная исходная вершина, из которой начинался обход.

Таким образом, алгоритмы~\ref{BFSRPQ1}~и~\ref{BFSRPQ2} решают сформулированные в пункте \ref{sec:3.3} задачи достижимости.



\section{Эксперимент}
% !TeX spellcheck = ru_RU
% !TEX root = vkr.tex

В данном разделе приведены описание и результаты экспериментов по оценке производительности предложенных решений в сравнении с аналогами, в которых реализована соответствующая функциональность. Для сравнения были выбраны библиотеки FSCL и ILGPU, поскольку они наиболее близки к предлагаемому решению по набору особенностей. Их сравнение приведено в таблице \ref{tab:comp}.

\begin{table}[h]
    \begin{tabularx}{\textwidth}{|X|c|c|c|}
      \hline
      \textbf{Характеристика} & \textbf{Brahma.FSharp} & \textbf{FSCL} & \textbf{ILGPU} \\
      \hline
      Поддержка непреобразуемых типов & Да & Да & Да \\
      Поддержка обобщенных структур данных & Да & \textit{Нет} & Да \\
      Поддержка преобразуемых типов & Да & \textit{Нет} & \textit{Нет} \\
      Поддержка произвольных атомарных операций & Да & \textit{Нет} & Да \\
      Поддержка параллельного исполнения команд & Да & Да & Да \\
      \hline
    \end{tabularx}
  \caption{Сравнение}
  \label{tab:comp}
\end{table}

\subsection{Условия эксперимента}
Сравнение поводилось с помощью библиотеки BenchmarkDotNet на ПК с операционной
системой Ubuntu 20.04 в конфигурации Intel core i7-4790 CPU,
3.6GHz, DDR4 32Gb RAM и GeForce GTX 2070 GPU, 8Gb VRAM.

\subsection{Оценка затрат на трансфер данных}
Для оценки времени, затрачиваемого на трансфер данных между хостом и OpenCL устройством, были проведены эксперименты, сравнивающие в отдельности следующие характеристики:
\begin{enumerate}
    \item запись данных в видеопамять;
    % \item чтение данных из видеопамяти;
    \item выделение памяти на OpenCL устройстве.
\end{enumerate}
Сущность экспериментов заключалась в оценке времени работы соответствующих операций применительно к некоторому массиву данных. Параметрами эксперимента послужили следующие свойства:
\begin{itemize}
    \item тип данных массива (непреобразуемый или преобразуемый);
    \item длина массива;
    \item используемая библиотека (Brahma.FSharp, FSCL или ILGPU).
\end{itemize}

\subsubsection{Запись данных в видеопамять}
Результаты эксперимента приведены в таблицах~\ref{tab:blit-w} и~\ref{tab:nonblit-w}.

\begin{table}
    \begin{tabularx}{\textwidth}{|l|l|X|X|}
      \hline
      \textbf{Библиотека} & \textbf{Длина массива} & \textbf{Среднее, мкс} & \textbf{Стандартная ошибка, мкс} \\
      \hline
      Brahma.FSharp & 1000 & 47.28  & 20.312  \\
      ILGPU & 1000 & 45.87   & 2.561   \\
      FSCL & 1000 & 1,506.0  & 152.3   \\
      \hline
      Brahma.FSharp & 1\,000\,000 & 1,214.76  & 310.448  \\
      ILGPU & 1\,000\,000 & 542.95   & 40.386  \\
      FSCL & 1\,000\,000 & 4,567.0  & 885.3   \\
      \hline
    \end{tabularx}
  \caption{Оценка времени записи данных в видеопамять, непреобразуемый тип данных, int}
  \label{tab:blit-w}
\end{table}

\begin{table}
    \begin{tabularx}{\textwidth}{|l|l|X|X|}
      \hline
      \textbf{Библиотека} & \textbf{Длина массива} & \textbf{Среднее, мкс} & \textbf{Стандартная ошибка, мкс} \\
      \hline
      Brahma.FSharp & 1000 & 49.66  & 19.347  \\
      ILGPU & 1000 & 45.80   & 2.730   \\
      FSCL & 1000 & 1,664.0 & 220.4\\
      \hline
      Brahma.FSharp & 1\,000\,000 & 5,327.74  & 1,262.829 \\
      ILGPU & 1\,000\,000 & 2,320.40   & 40.209   \\
      FSCL & 1\,000\,000 & 13,429.0  & 2,06.0   \\
      \hline
    \end{tabularx}
  \caption{Оценка времени записи данных в видеопамять, непреобразуемый тип данных, структура}
  \label{tab:blit-w}
\end{table}

\begin{table}
    \begin{tabularx}{\textwidth}{|l|l|X|X|}
      \hline
      \textbf{Библиотека} & \textbf{Длина массива} & \textbf{Среднее, мкс} & \textbf{Стандартная ошибка, мкс} \\
      \hline
      Brahma.FSharp & 1000 & 39.47 & 3.677  \\
      ILGPU & 1000 & 45.07   & 2.172   \\
      \hline
      Brahma.FSharp & 1\,000\,000 & 2,507.69  & 292.934  \\
      ILGPU & 1\,000\,000 & 1,082.90 & 94.608 \\
      \hline
    \end{tabularx}
  \caption{Оценка времени записи данных в видеопамять, непреобразуемый тип данных, ValueOption<int>}
  \label{tab:blit-w}
\end{table}

\begin{table}
    \begin{tabularx}{\textwidth}{|X|X|X|X|}
      \hline
      \textbf{Библиотека} & \textbf{Длина массива} & \textbf{Среднее, мкс} & \textbf{Стандартная ошибка, мкс} \\
      \hline
      Brahma.FSharp & 1000 & 2,685.18 & 418.878  \\
      \hline
      Brahma.FSharp & 1\,000\,000 & 2,044,113.55  & 48,098.780\\
      \hline
    \end{tabularx}
  \caption{Оценка времени записи данных в видеопамять, преобразуемый тип данных, bool}
  \label{tab:nonblit-w}
\end{table}

% \subsubsection{Чтение данных из видеопамяти}
% Результаты эксперимента приведены в таблицах~\ref{tab:blit-r} и~\ref{tab:nonblit-r}.

% \begin{table}
%     \begin{tabularx}{\textwidth}{|X|X|X|X|}
%       \hline
%       \textbf{Библиотека} & \textbf{Длина массива} & \textbf{Среднее, мкс} & \textbf{Стандартная ошибка, мкс} \\
%       \hline
%       Brahma.FSharp & 1000 & 29.42  & 4.730  \\
%       ILGPU & 1000 & 11.34  & 0.430  \\
%       \hline
%       Brahma.FSharp & 1\,000\,000 & 39.44   & 20.105   \\
%       ILGPU & 1\,000\,000 & 1,637.99  & 22.710   \\
%       \hline
%     \end{tabularx}
%   \caption{Оценка времени чтения данных из видеопамяти, непреобразуемый тип данных int}
%   \label{tab:blit-r}
% \end{table}

% \begin{table}
%     \begin{tabularx}{\textwidth}{|X|X|X|X|}
%       \hline
%       \textbf{Библиотека} & \textbf{Длина массива} & \textbf{Среднее, мкс} & \textbf{Стандартная ошибка, мкс} \\
%       \hline
%       Brahma.FSharp & 1000 & 29.42  & 4.730  \\
%       ILGPU & 1000 & 13.80  & 0.424  \\
%       \hline
%       Brahma.FSharp & 1\,000\,000 & 39.44   & 20.105   \\
%       ILGPU & 1\,000\,000 & 6,342.18   & 147.048   \\
%       \hline
%     \end{tabularx}
%   \caption{Оценка времени чтения данных из видеопамяти, непреобразуемый тип данных str}
%   \label{tab:blit-r}
% \end{table}

% \begin{table}
%     \begin{tabularx}{\textwidth}{|X|X|X|X|}
%       \hline
%       \textbf{Библиотека} & \textbf{Длина массива} & \textbf{Среднее, мкс} & \textbf{Стандартная ошибка, мкс} \\
%       \hline
%       Brahma.FSharp & 1000 & 29.42  & 4.730  \\
%       ILGPU & 1000 & 12.19  & 0.320  \\
%       \hline
%       Brahma.FSharp & 1\,000\,000 & 39.44   & 20.105   \\
%       ILGPU & 1\,000\,000 & 3,203.42    & 40.119 \\
%       \hline
%     \end{tabularx}
%   \caption{Оценка времени чтения данных из видеопамяти, непреобразуемый тип данных vo}
%   \label{tab:blit-r}
% \end{table}

% \begin{table}
%     \begin{tabularx}{\textwidth}{|X|X|X|X|}
%       \hline
%       \textbf{Библиотека} & \textbf{Длина массива} & \textbf{Среднее, мкс} & \textbf{Стандартная ошибка, мкс} \\
%       \hline
%       Brahma.FSharp & 1000 & 26.48  & 16.094  \\
%       \hline
%       Brahma.FSharp & 1\,000\,000 & 61.18  & 47.497  \\
%       \hline
%     \end{tabularx}
%   \caption{Оценка времени чтения данных из видеопамяти, преобразуемый тип данных bool}
%   \label{tab:nonblit-r}
% \end{table}

\subsubsection{Выделение памяти на OpenCL устройстве}
Результаты эксперимента приведены в таблицах~\ref{tab:blit-a} и~\ref{tab:nonblit-a}.

\begin{table}
    \begin{tabularx}{\textwidth}{|X|X|X|X|}
      \hline
      \textbf{Библиотека} & \textbf{Длина массива} & \textbf{Среднее, мкс} & \textbf{Стандартная ошибка, мкс} \\
      \hline
      Brahma.FSharp & 1000 & 43.11  & 11.056   \\
      ILGPU & 1000 & 43.75  & 8.165  \\
    %   FSCL & 100 &  &  \\
      \hline
      Brahma.FSharp & 1\,000\,000 & 41.61  & 15.903   \\
      ILGPU & 1\,000\,000 & 88.97   & 9.878  \\
    %   FSCL & 100\,000\,000 & &  \\
      \hline
    \end{tabularx}
  \caption{Оценка времени выделения памяти на OpenCL устройстве, непреобразуемый тип данных, int}
  \label{tab:blit-a}
\end{table}

\begin{table}
    \begin{tabularx}{\textwidth}{|X|X|X|X|}
      \hline
      \textbf{Библиотека} & \textbf{Длина массива} & \textbf{Среднее, мкс} & \textbf{Стандартная ошибка, мкс} \\
      \hline
      Brahma.FSharp & 1000 & 39.30  & 16.682    \\
      ILGPU & 1000 & 37.92  & 4.928  \\
    %   FSCL & 100 &  &  \\
      \hline
      Brahma.FSharp & 1\,000\,000 & 53.89   & 45.284   \\
      ILGPU & 1\,000\,000 & 111.61  & 26.629   \\
    %   FSCL & 100\,000\,000 & &  \\
      \hline
    \end{tabularx}
  \caption{Оценка времени выделения памяти на OpenCL устройстве, непреобразуемый тип данных, структура}
  \label{tab:blit-a}
\end{table}

\begin{table}
    \begin{tabularx}{\textwidth}{|X|X|X|X|}
      \hline
      \textbf{Библиотека} & \textbf{Длина массива} & \textbf{Среднее, мкс} & \textbf{Стандартная ошибка, мкс} \\
      \hline
      Brahma.FSharp & 1000 & 37.89  & 19.484   \\
      ILGPU & 1000 & 38.47  & 5.896 \\
    %   FSCL & 100 &  &  \\
      \hline
      Brahma.FSharp & 1\,000\,000 & 30.46  & 13.185   \\
      ILGPU & 1\,000\,000 & 92.77  & 10.086   \\
    %   FSCL & 100\,000\,000 & &  \\
      \hline
    \end{tabularx}
  \caption{Оценка времени выделения памяти на OpenCL устройстве, непреобразуемый тип данных, \text{ValueOption<int>}}
  \label{tab:blit-a}
\end{table}

\begin{table}
    \begin{tabularx}{\textwidth}{|X|X|X|X|}
      \hline
      \textbf{Библиотека} & \textbf{Длина массива} & \textbf{Среднее, мкс} & \textbf{Стандартная ошибка, мкс} \\
      \hline
      Brahma.FSharp & 1000 & 36.19 & 10.077   \\
    %   FSCL & 100 & &  \\
      \hline
      Brahma.FSharp & 1\,000\,000 &  39.71 & 10.528  \\
    %   FSCL & 100\,000\,000 & &  \\
      \hline
    \end{tabularx}
  \caption{Оценка времени выделения памяти на OpenCL устройстве, преобразуемый тип данных, bool}
  \label{tab:nonblit-a}
\end{table}

\subsubsection{Выводы}
Из приведенных результатов видно, что предлагаемое решение сравнимо по производительности c аналогичными инструментами в операциях выделения памяти и записи данных непреобразуемых типов. При этом в операциях записи данных преобразумых типов решение существенно медленнее.

\subsection{Оценка затрат на использование атомарных операций}
Для оценки временных затрат на применение атомарных операций, основанных на использовании спинлока, были проведены эксперименты, сравнивающие время выполнения такого ядра с временем выполнения ядра, использующего аналогичную нативную реализацию. Кроме того было проведено сравнение с аналогичным решением в библиотеке ILGPU. Параметром эксперимента послужил глобальный размер индексного пространства ядра. Результаты эксперимента приведены в таблице~\ref{tab:atom}.

\begin{table}
    \begin{tabularx}{\textwidth}{|l|X|l|X|}
      \hline
      \textbf{Реализация} & \textbf{Размер пространства} & \textbf{Среднее, мкс} & \textbf{Стд. ошибка, мкс} \\
      \hline
      Brahma.FSharp (нативная) & 1000 & 138.96  & 65.74  \\
      Brahma.FSharp (кастомная) & 1000 & 96.41   & 26.31  \\
      ILGPU (нативная) & 1000 & 20.40  & 8.876   \\
      ILGPU (кастомная) & 1000 & 631.24  & 88.88  \\
      \hline
      Brahma.FSharp (нативная) & 100\,000 & 126.49   & 40.55  \\
      Brahma.FSharp (кастомная) & 100\,000 & 298.12   & 42.93   \\
      ILGPU (нативная) & 100\,000 & 14.21  & 4.548   \\
      ILGPU (кастомная) & 100\,000 & 541,701.94   & 7,757.32 \\
      \hline
      Brahma.FSharp (нативная) & 10\,000\,000 & 366.2 & 45.58    \\
      Brahma.FSharp (кастомная) & 10\,000\,000 & 14,484.7  & 4,588.92    \\
      \hline
    \end{tabularx}
  \caption{Оценка затрат на использование атомарных операций}
  \label{tab:atom}
\end{table}

\subsubsection{Выводы}
Из приведенных результатов видно, что разница в производительности между нативной реализацией атомарных операций и реализацией, с использованием спинлока, в библиотеке Brahma.FSharp незначительна для малых размеров индексного пространства. Однако с ростом числа агентов, которым необходим эксклюзивный доступ к памяти, производительность сильно снижается. Предлагаемое решение производительнее аналогичного в библиотеке ILGPU и, кроме того, оно не так сильно деградирует с ростом числа агентов. По это причине производительность кастомной атомарной операции ILGPU при размере массива 10\,000\,000 установить не удалось.

% \subsection{Оценка производительности при параллельном исполнении команд}
% В настоящей работе была изменена модель параллельного исполнения команд. Для оценки производительности приведенного решения был проведен эксперимент, сравнивающий время, затраченное на параллельное и последовательное исполнение некоторого набора ядер. Кроме того, было проведено сравнение с аналогичными механизмами в библиотеке FSCL и ILGPU. Результаты эксперимента приведены в таблице~\ref{tab:parallel}.

% \begin{table}
%     \begin{tabularx}{\textwidth}{|X|X|X|}
%       \hline
%       \textbf{Библиотека} & \textbf{Среднее, мкс} & \textbf{Стандартная ошибка, мкс} \\
%       \hline
%       Brahma.FSharp & &  \\
%       ILGPU & & \\
%       FSCL & &  \\
%       \hline
%     \end{tabularx}
%   \caption{Оценка производительности при параллельном исполнении команд}
%   \label{tab:parallel}
% \end{table}

% \subsubsection{Выводы}
% Из приведенных результатов видно, что

% \subsection{Обсуждение результатов}
% ...


% \section{Применение того, что сделано на практике (опциональный)}

% Если применение в лоб не работает, потому что всё изложено чуть более сжато и теоретично, надо рассказать тонкости и правильный метод применения результатов. 

% \section{Угрозы нарушения корректности (опциональный)}

% Если основная заслуга метода, это то, что он дает лучшие цифры, то стоит сказать, где мы могли облажаться, когда
% \begin{enumerate}
% \item проводили численные замеры;
% \item выбирали тестовый набор (см. \emph{confirmation bias})
% \end{enumerate} 

\section*{Заключение}
% !TeX spellcheck = ru_RU
% !TEX root = vkr.tex

В рамках данной работы\footnote{Репозиторий проекта: \url{https://github.com/YaccConstructor/Brahma.FSharp}, имя аккаунта:
dpanfilyonok.} были получены перечисленные ниже результаты.
\begin{itemize}
\item Реализована поддержка обобщенных атомарных операций.
\item Реализована поддержка трансфера обобщенных типов данных, таких как структуры, кортежи и размеченные объединения.
\item Улучшена модель управления памятью.
\item Реализована возможность параллельного исполнения OpenCL ядер.
\item Проведено экспериментальное сравнение предложенной реализации с аналогами. 
\end{itemize}



\setmonofont[Mapping=tex-text]{CMU Typewriter Text}
  \bibliographystyle{ugost2008ls}
  \bibliography{vkr}
\end{document}
