% !TeX spellcheck = ru_RU
% !TEX root = vkr.tex

В данном разделе приведены описание и результаты экспериментов по оценке производительности предложенных решений в сравнении с аналогами, в которых реализована соответствующая функциональность. Для сравнения были выбраны библиотеки FSCL и ILGPU, поскольку они наиболее близки к предлагаемому решению по набору особенностей. Их сравнение приведено в таблице \ref{tab:comp}.

\begin{table}[h]
    \begin{tabularx}{\textwidth}{|X|c|c|c|}
      \hline
      \textbf{Характеристика} & \textbf{Brahma.FSharp} & \textbf{FSCL} & \textbf{ILGPU} \\
      \hline
      Поддержка непреобразуемых типов & Да & Да & Да \\
      Поддержка обобщенных структур данных & Да & \textit{Нет} & Да \\
      Поддержка преобразуемых типов & Да & \textit{Нет} & \textit{Нет} \\
      Поддержка произвольных атомарных операций & Да & \textit{Нет} & Да \\
      Поддержка параллельного исполнения команд & Да & Да & Да \\
      \hline
    \end{tabularx}
  \caption{Сравнение}
  \label{tab:comp}
\end{table}

\subsection{Условия эксперимента}
Сравнение поводилось с помощью библиотеки BenchmarkDotNet на ПК с операционной
системой Ubuntu 20.04 в конфигурации Intel core i7-4790 CPU,
3.6GHz, DDR4 32Gb RAM и GeForce GTX 2070 GPU, 8Gb VRAM.

\subsection{Оценка затрат на трансфер данных}
Для оценки времени, затрачиваемого на трансфер данных между хостом и OpenCL устройством, были проведены эксперименты, сравнивающие в отдельности следующие характеристики:
\begin{enumerate}
    \item запись данных в видеопамять;
    % \item чтение данных из видеопамяти;
    \item выделение памяти на OpenCL устройстве.
\end{enumerate}
Сущность экспериментов заключалась в оценке времени работы соответствующих операций применительно к некоторому массиву данных. Параметрами эксперимента послужили следующие свойства:
\begin{itemize}
    \item тип данных массива (непреобразуемый или преобразуемый);
    \item длина массива;
    \item используемая библиотека (Brahma.FSharp, FSCL или ILGPU).
\end{itemize}

\subsubsection{Запись данных в видеопамять}
Результаты эксперимента приведены в таблицах~\ref{tab:blit-w} и~\ref{tab:nonblit-w}.

\begin{table}
    \begin{tabularx}{\textwidth}{|l|l|X|X|}
      \hline
      \textbf{Библиотека} & \textbf{Длина массива} & \textbf{Среднее, мкс} & \textbf{Стандартная ошибка, мкс} \\
      \hline
      Brahma.FSharp & 1000 & 47.28  & 20.312  \\
      ILGPU & 1000 & 45.87   & 2.561   \\
      FSCL & 1000 & 1,506.0  & 152.3   \\
      \hline
      Brahma.FSharp & 1\,000\,000 & 1,214.76  & 310.448  \\
      ILGPU & 1\,000\,000 & 542.95   & 40.386  \\
      FSCL & 1\,000\,000 & 4,567.0  & 885.3   \\
      \hline
    \end{tabularx}
  \caption{Оценка времени записи данных в видеопамять, непреобразуемый тип данных, int}
  \label{tab:blit-w}
\end{table}

\begin{table}
    \begin{tabularx}{\textwidth}{|l|l|X|X|}
      \hline
      \textbf{Библиотека} & \textbf{Длина массива} & \textbf{Среднее, мкс} & \textbf{Стандартная ошибка, мкс} \\
      \hline
      Brahma.FSharp & 1000 & 49.66  & 19.347  \\
      ILGPU & 1000 & 45.80   & 2.730   \\
      FSCL & 1000 & 1,664.0 & 220.4\\
      \hline
      Brahma.FSharp & 1\,000\,000 & 5,327.74  & 1,262.829 \\
      ILGPU & 1\,000\,000 & 2,320.40   & 40.209   \\
      FSCL & 1\,000\,000 & 13,429.0  & 2,06.0   \\
      \hline
    \end{tabularx}
  \caption{Оценка времени записи данных в видеопамять, непреобразуемый тип данных, структура}
  \label{tab:blit-w}
\end{table}

\begin{table}
    \begin{tabularx}{\textwidth}{|l|l|X|X|}
      \hline
      \textbf{Библиотека} & \textbf{Длина массива} & \textbf{Среднее, мкс} & \textbf{Стандартная ошибка, мкс} \\
      \hline
      Brahma.FSharp & 1000 & 39.47 & 3.677  \\
      ILGPU & 1000 & 45.07   & 2.172   \\
      \hline
      Brahma.FSharp & 1\,000\,000 & 2,507.69  & 292.934  \\
      ILGPU & 1\,000\,000 & 1,082.90 & 94.608 \\
      \hline
    \end{tabularx}
  \caption{Оценка времени записи данных в видеопамять, непреобразуемый тип данных, ValueOption<int>}
  \label{tab:blit-w}
\end{table}

\begin{table}
    \begin{tabularx}{\textwidth}{|X|X|X|X|}
      \hline
      \textbf{Библиотека} & \textbf{Длина массива} & \textbf{Среднее, мкс} & \textbf{Стандартная ошибка, мкс} \\
      \hline
      Brahma.FSharp & 1000 & 2,685.18 & 418.878  \\
      \hline
      Brahma.FSharp & 1\,000\,000 & 2,044,113.55  & 48,098.780\\
      \hline
    \end{tabularx}
  \caption{Оценка времени записи данных в видеопамять, преобразуемый тип данных, bool}
  \label{tab:nonblit-w}
\end{table}

% \subsubsection{Чтение данных из видеопамяти}
% Результаты эксперимента приведены в таблицах~\ref{tab:blit-r} и~\ref{tab:nonblit-r}.

% \begin{table}
%     \begin{tabularx}{\textwidth}{|X|X|X|X|}
%       \hline
%       \textbf{Библиотека} & \textbf{Длина массива} & \textbf{Среднее, мкс} & \textbf{Стандартная ошибка, мкс} \\
%       \hline
%       Brahma.FSharp & 1000 & 29.42  & 4.730  \\
%       ILGPU & 1000 & 11.34  & 0.430  \\
%       \hline
%       Brahma.FSharp & 1\,000\,000 & 39.44   & 20.105   \\
%       ILGPU & 1\,000\,000 & 1,637.99  & 22.710   \\
%       \hline
%     \end{tabularx}
%   \caption{Оценка времени чтения данных из видеопамяти, непреобразуемый тип данных int}
%   \label{tab:blit-r}
% \end{table}

% \begin{table}
%     \begin{tabularx}{\textwidth}{|X|X|X|X|}
%       \hline
%       \textbf{Библиотека} & \textbf{Длина массива} & \textbf{Среднее, мкс} & \textbf{Стандартная ошибка, мкс} \\
%       \hline
%       Brahma.FSharp & 1000 & 29.42  & 4.730  \\
%       ILGPU & 1000 & 13.80  & 0.424  \\
%       \hline
%       Brahma.FSharp & 1\,000\,000 & 39.44   & 20.105   \\
%       ILGPU & 1\,000\,000 & 6,342.18   & 147.048   \\
%       \hline
%     \end{tabularx}
%   \caption{Оценка времени чтения данных из видеопамяти, непреобразуемый тип данных str}
%   \label{tab:blit-r}
% \end{table}

% \begin{table}
%     \begin{tabularx}{\textwidth}{|X|X|X|X|}
%       \hline
%       \textbf{Библиотека} & \textbf{Длина массива} & \textbf{Среднее, мкс} & \textbf{Стандартная ошибка, мкс} \\
%       \hline
%       Brahma.FSharp & 1000 & 29.42  & 4.730  \\
%       ILGPU & 1000 & 12.19  & 0.320  \\
%       \hline
%       Brahma.FSharp & 1\,000\,000 & 39.44   & 20.105   \\
%       ILGPU & 1\,000\,000 & 3,203.42    & 40.119 \\
%       \hline
%     \end{tabularx}
%   \caption{Оценка времени чтения данных из видеопамяти, непреобразуемый тип данных vo}
%   \label{tab:blit-r}
% \end{table}

% \begin{table}
%     \begin{tabularx}{\textwidth}{|X|X|X|X|}
%       \hline
%       \textbf{Библиотека} & \textbf{Длина массива} & \textbf{Среднее, мкс} & \textbf{Стандартная ошибка, мкс} \\
%       \hline
%       Brahma.FSharp & 1000 & 26.48  & 16.094  \\
%       \hline
%       Brahma.FSharp & 1\,000\,000 & 61.18  & 47.497  \\
%       \hline
%     \end{tabularx}
%   \caption{Оценка времени чтения данных из видеопамяти, преобразуемый тип данных bool}
%   \label{tab:nonblit-r}
% \end{table}

\subsubsection{Выделение памяти на OpenCL устройстве}
Результаты эксперимента приведены в таблицах~\ref{tab:blit-a} и~\ref{tab:nonblit-a}.

\begin{table}
    \begin{tabularx}{\textwidth}{|X|X|X|X|}
      \hline
      \textbf{Библиотека} & \textbf{Длина массива} & \textbf{Среднее, мкс} & \textbf{Стандартная ошибка, мкс} \\
      \hline
      Brahma.FSharp & 1000 & 43.11  & 11.056   \\
      ILGPU & 1000 & 43.75  & 8.165  \\
    %   FSCL & 100 &  &  \\
      \hline
      Brahma.FSharp & 1\,000\,000 & 41.61  & 15.903   \\
      ILGPU & 1\,000\,000 & 88.97   & 9.878  \\
    %   FSCL & 100\,000\,000 & &  \\
      \hline
    \end{tabularx}
  \caption{Оценка времени выделения памяти на OpenCL устройстве, непреобразуемый тип данных, int}
  \label{tab:blit-a}
\end{table}

\begin{table}
    \begin{tabularx}{\textwidth}{|X|X|X|X|}
      \hline
      \textbf{Библиотека} & \textbf{Длина массива} & \textbf{Среднее, мкс} & \textbf{Стандартная ошибка, мкс} \\
      \hline
      Brahma.FSharp & 1000 & 39.30  & 16.682    \\
      ILGPU & 1000 & 37.92  & 4.928  \\
    %   FSCL & 100 &  &  \\
      \hline
      Brahma.FSharp & 1\,000\,000 & 53.89   & 45.284   \\
      ILGPU & 1\,000\,000 & 111.61  & 26.629   \\
    %   FSCL & 100\,000\,000 & &  \\
      \hline
    \end{tabularx}
  \caption{Оценка времени выделения памяти на OpenCL устройстве, непреобразуемый тип данных, структура}
  \label{tab:blit-a}
\end{table}

\begin{table}
    \begin{tabularx}{\textwidth}{|X|X|X|X|}
      \hline
      \textbf{Библиотека} & \textbf{Длина массива} & \textbf{Среднее, мкс} & \textbf{Стандартная ошибка, мкс} \\
      \hline
      Brahma.FSharp & 1000 & 37.89  & 19.484   \\
      ILGPU & 1000 & 38.47  & 5.896 \\
    %   FSCL & 100 &  &  \\
      \hline
      Brahma.FSharp & 1\,000\,000 & 30.46  & 13.185   \\
      ILGPU & 1\,000\,000 & 92.77  & 10.086   \\
    %   FSCL & 100\,000\,000 & &  \\
      \hline
    \end{tabularx}
  \caption{Оценка времени выделения памяти на OpenCL устройстве, непреобразуемый тип данных, \text{ValueOption<int>}}
  \label{tab:blit-a}
\end{table}

\begin{table}
    \begin{tabularx}{\textwidth}{|X|X|X|X|}
      \hline
      \textbf{Библиотека} & \textbf{Длина массива} & \textbf{Среднее, мкс} & \textbf{Стандартная ошибка, мкс} \\
      \hline
      Brahma.FSharp & 1000 & 36.19 & 10.077   \\
    %   FSCL & 100 & &  \\
      \hline
      Brahma.FSharp & 1\,000\,000 &  39.71 & 10.528  \\
    %   FSCL & 100\,000\,000 & &  \\
      \hline
    \end{tabularx}
  \caption{Оценка времени выделения памяти на OpenCL устройстве, преобразуемый тип данных, bool}
  \label{tab:nonblit-a}
\end{table}

\subsubsection{Выводы}
Из приведенных результатов видно, что предлагаемое решение сравнимо по производительности c аналогичными инструментами в операциях выделения памяти и записи данных непреобразуемых типов. При этом в операциях записи данных преобразумых типов решение существенно медленнее.

\subsection{Оценка затрат на использование атомарных операций}
Для оценки временных затрат на применение атомарных операций, основанных на использовании спинлока, были проведены эксперименты, сравнивающие время выполнения такого ядра с временем выполнения ядра, использующего аналогичную нативную реализацию. Кроме того было проведено сравнение с аналогичным решением в библиотеке ILGPU. Параметром эксперимента послужил глобальный размер индексного пространства ядра. Результаты эксперимента приведены в таблице~\ref{tab:atom}.

\begin{table}
    \begin{tabularx}{\textwidth}{|l|X|l|X|}
      \hline
      \textbf{Реализация} & \textbf{Размер пространства} & \textbf{Среднее, мкс} & \textbf{Стд. ошибка, мкс} \\
      \hline
      Brahma.FSharp (нативная) & 1000 & 138.96  & 65.74  \\
      Brahma.FSharp (кастомная) & 1000 & 96.41   & 26.31  \\
      ILGPU (нативная) & 1000 & 20.40  & 8.876   \\
      ILGPU (кастомная) & 1000 & 631.24  & 88.88  \\
      \hline
      Brahma.FSharp (нативная) & 100\,000 & 126.49   & 40.55  \\
      Brahma.FSharp (кастомная) & 100\,000 & 298.12   & 42.93   \\
      ILGPU (нативная) & 100\,000 & 14.21  & 4.548   \\
      ILGPU (кастомная) & 100\,000 & 541,701.94   & 7,757.32 \\
      \hline
      Brahma.FSharp (нативная) & 10\,000\,000 & 366.2 & 45.58    \\
      Brahma.FSharp (кастомная) & 10\,000\,000 & 14,484.7  & 4,588.92    \\
      \hline
    \end{tabularx}
  \caption{Оценка затрат на использование атомарных операций}
  \label{tab:atom}
\end{table}

\subsubsection{Выводы}
Из приведенных результатов видно, что разница в производительности между нативной реализацией атомарных операций и реализацией, с использованием спинлока, в библиотеке Brahma.FSharp незначительна для малых размеров индексного пространства. Однако с ростом числа агентов, которым необходим эксклюзивный доступ к памяти, производительность сильно снижается. Предлагаемое решение производительнее аналогичного в библиотеке ILGPU и, кроме того, оно не так сильно деградирует с ростом числа агентов. По это причине производительность кастомной атомарной операции ILGPU при размере массива 10\,000\,000 установить не удалось.

% \subsection{Оценка производительности при параллельном исполнении команд}
% В настоящей работе была изменена модель параллельного исполнения команд. Для оценки производительности приведенного решения был проведен эксперимент, сравнивающий время, затраченное на параллельное и последовательное исполнение некоторого набора ядер. Кроме того, было проведено сравнение с аналогичными механизмами в библиотеке FSCL и ILGPU. Результаты эксперимента приведены в таблице~\ref{tab:parallel}.

% \begin{table}
%     \begin{tabularx}{\textwidth}{|X|X|X|}
%       \hline
%       \textbf{Библиотека} & \textbf{Среднее, мкс} & \textbf{Стандартная ошибка, мкс} \\
%       \hline
%       Brahma.FSharp & &  \\
%       ILGPU & & \\
%       FSCL & &  \\
%       \hline
%     \end{tabularx}
%   \caption{Оценка производительности при параллельном исполнении команд}
%   \label{tab:parallel}
% \end{table}

% \subsubsection{Выводы}
% Из приведенных результатов видно, что

% \subsection{Обсуждение результатов}
% ...
