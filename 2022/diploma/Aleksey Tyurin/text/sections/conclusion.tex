\section{Conclusion}

In the course of this work, the following results were obtained:

\begin{itemize}
    \item Approaches for fusion in different area were studied. It was shown that distillation might be a good choice for fusion automation in the area of sparse linear algebra programs.
    \item Hardware generation was implemented by combining the distiller and FHW compiler. Both the distiller and the compiler were refined to have the needed functionality.
    \item The support for passing arguments to the \texttt{main} function in FHW was added. The passing itself works under AXI-Stream protocol and supports tree-like structures.
    \item A testbench was implemented by using SystemVerilog generation and TCL scripts. It allows to query execution times and collect memory usage statistics. Configurations and results of all benchmarks are stored on GitHub. The evaluation showed the success of fusion by means of distillation and slight performance degradation from hardware generation. Future work directions were outlined.
    \item Two posters were accepted at ICFP SRC 2021, and VPT 2022 respectively.
\end{itemize}