\section{Правила работы на курсе}

Коротко зафиксируем основные правила работы на курсе: из чего формируется оценка, как сдавать домашние работы и т.д.

\subsection{Оценка за курс}

Оценка за курс складывается из баллов, полученных за работу в семестре. Баллы начисляются за следующее.
\begin{itemize}
    \item За домашние работы (балл за каждую задачу указывается отдельно). При этом у каждой работы есть жёсткий дедлайн и после него балл уменьшается вдвое.
    \item За летучки (короткие, 5-10 минут, контрольные работы). Летучка оценивается от 1 до 0 баллов с шагом 0.25. Сами по себе баллы за летучки не суммируются, но служат для корректировки баллов за домашние работы.
\end{itemize}

Итоговая оценка за курс --- это взвешенная сумма баллов за задачи, где вес --- баллы за ближайшую справа летучку.

Для трансляции баллов в оценку используется таблица~\ref{tbl:ects}.

\begin{table}[h]
    \caption{Конвертация баллов в оценки}
    \label{tbl:ects}
\begin{center}
    \begin{tabular}{ | c | c | c |}
        \hline
        Балл & ECTS & Классика \\ 
        \hline
        \hline
        91--100 & A & 5 \\  
        81--90  & B & 4 \\  
        71--80  & C & 4 \\  
        61--70  & D & 3 \\  
        51--60  & E & 3 \\  
         0--50  & F & 2 \\   
        \hline
    \end{tabular}
\end{center}
\end{table}

Решение задач можно продолжать до тех пор, пока не будет набрано достаточно баллов для получения неотрицательной оценки (с учётом понижения баллов после дедлайнов и с учётом летучек\footnote{Поэтому летучки имеет смысл писать всегда. Они могут влияют на задачи, даже если те сданы после написания летучки.}). 
Если даже при всех решённых задачах баллов не достаточно для получения неотрицательной оценки, то выдаются дополнительные задачи. 
Баллы за дополнительные задачи таковы, чтобы обеспечить разве что получение минимальной положительной оценки. 

Все текущие результаты оформляются в виде таблицы на Google Drive.
Таблица доступна всем студентам на чтение.

\subsection{Домашние задачи}

Домашние задачи представляют из себя практические задачи на программирование.
Они анонсируются по ходу семестра вместе с баллами и дедлайнами.
Дедлайн не может наступить раньше, чем через 6 дней с момента анонса задачи.

Полный балл за все задачи --- 100. После дедлайна баллы снижаются в два раза. Любая задача может быть либо зачтена полностью, либо не зачтена вообще\footnote{То есть частично решённых задач быть не может. А значит, и дробных баллов.}.

Основной язык программирования --- Python. Выбор инструментов и библиотек не ограничен. По ходу курса будут даваться некоторые рекомендации, однако не обязательно им следовать.

Задачи сдаются через GitHub. Процедура выглядит следующим образом.
\begin{enumerate}
    \item Студент создаёт репозиторий, в который приглашает ассистентов, помогающих с проверкой домашних заданий (анонсируются на первой паре). Запись об этом репозитории добавляется в таблицу с результатами. 
    \item Студенты распределяются преподавателем среди ассистентов.
    \item Репозиторий снабжается readme, системой автоматической сборки, системой автоматического тестирования, проверкой качества кода. 
    \item Новая задача решается в отдельной ветке. Задача снабжается тестами. Качество тестов --- обязанность студента. Задача может быть не принята из-за плохого тестового покрытия. После того, как все автоматические проверки пройдены, открывается реквест в основную ветку. Если и реквест прошёл все автоматические проверки, то запрашивается ревью у соответствующего ассистента. 
    \item Ассистент выполняет проверку, оставляет комментарии, выносит вердикт. Если задача зачтена, то реквест мёржится и закрывается студентом, а балл за задачу заносится в таблицу ассистентом. Иначе вносятся исправления, добавляются к этому же реквесту\footnote{Реквест не закрывать. Это позволяет отслеживать историю замечаний.}, заново запрашивается ревью\footnote{Запрашивать ревью обязательно, так как каждый коммит проверять никто не будет.}. и так до тех пор, пока задача не будет зачтена\footnote{Ну, или до тех пор, пока не отпадёт необходимость её решать.}.
\end{enumerate}

Решать задачи не обязательно в том порядке, в каком они были заданы. Однако, надо иметь ввиду, что многие задачи связаны между собой (следующая использует результаты предыдущих).

\subsection{Летучки}

Летучка --- небольшая, на 5--10 минут, проверочная работа, проводимая в начале пары. 
Летучка проверяет базовые знания: определения, теоремы, алгоритмы, свойства алгоритмов. 
Она может содержать несколько заданий, каждое из которых может быть как теоретическим (написать определение чего либо), так и практическим (показать шаги какого-то алгоритма для заданного входа).

Летучка оценивается от 0 до 1 балла с шагом в 0.25 и используется как вес для домашних работ. Оценки летучек не обсуждаются и не корректируются. 

Точная дата летучки заранее не анонсируется, однако их примерное местоположение может быть известно заранее, так как обычно они привязываются к тем или иным блокам материала.

Точное количество летучек также заранее не известно, но обычно это 4--5. Так как могут быть уважительные причины пропустить летучку, то в конце семестра за две случайно выбранные летучки с нулём баллов выставляется 0.75 баллов\footnote{Просто чтобы написать самому было лучше, чем прогулять и понадеяться на случайность.}. 

При занятиях в аудитории летучка пишется на листке бумаги, на котором, кроме решения задач, указывается ФИО студента и вариант (который назначается преподавателем). Листочки сдаются строго по истечению отведённого времени.

При занятии в удалённом формате всё происходит точно также, только сдаётся фотография (скан) листочка. Способ сдачи указывается преподавателем перед началом летучки. Время необходимое на отправку фото входит во время всей летучки\footnote{Так что будьте осторожны. Если на летучку дали 10 минут, то через 10 минут фото листочка должно быть у преподавателя. А не через 10 минут срочно начнётся поиск внезапно пропавшего телефона и т.д.}.