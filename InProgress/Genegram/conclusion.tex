We describe a new approach for RNA secondary structure prediction that employs the combination of formal grammars and neural networks. We provide an implementation, console tool Genegram, that allows to use several models designed for slightly different purposes and one can choose which model is more suitable for specific research. Genegram shows high quality and performance even compared with leading tools which makes it applicable for RNA secondary structure prediction. Moreover, our approach is quite flexible, because the final step of its pipeline is a neural network that has no limits in features presented in data, so, we can process such elements as pseudoknots, wobble base pairs, and multiplets if a proper, representable, and clean dataset is provided. And manipulating with grammar, network architecture and hyperparameters may reveal more possibilities of our approach in the secondary structure prediction filed.

We can provide several directions for future research. Firstly, all parameters tuning may lead to even better results for our models and we are planning to continue supporting and improving our tool, besides, it is possible that we will soon introduce the web application. Secondly, the current Genegram version can process only small RNA sequences having lengths up to 200 nucleotides, so, we need a solution for longer sequences processing. That requires searching for a big dataset of fully modeled high-quality long sequences and, perhaps, some architecture changes because bigger images processing is a non-trivial task demanding a lot of GPU memory. Moreover, we plan to run more experiments on RNA crystallography data from PDB database, especially concerning multiplets processing, in order to check the possibilities of our models in complicated secondary and tertiary features processing. And finally, in a long-term perspective, we want to adopt this approach for protein secondary structure prediction by similarly encoding main structural features using grammar and developing the corresponding neural network architecture.