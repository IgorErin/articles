\section{CFPQ on planar digraphs}

We want to consider algorithms for semi-dynamic transitive closure for the case of a planar graph. Our algorithm needs to support only edge insertions that do not violate planarity.

There are several algorithms that look into the problem of dynamic reachability in planar digraphs and have sublinear both update and query time~\cite{10.5555/647903.739284},~\cite{karczmarz2018data},~\cite{kao2008encyclopedia}. 

Some algorithms~\cite{10.5555/1778580.1778635},~\cite{karczmarz2018data} restrict insertions to only that edges that respect the specific embedding of the planar graph. 

\todo[inline]{maybe we can choose one embedding at random and hope that nothing will violate it. So it will be monte-carlo (why it will have good probability? where from can we get all the possible non-isomorphic embeddings?)}  

The algorithms that allow edge insertions not with respect to the embedding of the graph, but to the planarity, are described in~\cite{10.5555/647903.739284},~\cite{10.1145/300515.300517}. The main idea of both articles is the same: a planar graph is separated into \textit{clusters} (edge induced subgraphs) for which reachability is shown by their sparse substitute $S$. \textit{Sparse substitute} is a graph that contains the reachability information between some set of vertices that lie in the same face of an embedded graph.

We show the results of~\cite{10.5555/647903.739284} more closely as they differ from the results of~\cite{10.1145/300515.300517} only by a logarithmic factor, have the same idea and are easier for understanding. 

Updates and queries are the following: adding or deleting the edge between vertices $u, v$, checking reachability, checking if new edge $(u, v)$ violates planarity. 

Every update procedure modifies a constant number of clusters, for which we reconstruct their sparse substitutes. If the procedure added the edge, new edge forms its own cluster. Moreover after every $\mathcal{O}(n^{1/3})$ insertions we rebuild cluster partition and sparse substitutes from the scratch so that clusters do not lose their properties, this rebuilding time is distributed between these $\mathcal{O}(n^{1/3})$ insertions, so insertion time is amortized. 

After splitting graph into clusters and looking into its sparse substitute update or query about vertices $u, v$ can be done by placing clusters of the vertices $u, v$ in the original graph $G$ into the graph of sparse substitutes $S$.

For maintaining dynamic transitive closure in this way we only need planarity to divide graph $G$ into clusters and for each cluster to build its sparse substitute. As we rebuild cluster partition and substitutes at the beginning and after every $\mathcal{O}(n^{\frac{1}{3}})$ edge-insertions, graph needs to remain planar during the updates. 

The results are the following.
 
\begin{enumerate}
	\item The amount of time for preprocessing is $\mathcal{O}(n \log n)$ --- in this time we find the separation of the given graph $G$ into $\mathcal{O}(n^{\frac{1}{3}})$ clusters (each consists of no more than $n^{\frac{2}{3}}$ edges of $G$) and build sparse substitution graph $S$ of total size $\mathcal{O}(n^{\frac{2}{3}} \log n)$ for them.
	
	\item After that we can perform add operation in $\mathcal{O}(n^{\frac{2}{3}} \log n)$ amortized time, delete operation in $\mathcal{O}(n^{\frac{2}{3}} \log n)$ worst-case time.
	
	\item Reachability query takes $\mathcal{O}(n^{\frac{2}{3}} \log n)$ worst-case time.
	
	\item Checking if the graph is planar can be done in $\mathcal{O}(n^{\frac{1}{2}})$ amortized time. 
\end{enumerate}

Let us look closely on how can we use the power of sparse substitution on our incremental transitive closure problem.

\begin{proposition}
We can maintain the structure described in~\cite{10.5555/647903.739284} not only for planar graphs, but for planar graphs with $\mathcal{O}(n^{\frac{1}{3}})$ edges that violate planarity. 
\end{proposition}

$\square$
In~\cite{10.5555/647903.739284} the number of clusters is $\mathcal{O}(n^{\frac{1}{3}})$. In the beginning and after every $\mathcal{O}(n^{\frac{1}{3}})$ edge insertions we rebuild sparse substitute graph $S$. and in these moments we need planarity. We can additionally maintain the set of non-planar edges, each edge of this set will present its own cluster and will not take part in the rebuilding of sparse substitution. Edge is \textit{non-planar} if its insertion to current planar graph $G$ makes it non-planar.
$\hfill\blacksquare$

\begin{proposition}
We can add edges in the graph and print out new pairs of reachable vertices for every insertion in $\mathcal{O}(n^{\frac{5}{3}})$ total time for every sequence of insertions if our model allows parallel computations.
\end{proposition}

$\square$
From~\cite{10.5555/647903.739284} we can add an edge $(u, v)$ in $\mathcal{O}(n^{\frac{2}{3}}\log n)$ amortized time. We want to get all pairs of vertices that are connected through the new edge.

We can run DFS from $v$ and DFS on reversed edges from $u$ in graph $S$ of sparse substitutes. DFS runs in time proportional to the size of the graph $S$ --- $\mathcal{O}(n^{\frac{2}{3}} \log n)$. After that for every cluster in the original graph $G$ we create dummy vertex $s$ and edges from $s$ to all boundary vertices in the cluster that were reached from $u$ and $v$ respectively. Then run DFS from this dummy vertex $s$. We print out every vertex that was reached by our DFSs, so we get two lists $U$ and $V$ for beginnings and the endings of the paths, going through $(u, v)$.

If any vertex $w$ is reachable from $v$ (without loss of generation), then it is a boundary vertex or lie in some cluster and is reachable from some boundary vertex of this cluster. In both cases, one of DFS's will print it out and the algorithm is correct. 

If we can run these DFS's in parallel (there are $\mathcal{O}(n^{\frac{1}{3}})$ clusters and the same number of DFS's), then each of them will take $\mathcal{O}(n^{\frac{2}{3}})$ amount of time (all cluster have $\mathcal{O}(n^{\frac{2}{3}})$ edges by definition in~\cite{10.5555/647903.739284}). As maximum number of edges in the planar graph is linear, the total amount of time will be $\mathcal{O}(n^{\frac{5}{3}})$ for every sequence of edge insertions.
$\hfill\blacksquare$

\begin{conjecture}
If our model does not allow parallel computations total amount of time for every sequence of insertions is at most $\mathcal{O}(n^2)$ and planarity does not get us any advantage in solving the problem of dynamic reachability (in amortized time per edge and query).
\end{conjecture}

[Thoughts]
If our model can not run algorithms in parallel, then the amount of time taken by iterating DFS's described above for every cluster is linear --- $\mathcal{O}(n)$. This means that usual DFS on the original graph $G$ has the same complexity and we will spend in total $\mathcal{O}(n^2)$ time.
$\hfill\blacksquare$

\todo[inline]{maybe we can spare some space ($\mathcal{O}(n^2)$?) so we can store list of reachable vertices from any boundary vertex and somehow update them during the edge additions?}
