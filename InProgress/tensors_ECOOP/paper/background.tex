\section{Background}
\label{sec:background}
\subsection{CFL-reachability}

\subsection{Recursive State Machines}
In this work we use the notion of \textit{Finite-State Machine} (FSM).

\begin{definition}
A \textit{deterministic finite-state machine without $\varepsilon$-transitions} $T$ is a tuple $\langle \Sigma, Q, Q_s, Q_f, \delta \rangle$, where:
\begin{itemize}
    \item $\Sigma$ is an input alphabet,
    \item $Q$ is a finite set of states,
    \item $Q_s \subseteq Q$ is a set of start (or initial) states,
    \item $Q_f \subseteq Q$ is a set of final states,
    \item $\delta: Q \times \Sigma \to Q$ is a transition function.
\end{itemize}
\end{definition}

It is well known, that every regular expression can be converted to deterministic FSM without $\varepsilon$-transitions~\cite{automata:theory:10.5555/1177300}.

While a regular expression can be transformed to an FSM, a context-free grammar can be transformed to a \textit{Recursive State Machine} (RSM) in a similar fashion.
In our work, we use the following definition of RSM based
on~\cite{rsm:analysis:10.1007/3-540-44585-4_18}.

\begin{definition}
A \textit{recursive state machine} $R$ over a finite alphabet $\Sigma$ is defined as a tuple of elements $\langle B,m,\{C_i\}_{i \in B} \rangle$, where:

\begin{itemize}
    \item $B$ is a finite set of labels of boxes,
    \item $m \in B$ is an initial box label,
    \item Set of \textit{component state machines} or \textit{boxes},
          where $C_i=(\Sigma \cup B, Q_i,q_i^0,F_i,\delta_i)$:
    \begin{itemize}
        \item $\Sigma \cup B$ is a set of symbols, $\Sigma \cap B = \varnothing$,
        \item $Q_i$ is a finite set of states,
              where $Q_i \cap Q_j =  \varnothing, \forall i \neq j$,
        \item $q_i^0$ is an initial state for $C_i$,
        \item $F_i$ is a set of final states for $C_i$, where $F_i \subseteq Q_i$,
        \item $\delta_i: Q_i \times (\Sigma \cup B) \to Q_i$ is a transition function. %for $C_i$
    \end{itemize}
\end{itemize}

\end{definition}

\begin{definition}
    The \textit{size of RSM} $|R|$ is defined as the sum of the number of states in all boxes.
\end{definition}

RSM behaves as a set of finite state machines (or FSM).
Each such FSM is called a \textit{box} or a \textit{component state machine}.
A box works similarly to the classic FSM, but it also handles additional \textit{recursive calls} and employs an implicit \textit{call stack} to \textit{call} one component from another and then return execution flow back.

\subsection{Linear algebra}
\paragraph*{Graph Kronecker product and machines intersection}
\begin{definition}
Given two matrices $A$ and $B$ of sizes $m_1 \times n_1$ and $m_2 \times n_2$
respectively, with element-wise product operation $\cdot$, the Kronecker product of these two matrices is a new matrix $C = A \otimes B$ of size $m_1 * m_2 \times n_1 * n_2$ and \[C[u * m_2 + v,n_2 * p + q] = A[u,p] \cdot B[v,q].\]
\end{definition}
\begin{definition}
\label{def:graph:product}
Given two edge-labeled directed graphs $\mathcal{G}_1=\langle V_1, E_1, L_1 \rangle$
and $\mathcal{G}_2=\langle V_2, E_2, L_2 \rangle$,
the \textit{Kronecker product} of these two graphs is a edge-labeled directed graph
$\mathcal{G}=\mathcal{G}_1 \otimes \mathcal{G}_2$,
where $\mathcal{G}= \langle V, E, L \rangle$:
\begin{itemize}
    \item $V = V_1 \times V_2$
    \item $E = \{((u,v),l,(p,q)) \mid (u,l,p) \in E_1 \wedge (v,l,q) \in E_2 \}$
    \item $L = L_1 \cap L_2$
\end{itemize}
\end{definition}

The Kronecker product for graphs produces a new graph with a property
that if and only if some path $(u,v)\pi(p,q)$ exists in the result graph
then paths $u\pi_1p$ and $v\pi_2q$ exist in the input graphs,
and $\omega(\pi) = \omega(\pi_1) = \omega(\pi_2)$.
These paths $\pi_1$ and $\pi_2$ can easily be found from $\pi$ by its definition.

The Kronecker product for directed graphs can be described as
the Kronecker product of the corresponding adjacency matrices of graphs,
what gives the following definition:

\begin{definition}
\label{def:graph:adjproduct}
Given two adjacency matrices $M_1$ and $M_2$ of sizes
$m_1 \times n_1$ and $m_2 \times n_2$ respectively
for some directed graphs $\mathcal{G}_1$ and $\mathcal{G}_2$,
the \textit{Kronecker product} of these two adjacency matrices is the adjacency matrix $M$
of some graph $\mathcal{G}$, where $M$ has size $m_1 * m_2 \times n_1 * n_2$ and
\[M[u * m_2 + v,n_2 * p + q] = M_1[u,p] \cap M_2[v,q].\]
\end{definition}

By definition, the Kronecker product for adjacency matrices gives an
adjacency matrix with the same set of edges as in the resulting graph in the
Definition~\ref{def:graph:product}. Thus, $M(\mathcal{G}) = M(\mathcal{G}_1) \otimes
M(\mathcal{G}_2)$, where $\mathcal{G} = \mathcal{G}_1 \otimes \mathcal{G}_2$.

\begin{definition}
\label{def:fsm:intersection}
Given two finite state machines 
$T_1 = \langle \Sigma, Q^1, Q_S^1, Q_F^1, \delta^1 \rangle$ and \\
$T_2 = \langle \Sigma, Q^2, Q_S^2, Q_F^2, \delta^2 \rangle$, the \textit{intersection} of these two machines is a new FSM $T = \langle \Sigma, Q, Q_S, Q_F, \delta \rangle$, where:
\begin{itemize}
    \item $Q = Q^1 \times Q^2$
    \item $Q_S = Q_S^1 \times Q_S^2$
    \item $Q_F = Q_F^1 \times Q_F^2$
    \item $\delta: Q \times \Sigma \to Q$,
    $\delta (\langle q_1, q_2 \rangle, s) = \langle q_1', q_2' \rangle$, if $\delta(q_1,s)=q_1'$ and $\delta(q_2,s)=q_2'$
\end{itemize}
\end{definition}

According to~\cite{automata:theory:10.5555/1177300} an FSM intersection defines the machine for which $L(T) = L(T_1) \cap L(T_2)$.



\subsection{Pointer analysis as CFL-reachability problem}
\paragraph*{Memory alias}

\paragraph*{Points-to analysis for Java}