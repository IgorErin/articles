\section{Introduction}
\label{sec:intro}
Interprocedural static program analyses are widely applied to support the development of high-quality software. It helps developers detect potential bugs and security vulnerabilities in a program's source code. The popular approach to formualate a large body of interprocedural sratic analysis problems, such as points-to and dataflow analysis, is to use the context-free (CFL) reachability framework~\cite{10.5555/271338.271343}. The CFL-reachability problem is to find realizable paths in the graph using a context-free language. The widely used example of such context-free language is Dyck language, which treats method calls and returns as pairs of balanced parentheses.

\paragraph*{Motivation}
The CFL-reachability problem has cubic $O(n^3)$ time complexity in general case, and, despite all efforts, no algorithm faster than $O(n^3/\log n)$~\cite{10.1145/1328438.1328460} has been obtained. Therefore, the CFL-reachability is known to have a so called ``cubic bottleneck'', which is often reffered to as ``cubic bottleneck in static analysis''~\cite{10.5555/788019.788876}.
All this leads to the fact that precise CFL-reachability-based  analysis  can be expensive when applied to large programs. 

\paragraph*{Our approach}
One promising way to achieve high-performance solutions for graph analysis problems is to reduce them to linear algebra operations.
To facilitate this approach, the description of basic linear algebra primitives GraphBLAS~API~\cite{7761646} was proposed.
Evaluation of the libraries that implement this API, such as SuiteSparce~\cite{10.1145/3322125} and CombBLAS~\cite{10.1177/1094342011403516}, show that reduction to linear algebra is a good way to utilize high-performance parallel and distributed computations for graph analysis. A matrix-based approach to graph algorithms allows the graph algorithms community to leverage the decades of work in creating optimized parallel algorithms for matrix computations. Moreover, the bulk graph/matrix operations allow a serial, parallel, or GPU-based library to optimize the graph operations.

\paragraph*{Our contributions}
To summarize, we make the following contributions in this paper.
\begin{enumerate}
\item We obtain the  linear algebra based formulation of the CFL-reachability problem and show that our solution has state-of-the-art theoretical time complexity.
\item We implement the described algorithm on top of pygraphblas library, which is full implementation of GraphBLAS~API.
\item To validate scalability, high perfomance and generality, we use our tool for running CFL-reachability based alias analysis for C~\cite{Zheng:2008:DAA:1328897.1328464} and field-sensitive points-to analysis for Java~\cite{10.1145/1103845.1094817}. We analyzed large-scale software systems: ???. Our experiments show promising results: ???.
\end{enumerate}
