
\documentclass[a4paper,UKenglish,cleveref, autoref, thm-restate]{lipics-v2021}
\usepackage[ruled]{algorithm}
\usepackage{algorithmicx}
\usepackage{algpseudocode}
\usepackage{tikz}
\usetikzlibrary{fit,automata,positioning}
\usetikzlibrary{decorations.pathmorphing}
\tikzset{snake it/.style={decorate, decoration=snake}}
%This is a template for producing LIPIcs articles. 
%See lipics-v2021-authors-guidelines.pdf for further information.
%for A4 paper format use option "a4paper", for US-letter use option "letterpaper"
%for british hyphenation rules use option "UKenglish", for american hyphenation rules use option "USenglish"
%for section-numbered lemmas etc., use "numberwithinsect"
%for enabling cleveref support, use "cleveref"
%for enabling autoref support, use "autoref"
%for anonymousing the authors (e.g. for double-blind review), add "anonymous"
%for enabling thm-restate support, use "thm-restate"
%for enabling a two-column layout for the author/affilation part (only applicable for > 6 authors), use "authorcolumns"
%for producing a PDF according the PDF/A standard, add "pdfa"

%\pdfoutput=1 %uncomment to ensure pdflatex processing (mandatatory e.g. to submit to arXiv)
%\hideLIPIcs  %uncomment to remove references to LIPIcs series (logo, DOI, ...), e.g. when preparing a pre-final version to be uploaded to arXiv or another public repository

%\graphicspath{{./graphics/}}%helpful if your graphic files are in another directory

\bibliographystyle{plainurl}% the mandatory bibstyle

\title{Dummy title} %TODO Please add

%\titlerunning{Dummy short title} %TODO optional, please use if title is longer than one line

\author{Ekaterina Shemetova. {Open Access}}{Dummy University Computing Laboratory, [optional: Address], Country \and My second affiliation, Country \and \url{http://www.myhomepage.edu} }{johnqpublic@dummyuni.org}{https://orcid.org/0000-0002-1825-0097}{(Optional) author-specific funding acknowledgements}%TODO mandatory, please use full name; only 1 author per \author macro; first two parameters are mandatory, other parameters can be empty. Please provide at least the name of the affiliation and the country. The full address is optional. Use additional curly braces to indicate the correct name splitting when the last name consists of multiple name parts.

\author{Vladimir Kutuev. Public\footnote{Optional footnote, e.g. to mark corresponding author}}{Department of Informatics, Dummy College, [optional: Address], Country}{joanrpublic@dummycollege.org}{[orcid]}{[funding]}

\author{Rustam Azimov. Public\footnote{Optional footnote, e.g. to mark corresponding author}}{Department of Informatics, Dummy College, [optional: Address], Country}{joanrpublic@dummycollege.org}{[orcid]}{[funding]}

\author{Egor Orachev. Public\footnote{Optional footnote, e.g. to mark corresponding author}}{Department of Informatics, Dummy College, [optional: Address], Country}{joanrpublic@dummycollege.org}{[orcid]}{[funding]}

\author{Ilya Epelbaum. Public\footnote{Optional footnote, e.g. to mark corresponding author}}{Department of Informatics, Dummy College, [optional: Address], Country}{joanrpublic@dummycollege.org}{[orcid]}{[funding]}

\author{Semyon Grigorev. Public\footnote{Optional footnote, e.g. to mark corresponding author}}{Department of Informatics, Dummy College, [optional: Address], Country}{joanrpublic@dummycollege.org}{[orcid]}{[funding]}

\authorrunning{J. Open Access and J.\,R. Public} %TODO mandatory. First: Use abbreviated first/middle names. Second (only in severe cases): Use first author plus 'et al.'

\Copyright{Jane Open Access and Joan R. Public} %TODO mandatory, please use full first names. LIPIcs license is "CC-BY";  http://creativecommons.org/licenses/by/3.0/

\ccsdesc[100]{\textcolor{red}{Replace ccsdesc macro with valid one}} %TODO mandatory: Please choose ACM 2012 classifications from https://dl.acm.org/ccs/ccs_flat.cfm 

\keywords{CFL-reachability, interprocedural program analysis, linear algebra } %TODO mandatory; please add comma-separated list of keywords

\category{} %optional, e.g. invited paper

\relatedversion{} %optional, e.g. full version hosted on arXiv, HAL, or other respository/website
%\relatedversiondetails[linktext={opt. text shown instead of the URL}, cite=DBLP:books/mk/GrayR93]{Classification (e.g. Full Version, Extended Version, Previous Version}{URL to related version} %linktext and cite are optional

%\supplement{}%optional, e.g. related research data, source code, ... hosted on a repository like zenodo, figshare, GitHub, ...
%\supplementdetails[linktext={opt. text shown instead of the URL}, cite=DBLP:books/mk/GrayR93, subcategory={Description, Subcategory}, swhid={Software Heritage Identifier}]{General Classification (e.g. Software, Dataset, Model, ...)}{URL to related version} %linktext, cite, and subcategory are optional

%\funding{(Optional) general funding statement \dots}%optional, to capture a funding statement, which applies to all authors. Please enter author specific funding statements as fifth argument of the \author macro.

\acknowledgements{I want to thank \dots}%optional

%\nolinenumbers %uncomment to disable line numbering



%Editor-only macros:: begin (do not touch as author)%%%%%%%%%%%%%%%%%%%%%%%%%%%%%%%%%%
\EventEditors{John Q. Open and Joan R. Access}
\EventNoEds{2}
\EventLongTitle{42nd Conference on Very Important Topics (CVIT 2016)}
\EventShortTitle{CVIT 2016}
\EventAcronym{CVIT}
\EventYear{2016}
\EventDate{December 24--27, 2016}
\EventLocation{Little Whinging, United Kingdom}
\EventLogo{}
\SeriesVolume{42}
\ArticleNo{23}
%%%%%%%%%%%%%%%%%%%%%%%%%%%%%%%%%%%%%%%%%%%%%%%%%%%%%%

\begin{document}

\maketitle

%TODO mandatory: add short abstract of the document
\begin{abstract}
Lorem ipsum dolor sit amet, consectetur adipiscing elit. Praesent convallis orci arcu, eu mollis dolor. Aliquam eleifend suscipit lacinia. Maecenas quam mi, porta ut lacinia sed, convallis ac dui. Lorem ipsum dolor sit amet, consectetur adipiscing elit. Suspendisse potenti. 
\end{abstract}

\section{Introduction}

Scalable high-performance graph analysis is an actual challenge.
There is a big number of ways to attack this challenge~\cite{Coimbra2021} and the first promising idea is to utilize general-purpose graphic processing units (GPGPU-s).
Such existing solutions, as CuSha~\cite{10.1145/2600212.2600227} and Gunrock~\cite{7967137} show that utilization of GPUs can improve the performance of graph analysis, moreover it is shown that solutions may be scaled to multi-GPU systems.
But low flexibility and high complexity of API are problems of these solutions.

The second promising thing which provides a user-friendly API for high-performance graph analysis algorithms creation is a GraphBLAS API~\cite{7761646} which provides linear algebra based building blocks to create graph analysis algorithms.
The idea of GraphBLAS is based on is a well-known fact that linear algebra operations can be efficiently implemented on parallel hardware.
Along with this, a graph can be natively represented using matrices: adjacency matrix, incidence matrix, etc.
While reference CPU-based implementation of GraphBLAS, SuiteSparse:GraphBLAS~\cite{10.1145/3322125}, demonstrates good performance in real-world tasks, GPU-based implementation is challenging.

One of the challenges in this way is that real data are often sparse, thus underlying matrices and vectors are also sparse, and, as a result, classical dense data structures and respective algorithms are inefficient. 
So, it is necessary to use advanced data structures and procedures to implement sparse linear algebra, but the efficient implementation of them on GPU is hard due to the irregularity of workload and data access patterns.
Though such well-known libraries as cuSparse show that sparse linear algebra operations can be efficiently implemented for GPGPU-s, it is not so trivial to implement GraphBLAS on GPGPU. 
First of all, it requires \textit{generic} sparse linear algebra, thus it is impossible just to reuse existing libraries which are almost all specified for operations over floats.
The second problem is specific optimizations, such as maskings fusion, which can not be natively implemented on top of existing kernels.
Nevertheless, there is a number of implementations of GraphBLAS on GPGPU, such as GraphBLAST:~\cite{yang2019graphblast}, GBTL~\cite{7529957}, which show that GPGPUs utilization can improve the performance of GraphBLAS-based graph analysis solutions.
But these solutions are not portable because they are based on Nvidia Cuda stack.
Moreover, the scalability problem is not solved: all these solutions support only single-GPU, not multi-GPU computations.

To provide portable GPU implementation of GraphBLAS API we developed a \textit{SPLA} library (sources are published on GitHub: \url{https://github.com/JetBrains-Research/spla}).
This library utilizes OpenCL for GPGPU computing to be portable across devices of different vendors.
Moreover, it is initially designed to utilize multiple GPGPUs to be scalable.
To sum up, the contribution of this work is the following.
\begin{itemize}
    \item Design of portable GPU GraphBLAS implementation proposed. The design involves the utilization of multipole GPUS. Additionally, the proposed design is aimed to simplify library tuning and wrappers for different high-level platforms and languages creation. 
    \item Subset of GraphBLAS API, including such operations as masking, matrix-matrix multiplication, matrix-matrix e-wise addition, is implemented. The current implementation is limited by COO and CSR matrix representation format and uses basic algorithms for some operations, but work in progress and more data formats will be supported and advanced algorithms will be implemented in the future.
    \item Preliminary evaluation on such algorithms as breadth-first search (BFS) and triangles counting (TC), and real-world graphs shows portability across different vendors and promising performance: for some problems Spla is comparable with GraphBLAST. Surprisingly, for some problems, the proposed solution on embedded Intel graphic card shows better performance than SuiteSparse:GraphBLAS on the same CPU. At the same time, the evaluation shows that further optimization is required.
\end{itemize} 
\section{Background}
\label{sec:background}
\subsection{CFL-reachability}

\subsection{Recursive State Machines}
In this work we use the notion of \textit{Finite-State Machine} (FSM).

\begin{definition}
A \textit{deterministic finite-state machine without $\varepsilon$-transitions} $T$ is a tuple $\langle \Sigma, Q, Q_s, Q_f, \delta \rangle$, where:
\begin{itemize}
    \item $\Sigma$ is an input alphabet,
    \item $Q$ is a finite set of states,
    \item $Q_s \subseteq Q$ is a set of start (or initial) states,
    \item $Q_f \subseteq Q$ is a set of final states,
    \item $\delta: Q \times \Sigma \to Q$ is a transition function.
\end{itemize}
\end{definition}

It is well known, that every regular expression can be converted to deterministic FSM without $\varepsilon$-transitions~\cite{automata:theory:10.5555/1177300}.

While a regular expression can be transformed to an FSM, a context-free grammar can be transformed to a \textit{Recursive State Machine} (RSM) in a similar fashion.
In our work, we use the following definition of RSM based
on~\cite{rsm:analysis:10.1007/3-540-44585-4_18}.

\begin{definition}
A \textit{recursive state machine} $R$ over a finite alphabet $\Sigma$ is defined as a tuple of elements $\langle B,m,\{C_i\}_{i \in B} \rangle$, where:

\begin{itemize}
    \item $B$ is a finite set of labels of boxes,
    \item $m \in B$ is an initial box label,
    \item Set of \textit{component state machines} or \textit{boxes},
          where $C_i=(\Sigma \cup B, Q_i,q_i^0,F_i,\delta_i)$:
    \begin{itemize}
        \item $\Sigma \cup B$ is a set of symbols, $\Sigma \cap B = \varnothing$,
        \item $Q_i$ is a finite set of states,
              where $Q_i \cap Q_j =  \varnothing, \forall i \neq j$,
        \item $q_i^0$ is an initial state for $C_i$,
        \item $F_i$ is a set of final states for $C_i$, where $F_i \subseteq Q_i$,
        \item $\delta_i: Q_i \times (\Sigma \cup B) \to Q_i$ is a transition function. %for $C_i$
    \end{itemize}
\end{itemize}

\end{definition}

\begin{definition}
    The \textit{size of RSM} $|R|$ is defined as the sum of the number of states in all boxes.
\end{definition}

RSM behaves as a set of finite state machines (or FSM).
Each such FSM is called a \textit{box} or a \textit{component state machine}.
A box works similarly to the classic FSM, but it also handles additional \textit{recursive calls} and employs an implicit \textit{call stack} to \textit{call} one component from another and then return execution flow back.

\subsection{Linear algebra}
\paragraph*{Graph Kronecker product and machines intersection}
\begin{definition}
Given two matrices $A$ and $B$ of sizes $m_1 \times n_1$ and $m_2 \times n_2$
respectively, with element-wise product operation $\cdot$, the Kronecker product of these two matrices is a new matrix $C = A \otimes B$ of size $m_1 * m_2 \times n_1 * n_2$ and \[C[u * m_2 + v,n_2 * p + q] = A[u,p] \cdot B[v,q].\]
\end{definition}
\begin{definition}
\label{def:graph:product}
Given two edge-labeled directed graphs $\mathcal{G}_1=\langle V_1, E_1, L_1 \rangle$
and $\mathcal{G}_2=\langle V_2, E_2, L_2 \rangle$,
the \textit{Kronecker product} of these two graphs is a edge-labeled directed graph
$\mathcal{G}=\mathcal{G}_1 \otimes \mathcal{G}_2$,
where $\mathcal{G}= \langle V, E, L \rangle$:
\begin{itemize}
    \item $V = V_1 \times V_2$
    \item $E = \{((u,v),l,(p,q)) \mid (u,l,p) \in E_1 \wedge (v,l,q) \in E_2 \}$
    \item $L = L_1 \cap L_2$
\end{itemize}
\end{definition}

The Kronecker product for graphs produces a new graph with a property
that if and only if some path $(u,v)\pi(p,q)$ exists in the result graph
then paths $u\pi_1p$ and $v\pi_2q$ exist in the input graphs,
and $\omega(\pi) = \omega(\pi_1) = \omega(\pi_2)$.
These paths $\pi_1$ and $\pi_2$ can easily be found from $\pi$ by its definition.

The Kronecker product for directed graphs can be described as
the Kronecker product of the corresponding adjacency matrices of graphs,
what gives the following definition:

\begin{definition}
\label{def:graph:adjproduct}
Given two adjacency matrices $M_1$ and $M_2$ of sizes
$m_1 \times n_1$ and $m_2 \times n_2$ respectively
for some directed graphs $\mathcal{G}_1$ and $\mathcal{G}_2$,
the \textit{Kronecker product} of these two adjacency matrices is the adjacency matrix $M$
of some graph $\mathcal{G}$, where $M$ has size $m_1 * m_2 \times n_1 * n_2$ and
\[M[u * m_2 + v,n_2 * p + q] = M_1[u,p] \cap M_2[v,q].\]
\end{definition}

By definition, the Kronecker product for adjacency matrices gives an
adjacency matrix with the same set of edges as in the resulting graph in the
Definition~\ref{def:graph:product}. Thus, $M(\mathcal{G}) = M(\mathcal{G}_1) \otimes
M(\mathcal{G}_2)$, where $\mathcal{G} = \mathcal{G}_1 \otimes \mathcal{G}_2$.

\begin{definition}
\label{def:fsm:intersection}
Given two finite state machines 
$T_1 = \langle \Sigma, Q^1, Q_S^1, Q_F^1, \delta^1 \rangle$ and \\
$T_2 = \langle \Sigma, Q^2, Q_S^2, Q_F^2, \delta^2 \rangle$, the \textit{intersection} of these two machines is a new FSM $T = \langle \Sigma, Q, Q_S, Q_F, \delta \rangle$, where:
\begin{itemize}
    \item $Q = Q^1 \times Q^2$
    \item $Q_S = Q_S^1 \times Q_S^2$
    \item $Q_F = Q_F^1 \times Q_F^2$
    \item $\delta: Q \times \Sigma \to Q$,
    $\delta (\langle q_1, q_2 \rangle, s) = \langle q_1', q_2' \rangle$, if $\delta(q_1,s)=q_1'$ and $\delta(q_2,s)=q_2'$
\end{itemize}
\end{definition}

According to~\cite{automata:theory:10.5555/1177300} an FSM intersection defines the machine for which $L(T) = L(T_1) \cap L(T_2)$.



\subsection{Pointer analysis as CFL-reachability problem}
\paragraph*{Memory alias}

\paragraph*{Points-to analysis for Java}
\section{CFL-reachability in terms of linear algebra}
\label{sec:algo}

\subsection{Algorithm description}
The algorithm is based on the generalization of the FSM intersection for an RSM,  and the edge-labeled directed input graph.
Since the RSM is composed as a set of FSMs, it could easily be presented as an adjacency matrix for some graph over the set of labels.
As shown in the Definition~\ref{def:graph:adjproduct}, we can apply the Kronecker product for matrices to \textit{intersect} the RSM and the input graph to some extent.
But the RSM contains nonterminal symbols with the additional logic of \textit{recursive calls}, which requires a \textit{transitive closure} step to extract such symbols.

The core idea of the algorithm comes from the Kronecker product and transitive closure.
The algorithm boils down to the evaluation of the iterative Kronecker product for the adjacency matrix $\mathcal{M}_1$ of the RSM $R$ and the adjacency matrix $\mathcal{M}_2$ of the  input graph $\mathcal{G}$, followed by the transitive closure, extraction of new reachability information and updating the graph adjacency matrix $\mathcal{M}_2$. These steps are described in Algorithm~\ref{tensor:cflr}.

New elements of the Kronecker product are computed in Line 9 of the Algorithm~\ref{tensor:cflr}. Function $DTC(T, K)$ from Algorithm~\ref{tensor:cflr} takes transitive closure matrix $T$ and a matrix $K$ with edges to be inserted, maintains $T$ under edge insertions and returns pairs of vertices $(i,j)$ such that $j$ \textit {became reachable} from $i$ after the insertion of some edge from $K$. Then the new reachable pairs are validated in the lines 12-18: we are interested only in paths from start to final state of some box, therefore some pairs can be excluded from adding to $\mathcal{M}_2$. If $\mathcal{M}_2$ has changed after the insertion of the elements, we calculate the new elements of the Kronecker product and so on. 

Notice that Algorithm~\ref{tensor:cflr} naturally allows one to calculate regular reachability or FSM intersection (in this case the main while loop takes only one iteration to actually append data). This feature may be useful for regular over-approximation of the CFL-reachability, for example, when one needs to make the finding of the point-to information less expensive~\cite{10.1145/2814270.2814307, 10.1145/1103845.1094817}.

\begin{algorithm}[h]
\floatname{algorithm}{Listing}
\begin{algorithmic}[1]
\footnotesize
\caption{Kronecker product-based CFL-reachability}
\label{tensor:cflr}
\Function{LA-CFL-Reachability}{G, $\mathcal{G}$}
    % Input data preparation
    \State{$R \gets$ Recursive automata for $G$ with $r$ states}
    \State{$n \gets$ The number of vertices in $\mathcal{G}$}
    \State{$\mathcal{M}_1 \gets$ Adjacency matrix for $R$}
    \State{$\mathcal{M}_2 \gets$ Adjacency matrix for $\mathcal{G}$}
    \State{$\Delta\mathcal{M}_2 \gets \mathcal{M}_2 $}
     \State{$K, T \gets$ The empty matrices of size $rn \times rn$}
    \While{Matrix $\mathcal{M}_2$ is changing}
        % Kronecker product (i.e. partial intersection)
        \State{$K \gets \mathcal{M}_1 \otimes \Delta \mathcal{M}_2$}
        \Comment{Evaluate Kronecker product}
           \State{$\Delta \mathcal{M}_2 \gets$ The empty matrix}
        
       \State{$\Delta T \gets DTC(T, K)$}
      \Comment{Dynamic transitive closure, $\Delta T$ contains new reachable pairs}


      \For{$(i,j) \in  \Delta T $}
      \State{$s, f \gets \left\lfloor{i / r}\right\rfloor, \left\lfloor{j / r}\right\rfloor$}
      \State{$x, y \gets i \bmod n, j \bmod n$}
            \If{$s$ is start  state and $f$ is a final state for box $A$}  
          \Comment{Getting only accepting runs}
             \State{$\Delta \mathcal{M}_2[x, y] \gets \Delta \mathcal{M}_2[x, y] \cup \{A\}$}
            \EndIf
       \EndFor
        \State{$ \mathcal{M}_2 \gets \mathcal{M}_2 + \Delta \mathcal{M}_2$}
    \EndWhile
\State \Return $\mathcal{M}_2$
\EndFunction
\end{algorithmic}
\end{algorithm}


\paragraph*{Graph Kronecker product and machines intersection}

To effectively recompute the Kronecker product on each iteration, we employ the fact that it is left-distributive.
Let $\mathcal{A}_2$ be a matrix with newly added elements and $\mathcal{B}_2$ be a matrix with all previously found elements, such that $\mathcal{M}_2 = \mathcal{A}_2 + \mathcal{B}_2$.
Then by left-distributivity of the Kronecker product we have $K = \mathcal{M}_1 \otimes \mathcal{M}_2 = \mathcal{M}_1 \otimes (\mathcal{A}_2 + \mathcal{B}_2) = \mathcal{M}_1\otimes \mathcal{A}_2 + \mathcal{M}_1 \otimes \mathcal{B}_2$.
Note that $\mathcal{M}_1 \otimes \mathcal{B}_2$ is known from the previous iteration, so it is left to update some elements of $K$ by computing $\mathcal{M}_1\otimes \mathcal{A}_2$.

\paragraph*{Dynamic transitive closure}
Note that the adjacency matrix $\mathcal{M}_2$ is changed incrementally i.e. elements (edges) are added to $\mathcal{M}_2$ at each iteration of the algorithm and are never deleted from it.
So it is not necessary to recompute the whole product or transitive closure if some appropriate data structure is maintained.
The fast computation of transitive closure can be obtained by using an incremental transitive closure technique.
Let $T$ be a transitive closure matrix of the graph $\mathcal{G}$ with $n$ vertices.
We use an approach by Ibaraki and Katoh~\cite{IBARAKI198395} to maintain dynamic transitive closure.
The key idea of their algorithm is to recalculate reachability information only for those vertices which become reachable after insertion of a certain edge. For each newly inserted edge $(i, j)$ and every node $u \neq j$ of $G$ such that $T[u, i] = 1$ and $T[u, j]=0$, one needs to perform operation $T[u,v] = T[u, v] \wedge T[j, v]$ for every node $v$, where $1 \wedge 1 = 0 \wedge 0 = 1 \wedge 0 = 0$ and $0 \wedge 1 = 1$. In this way, transitive closure matrix $T$ can be maintained under edge insertions in $O(n^3)$ total time. 

We have modified this algorithm to achieve a logarithmic speed-up on a word RAM with word size $w= \theta(\log n)$. Notice that operations above are equivalent to the element-wise product of two vectors of size $n$, where multiplication operation is denoted as $\wedge$. To check whether $T[u, i] = 1$ and $T[u, j]=0$ one needs to multiply two vectors: the first vector represents reachability of the given vertex $i$ from other vertices $\{u_1, u_2, ..., u_n\}$ of the graph and the second vector represents the same for the given vertex $j$. The operation $T[u, v] \wedge T[j, v]$ also can be reduced to the computation of the element-wise product of two vectors of size $n$ for the given $u_k$. The first vector contains the information whether vertices  $\{v_1, v_2, ..., v_n\}$ of the graph are reachable from the given vertex $u_k$ and the second vector represents the same for the given vertex $j$. The element-wise product of two vectors can be calculated naively in time $O(n)$. Thus, the time complexity of the transitive closure can be reduced by speeding up the element-wise product of two vectors of size $n$.

To achieve logarithmic speed-up, we use the Four Russians' trick \cite{arlazarov1970economical}.
Let us assume an architecture with word size $w= \theta(\log n)$.
First, we split each vector into $n/\log n$ parts of size $\log n$.
Then we create a table $\mathcal{T}$ such that $\mathcal{T}(a, b)$ = $a \wedge b$ where $a, b \ \in {\{0,1\}}^{\log n}$.
This takes time $O(n^2 \log n)$, since there are $2^{\log n} = n$ variants of Boolean vectors of size $\log n$ and hence $n^2$ possible pairs of vectors $(a, b)$ in total, and each component takes $O(\log n)$ time.
Assuming constant-time logical operations on words, we can store a polynomial number of lookup tables (arrays) $\mathcal{T}_i$ (one array for each vector of size $\log n$), such that given an index of a table $\mathcal{T}_i$, and any $O(\log n)$ bit vector $b$, we can look up $\mathcal{T}_i(b)$ in constant time. The index of each array $\mathcal{T}_a$ is stored in array $\mathcal{T}$, which can be accessed in constant time for a given $\log$-size vector $a$. Thus, we can calculate the product of two parts $a$ and $b$ of size $\log n$ in constant time using the table $\mathcal{T}$.
There are $n/\log n$ such parts, so the element-wise product of two vectors of size $n$ can be calculated in time $O(n/\log n)$ with $O(n^2 \log n)$ preprocessing.



\subsection{Correctness and complexity}

\paragraph*{Correctness}
\begin{theorem}
    Let $\mathcal{G} = (V,E,L)$ be a graph and $G = \langle\Sigma, N, S, P\rangle$ be a grammar.
    Let $\mathcal{M}_{2}$ be a resulting adjacency matrix after the execution of the algorithm in Algorithm~\ref{tensor:cflr}. Then for any valid indices $i, j$ and for each nonterminal $A \in N$ the following statement holds: the non-terminal $A \in \mathcal{M}_2[i,j]$, iff there is a $A$-path from node $i$ to node $j$ in the graph $\mathcal{G}$.
\end{theorem}
\begin{proof}
    The main idea of the proof is to use induction on the height of the derivation tree obtained on each iteration.
\end{proof}


\paragraph*{Complexity}

\begin{theorem}{}
\label{theorem: subcubic}
    Let $\mathcal{G} = \langle V,E,L \rangle$ be a graph and $G = \langle\Sigma, N, S, P\rangle$ be a grammar.
    The Algorithm~\ref{tensor:cflr} calculates the resulting matrix $\mathcal{M}_2$ in $O({|P|}^3n^3/\log (|P|n))$ time on a word RAM with word size $w= \theta(\log |P|n)$, where $n = |V|$. Moreover, maintaining of the dynamic transitive closure dominates the cost of the algorithm.
\end{theorem}


\begin{proof}
The most time-consuming steps of the algorithm are the computations of the Kronecker product and transitive closure.

 Let $|\Delta\mathcal{M}_2|$ be the number of non-zero elements in a matrix $\Delta\mathcal{M}_2$. Consider the total time which is needed for computing the Kronecker products. The elements of the matrices $\Delta\mathcal{M}_2^{(i)}$ are pairwise distinct on every $i$-th iteration of the algorithm because $\Delta T$ contains only new reachable pairs of vertices.  Therefore the total number of operations is $\sum\limits_i{\text{\# of operations }(\mathcal{M}_1 \otimes \Delta\mathcal{M}_2^{(i)})} = |\mathcal{M}_1| \sum\limits_i {|\Delta\mathcal{M}_2^{(i)}|} = (|N| + |\Sigma|){|P|}^2 \sum\limits_i {|\Delta\mathcal{M}_2^{(i)}|} = O({(|N| + |\Sigma|)}^2{|P|}^2 n^2)$.

Now we derive the time complexity of maintaining the dynamic transitive closure.
Notice that $K$ has the size of the Kronecker product of $\mathcal{M}_1 \otimes \mathcal{M}_2$, which is equal to $r n \times r n = |P|n \times |P|n$ so no more than ${|P|}^2n^2$ edges will be added during all iterations of the Algorithm~\ref{tensor:cflr}.
Checking whether $T[u, i] = 1$ and $T[u, j]=0$ for every node $u \in V$ for each newly inserted edge $(i, j)$ requires one multiplication of vectors per insertion, thus total time is $O({|P|}^3n^3/\log (|P|n))$.
Note that after checking the condition, at least one element $T[u', j]$ changes value from 0 to 1 and then never becomes 0 for some $u'$ and $j$.
Therefore the operation $T[u',v] = T[u', v] \wedge T[j, v]$ for all $v \in V$ is executed at most once for every pair of vertices $(u',j)$ during the entire computation implying that the total time is equal to $O({|P|}^2n^2|P|n/\log (|P|n))=O({|P|}^3n^3/\log (|P|n))$, using the  multiplication of vectors.

The matrix $\Delta T$ contains only new elements, therefore $T$ can be updated directly using only $|\Delta T|$ operations and hence ${|P|}^2n^2$ operations in total.
The same holds for the lines 12-18 of the Algorithm~\ref{tensor:cflr}, because operations are performed only for non-zero elements of $|\Delta T|$.
Finally, the time complexity of the Algorithm~\ref{tensor:cflr} is $O({(|N| + |\Sigma|)}^2{|P|}^2 n^2) + O({|P|}^2n^2) + O({|P|}^2n^2 \log (|P|n)) + O({|P|}^3n^3/\log (|P|n)) + O({|P|}^2n^2)= O({|P|}^3n^3/\log (|P|n))$. 
\end{proof}
The complexity analysis of the Algorithm~\ref{tensor:cflr} shows that the maintaining of the incremental transitive closure dominates the cost of the algorithm. Thus, CFL-reachability can be solved in truly subcubic $O(n^{3-\varepsilon})$ time if there exists an incremental dynamic algorithm for the transitive closure for a graph with $n$ vertices with preprocessing time $O(n^{3-\varepsilon})$ and total update time $O(n^{3-\varepsilon})$. Unfortunately, such an algorithm is unlikely to exist: it was shown that there is no incremental dynamic transitive closure algorithm for a graph with $n$ vertices and at most $m$ edges with preprocessing time $poly(m)$, total update time $mn^{1-\varepsilon}$, and query time $m^{\delta-\varepsilon}$ for any $\delta \in (0, 1/2]$ per query that has an error probability of at most 1/3 assuming the widely believed Online Boolean Matrix-Vector Multiplication (OMv) Conjecture~\cite{10.1145/2746539.2746609}. OMv Conjecture states that for any constant $ \varepsilon>0$, there is no $O(n^{3-\varepsilon})$-time algorithm that solves OMv with an error probability of at most 1/3.

\section{Implementation}
\label{sec:implementation}
\subsection{SuiteSparse:GraphBLAS}

GraphBLAS~\cite{7761646} is an API specification that defines standard building blocks for graph algorithms in the language of linear algebra. SuiteSparse:GraphBLAS~\cite{10.1145/3322125} is a full implementation of the GraphBLAS standard, which defines a set of sparse matrix operations on an extended algebra of semirings using an almost unlimited variety of operators and types.
We use pygraphblas\footnote{GitHub repository of PyGraphBLAS, a Python wrapper for GraphBLAS
API: https://github.com/Graphegon/pygraphblas. Access date: 21.11.2021.}~\cite{pygraphblas}: a Python wrapper for SuiteSparse:GraphBLAS.

The building blocks of our implementation are Kronecker product and sparse matrix multiplication, which are built-in primitives of pygraphblas.
\subsection{Input representation}
GraphBLAS provides a wide range of built-in types and operators, and allows the user application to create new types and operators. In our work we use Boolean and integer matrix representation of the input.


\paragraph*{Boolean matrices}
Since RSMs and FSMs can be represented as a labeled graph, and, hence, adjacency matrix, one can represent such matrix as a set of Boolean matrices containing a single Boolean matrix for every label. For example, the adjacency matrix $\mathcal{M}_2$ of the graph from Figure~\ref{fig:example_fsm} can be represented as follows.
{\small
\begin{align*}
M_2^a =
\begin{pmatrix}
. & . & .  &.  & .  &.     \\
.& .&.  &. & 1   &.     \\
. & . & .  & . &.  &.     \\
. & .& . &. & . &.   \\
. &  . &. & .   & . &.      \\
. & . & .  & .  &. &.    \\
\end{pmatrix},~M_2^{\bar{a}} =
\begin{pmatrix}
. & . & .  &.  & .  &.     \\
.& .&.  &. & .   &.     \\
. & . & .  & . &.  &.     \\
. & .& . &. & . &.   \\
. &  1 &. & .   & . &.      \\
. & . & .  & .  &. &.    \\
\end{pmatrix},
~M_2^{d} =
\begin{pmatrix}
. & . & .  &.  & .  &.     \\
.& 1&.  &. & .   &.     \\
. & . & 1  & . &.  &.     \\
. & .& . &. & . &.   \\
. &  . &. & 1   & . &.      \\
. & . & .  & .  & 1 &.    \\
\end{pmatrix},
\end{align*}
}
{\small
\begin{align*}M_2^{\bar{d}}=
\begin{pmatrix}
. & 1 & .  &.  & .  &.     \\
.& .&.  & 1 & .   &.     \\
. & . & .  & . &.  &.     \\
. & .& . &. & 1 &.   \\
. &  . &. & .   & . & 1      \\
. & . & .  & .  &. &.    \\
\end{pmatrix}.
\end{align*}
}
 

Using Boolean adjacency matrices representation, we can reformulate the Kronecker product of such matrices.

\begin{definition}
\label{def:bool:product}
Given two sets of Boolean adjacency matrices $\mathcal{M}_1$ and $\mathcal{M}_2$, the  Kronecker product of these matrices is a new matrix
$\mathcal{M} = \mathcal{M}_1 \otimes \mathcal{M}_2$, where $\mathcal{M} = \{ M_1^a \otimes M_2^a~|~a \in \Sigma \}$ and the element-wise operation is a conjunction over Boolean values ($\wedge$).
\end{definition}

\paragraph*{Integer matrices}


\section{Evaluation}

For performance analysis of proposed solution we evaluated some most common graph algorithms using real-world sparse matrix data. 
As a baseline for comparison we chose LAGraph~\cite{szarnyas2021lagraph} in connection with SuiteSparse~\cite{10.1145/3322125} as a CPU tool, Gunrock~\cite{7967137} and GraphBLAST~\cite{yang2019graphblast} as a Nvidia GPU tools. 
Also, we tested algorithms on several devices with distinct OpenCL vendors in order to validate portability of the proposed solution. 
In general, these evaluation intentions are summarized in the following research questions. 

\vspace{0.2cm}
\begin{itemize}
    \item[\textbf{RQ1}] What is the performance of the proposed solution relative to existing tools for both CPU and GPU analysis?
    
    \item[\textbf{RQ2}] What is the portability of the proposed solution with respect to various device vendors and OpenCL runtimes?
\end{itemize}

\subsection{Evaluation Setup}

For evaluation, we use a PC with Ubuntu 20.04 installed, which has 3.40Hz Intel Core i7-6700 4-core CPU, DDR4 64Gb RAM, and Nvidia GeForce GTX 1070 GPU with 8Gb VRAM. 
Host programs were compiled with GCC 9.3.0 compiler. Programs using CUDA were compiled with GCC 8.4.0 and Nvidia NVCC 10.1.243 compiler.
Release mode and maximum optimization level was enabled for all tested programs. 
Data loading time, preparation, format transformations and host-device initial communications are excluded from time measurements. 
All tests are averaged across 10 runs.
Additional warm-up run for each test execution is excluded from measurements.

\subsection{Graph Algorithms}

For preliminary study \textit{breadth-first search} (bfs) and \textit{triangles counting} (tc) algorithms were chosen, since they allows analyse the performance of \textit{vxm} and \textit{mxm} operations, rely heavily on \textit{masking}, and utilize \textit{reduction} or \textit{assignment}. 
BFS implementation utilizes automated vector storage from sparse to dense switch and only \textit{}{push optimization}. 
TC implementation uses masked \textit{mxm} of source lower-triangular matrix with second transposed argument.

\subsection{Dataset}

Nine graph matrices were selected from the Sparse Matrix Collection at University of Florida~\cite{dataset:10.1145/2049662.2049663}. 
Information about graphs is summarized in Table~\ref{dataset:info}. 
All datasets are converted to undirected graphs. 
Self-loops and duplicated edges are removed.

\begin{table}[htbp]
\caption{Dataset description.} 
\begin{center}
    \rowcolors{2}{black!2}{black!10}
    \begin{tabular}{|l|r|r|r|}
    \hline
    Dataset & Vertices  & Edges & Max Degree \\
    \hline
    \hline
    coAuthorsCiteseer & 227.3K &   1.6M &    1372 \\
    coPapersDBLP      & 540.4K &  30.4M &    3299 \\
    hollywood-2009    &   1.1M & 113.8M &  11,467 \\
    roadNet-CA        &   1.9M &   5.5M &      12 \\
    com-Orkut         &     3M &   234M &   33313 \\
    cit-Patents       &   3.7M &  16.5M &     793 \\
    rgg\_n\_2\_22\_s0 &   4.1M &  60.7M &      36 \\
    soc-LiveJournal   &   4.8M &  68.9M &  20,333 \\
    indochina-2004    &   7.5M & 194.1M & 256,425 \\
    \hline
    \end{tabular}
    \label{dataset:info}
\end{center}
\end{table}

\subsection{Results}

Table~\ref{results} presents results of the evaluation and compares performance of Spla against other tool on different execution platforms.
Tools are grouped by the type of the device for the execution, where either Nvidia GPU or Intel CPU are used. 
Cell left empty if tested tool failed to analyse graph due to \textit{out of memory} exception.

In general, Spla BFS shows acceptable performance, especially on graphs with large vertex degree, such as soc-LiveJournal and com-Orkut.
On graphs roadNet-CA and rgg it has a significant performance drop due to the nature of underlying algorithms and data structures. 
Firstly, library utilizes immutable data buffers. Thus, iteratively updated dense vector of reached vertices must be copied for each modification, what dominates the performance of the library on a graph with large search depth. 
Secondly, Spla BFS does not utilise \textit{pull optimization}, what is critical in a graph with relatively small search frontier. 

Spla TC has a good performance on GPU, which is better in all cases that reference SuiteSparse solution. 
But in most tests GPU competitors, especially Gunrock, show smaller processing times. 
GraphBLAST shows better performance as well. 
Library utilises masked SpGEMM algorithm, the same as in GraphBLAST, but without \textit{identity} element to fill gaps. 
Library explicitly stores all non-zero elements, and uses mask to reduce only non-zero while evaluating dot products of rows and columns. 
What causes extra divergence inside work groups. 
On Intel device Spla shows better performance compared to SuiteSparse on com-Orkut, cit-Patents and soc-LiveJournal. 
A possible reason is the large lengths of processed rows and columns in the product of matrices.

Gunrock shows nearly best average performance due to its specialized and optimized algorithms.
Also, it has good time characteristics on a mentioned earlier roadNet-CA and rgg in BFS algortihm. 
GraphBLAST follows Gunrock and show good performance as well. 
But it runs out of memory on a two significantly large graphs con-Orkut and indochina-2004. 
Spla does not rut out of memory on any test due to simplified storage scheme.

\begin{table}[htbp]
\caption{Graph algorithms evaluation results.\\Time in milliseconds (lower is better).} 
\begin{center}
    \begin{tabular}{|l|r|r|r|r|r|}
    \hline
    \multirow{2}{*}{Dataset} & \multicolumn{3}{c|}{Nvidia} & \multicolumn{2}{c|}{Intel} \\
    \cline{2-6}
    & GR & GB & SP & SS & SP \\
    \hline
    \hline
    \multicolumn{6}{|c|}{BFS} \\
    \hline
    \rowcolor{black!10} hollywood-2009    &  20.3 &  82.3 &   36.9 &   23.7 &   303.4 \\
    \rowcolor{black!2 } roadNet-CA        &  33.4 & 130.8 & 1456.4 &  168.2 &   965.6 \\
    \rowcolor{black!10} soc-LiveJournal   &  60.9 &  80.6 &   90.6 &   75.2 &  1206.3 \\
    \rowcolor{black!2 } rgg\_n\_2\_22\_s0 &  98.7 & 414.9 & 4504.3 & 1215.7 & 15630.1 \\
    \rowcolor{black!10} com-Orkut         & 205.2 & -- -- &  117.9 &   43.2 &   903.6 \\
    \rowcolor{black!2 } indochina-2004    &  32.7 & -- -- &  199.6 &  227.1 &  2704.6 \\
    \hline
    \hline
    \multicolumn{6}{|c|}{TC} \\
    \hline
    \rowcolor{black!10} coAuthorsCiteseer &   2.1 &    2.0 &    9.5 &    17.5 &    64.9 \\
    \rowcolor{black!2 } coPapersDBLP      &   5.7 &   94.4 &  201.9 &   543.1 &  1537.8 \\
    \rowcolor{black!10} roadNet-CA        &  34.3 &    5.8 &   16.1 &    47.1 &   357.6 \\
    \rowcolor{black!2 } com-Orkut         & 218.1 & 1583.8 & 2407.4 & 23731.4 & 15049.5 \\
    \rowcolor{black!10} cit-Patents       &  49.7 &   52.9 &   90.6 &   698.3 &   684.1 \\
    \rowcolor{black!2 } soc-LiveJournal   &  69.1 &  449.6 &  673.9 &  4002.6 &  3823.9 \\
    \hline
    \hline
    \multicolumn{6}{l}{Tools: Gunrock (GR), GraphBLAST (GB), SuiteSparse (SS), Spla (SP).} \\
    \end{tabular}
    \label{results}
\end{center}
\end{table}
 
% Two GPU

% \begin{table}[htbp]
%     \caption{Table Type Styles}
%     \begin{center}
%     \begin{tabular}{|c|c|c|c|}
%     \hline
%     \textbf{Table}&\multicolumn{3}{|c|}{\textbf{Table Column Head}} \\
%     \cline{2-4} 
%     \textbf{Head} & \textbf{\textit{Table column subhead}}& \textbf{\textit{Subhead}}& \textbf{\textit{Subhead}} \\
%     \hline
%     copy& More table copy$^{\mathrm{a}}$& &  \\
%     \hline
%     \multicolumn{4}{l}{$^{\mathrm{a}}$Sample of a Table footnote.}
%     \end{tabular}
%     \label{tab2}
%     \end{center}
% \end{table}

\section{Related work}
\label{sec:related}
\subsection{CFL-reachability}
The CFL-reachability problem was introduced by Yannakakis~\cite{Yannakakis} to describe the Datalog chain query evaluation problem. Later, Reps et al.~\cite{10.1145/222124.222146, 10.5555/271338.271343, SAGIV1996131} proposed the CFL-reachability framework for interprocedural program analysis. Since then the CFL-reachability has been used to formulate a variety of static analyses, such as points-to and alias analysis~\cite{ 10.1145/3158118, 10.1145/2814270.2814307,  10.1145/3450492, 10.1145/3360574, 10.1007/978-3-642-37051-9_4, 10.1145/2351676.2351720, 10.1145/1103845.1094817, 10.1145/2491956.2462159, 10.1145/2660193.2660213, Zheng:2008:DAA:1328897.1328464}, data-dependence analysis~\cite{10.1145/3158118}, type inference analysis~\cite{10.1145/2647508.2647522}, type-base flow analysis~\cite{10.1145/360204.360208} and program slicing~\cite{10.1145/193173.195287}.

A cubic $O(n^3)$ algorithm for the CFL-reachability which uses dynamic programming technique, was proposed by Melski and Reps~\cite{10.1145/258994.259006}. This result was improved by a logarithmic factor by Chaudhuri~\cite{Chaudhuri2008SubcubicAF}, giving the worst-case runtime complexity $O(n^3/\log n)$. Unfortunately, no algorithm faster has been discovered, for general graphs with $n$ vertices and general context-free grammars, so the CFL-reachability is known to have a ``cubic bottleneck''~\cite{10.5555/788019.788876}. Recent result by Chatterjee et al.~\cite{10.1145/3158118} shows that the CFL-reachability in cubic time is optimal under combinatorial Boolean Matrix Multiplication (BMM) hypothesis. The cubic lower bound under the same hypothesis was also established for Andersen's Pointer Analysis directly~\cite{pavlogiannis2020finegrained}. The cubic runtime can be improved substantially in specific cases, by taking advantage of certain properties of the underlying graph (i.e. bidirected graphs)~\cite{10.1145/3158118, 10.1145/2491956.2462159} or grammar/context-free language (i.e. Dyck language of 1 parenthesis)~\cite{8249039, pavlogiannis2020finegrained}.

There are some algorithms in the context of database theory, where exists the equivalent problem called Context-Free Path Querying (CFPQ)~\cite{Azimov:2018:CPQ:3210259.3210264, Grigorev:2017:CPQ:3166094.3166104, hellingsPathQuerying, Medeiros:2018:EEC:3167132.3167265, 10.1007/978-3-030-54832-2_6, 10.1007/978-3-319-91662-0_17, 10.1145/3398682.3399163, 10.1007/978-3-319-41579-6_22}. It is important to mention that some of these algorithms reduce CFPQ evaluation to linear algebra operations: Azimov et al.~\cite{Azimov:2018:CPQ:3210259.3210264} reduce CFPQ to matrix multiplication and Orachev et al.~\cite{10.1007/978-3-030-54832-2_6} reduce CFPQ to Kronecker product. Additionally, recently Sato~\cite{sato_2017} proposed linear algebraic approach to Datalog evaluation. This approach is based on the transformation of Datalog program to a set of matrix equations, and can be used for Datalog chain queries evaluation which is equivalent to the CFL-reachability problem. Unfortunately, all three mentioned algorithms have worse than cubic $O(n^5)$ theoretical time complexity, whereas our algorithm has state-of-the-art theoretical time complexity, having all the advantages of linear algebra formulation at the same time.

\subsection{Graph processing systems}

State-of-the-art systems for large graph proccessing use different architectures including single‑machine and shared‑memory parallel ones~\cite{10.1145/3064176.3064191, 10.1145/2723372.2735369, 10.1145/2442516.2442530, Wang2013AsynchronousLG, 10.1145/2688500.2688507}, multi-core and multi-processor architectures \cite{10.1177/1094342011403516, Gregor2005ThePB, 6569865}, and distributed graph processing systems~\cite{ 10.1145/3087556.3087580, 10.1145/2621934.2621936, Jia2017ADM, Khorasani2014CuShaVG, 10.14778/2212351.2212354, 10.1145/2517349.2522740, Sengupta2016GraphInAO, 10.1145/3016078.2851145, Yan2018GraphDDV, 10.5555/1863103.1863113}. However, it is hard to use these engines for the implementation of the interprocedural program analysis tool without ground-up redesign~\cite{10.1145/3037697.3037744}.

There are many works which formulate specific graph algorithms in terms of linear algebra, for example, such algorithms as for computing transitive closure and all-pairs shortest paths.
Recently this direction was summarized in GraphBLAS API~\cite{7761646} which provides building blocks to develop a graph analysis algorithm in terms of linear algebra.
There is a number of implementations of this API, such as SuiteSparse:GraphBLAS~\cite{10.1145/3322125}, CombBLAS~\cite{10.1177/1094342011403516}, GraphBLAST~\cite{yang2019graphblast}, GraphMat~\cite{10.14778/2809974.2809983}, GraphPad~\cite{7516027}. 

We implemented our tool on top of SuiteSparse:GraphBLAS because it gives a very flexible and convenient way to construct graph algorithms by using primitive and highly-optimized building blocks based on the set of of sparse matrix operations.
\subsection{CFL-reachability-based code analysis tools}
Since CFL-reachability captures a certain sub-class of Datalog, Datalog can be employed as a domain specific language to express custom program analyses, reducing the complexity of developing program analyzers. Such Datalog-powered tools, which are able to run sophisticated static analysis include bddbddb~\cite{10.1007/11575467_8}, DOOP~\cite{10.1145/1640089.1640108}, LogicBlox~\cite{10.1145/2723372.2742796}, $\mu$Z~\cite{10.1007/978-3-642-22110-1_36}, Souffl{\'e}~\cite{10.1007/978-3-319-41540-6_23}. However, such engines are known to be fundamentally limited by the size of main memory and, therefore, are not able to scale well on a large code systems~\cite{10.1145/3453483.3454085}, and experience reduced performance compared to manually implemented tools~\cite{10.1007/978-3-319-41540-6_23}.

A single-machine, disk-based graph systems Grapple~\cite{10.1145/3302424.3303972}, Graspan~\cite{10.1145/3037697.3037744} and Chianina~\cite{10.1145/3453483.3454085} turn code analysis into bigdata analytics. The main goal of Graspan is to scale context-free CFL-reachability based analyses to large programs with disk support. A piece of work Chianina~\cite{10.1145/3453483.3454085} supports easy development of any context- and flow-sensitive analysis for C. Unfortunately, massive expensive disk I/Os remain the major performance bottleneck of disk-based graph processing.


\section{Conclusion}

In this paper we present a library for sparse Boolean linear algebra which implements such basic operations as matrix-matrix multiplication and element-wise matrix-matrix addition in both Cuda and OpenCL.
Evaluation shows that our Boolean-specific implementations faster and require less memory than generic, not the Boolean optimized, operations from state-of-the-art libraries. 
Thus, the specialization of operations for this data type makes sense. 

The first direction of the future work is to integrate all parts (OpenCL and Cuda backends) into a single library and improve its documentation and prepare to publish.
Moreover, it is necessary to extend the library with other operations, including matrix-vector operations, masking, and so on.
As a result a Python package should be published.

Another important step is to evaluate the library on different algorithms and devices.
Namely, algorithms for RPQ and CFPQ should be implemented and evaluated on related data sets.
Also, it is necessary to evaluate OpenCL version on FPGA which may require additional technical effort and code changes.

Finally, we plan to discuss with GraphBLAS community possible ways to use our library as a backend for GraphBLAST or SuiteSparse in case of Boolean computations.
Moreover, it may be possible to use implemented algorithms as a foundation for generalization to arbitrary semirings.



\bibliography{paper}
\end{document}
