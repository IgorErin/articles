\section{Implementation}
\label{sec:implementation}
\subsection{SuiteSparse:GraphBLAS}

GraphBLAS~\cite{7761646} is an API specification that defines standard building blocks for graph algorithms in the language of linear algebra. SuiteSparse:GraphBLAS~\cite{10.1145/3322125} is a full implementation of the GraphBLAS standard, which defines a set of sparse matrix operations on an extended algebra of semirings using an almost unlimited variety of operators and types.
We use pygraphblas\footnote{GitHub repository of PyGraphBLAS, a Python wrapper for GraphBLAS
API: https://github.com/Graphegon/pygraphblas. Access date: 21.11.2021.}~\cite{pygraphblas}: a Python wrapper for SuiteSparse:GraphBLAS.

The building blocks of our implementation are Kronecker product and sparse matrix multiplication, which are built-in primitives of pygraphblas.
\subsection{Input representation}
GraphBLAS provides a wide range of built-in types and operators, and allows the user application to create new types and operators. In our work we use Boolean and integer matrix representation of the input.


\paragraph*{Boolean matrices}
Since RSMs and FSMs can be represented as a labeled graph, and, hence, adjacency matrix, one can represent such matrix as a set of Boolean matrices containing a single Boolean matrix for every label. For example, the adjacency matrix $\mathcal{M}_2$ of the graph from Figure~\ref{fig:example_fsm} can be represented as follows.
{\small
\begin{align*}
M_2^a =
\begin{pmatrix}
. & . & .  &.  & .  &.     \\
.& .&.  &. & 1   &.     \\
. & . & .  & . &.  &.     \\
. & .& . &. & . &.   \\
. &  . &. & .   & . &.      \\
. & . & .  & .  &. &.    \\
\end{pmatrix},~M_2^{\bar{a}} =
\begin{pmatrix}
. & . & .  &.  & .  &.     \\
.& .&.  &. & .   &.     \\
. & . & .  & . &.  &.     \\
. & .& . &. & . &.   \\
. &  1 &. & .   & . &.      \\
. & . & .  & .  &. &.    \\
\end{pmatrix},
~M_2^{d} =
\begin{pmatrix}
. & . & .  &.  & .  &.     \\
.& 1&.  &. & .   &.     \\
. & . & 1  & . &.  &.     \\
. & .& . &. & . &.   \\
. &  . &. & 1   & . &.      \\
. & . & .  & .  & 1 &.    \\
\end{pmatrix},
\end{align*}
}
{\small
\begin{align*}M_2^{\bar{d}}=
\begin{pmatrix}
. & 1 & .  &.  & .  &.     \\
.& .&.  & 1 & .   &.     \\
. & . & .  & . &.  &.     \\
. & .& . &. & 1 &.   \\
. &  . &. & .   & . & 1      \\
. & . & .  & .  &. &.    \\
\end{pmatrix}.
\end{align*}
}
 

Using Boolean adjacency matrices representation, we can reformulate the Kronecker product of such matrices.

\begin{definition}
\label{def:bool:product}
Given two sets of Boolean adjacency matrices $\mathcal{M}_1$ and $\mathcal{M}_2$, the  Kronecker product of these matrices is a new matrix
$\mathcal{M} = \mathcal{M}_1 \otimes \mathcal{M}_2$, where $\mathcal{M} = \{ M_1^a \otimes M_2^a~|~a \in \Sigma \}$ and the element-wise operation is a conjunction over Boolean values ($\wedge$).
\end{definition}

\paragraph*{Integer matrices}

