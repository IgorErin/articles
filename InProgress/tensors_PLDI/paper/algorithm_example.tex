\subsection{An example}
\label{example:section}
In this section, we introduce a detailed example to demonstrate the steps taken by the proposed algorithms.
Namely, consider the graph $\mathcal{G}$ presented in Figure~\ref{fig:example_input_graph} and the RSM $R$ presented in Figure~\ref{example:automata}.

In the first step, we represent both the graph and RSM as a set of Boolean matrices.
Notice that we should add a new empty matrix $M_2^{S}$ to $\mathcal{M}_2$,
where edges labeled by $S$ will be added at the time of the computation.

After the initialization, the algorithm handles the $\varepsilon$-case.
The input RSM does not have any $\varepsilon$-transitions and does not have any states that are both start and final, therefore, no edges are added at this stage.
After that, we should compute $\mathcal{M}_2$ and $C_3$ iteratively.
We denote the iteration number of the loop of matrices evaluation as the number in parentheses in the subscript.

\textbf{The first iteration.} First of all, we compute the Kronecker product of the
$\mathcal{M}_1$ and $\mathcal{M}_{2,(0)}$ matrices and collapse the result to the single Boolean matrix
$M_{3,(1)}$. For the sake of simplicity, we provide only
$M_{3,(1)}$, which is evaluated as follows.
{
    \renewcommand{\arraystretch}{0.5}
    \setlength\arraycolsep{0.1pt}
\begin{align*}
  \centering
& M_{3,(1)} = M_1^a \otimes M_{2,(0)}^a +  M_1^b \otimes M_{2,(0)}^b + M_1^S \otimes M_{2,(0)}^S =\\
& \kbordermatrix{
          & (0,0) & (0,1) & \vrule & (1,0) & (1,1) & \vrule &  (2,0) & (2,1) & \vrule &  (3,0) & (3,1) &\\
    (0,0) & . & .  & \vrule & . & 1  & \vrule & . & .  &  \vrule & . & .  \\
    (0,1) & . & .  & \vrule & 1 & .   & \vrule & . & .  &  \vrule & . & .  \\
    \hline
    (1,0) & . & .   & \vrule & . & .  & \vrule & . & .  & \vrule & . & . \\
    (1,1) & . & .   & \vrule & . & .  & \vrule & . & .  & \vrule & .  &1   \\
    \hline
    (2,0) & . & .   & \vrule & . & .  & \vrule & . & .  & \vrule & . & .  \\
    (2,1) & . & .   & \vrule & . & .  & \vrule & . & .  & \vrule & . &1  \\
    \hline
    (3,0) & . & .   & \vrule & . & .  & \vrule & . & .  & \vrule & . & .  \\
    (3,1) & . & .   & \vrule & . & .  & \vrule & . & .  & \vrule & . & .  \\
}
\end{align*}
}

As far as the input graph has no edges with label $S$, the correspondent block of the Kronecker product will be empty. The Kronecker product graph of the input graph $\mathcal{G}$ and RSM $R$ is shown in Figure~\ref{fig:example_1_product}. Then, the transitive closure evaluation result, stored in the matrix $C_{3,(1)}$, introduces one new path of length 2 (the thick edges in Figure~\ref{fig:example_1_product}).

\begin{figure}[h]
    \centering
   \begin{tikzpicture}[->,auto,node distance=0.5cm]
       \node[state] (q_0)                      {$(0, 0)$};
       \node[state] (q_1) [right=of q_0] {$(1, 0)$};
       \node[state] (q_2)  [right=of q_1] {$(2, 0)$};
       \node[state] (q_3) [right=of q_2] {$(3, 0)$};
       \node[state] (q_4)  [below=of q_0] {$(0, 1)$};
       \node[state] (q_5)  [right=of q_4] {$(1, 1)$};
      \node[state] (q_6)  [right=of q_5] {$(2, 1)$};
       \node[state] (q_7)  [right=of q_6] {$(3, 1)$};
       \path[->]
        (q_0) edge[very thick]  node {} (q_5)
        (q_4) edge  node {} (q_1)
        (q_5) edge [bend left, out=40, in=140, below, very thick]  node {} (q_7)
        (q_6) edge   node {} (q_7)
        ;
    \end{tikzpicture}
    \caption{The Kronecker product graph of RSM $R$ and the input graph $\mathcal{G}$ (edges which form new paths are thick)}
    \label{fig:example_1_product}
\end{figure}

This path starts in the vertex $(0,0)$ and finishes in the vertex $(3,1)$.
We can see, that 0 and 3 are the start and final states of some component
state machine for label $S$ in $R$ respectively. Thus we can conclude that
there exists a path between vertices 0 and 1 in the graph, such that the
respective word is derivable from $S$ in the $R$ execution flow.

As a result, we can add the edge $(0,S,1)$ to the resulting graph, what is done by updating the matrix $M_2^S$.

\textbf{The second iteration.} The modified graph Boolean adjacency matrices contain
an edge with label $S$. Therefore, this label contributes to the non-empty
corresponding matrix block in the evaluated matrix $M_{3,{2}}$. The transitive closure
evaluation introduces three new paths $(0, 1) \rightarrow (2,1), (1, 0) \rightarrow (3,1)$ and $(0, 1) \rightarrow (3,1)$ (see Figure~\ref{fig:example_2_product}). Since only the path between vertices $(0,1)$ and
$(3,1)$ connects the start and final states in the automaton, the edge $(1,S,1)$ is added to the resulting graph.
{
    \renewcommand{\arraystretch}{0.5}
    \setlength\arraycolsep{0.1pt}
\begin{align*}
  \centering
& M_{3,(2)} = M_1^a \otimes M_{2,(2)}^a +  M_1^b \otimes M_{2,(2)}^b + M_1^S \otimes M_{2,(2)}^S = \\
& \kbordermatrix{
          & (0,0) & (0,1) & \vrule & (1,0) & (1,1) & \vrule &  (2,0) & (2,1) & \vrule &  (3,0) & (3,1) &\\
    (0,0) & . & .  & \vrule & . & 1  & \vrule & . & .  &  \vrule & . & .  \\
    (0,1) & . & .  & \vrule & 1 & .   & \vrule & . & .  &  \vrule & . & .  \\
    \hline
    (1,0) & . & .   & \vrule & . & .  & \vrule & . & \mc  & \vrule & . & . \\
    (1,1) & . & .   & \vrule & . & .  & \vrule & . & .  & \vrule & .  &1   \\
    \hline
    (2,0) & . & .   & \vrule & . & .  & \vrule & . & .  & \vrule & . & .  \\
    (2,1) & . & .   & \vrule & . & .  & \vrule & . & .  & \vrule & . &1  \\
    \hline
    (3,0) & . & .   & \vrule & . & .  & \vrule & . & .  & \vrule & . & .  \\
    (3,1) & . & .   & \vrule & . & .  & \vrule & . & .  & \vrule & . & .  \\
}
\end{align*}
}
\begin{figure}[h]
    \centering
   \begin{tikzpicture}[->,auto,node distance=0.5cm]
       \node[state] (q_0)                      {$(0, 0)$};
       \node[state] (q_1) [right=of q_0] {$(1, 0)$};
       \node[state] (q_2)  [right=of q_1] {$(2, 0)$};
       \node[state] (q_3) [right=of q_2] {$(3, 0)$};
       \node[state] (q_4)  [below=of q_0] {$(0, 1)$};
       \node[state] (q_5)  [right=of q_4] {$(1, 1)$};
      \node[state] (q_6)  [right=of q_5] {$(2, 1)$};
       \node[state] (q_7)  [right=of q_6] {$(3, 1)$};
       \path[->]
        (q_1) edge[very thick] node {} (q_6)
        (q_0) edge node {} (q_5)
        (q_4) edge[very thick]  node {} (q_1)
        (q_5) edge [bend left, in=140, out=40, below]  node {} (q_7)
        (q_6) edge[very thick]   node {} (q_7)
        ;
    \end{tikzpicture}
    \caption{The Kronecker product graph of RSM $R$ and the updated graph $\mathcal{G}$ (edges which form new paths are thick)}
    \label{fig:example_2_product}
\end{figure}
The result graph is presented in Figure~\ref{fig:example_result}.
\begin{figure}[h]
    \centering
    \begin{tikzpicture}[shorten >=1pt,auto]
       \node[state] (q_0)                      {$0$};
       \node[state] (q_1) [right=of q_0]       {$1$};
        \path[->]
        (q_0) edge[bend left, above]   node {a} (q_1)
         (q_0) edge[in=190, out=-10, red, very thick]   node {S} (q_1)
        (q_1) edge [bend left, below] node {a} (q_0)
         (q_1) edge[loop above, red, thick] node {S} (q_1)
        (q_1) edge[loop right] node {b} (q_1);

    \end{tikzpicture}
    \caption{The result graph $\mathcal{G}$}
    \label{fig:example_result}
\end{figure}


\begin{figure}[h]
	\centering
	\begin{tikzpicture}[->,auto,node distance=0.5cm]
		\node[state] (q_0)                      {$(0, 0)$};
		\node[state] (q_1) [right=of q_0] {$(1, 0)$};
		\node[state] (q_2)  [right=of q_1] {$(2, 0)$};
		\node[state] (q_3) [right=of q_2] {$(3, 0)$};
		\node[state] (q_4)  [below=of q_0] {$(0, 1)$};
		\node[state] (q_5)  [right=of q_4] {$(1, 1)$};
		\node[state] (q_6)  [right=of q_5] {$(2, 1)$};
		\node[state] (q_7)  [right=of q_6] {$(3, 1)$};
		\path[->]
		(q_1) edge node {} (q_6)
		(q_0) edge[very thick] node {} (q_5)
		(q_4) edge  node {} (q_1)
		(q_5) edge [bend left, in=140, out=40, below]  node {} (q_7)
		(q_5) edge[very thick]   node {} (q_6)
		(q_6) edge[very thick]   node {} (q_7)
		;
	\end{tikzpicture}
	\caption{The Kronecker product graph of RSM $R$ and the final graph $\mathcal{G}$ (edges which form new paths are thick)}
	\label{fig:example_3_product}
\end{figure}


No more edges will be added to the graph $\mathcal{G}$ at the last iteration.
However, the new edge $(1, 1) \rightarrow (2,1)$ will be added to the resulting Kronecker product, and the transitive closure
evaluation introduces one new path $(0, 0) \rightarrow (3,1)$ that connects the start and final states in the automaton (see Figure~\ref{fig:example_3_product}).
At this point, the index creation is finished.
One can use it to answer reachability queries, but it also can be used
to restore paths for some reachable vertices. The resulting Kronecker product matrix
$M_3$, or so-called \textit{index}, can be used for it.
For example, we can restore paths from vertex 1 to vertex 1 derived from $S$ in the resulting graph.

To get these paths we should call \verb|getPaths(1, 1, S)| function.
A partial trace of this call is presented in Figure~\ref{trc:example}.
First, we query paths for all possible start and final states of the
machine for the provided graph vertices.
Since the component state machine with label $S$ in the example RSM has the single final state, the function \verb|genPaths| is called with the arguments $(0,1)$ and $(3,1)$.
Note, that the values passed to the functions in the path extraction algorithm are the
pairs of the machine state and graph vertex, which uniquely identify a cell of
the index matrix $M_3$. As a result,
we get the set of all possible paths in the graph from $1$ to $1$ derived from $S$.

\begin{figure}
\begin{minipage}[t]{0.48\textwidth}
{
\scriptsize
\setlength{\DTbaselineskip}{8pt}
\DTsetlength{0.2em}{0.5em}{0.2em}{0.4pt}{1.6pt}
\dirtree{%
.1 getPaths($1,1,S$).
.2 genPaths($(0,1),(3,1)$).
.3 return $\{[((0,1),(1,0)), ((1,0),(2,1)), ((2,1),(3,1))]\}$.
.2 currentSubPaths = $\{[1 \xrightarrow{a} 0]\}$.
.2 currentPaths = $\{[(1 \xrightarrow{a} 0)]\}$.
.2 getPaths($0,1,S$).
.3 genPaths($(0,0),(3,1)$).
.4 return $\{[((0,0),(1,1)), ((1,1),(3,1))], [((0,0),(1,1)),
\\ ((1,1),(2,1)), ((2,1),(3,1))]\}$.
.3 path = $[((0,0),(1,1)), ((1,1),(3,1))]$.
.3 currentSubPaths = $\{[0 \xrightarrow{a} 1]\}$.
.3 currentPaths = $\{[0 \xrightarrow{a} 1]\}$.
.3 currentSubPaths = $\{[1 \xrightarrow{b} 1]\}$.
.3 $\cdots$.
.3 resultPaths = $\{[0 \xrightarrow{a} 1 \xrightarrow{b} 1]\}$.
.3 path = $[((0,0),(1,1)), ((1,1),(2,1)), ((2,1),(3,1))]$.
.3 currentSubPaths = $\{[0 \xrightarrow{a} 1]\}$.
.3 currentPaths = $\{[0 \xrightarrow{a} 1]\}$.
.3 getPaths(1, 1, S) // \begin{minipage}[t]{14cm} An alternative way to get paths from 1 to 1\\ (leads to infinite set of paths) \end{minipage}.
.3 $\cdots$.
.8 return $r_\infty^{1 \leadsto 1}$ // \begin{minipage}[t]{5cm} An infinite set of path from 1 to 1 \end{minipage}.
.3 currentPaths = $\{[0 \xrightarrow{a} 1] \} \cdot r_\infty^{1\leadsto 1}$.
.3 currentSubPaths = $\{[1 \xrightarrow{b} 1]\}$.
.3 $\cdots$.
.3 return $\{[0 \xrightarrow{a} 1 \xrightarrow{b} 1]\} \cup (\{[0 \xrightarrow{a} 1] \} \cdot r_\infty^{1\leadsto 1} \cdot \{[1 \xrightarrow{b} 1]\})$.
.2 currentPaths = $\{[1 \xrightarrow{a} 0 \xrightarrow{a} 1 \xrightarrow{b} 1]\} \cup (\{[1 \xrightarrow{a} 0 \xrightarrow{a} 1] \} \cdot r_\infty^{1\leadsto 1} \cdot \{[1 \xrightarrow{b} 1]\})$.
.2 currentSubPaths = $\{[1 \xrightarrow{b} 1]\}$.
.2 $\cdots$.
.2 return = $\{[1 \xrightarrow{a} 0 \xrightarrow{a} 1 \xrightarrow{b} 1 \xrightarrow{b} 1]\} \cup (\{[1 \xrightarrow{a} 0 \xrightarrow{a} 1] \} \cdot r_\infty^{1\leadsto 1} \cdot \{[1 \xrightarrow{b} 1 \xrightarrow{b} 1]\})$.
}
}
\caption{Example of call stack trace}
\label{trc:example}
\end{minipage}
\end{figure}