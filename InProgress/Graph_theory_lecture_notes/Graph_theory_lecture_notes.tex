%\documentclass[a4paper,12pt]{article}  % standard LaTeX, 12 point type
\documentclass[12pt, a4paper]{article}

\usepackage{algpseudocode}
\usepackage{algorithm}
\usepackage{algorithmicx}

\usepackage{geometry}
\usepackage{amsfonts,latexsym}
\usepackage{amsthm}
\usepackage{amssymb}
\usepackage[utf8]{inputenc} % Кодировка
\usepackage[english,russian]{babel} % Многоязычность
\usepackage{mathtools}
\usepackage{hyperref}
\usepackage{tikz}
\usepackage{dsfont}
\usepackage{multicol}

\usepackage[bb=boondox]{mathalfa}

\usepackage{tkz-euclide}

\usetikzlibrary{fit,calc,automata,positioning}

\theoremstyle{definition}
\newtheorem{definition}{Определение}[section]
\newtheorem{example}{Пример}[section]
\newtheorem{theorem}{Теорема}[section]
\newtheorem{proposition}[theorem]{Proposition}
\newtheorem{lemma}[theorem]{Лемма}
\newtheorem{corollary}[theorem]{Corollary}
\newtheorem{conjecture}[theorem]{Conjecture}


% unnumbered environments:

\theoremstyle{remark}
\newtheorem*{remark}{Remark}
\newtheorem*{notation}{Notation}
\newtheorem*{note}{Note}



\setlength{\parskip}{5pt plus 2pt minus 1pt}
%\setlength{\parindent}{0pt}


\algtext*{EndWhile}% Remove "end while" text
\algtext*{EndIf}% Remove "end if" text
\algtext*{EndFor}% Remove "end for" text
\algtext*{EndFunction}% Remove "end function" text


\usepackage{color}
\usepackage{listings}
\usepackage{caption}
\usepackage{graphicx}
\usepackage{ucs}

\graphicspath{{pics/}}

\geometry{left=1cm}
\geometry{right=1cm}
\geometry{top=2cm}
\geometry{bottom=2cm}




%\lstnewenvironment{algorithm}[1][]
%{
%    \lstset{
%        frame=tB,
%        numbers=left,
%        mathescape=true,
%        numberstyle=\small,
%        basicstyle=\small,
%        inputencoding=utf8,
%        extendedchars=\true,
%        keywordstyle=\color{black}\bfseries,
%        keywords={,function, procedure, return, datatype, function, in, if, else, for, foreach, while, denote, do, and, then, assert,}
%        numbers=left,
%        xleftmargin=.04\textwidth,
%        #1 % this is to add specific settings to an usage of this environment (for instnce, the caption and referable label)
%    }
%}
%{}

\newcommand{\tab}[1][0.3cm]{\ensuremath{\hspace*{#1}}}

\newcommand{\rvline}{\hspace*{-\arraycolsep}\vline\hspace*{-\arraycolsep}}

\newcommand{\derives}[1][*]{\xRightarrow[]{#1}}
\newcommand{\first}[1][1]{\textsc{first}_{#1}}
\newcommand{\follow}[1][1]{\textsc{follow}_{#1}}

\setcounter{MaxMatrixCols}{20}


\tikzset{
%->, % makes the edges directed
%>=stealth’, % makes the arrow heads bold
node distance=4cm, % specifies the minimum distance between two nodes. Change if necessary.
%every state/.style={thick, fill=gray!10}, % sets the properties for each ’state’ node
initial text=$ $, % sets the text that appears on the start arrow
}

\tikzstyle{symbol_node} = [shape=rectangle, rounded corners, draw, align=center]
\tikzstyle{prod_node} = [shape=rectangle, draw, align=center]

\tikzset{
    between/.style args={#1 and #2}{
         at = ($(#1)!0.5!(#2)$)
    }
}

%every node/.style = {shape=rectangle, rounded corners,
%      draw, align=center,
%      top color=white, bottom color=blue!20}

\title{Теория графов. Лекции. Заметки.}
\author{Семён Григорьев}
\date{\today}

\begin{document}
\maketitle
\newpage
\tableofcontents
\newpage

%\section{План занятий}
\begin{enumerate}
  \item Введение. О чём курс: общая структура, что будет и чего не будет. Правила получения оценки за курс. Базовые определения.
  \item Иерархия Хомского. Основные классы языков. За пределами иерархии Хомского. Нестроковые языки.
  \item Задача поиска пути с ограничениями в терминах формальных языков. Варианты постановки, прикладное значение, теоретические вопросы.  
  \item Регулярные языки, конечные автоматы (детерминированные, недетерминированные), регулярные выражения. Операции над ними. Операции над автоматами как операции над их матрицами смежности. Поиск путей с регулярными ограничениями.
  \item Контекстно-свободные языки. Нормальная и ослабленная нормальная формы Хомского. Поиск путей с КС ограничениями. CYK и Hellings.
  \item Матричный алгоритм КС запросов.
  \item Тензорный алгоритм КС запросов.
  \item Дерево разбора и поиск путей. SPPF. 
  \item Синтаксический анализ языков программирования. Лексика и синтаксис. Тонкости, проблемы, инструменты.
  \item ANTLR, LL, ещё раз про неоднозначности.
  \item Семантика языков программирования. Интерпретаторы. Что делать с деревом разбора. 
  \item Атрибутные грамматики.
  \item Немного о том, что за КС тоже есть жизнь.
\end{enumerate}


Общая цель курса --- посмотреть на формальные языки с прикладной точки зрения. При этом предлагается попробовать применить их сразу в двух областях: классический синтаксический анализ языков программирования и анализ графов.

В ходе курса будет предложено разработать небольшой инструментарий для выполнения запросов к графам. Окажется, что алгоритмы для некоторых задач анализа графов непосредственно основаны на алгоритмах из теории формальных языков и синтаксического анализа. Далее, будет предложено разработать язык запросов, позволяющий использовать разработанные алгоритмы. Необходимо будет разработать сам язык, лексический и синтаксический  анализаторы для него, интерпретатор. Интерпретатор будет использовать разработанные алгоритмы выполнения запросов к графам.

Примерные темы задач с баллами.
\begin{enumerate}
  \item [5] Развернуть репозиторий. Научиться подгружать графы, запрашивать у них вершины и рёбра. Консольный клиент.
  \item [2] Регулярка в ДКА. 
  \item [2] Граф в НКА.
  \item [5] Регулярные запросы через тензорное произведение. 
  \item [11] Сравнение производительности пересечения автоматов.
  \item [2] КС граммтики. Преобразование в ОНФХ. 
  \item [5] Построение рекурсивного конечного автомата и его минимизация.
  \item [5] CTK
  \item [5] Hellings
  \item [5] Матрицы
  \item [5] Тензоры
  \item [15] Сравнение производительности КС алгоритмов.
  \item [5] Разработать конкретный синтаксис языка запросов. Документация. 
  \item [5] Реализовать его парсер.
  \item [3] Печать дерева в DOT.
  \item [20] Реализовать интерпретатор языка запросов.
\end{enumerate}

\section{Лекция 1: Введение}

Программирование --- не только написание кода.
Документация, сборка, тестирование, версионирование, обработка отзывов пользователей.

Инфраструктура проекта, рабочее окружение, система контроля версий, непрерывная сборка.

Соответствующие решения на примере инфраструктуры вокруг GitHub. GithubActions, внешние сервисы для CI (\url{https://travis-ci.org/}, \url{https://www.appveyor.com/}, \url{https://circleci.com/}). 

Практика развёртывания соответствующей инфраструктуры. 
\begin{enumerate}
  \item Для начала, завести аккаунт на GitHub (\url{https://github.com/}). 
Важно, чтобы имя аккаунта (логин) было \texttt{NameSurname} или \texttt{Name\_Surname}.
   
  \item Создаём репозиторий для проекта (для всех домашних работ). Название должно отражать сожержимое репозитория. Не забываем добавить описание. Лицензию, readme и gitignore лучше не добавлять.
  \item Устанавливаем git (\url{https://git-scm.com/}) и графическую оболочку для работы с ним (если кому нужно).
  \item Теперь пора приступать к созданию проекта.
Так как дальше мы будем пользоваться F\#, то в качестве шаблона предлагается использовать \url{https://github.com/TheAngryByrd/MiniScaffold}. 
\begin{enumerate}
  \item Установить .NET Core: \url{https://dotnet.microsoft.com/download}
  \item Прочитать инструкции (\href{https://github.com/TheAngryByrd/MiniScaffold#install-the-dotnet-template-from-nuget}{https://github.com/TheAngryByrd/MiniScaffold\#install-the-dotnet-template-from-nuget}) и выполнить соостветсвующие шаги. Нам нужно создать консольное приложение. Это может занять некоторое время.
\end{enumerate}
\item Устанавливаем связь только что созданного локального репозитория с удалённым репозиторием: \href{https://docs.github.com/en/github/importing-your-projects-to-github/adding-an-existing-project-to-github-using-the-command-line}{https://docs.github.com/en/github/importing-your-projects-to-github/adding-an-existing-project-to-github-using-the-command-line}

\end{enumerate}

С этого момента домашние работы только через GitHub с налаженной сборкой.
%\section{Домашняя работа 1}

Задачи:

\begin{enumerate}
\item \textbf{(1 балл)} Инициализировать рабочее окружение: репозиторий на GitHub, CI, readme, лицензия. 
Добавить преподавателя в совладельцы. Оформить тестовый pull request: например, оформленное readme (поправить описание проекта, удалить лишнее, что досталось от шаблона, поправить ссылку на статус сборки). Запросить ревью у преподавателя.
\end{enumerate}

\section{Спектральная теория графов. Введение}

Матрица смежности.

Матрица Кирхгофа. Оператор Лапласа. Для неориентированного графа без кратных рёбер и петель.
\begin{definition}
Пусть неориентированный граф без кратных рёбер и петель (простой граф) $G=\langle V, E \rangle, |V| = n$. Тогда матрица Кирхгофа $ K=(k_{i,j})_{n \times n}$. 
$$ k_{i,j}:={\begin{cases}\deg(v_{i})&{\text{при}}\ i=j,\\-1&{\text{при}}\ (v_{i},v_{j})\in E(G),\\0&{\text{в противном случае}}.\end{cases}}
$$
\end{definition}

$K = D - A$, где $A$ --- матрица смежности графа, а $D$ --- матрица, на диагонали которой строят степени вершин, а остальные элементы равны нулю.


\begin{example}[Пример графа и его матрицы Кирхгофа]
Пусть дан граф:
  \begin{center}
  \begin{tikzpicture}[on grid, auto]
     \node[state] (q_0)   {$0$};
     \node[state] (q_1) [above right=1.4cm and 1.0cm of q_0] {$1$};
     \node[state] (q_2) [right=2.0cm of q_0] {$2$};
     \node[state] (q_3) [right=2.0cm of q_2] {$3$};
      \path[-]
      (q_0) edge (q_1)
      (q_1) edge (q_2)
      (q_2) edge (q_0)
      (q_2) edge (q_3);
  \end{tikzpicture}
  \end{center}
  $$ D =
  \left({
  \begin{array}{rrrrrr}
  2 & 0 & 0 & 0 \\
  0 & 2 & 0 & 0 \\
  0 & 0 & 3 & 0 \\
  0 & 0 & 0 & 1 \\
  \end{array}
  }\right)
$$
$$ A =
  \left({
  \begin{array}{rrrrrr}
  0 & 1 & 1 & 0 \\
  1 & 0 & 1 & 0 \\
  1 & 1 & 0 & 1 \\
  0 & 0 & 1 & 0 \\
  \end{array}
  }\right)
$$

$$ K = D - A =
  \left({
  \begin{array}{rrrrrr}
  2  & -1 & -1 & 0  \\
  -1 & 2  & -1 & 0  \\
  -1 & -1 & 3  & -1 \\
  0  & 0  & -1 & 1  \\
  \end{array}
  }\right)
$$
\end{example}

Определитель матрицы.


\begin{definition}[Дополнительный минор]
$M_{i,j}$ --- дополнительный минор, определитель матрицы, получающейся из исходной матрицы $A$ путём вычёркивания $i$-й строки и $j$-го столбца.
\end{definition}


\begin{definition}[Определитель матрицы $2 \times 2$]
Для матрицы $2\times 2$ определитель вычисляется как:

$$\Delta ={\begin{vmatrix}a&c\\b&d\end{vmatrix}} = ad - bc$$

\end{definition}

\begin{definition}[Определитель матрицы $N \times N$]

$$\Delta =\sum _{j=0}^{n-1}(-1)^{j}a_{0,j}{M}_{0,j}$$
, где $M_{0,j}$ --- дополнительный минор к элементу $a_{0,j}$. 


\end{definition}

Это было разложение по строке и, вообще говоря, подобная операция может быть проделяна для любой строки.
Аналогично можно использовать разложение по столбцу.


\begin{definition}[Определитель матрицы $3 \times 3$]
\begin{align*}
&\Delta =
{\begin{vmatrix}
  a_{0,0}&a_{0,1}&a_{0,2}\\
  a_{1,0}&a_{1,1}&a_{1,2}\\
  a_{2,0}&a_{2,1}&a_{2,2}
 \end{vmatrix}}
 =
  a_{0,1}{\begin{vmatrix}a_{1,1}&a_{1,2}\\a_{2,1}&a_{2,2}\end{vmatrix}}
 -a_{0,2}{\begin{vmatrix}a_{1,0}&a_{1,2}\\a_{2,0}&a_{2,2}\end{vmatrix}}
 +a_{0,3}{\begin{vmatrix}a_{1,0}&a_{1,1}\\a_{2,0}&a_{2,1}\end{vmatrix}} 
 =\\
&a_{0,0}a_{1,1}a_{2,2}-a_{0,0}a_{1,2}a_{2,1}-a_{0,1}a_{1,0}a_{2,2}+a_{0,1}a_{1,2}a_{2,0}+a_{0,2}a_{1,0}a_{2,1}-a_{0,2}a_{1,1}a_{2,0}
\end{align*}
\end{definition}


\begin{definition}[Алгебраическое дополнение]
Алгебраическим дополнением элемента $ a_{i,j}$ матрицы $A$ называется число $A_{i,j}=(-1)^{i+j}M_{i,j}$,
где $M_{i,j}$ --- дополнительный минор.
\end{definition}

\begin{example}[Определитель]
Найдём определитель матрицы Кирхгофа для нашего графа. Будем использовать разложение по 3-й строке.
\begin{align*}
&\Delta\left({
  \begin{array}{rrrrrr}
  2  & -1 & -1 & 0  \\
  -1 & 2  & -1 & 0  \\
  -1 & -1 & 3  & -1 \\
  0  & 0  & -1 & 1  \\
  \end{array}
  }\right) = (-1)^5 (-1) {\begin{vmatrix}2&-1&0\\-1&2&0\\-1&-1&-1\end{vmatrix}} + (-1)^6 1 {\begin{vmatrix}2&-1&-1\\-1&2&-1\\-1&-1&3\end{vmatrix}}=\\
  &1((2\cdot2\cdot-1) - (2\cdot0\cdot-1) - (-1\cdot-1\cdot-1) + (-1\cdot0\cdot-1) + (0\cdot-1\cdot-1) - (0\cdot2\cdot-1)) +\\
  & 1((2\cdot2\cdot3) - (2\cdot-1\cdot-1) - (-1\cdot-1\cdot3) + (-1\cdot-1\cdot-1) + (-1\cdot-1\cdot-1) - (-1\cdot2\cdot-1) ) =\\
  & (-4 + 1) + (12 - 2 - 3 - 1 - 1 - 2) = -3 + 3 = 0 
\end{align*}
\end{example}


\begin{theorem}[Матричная теорема об остовных деревьях]
Пусть $G$ --- связный простой граф с матрицей Кирхгофа $K$. Все алгебраические дополнения матрицы Кирхгофа $K$ равны между собой и их общее значение равно количеству остовных деревьев графа $G$.
\end{theorem}

\begin{example}[Количество остовных деревьев]
Из примера выше, значения миноров равно 3.

Деревья:
\begin{center}
  \begin{tikzpicture}[on grid, auto]
     \node[state] (q_0)   {$0$};
     \node[state] (q_1) [above right=1.4cm and 1.0cm of q_0] {$1$};
     \node[state] (q_2) [right=2.0cm of q_0] {$2$};
     \node[state] (q_3) [right=2.0cm of q_2] {$3$};
      \path[-]
      (q_0) edge (q_1)
      (q_2) edge (q_0)
      (q_2) edge (q_3);
  \end{tikzpicture}
  \end{center}

  \begin{center}
  \begin{tikzpicture}[on grid, auto]
     \node[state] (q_0)   {$0$};
     \node[state] (q_1) [above right=1.4cm and 1.0cm of q_0] {$1$};
     \node[state] (q_2) [right=2.0cm of q_0] {$2$};
     \node[state] (q_3) [right=2.0cm of q_2] {$3$};
      \path[-]
      (q_1) edge (q_2)
      (q_2) edge (q_0)
      (q_2) edge (q_3);
  \end{tikzpicture}
  \end{center}

  \begin{center}
  \begin{tikzpicture}[on grid, auto]
     \node[state] (q_0)   {$0$};
     \node[state] (q_1) [above right=1.4cm and 1.0cm of q_0] {$1$};
     \node[state] (q_2) [right=2.0cm of q_0] {$2$};
     \node[state] (q_3) [right=2.0cm of q_2] {$3$};
      \path[-]
      (q_0) edge (q_1)
      (q_1) edge (q_2)
      (q_2) edge (q_3);
  \end{tikzpicture}
  \end{center}
\end{example}


Сумма элементов каждой строки (столбца) матрицы Кирхгофа равна нулю: $\sum _{i=0}^{|V|-1}k_{i,j}=0$.

Определитель матрицы Кирхгофа равен нулю: $\Delta K = 0$.

Собственные числа и собственные вектора.

Нам понядобится поле $F$. 

\begin{definition}[Собственный вектор]
Ненулевой вектор $x$ называется собственным вектором матрицы $A$ для некоторого элемента $\lambda \in F$, если $Ax = \lambda x$
\end{definition}

\begin{definition}[Собственное число (Собственное значение)]
Собственным числом матрицы $A$ называется такое $\lambda \in F$, что существует ненулевое решение уравнения $Ax = \lambda x$.
\end{definition}

Как видно, собственные числа и собственные вектора ``ходят парами''.

\begin{example}[Собственные числа и вектора]

$$
A = \left({
  \begin{array}{rrrrrr}
  2  & -1 & -1 & 0  \\
  -1 & 2  & -1 & 0  \\
  -1 & -1 & 3  & -1 \\
  0  & 0  & -1 & 1  \\
  \end{array}
  }\right)
$$

По определению $Ax = \lambda x$. 
$$Ax - \lambda x = 0$$
$$(A - \lambda E) x = 0$$
$$
(\left({
  \begin{array}{rrrrrr}
  2  & -1 & -1 & 0  \\
  -1 & 2  & -1 & 0  \\
  -1 & -1 & 3  & -1 \\
  0  & 0  & -1 & 1  \\
  \end{array}
  }\right) -  \left({
  \begin{array}{rrrrrr}
  \lambda  & 0 & 0 & 0  \\
  0 & \lambda  & 0 & 0  \\
  0 & 0 & \lambda  & 0 \\
  0  & 0  & 0 & \lambda  \\
  \end{array}
  }\right)) x = 0
$$

Данное уравнение имеет ненулевое решение тогда и только тогда, когда $|A - \lambda E| = 0$
\begin{align*}
|A - \lambda E| =
  \begin{vmatrix}
  2 - \lambda & -1 & -1 & 0  \\
  -1 & 2 - \lambda & -1 & 0  \\
  -1 & -1 & 3 - \lambda & -1 \\
  0  & 0  & -1 & 1 - \lambda \\
  \end{vmatrix} = \lambda^4-8\lambda^3+19\lambda^2-12\lambda
\end{align*}
То есть надо решить уравнение 
$$
\lambda^4-8\lambda^3+19\lambda^2-12\lambda = 0
$$

$$
\lambda^4-8\lambda^3+19\lambda^2-12\lambda  = \lambda (\lambda^3 - 8\lambda^2 + 19\lambda - 12) = \lambda(\lambda - 1)(\lambda^2 - 7\lambda + 12) = \lambda(\lambda - 1)(\lambda - 3)(\lambda - 4) = 0
$$

Корни: $\lambda \in \{0,1,3,4\}$.

Далее для каждого совственного числа нужно найти соответсвующий вектор. Для этого решаем системы линейных уравнений (метод Гаусса в помощь).
$$(A - 0\cdot E) x = 0$$
$$(A - 1\cdot E) x = 0$$
$$(A - 3\cdot E) x = 0$$
$$(A - 4\cdot E) x = 0$$

$$x_0 = \left(\begin{array}{r}1\\1\\1\\1\end{array}\right)$$
$$x_1 = \left(\begin{array}{r}-\frac{1}{2}\\-\frac{1}{2}\\0\\1\end{array}\right)$$
$$x_2 = \left(\begin{array}{r}-1\\1\\0\\0\end{array}\right)$$
$$x_3 = \left(\begin{array}{r}1\\1\\-3\\1\end{array}\right)$$


\end{example}

\begin{definition}[Спектр графа]
Спектром графа называется упорядоченное по возростанию мультимножество собственных значений его матрицы смежности.
\end{definition}

Так как мы говорим о неориентированном графе, то собственные значения всегд вещественные числа (почему?).

Хотя матрица смежности и зависит от нумерации вешин, спектр является инвариантом графа (почему?).

Следствие: изоморфные графы имеют одинаковый спектр.

Графы с одинаковым спектром --- изоспектральные (или коспектральные).

\begin{theorem}
Изоморфные графы всегда изоспектральны. Обратное не верно (изоспектральные графы не обязательно изоморфны).
\end{theorem}

\subsection{Лекция 2}

Вернёмся к собственным числам матрицы Кирхгофа. 

Минимальное значение собственных чисел --- это 0.

\begin{theorem}
$\lambda_1 = 0 $ тогда и только тогда, когда в графе больше одной компоненты связянности. 
\end{theorem}

Напомним, что нумерация с нуля. То есть речь про второе совственное число.

Кратность нуля как собственного числа --- количество компонент явязанности.

\begin{definition}[Алгебраическая связянность графа]
Занчение второго собственного числа матрицы Кирхгофа называют алгебраической связанностью графа.
Соотвтетсвующий собственный вектор --- вектор Фидлера (Fiedler).
\end{definition}

$\lambda_1$ монотонно неубывает при добавлении рёбер.

\subsubsection{Укладка графов}

\href{https://www.jstor.org/stable/2629091?seq=1}{Hall, Kenneth M. “An r-Dimensional Quadratic Placement Algorithm.” Management Science, vol. 17, no. 3, 1970, pp. 219–229. JSTOR, www.jstor.org/stable/2629091. Accessed 4 Apr. 2021.}

Цель: нарисовать граф. Построить отоброжение из $V$ $\mathbb{R}^n$.

Для начала, на прямой ($n=1$). То есть хотим получить вектор координат вершин $x$. 
Для этого минимизируем $$\sum_{(i,j) \in E} (x_i + x_j)^2 = x^T K x$$
Тривиальное решение: $x = \mathbb{1}$. 
Чтобы его не допускать, потребуем, чтобы $x^T\mathbb{1} = 0$.
Тогда решением будет второй собственный вектор.


 \begin{center}
  \begin{tikzpicture}[on grid, auto]
     \node[state] (q_0)   {$0$};
     \node[state] (q_1) [above right=1.4cm and 1.0cm of q_0] {$1$};
     \node[state] (q_2) [right=2.0cm of q_0] {$2$};
     \node[state] (q_3) [right=2.0cm of q_2] {$3$};
      \path[-]
      (q_0) edge (q_1)
      (q_1) edge (q_2)
      (q_2) edge (q_0)
      (q_2) edge (q_3);
  \end{tikzpicture}
  \end{center}

$$x_1 = \left(\begin{array}{r}-\frac{1}{2}\\-\frac{1}{2}\\0\\1\end{array}\right)$$


В $\mathbb{R}^2$ интереснее. Хотим минимизировать суму расстояний: $$\sum_{(i,j) \in E}(x_i - x_j)^2 + (y_i - y_j)^2$$

при условии $$\sum_{i \in V} (x_i,y_i) = (0,0)$$ чтобы избежать тривиального решения. Дополнительно потребуем ортогональность $x$ и $y$.

Тогда решение --- второй и третий собственные вектора.

$$x_2 = \left(\begin{array}{r}-1\\1\\0\\0\end{array}\right)$$

$$
a: (-\frac{1}{2},-1)
$$
$$
b: (-\frac{1}{2},1)
$$
$$
c: (0,0)
$$
$$
d: (1,0)
$$


\begin{figure}[h]
\begin{center}
\begin{tikzpicture}
\tkzInit[xmax=2,ymax=2,xmin=-2,ymin=-2]
\tkzGrid
\tkzAxeXY
\tkzSetUpLine[color=blue,line width=1pt]
\tkzDefPoint(-0.5,-1){a}
\tkzDefPoint(-0.5,1){b}
\tkzDefPoint(0,0){c}
\tkzDefPoint(1,0){d}
\tkzDrawSegments(a,b b,c c,a c,d)
\tkzLabelPoints[above left](b)
\tkzLabelPoints[above right](c,d)
\tkzLabelPoints[below left](a)
\tkzDrawPoints(a,b,c,d)
\end{tikzpicture}
\end{center}
\end{figure}


\subsubsection{Матрицы смежности}

Вернёмся к матрицам смежности. Те же собственные числа, те же собственные вектора. Только упорядочиваем наоборот: $\lambda_0 \geq \lambda_1 \ldots \geq \lambda_n$.

\begin{theorem}
$$d_{avg} \leq \lambda_0 \leq d_{max}$$ , где $d$ --- степень вершины.
\end{theorem}

Что будет, если начать удалять вешины с наименьшими степенями?

\begin{theorem}
Граф раскрашиваем в $\lfloor \lambda_0\rfloor + 1$ цвет
\end{theorem}

\begin{example}
$$ A =
  \left({
  \begin{array}{rrrrrr}
  0 & 1 & 1 & 0 \\
  1 & 0 & 1 & 0 \\
  1 & 1 & 0 & 1 \\
  0 & 0 & 1 & 0 \\
  \end{array}
  }\right)
$$

$$\lambda \in \{2.170,0.311,-1,-1.481\}$$

(все, кроме одного --- приближённые значения).

Раскрашиваем ли он наш граф в $\lfloor \lambda_0\rfloor + 1 = \lfloor 2.170 \rfloor + 1 = 2 + 1 = 3$ цвета? 
\end{example}

\begin{theorem}
Хроматическое число $\chi \geq 1 + \frac{\lambda_0}{-\lambda_n}$
\end{theorem}


\begin{theorem}
Граф двудольный тогда и только тогда, когда для любого собственного числа $\lambda_i$, величина $-\lambda_i$ 
также является собственным числом.
\end{theorem}


\begin{theorem}
Ещё немного характеристик на основе собственных значений.
\begin{itemize}
  \item Граф с одним собственным числом --- граф без рёбер.
  \item Граф с двумя собственными числами --- полный граф.
\end{itemize}
\end{theorem}


\begin{example}[Пример полного графа и его матриц смежности и Кирхгофа]
Пусть дан граф:
  \begin{center}
  \begin{tikzpicture}[on grid, auto]
     \node[state] (q_0)   {$0$};
     \node[state] (q_1) [below=2.0cm of q_0] {$1$};
     \node[state] (q_2) [right=2.0cm of q_0] {$2$};
     \node[state] (q_3) [right=2.0cm of q_1] {$3$};     
      \path[-]
      (q_0) edge (q_1)
      (q_1) edge (q_2)
      (q_2) edge (q_0)
      (q_1) edge (q_3)
      (q_0) edge (q_3)
      (q_2) edge (q_3);
  \end{tikzpicture}
  \end{center}
  $$ D =
  \left({
  \begin{array}{rrrrrr}
  3 & 0 & 0 & 0 \\
  0 & 3 & 0 & 0 \\
  0 & 0 & 3 & 0 \\
  0 & 0 & 0 & 3 \\
  \end{array}
  }\right)
$$
$$ A =
  \left({
  \begin{array}{rrrrrr}
  0 & 1 & 1 & 1 \\
  1 & 0 & 1 & 1 \\
  1 & 1 & 0 & 1 \\
  1 & 1 & 1 & 0 \\
  \end{array}
  }\right)
$$

$$ K = D - A =
  \left({
  \begin{array}{rrrrrr}
  3  & -1 & -1 & -1  \\
  -1 &  3 & -1 & -1  \\
  -1 & -1 &  3 & -1  \\
  -1 & -1 & -1 &  3  \\
  \end{array}
  }\right)
$$

Начнём с матрицы смежности. Посчитаем её собственные числа.

$$\lambda_0 = 3; \lambda_1 = -1 $$

\end{example}


\begin{example}[Пример двудольного графа и его матриц смежности и Кирхгофа]
Пусть дан граф:
  \begin{center}
  \begin{tikzpicture}[on grid, auto]
     \node[state] (q_0)   {$0$};
     \node[state] (q_1) [below=2.0cm of q_0] {$1$};
     \node[state] (q_2) [below=2.0cm of q_1] {$2$};
     \node[state] (q_3) [right=2.0cm of q_0] {$3$};     
     \node[state] (q_4) [right=2.0cm of q_1] {$4$};
     \node[state] (q_5) [right=2.0cm of q_2] {$5$};     
      \path[-]
      (q_0) edge (q_3)
      (q_0) edge (q_4)
      (q_1) edge (q_4)
      (q_1) edge (q_5)
      (q_2) edge (q_5)
      ;
  \end{tikzpicture}
  \end{center}
  $$ D =
  \left({
  \begin{array}{rrrrrrrr}
  2 & 0 & 0 & 0 & 0 & 0 \\
  0 & 2 & 0 & 0 & 0 & 0 \\
  0 & 0 & 1 & 0 & 0 & 0 \\
  0 & 0 & 0 & 1 & 0 & 0 \\
  0 & 0 & 0 & 0 & 2 & 0 \\
  0 & 0 & 0 & 0 & 0 & 2 \\
  \end{array}
  }\right)
$$
$$ A =
  \left({
  \begin{array}{rrrrrrrr}
  0 & 0 & 0 & 1 & 1 & 0 \\
  0 & 0 & 0 & 0 & 1 & 1 \\
  0 & 0 & 0 & 0 & 0 & 1 \\
  1 & 0 & 0 & 0 & 0 & 0 \\
  1 & 1 & 0 & 0 & 0 & 0 \\
  0 & 1 & 1 & 0 & 0 & 0 \\
  \end{array}
  }\right)
$$

$$ K = D - A =
  \left({
  \begin{array}{rrrrrrrr}
  2  & 0  & 0 & -1 & -1 & 0 \\
  0  & 2  & 0 & 0 & -1 & -1 \\
  0  & 0  & 1 & 0 & 0 & -1 \\
  -1 & 0  & 0 & 1 & 0 & 0 \\
  -1 & -1 & 0 & 0 & 2 & 0 \\
  0  & -1 & -1 & 0 & 0 & 2 \\
  \end{array}
  }\right)
$$


$$ Q = D + A =
  \left({
  \begin{array}{rrrrrrrr}
  2  & 0  & 0 & 1 & 1 & 0 \\
  0  & 2  & 0 & 0 & 1 & 1 \\
  0  & 0  & 1 & 0 & 0 & 1 \\
  1 & 0  & 0 & 1 & 0 & 0 \\
  1 & 1 & 0 & 0 & 2 & 0 \\
  0  & 1 & 1 & 0 & 0 & 2 \\
  \end{array}
  }\right)
$$

Начнём с матрицы смежности. Посчитаем её собственные числа.

$$\lambda_0 \approx 1.802; \lambda_1 \approx 1.247 ; \lambda_2 \approx 0.445; $$
$$\lambda_3 \approx -0.445; \lambda_4 \approx -1.247; \lambda_5 \approx -1.802$$

Собственные числа и собственные вектора матрицы Кирхгофа.

$$\lambda_0 =0; \lambda_1 = -\sqrt{3} + 2; \lambda_2 = 1; $$
$$\lambda_3 = 2; \lambda_4 = 3; \lambda_4 = \sqrt{3} + 2; $$


$$x_1 = \left(\begin{array}{c}-1 \\ \frac{\sqrt{3}-1}{2} \\ \frac{\sqrt{3}+1}{2} \\ \frac{-\sqrt{3}-1}{2} \\ \frac{-\sqrt{3}+1}{2} \\ 1 \end{array}\right)$$
$$x_2 = \left(\begin{array}{c}0\\1\\-1\\-1\\1\\0\end{array}\right)$$


Заметим, что спектр $K$ равен спектру $Q$.

Спектр Q:
$$\lambda^q_0 =0; \lambda^q_1 = -\sqrt{3} + 2; \lambda^q_2 = 1; $$
$$\lambda^q_3 = 2; \lambda^q_4 = 3; \lambda^q_4 = \sqrt{3} + 2; $$


\end{example}


\begin{theorem}
Граф двудольный тогда и только тогда, когда спектр матрицы Кирхгофа равен спектру беззнаковой матрицы Кирхгофа.
\end{theorem}

\subsection{Задачи}
\begin{enumerate}
\item \textbf{[2 балла]} Доказать формулу Кэли, пользуясь матричной теоремой об остовных деревьях. Формула Кэли даёт оценку числа остовных деревьев полного графа $K_{n}$: $n^{n-2}$.
\item \textbf{[2 балла]} Доказать, пользуясь матричной теоремой об остовных деревьях, что число остовных деревьев полного двудольнoгo графа $K_{m,n}$ равно $m^{n-1}\cdot n^{m-1}$.
\item \textbf{[2 балла]} Доказать, что спектр является инвариантом графа.
\item \textbf{[4 балла]} Реализовать визуализацию графов через собственные вектора. Граф принимается либо как матрица смежности в формате .mtx, либо как список рёбер (лучше предусмотреть обв варианта). Предусмотреть возможность "зума" результирующей картинки. Для экспорта лучше выбирать векторный формат, тогда проблем не будет. Можно использовать готовые компоненты.
\end{enumerate}
%\section{Домашняя работа 2}

В задачах, связанных с обработкой массивов на вход необходимо принимать длину массива и затем создавать случайный массив соответствующей длины. Для всех задач необходимо реализовать чтение входных данных из консоли и вывод результата в консоль.

Задачи:
\begin{enumerate}
    \item \textbf{(1 балл)} Реализовать функцию, вычисляющую значение выражения $x^4+x^3+x^2+x+1$ ``наивным'' способом. 
    \item \textbf{(1 балл)} Реализовать функцию, вычисляющую значение выражения $x^4+x^3+x^2+x+1$, применив минимальное число умножений и сложений.
    \item \textbf{(1 балл)} Вычислить индексы элементов массива, не больших, чем заданное число.
    \item \textbf{(1 балл)} Вычислить индексы элементов массива, лежащих вне диапазона, заданного двумя числами.
    \item \textbf{(1 балл)} Дан массив длины 2. Поменять местами нулевой и первый элементы, не используя дополнительной памяти/переменных.
    \item \textbf{(1 балл)} Поменять местами $i$-й и $j$-й элементы массива не используя дополнительной памяти/переменных.
\end{enumerate}

%\section{Лекция 3}

	Ещё раз произменения: в реквесте должно быть только то, что непосредственно относится к сдаваемой домашке.

	Ещё раз про функции, про то, как выделять и разделять функциональность, не надо запихивать всё в одну функцию. Про то, где должны быть проверки.

	Про консоль.

	Про обработку крайних случаев. Про исключения.

	Про тесты и ошибки: нашёл ошибку --- создал тест.

	Про стиль кодирования: про пробелы вокруг скобок и операций, про отступы и переводы строк. Про соглашения о наименовании. camlCase CamlCase


	Про единицы измерения.

    Базовые структуры данных, алгоритмы и их выражение в F\#. Функция. Рекурсия и итерация.  Базовые типы и основы работы с ними: матрицы, массивы, списки, структуры. 


    Числа Фибоначчи.
%\section{Домашняя работа 3}

Для всех задач обеспечить чтение $n$ из консоли и печать результата в консоль.

\begin{enumerate}
    \item \textbf{(1 балл)} Реализовать вычисление $n$-ого числа Фибоначчи рекурсивным методом. 
    \item \textbf{(1 балл)} Реализовать вычисление $n$-ого числа Фибоначчи итеративным методом. 
    \item \textbf{(1 балл)} Реализовать вычисление $n$-ого числа Фибоначчи используя хвостовую рекурсию (не используя \texttt{mutable} и других изменяемых структур). Подсказка: нужно использовать рекурсию с аккумулятором.
    \item \textbf{(2 балла)} Реализовать вычисление $n$-ого числа Фибоначчи через перемножение матриц ``наивным'' методом. Функции построения единичной матрицы, умножения и возведения в степень должны быть реализованы в общем виде.
    \item \textbf{(2 балла)} Реализовать вычисление $n$-ого числа Фибоначчи через перемножение матриц за логарифм.
    \item \textbf{(1 балл)} Реализовать вычисление всех чисел Фибоначчи до $n$-ого включительно.
\end{enumerate}

%\section{Лекция 4}
    Рассказать про то, что выделенные значения --- это плохо. Немного про исключительные ситуации.
    Про создание проектов. Про версии пакетов и вообще версии артефактов. Про то, что заимствование кода не поощрается. Тем более неправомерное заимствование. Про классный пример форматирования \verb|x*x*x + x*x + x + 1|.

    Работа с файлами.
    
    Сортировки: пузырьком, вставкой, Хоара. Различные сценарии использования: поддержание отсортированного набора, сортировка всего набора целиком. Некоторые особенности реализации: наивная функциональная реализация Хоара, реализация на массиве.
    
    Основы машинного представления данных. Представления чисел. Представление чисел с плавающей точкой. Проблемы переполнения. Битовые операции. Строки, кодировки.

%\section{Домашняя работа 4}

Во всех задачах на сортировку необходимо реализовать чтение массива из файла и печать результата в файл. Функции чтения и записи необходимо переиспользовать.

В задачах на битовые операции продолжаем рабоать с консолью: чтение данных с консоли и печать результата туда же.

Для данной домашней работы необходимо создать отдельный проект.

При создании тестов необходимо, в задачах на сортировку, убедиться, что, во-первых, сортировки ведут себя одинаоково на одинаковых данных, во-вторых, что они ведут себя так же, как системные сортировки для соответствующих коллекций. Для задачи о запаковке и распаковке надо проверить, что реализованные функции являются взаимно обратными. FsCheck (testProperty) в помощь.

\begin{enumerate}
    \item \textbf{(1 балл)} Реализовать сортировку пузырьком массива. 
    \item \textbf{(1 балл)} Реализовать сортировку пузырьком списка.
    \item \textbf{(1 балл)} Реализовать быструю сортировку для списка.
    \item \textbf{(1 балл)} Реализовать быструю сортировку для массива.
    \item \textbf{(1 балл)} Реализовать запаковку двух 32-битных чисел в одно 64-битное и распаковку обратно. 
    \item \textbf{(1 балл)} Реализовать запаковку четырёх 16-битных чисел в одно 64-битное и распаковку обратно. 
\end{enumerate}

%\section{Лекция 5}
 
Основы анализа алгоритмов. Модель вычислителя. Понятие элементарной операции. Асимптотика, ``О''-символика.

Постановка эксперимента и оформление результатов. Эксперимент по сравнению и анализу производительности. Точность проведения замеров. ``Масштабы времени'', цель эксперимента и точность измерений, инструменты измерений. Базовая статистическая обработка данных: медиана и среднее, выбросы, распределение. Проверка гипотез.  

Технические средства.
О системе вёрски \LaTeX. \href{https://www.tug.org/texlive/}{TexLive}, \href{https://www.overleaf.com/}{Overleaf}.
Способы визуализации результатов. Python + \href{https://matplotlib.org/3.3.1/index.html}{Matplotlib}, другие библиотеки. \href{http://www.gnuplot.info/}{GnuPlot}.  

Отдельно про скрипты сборки. Теперь мы руками попробуем сделать свой маленький скрипт. Shell, cmd (PowerShell).
%\section{Домашняя работа 5}

В данной задаче предолагается, что для обработки сырых данных и отрисовки графиков используется Python и Matplotlib, однако другие сравнимые по выразительности и качеству результата пакеты разрешены. Исользование офисных пакетов (типа LibreOffice, MS Office и т.д.) запрещено. Шрафики должны быть векторными. Внимательно ледите за тем, что в репозиторий должны быть только исходники, сгенерированные файлы в репозиторий попасть не должны.

\begin{enumerate}
    \item \textbf{(5 баллов)} Провести сравнительное исследование реализованных в предыдущей домашней работе сортировок и стандартных реализаций сортировок соответствующих коллекций. Оформить отчёт: введение, детали реализации, постановка эксперимента, результаты экспериментов, анализ результатов. Отчёт оформляется в \LaTeX, исходники выкладываются так же как и обычный код, снабжаются скриптом сборки (Shell), который генерирует графики и собирает pdf-документ.
\end{enumerate}


\bibliographystyle{abbrv}
\bibliography{Graph_theory_lecture_notes}


\end{document}
