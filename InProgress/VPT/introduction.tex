\section{Introduction}

Nowadays high-performance processing of a huge amount of data is indeed a challenge not only for scientific computing, but for applied systems as well.
Special types of hardware, such as General Purpose Graphic Processing Units (GPGPUs), Tensor Processing Units (TPUs), FPGA-based solutions, along with respective specialized software have been developed to provide appropriate solutions, and the development of new solutions continues.
In its turn, sparse linear algebra and GraphBLAS~\cite{buluc2017graphblas} in particular, are a way to utilize all these accelerators to provide high-performance solutions in many areas including machine learning~\cite{Kepner2017} and graph analysis~\cite{graphblast}.

However, evaluation of expressions over matrices generates intermediate data structures similar to the well-known example of a pipelined processing of collections: \verb| map g (map f data)|. Suppose \verb|data| is a list, then the first \verb|map| produces a new list which then will be traversed by the second \verb|map|. The same pattern could be observed in neural networks, where initial data flows through network layers, or in linear algebra expressions, where each subexpression produces an intermediate matrix. 
The last case occurs not only in scientific computations but also in graph analysis~\cite{graphblast}.
It is crucial that not only the data structures are traversed multiple times, while it is possible to traverse over them only once, but also the intermediate data populates memory (RAM).
Extra memory accesses are a big problem for real-world data analysis: the size of data is huge and memory accesses are expensive operations with noticable latency.
While a number of complex real-world cases including stream fusion and dense kernels fusion~\cite{fusion-boosting-memory-computations} could be successfully optimized using deforestation and other techniques, avoiding intermediate data structures in sparse data processing is still an open problem~\cite{graphblast}.