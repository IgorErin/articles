\section{Introduction}

\db{check the paragraph.}
Nowadays high-performance processing of a huge amount of data is an actual challenge not only for scientific computing but for applied systems.
Special types of hardware, such as General Purpose Graphic Processing Units (GPGPUs), Tensor Processing Units (TPUs), FPGA-based solutions, along with respective specialized software were developed to provide appropriate solutions, and the development of new solutions continues.
And sparse linear algebra, particularly GraphBLAS~\cite{buluc2017graphblas}, is a way to utilize all these accelerators to provide high-performance solutions in many areas including machine learning~\cite{Kepner2017} and graph analysis~\cite{graphblast}.

\db{TODO: connection between paragraphs?}

\db{maybe example with sequential matrix multiplication? Or it doesn't work?!?}
But evaluation of expressions over matrices generates intermediate data structures similar to the well-known example is a pipelined processing of collections: \verb| map g (map f data)|. Suppose \verb|data| is a list, then the first map produces a new list which then will be traversed by the second map. The same pattern can be observed in neural networks where initial data flow through network layers, or in linear algebra expressions where each subexpression produces an intermediate matrix. The last case occurs not only in scientific computations but in graph analysis~\cite{graphblast}. It is important that not only data structure will be traversed multiple times while can be traversed only once, but the intermediate data will also be stored (to RAM) which is a big problem for real-world data analysis: the size of data is huge and memory access is one of the expensive operations.
While a number of complex real-world cases including stream fusion and dense kernels fusion~\cite{fusion-boosting-memory-computations}, can be successfully optimized using deforestation~\cite{wadler-deforestation,wadler-deforestation-ho} and other techniques~\db{add_links}, avoiding intermediate data structures in sparse data processing is still an open problem~\cite{graphblast}.