\section{Implementation}

\at{Initially neither Reduceron nor FHW were stable enough to run our sparse routines. 
Reduceron supported only limited number of arguments for each function, while FHW had support only for algebraic data types with less than 64 number of fields which was not the case when translating our benchmarks. 
So both issues have been fixed.
On top of that, appropriate initial values were set for System Verilog signals in FHW to make it possible to utilize Vivado to handle further development.

FHW relies on External Core feature of GHC as a frontend, which has been removed since GHC $>$ 7.6.3. 
To mitigate this dependency, to make the code base more maintainable, and to overcome other issues, like overflows during CPS-transformation and support for partially applied functions, we have opted for GRIN~\cite{GRIN} as an intermediate representation between POT language and dataflow representation of FHW.
GRIN makes defunctionalization, which is essential for hardware generation, more convenient and provides extensive points-to analysis. Finally, FHW assumes the presence of a hardware garbage collector, but does not implement it. We also have not implemented this feature yet.

The distiller at the moment produces function duplicates during residualization, which is non-trivial to resolve. 
Such duplicates increase the consumption of logic blocks in hardware, so we remove them before translation. The usage of De Bruijn indexes makes it possible to rename only function names to determine whether two functions are duplicates.

Both Reduceron and FHW do not support external memory at the moment, and thus we store all the date inside our programs. It makes FHW-generated hardware larger, introduces code size overhead and unnecessary reductions.
}