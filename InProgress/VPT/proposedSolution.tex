\section{Proposed Solution}

The goal of our research is to figure out can distillation~\cite{hamilton2021700} be a solution of the intermediate data structures problem in linear algebra based programs.
To answer this question we developed a library of matrix operations in POT language: simple functional language used by Geoff Hamilton in his distiller.  

We use a quad-tree representation~\cite{qtree} for matrices because it avoids indexing and natural for functional programming because can be defined as an algebraic data type.
Moreover it allows one to represent both sparse and dense matrices naturally, and basic operations over such a representation can be natively expressed as recursive functions which traverse this tree-like structure.
Additionally, this structure allows natively exploiting divide-and-conquer parallelism in matrices handing functions.

We selected two different target hardware platforms.
The first one is a Reduceron~\cite{naylorRunciman2012} --- general-purpose functional-language-specific processor.
The second one is program-specific hardware for arbitrary functional programs FHW~\cite{Edwards2019FHWP} which utilizes the flexibility of FPGA to create hardware for a particular program.
While the first case is more typical, the second one may provide higher performance for specific tasks.

At the current stage, we propose to use distillation as the first step of program optimization which, we hope, should reduce memory traffic, and then compile a distilled program to two different hardware platforms using the respective compiler with platform-specific optimizations.
For evaluation, we propose to create a library of linear algebraic operations, such as matrix-matrix, matrix-vector, and matrix-scalar operations.
Programs of interest are compositions of these basic functions.