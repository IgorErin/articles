\section{Proposed Solution}

% The goal of our research is to figure out can distillation~\cite{hamilton2021700} be a solution of the intermediate data structures problem in linear algebra based programs.
The goal of our research is to find out if the linear algebra-based programs can be efficiently optimized by program distillation~\cite{hamilton2021700} by eliminating intermediate data structures and computations.
To answer this question, we implemented a library of matrix operations~\db{add link to our implementation} in POT language: a simple functional language used by Hamilton in his distiller~\db{add link to the Geoff's github}.  

% We use a quad-tree representation~\cite{qtree} for matrices because it avoids indexing and natural for functional programming because can be defined as an algebraic data type.
We use a quad-tree matrix representation since it both avoids indexing and can be implemented via algebraic data types, which itself is very natural for functional programming.
% Moreover it allows one to represent both sparse and dense matrices naturally, and basic operations over such a representation can be natively expressed as recursive functions which traverse this tree-like structure.
Besides, it provides a natural way to represent both sparse and dense matrices, as well as makes it possible to express basic operations over the representation via recursive functions traversing the tree-like structure.
% Additionally, this structure allows natively exploiting divide-and-conquer parallelism in matrices handing functions.
Finally, the quad-tree representation allows natively exploiting divide-and-conquer parallelism in matrices handing functions.

We selected two different target hardware platforms.
The first one is Reduceron~\cite{naylorRunciman2012} --- a general-purpose functional-language-specific processor.
The second is program-specific hardware for arbitrary functional programs FHW~\cite{Edwards2019FHWP} which utilizes the flexibility of FPGA to create hardware for a particular program.
While the first case is more typical, the second may provide higher performance for specific tasks.

% At the current stage,
For now, we propose to use program distillation as the first step of program optimization which, we hope, should reduce memory % traffic,
usage, and then compile a distilled program to the two different hardware platforms using the respective compiler with platform-specific optimizations.
For evaluation, we propose \db{\{propose or some of them are already implemented?!? =)) \}} to create a library of linear algebraic operations, such as matrix-matrix, matrix-vector, and matrix-scalar operations.
Programs of interest are compositions of these basic functions.