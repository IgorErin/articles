\section{Proposed Solution}

The goal of our research is to find out if linear algebra-based programs can be efficiently optimized by program distillation~\cite{hamilton2021700} through eliminating intermediate data structures and computations.
To answer this question, we have developed a library of matrix operations
%\footnote{Linear algebra operations in POT language: \url{https://github.com/YaccConstructor/Distiller/blob/master/inputs/LinearAlgebra.pot}}
 in POT language: a simple functional language used in distiller.%\footnote{Distiller implementation on GitHub: \url{https://github.com/poitin/Distiller}}.  

We use a quad-tree matrix representation~\cite{qtree} since it both avoids indexing and can be implemented via algebraic data type, which itself is very natural for functional programming.
Besides, it provides similar compression rate to widely adopted CSR and COO, and a natural way to represent both sparse and dense matrices, as well as makes it possible to express basic operations over the representation via recursive functions traversing the tree-like structure.
Finally, the quad-tree representation allows to natively exploit divide-and-conquer parallelism in matrices handling functions.

%\begin{listing}
%
%    \begin{minted}[]{Haskell}
%  
%    data QTree a = QNone  
%                 | QVal a 
%                 | QNode (QTree a) (QTree a)
%                         (QTree a) (QTree a) 
%    
%    \end{minted}
%    \caption{Quad-tree compressed representation}
%    \label{lst:qtree}
%    
%    \end{listing}

Since general purpose devices like CPUs and GPUs appear to be heavily underutilized when executing sparse routines due to low arithmetic intensity and memory boundness, custom hardware seems to be a promising solution towards mitigating these issues. 
Distillation provides all the needed optimizations for free for a functional language, so what remains is to translate a functional program into custom hardware. We have opted for two solutions here.
The first one is Reduceron~\cite{naylorRunciman2012} --- a processor, designed to perform highly performant reductions.
The second one is FHW~\cite{Edwards2019FHWP} --- a project, which is constituted of several compilers that help to translate an arbitrary Haskell progam into System Verilog to eventually provide a bitstream for a custom hardware.
It leverages a parallel and pipelined dataflow representation, which is abstracted over, e.g., nodes for memory operations, which makes it possible to come up with a specialized memory for our data structures.
While the first case is more typical, the second might provide higher performance for specific tasks.

For now, we propose to use program distillation as the first step of program optimization which, we hope, should reduce memory usage and other unnecessary computations, and then compile a distilled program to the two different hardware platforms by using the respective compiler with platform-specific optimizations.

For evaluation, we have been implementing a library of a subset of linear algebraic routines, only matrix-matrix operations for now: matrix-to-matrix element-wise addition (\verb|mtxAdd|),
matrix-to-scalar \emph{apply-to-all} operation (\verb|map|),
masking (\verb|mask|), which takes a subset of matrix elements, and Kronecker product (\verb|Kron|).% All operations are simplified, for example, they do not check matrices' size correctness, handling of incorrect input not provided.