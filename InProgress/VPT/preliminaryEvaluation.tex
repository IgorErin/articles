\section{Preliminary Evaluation}

% Compositions of such basic functions, matrix-matrix elementwise operations (\verb|mtxAdd|), matrix-scalar elementwise operation (\verb|map|), masking (\verb|mask|), Kronecker product (\verb|Kron|) were used for evaluation.
For now, we have implemented some basic functions for the proposed library, which are used in the current evaluation stage: matrix-to-matrix element-wise addition (\verb|mtxAdd|),
matrix\db{-to?}-scalar \emph{apply-to-all}\db{??} operation (\verb|map|),
masking (\verb|mask|), which takes a subset of matrix elements, and Kronecker product (\verb|Kron|).
% Namely, we use following compositions.
The following examples which are a combination of the implemented functions are used for the evaluation. The examples are fairly practical, for example, one could see a sequence of element-wise additions in a Luby's maximal independent set algorithm.

\begin{itemize}
\item Sequential matrix addition:\\
  \verb|seqAdd m1 m2 m3 m4 = mtxAdd (mtxAdd (mtxAdd m1 m2) m3) m4|
\item \db{todo}: \verb|addMask m1 m2 m3 = mask (mtxAdd m1 m2) m3|
\item \db{todo}: \verb|kronMask m1 m2 m3 = mask (kron m1 m2) m3 |  
\item \db{todo}: \verb|addMap m1 m2 = map f (mtxAdd m1 m2)|  
\item \db{todo}:\verb|kronMap m1 m2 = map f (kron m1 m2)|  
\end{itemize}

We compare original versions of these functions and distilled ones in three ways.
We use the interpreter of the POT language to measure the number of reductions and \at{memory reads inside} \verb|case| expressions.
We use the simulator shipped with Reduceron to measure the number of clock ticks necessary to evaluate a program, and Vivado's simulator for FHW-compiled programs to measure the number of both clock ticks and memory writes that a program produces.
% We use a set of randomly generated sparse matrices of appropreate size as a dynamic (not known statically) input for both versions.
A set of sparse matrices of appropriate sizes provided at~\cite{Matrices} is used. 
The matrices are converted into boolean ones since POT language lacks the needed primitives at the moment.
Average results for \at{several hundreds} of different inputs are presented in table~\ref{tbl:evaluationResults}.

\begin{table}[ht]
    \centering    
    \begin{tabular}{|c|c|c|c|c||c|c|c|c|c|}
        \hline
        \multirow{2}{*}{Function} &  \multicolumn{4}{c||}{Matrix size}  & \multicolumn{2}{c|}{Interpreter}            & Reduceron & \multicolumn{2}{c|}{FHW}\\
        \cline{2-10}
                                  &   m1 & m2 & m3 & m4                & Red-s & Reads                               & Ticks     & Ticks & Writes \\
        \hline
        \hline
        seqAdd   & $64 \times 64$ & $64 \times 64$ & $64 \times 64$ & $64 \times 64$ & 2.7          & 1.9        & 1.8 & 1.4 & 1.1 \\ 
        addMask  & $64 \times 64$ & $64 \times 64$ & $64 \times 64$ & --             & 2.1          & 1.8        & 1.4 & 1.4 & 1.1\\ 
        kronMask & $64 \times 64$ & $2 \times 2$   &$128 \times 128$& --             & 2.2          & 1.9        & 1.4 & 2.7 & 2.5\\ 
        addMap   & $64 \times 64$ & $64 \times 64$ & --             & --             & 2.5          & 1.7        & 1.7 & 1.5 & 1\\
        kronMap  & $64 \times 64$ & $2 \times 2$   & --             & --             & 2.9          & 2.2        & 1.8 & 2.0 & 1\\ 
        \hline
        
    \end{tabular}
    \caption{Evaluation results: original program to distilled one ratio of measured metrics is presented}
    \label{tbl:evaluationResults}
\end{table}

\at{We can see that on average distillation provides up to 3 and 2 times improvement in terms of reductions and memory reads respectively for the interpreter. 
The number of reductions is also considerably reduced for hardware benchmarks. 
The lack of matches between ticks for FHW and Reduceron is justified by different subsets of matrices being used in each benchmark and architecture distinction. 
Finally, from the last column one could see memory consumption reduction, which supports our approach.
All this hopefully makes the proposed solution viable, and we look forward to coming up with full-fledged experiments that would target real hardware and real life competitors like C++ implementations. }
% We can see that !!!.
% So, we can conclude !!!
