\section{Preliminary Evaluation}

Compositions of such basic functions, matrix-matrix elementwise operations (\verb|mtxAdd|), matrix-scalar elementwise operation (\verb|map|), masking (\verb|mask|), Kronecker product (\verb|Kron|) were used for evaluation.
Namely, we use following compositions.
\begin{itemize}
     \item Sequential addition of three matrices: \verb|Seq_add m1 m2 m3 = mtxAdd (mtxAdd m1 m2) m3 |  
     \item Sequential addition of four boolean matrices: \\ \verb|Seq_add m1 m2 m3 m4 = mtxAdd (mtxAdd (mtxAdd m1 m2) m3 m4|  
     \item Sequential addition of three matrices: \verb|Seq_add m1 m2 m3 = mtxAdd (mtxAdd m1 m2) m3 |  
     \item Sequential addition of three matrices: \verb|Seq_add m1 m2 m3 = mtxAdd (mtxAdd m1 m2) m3 |  
\end{itemize}
We compare original versions of these functions and distilled ones in three ways: using interpreter of POT language to measure a number of reductions and memory allocation, using simulator of Reduceron, and using simulation of FHW to measure a number of clock ticks necessary to evaluate a program.
We use a set of randomly generated sparse matrices of appropreate size as a dynamic (not known statically) input for both versions.
Average results for !!! different inputs are presented in table~\ref{tbl:evaluationResults}. 

\begin{table}[ht]
    \centering    
    \begin{tabular}{|c|c|c|c|c|c|c|c|c|}
        \hline
        \multirow{2}{*}{Function} &  \multicolumn{4}{c|}{Matrix size}  & \multicolumn{2}{c|}{Interpreter}             & Reduceron & FHW \\
        \cline{2-7}
                                  &   m1 & m2 & m3 & m4                & Red-s & Allocs                               & Ticks     & Ticks\\
        \hline
        \hline
        Seq\_add & $64 \times 64$ & $64 \times 64$ & $64 \times 64$ & $64 \times 64$ & 10          & 10        & 10 & 10 \\ 
        Seq\_add & $64 \times 64$ & $64 \times 64$ & $64 \times 64$ & $64 \times 64$ & 10          & 10        & 10 & 10 \\ 
        Seq\_add & $64 \times 64$ & $64 \times 64$ & $64 \times 64$ & $64 \times 64$ & 10          & 10        & 10 & 10 \\ 
        Seq\_add & $64 \times 64$ & $64 \times 64$ & $64 \times 64$ & $64 \times 64$ & 10          & 10        & 10 & 10 \\ 
        \hline
        
    \end{tabular}
    \caption{Evaluation results: original program to distilled one ratio of measured metrics is presented}
    \label{tbl:evaluationResults}
\end{table}

We can see that !!!.
So, we can conclude !!!