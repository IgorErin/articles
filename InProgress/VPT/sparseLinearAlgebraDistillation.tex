\documentclass[submission,copyright,creativecommons]{eptcs}
\pagestyle{plain}
\providecommand{\event}{VPT 2022} % Name of the event you are submitting to
\usepackage{breakurl}             % Not needed if you use pdflatex only.
\usepackage{underscore}           % Only needed if you use pdflatex.
\usepackage{lineno}
\usepackage{multirow}
\usepackage{xcolor}

\linenumbers
\title{Distillation of Sparse Linear Algebra}

%\author[1,4]{Aleksey Tyurin}
%\author[1,4]{Ekaterina Vinnik}
%\author[1,2,4]{Daniil Berezun}
%\author[1,2,4]{Semyon Grigorev}
%\author[3,4]{Geoff Hamilton}

\author{Aleksey Tyurin
\institute{Saint Petersburg University, Russia}
\institute{JetBrains Research, Russia}
\email{alekseytyurinspb@gmail.com}
\and
Ekaterina Vinnik
\institute{Saint Petersburg University, Russia}
\institute{JetBrains Research, Russia}
\email{catherine.vinnik@gmail.com}
\and
Mikhail Nikolukin
\institute{!!!}
\email{!!!}
\and
Daniil Berezun
\institute{Saint Petersburg University, Russia}
\institute{JetBrains Research, Russia}
\email{d.berezun@spbu.ru}
\email{daniil.berezun@jetbrains.com}
\and
Semyon Grigorev
\institute{Saint Petersburg University, Russia}
\institute{JetBrains Research, Russia}
\email{s.v.grigoriev@spbu.ru}
\email{semyon.grigorev@jetbrains.com}
\and
Geoff Hamilton
\institute{School of Computing, \\ Dublin City University, Ireland}
\email{geoffrey.hamilton@dcu.ie}
}

\def\titlerunning{Distillation of Sparse Linear Algebra}
\def\authorrunning{A. Tyurin, E. Vinnik, et al}


\newcommand{\db}[1]{{\color{violet} #1}}

\begin{document}

\maketitle

\begin{abstract}
    %% Linear algebra is a common language for many areas including machine learning and graph analysis. 
    %% It is a way to provide high-performance solutions by utilizing the highly parallelizable nature of linear algebra operations.
    %% However, intermediate data structures that arise during linear algebra expressions evaluation are one of the performance bottlenecks, especially for sparse data.
    %% We show that distillation appears to be a promising way to optimize linear algebra-based programs, and deals with such intermediate structures. 
  Linear algebra is a common language for many areas, including machine learning and graph analysis, providing high-performance solutions utilizing the highly parallelizable nature of linear algebra operations.
  However, a variety of intermediate data structures arising during linear algebra expressions evaluation is one of the primary performance bottlenecks, especially when sparse data are used.
  We show that program distillation can be efficiently used to optimize linear algebra-based programs, minimizing appearance and evaluation over intermediate data structures at runtime.
\end{abstract}


\section{Introduction}

Scalable high-performance graph analysis is an actual challenge.
There is a big number of ways to attack this challenge~\cite{Coimbra2021} and the first promising idea is to utilize general-purpose graphic processing units (GPGPU-s).
Such existing solutions, as CuSha~\cite{10.1145/2600212.2600227} and Gunrock~\cite{7967137} show that utilization of GPUs can improve the performance of graph analysis, moreover it is shown that solutions may be scaled to multi-GPU systems.
But low flexibility and high complexity of API are problems of these solutions.

The second promising thing which provides a user-friendly API for high-performance graph analysis algorithms creation is a GraphBLAS API~\cite{7761646} which provides linear algebra based building blocks to create graph analysis algorithms.
The idea of GraphBLAS is based on is a well-known fact that linear algebra operations can be efficiently implemented on parallel hardware.
Along with this, a graph can be natively represented using matrices: adjacency matrix, incidence matrix, etc.
While reference CPU-based implementation of GraphBLAS, SuiteSparse:GraphBLAS~\cite{10.1145/3322125}, demonstrates good performance in real-world tasks, GPU-based implementation is challenging.

One of the challenges in this way is that real data are often sparse, thus underlying matrices and vectors are also sparse, and, as a result, classical dense data structures and respective algorithms are inefficient. 
So, it is necessary to use advanced data structures and procedures to implement sparse linear algebra, but the efficient implementation of them on GPU is hard due to the irregularity of workload and data access patterns.
Though such well-known libraries as cuSparse show that sparse linear algebra operations can be efficiently implemented for GPGPU-s, it is not so trivial to implement GraphBLAS on GPGPU. 
First of all, it requires \textit{generic} sparse linear algebra, thus it is impossible just to reuse existing libraries which are almost all specified for operations over floats.
The second problem is specific optimizations, such as maskings fusion, which can not be natively implemented on top of existing kernels.
Nevertheless, there is a number of implementations of GraphBLAS on GPGPU, such as GraphBLAST:~\cite{yang2019graphblast}, GBTL~\cite{7529957}, which show that GPGPUs utilization can improve the performance of GraphBLAS-based graph analysis solutions.
But these solutions are not portable because they are based on Nvidia Cuda stack.
Moreover, the scalability problem is not solved: all these solutions support only single-GPU, not multi-GPU computations.

To provide portable GPU implementation of GraphBLAS API we developed a \textit{SPLA} library (sources are published on GitHub: \url{https://github.com/JetBrains-Research/spla}).
This library utilizes OpenCL for GPGPU computing to be portable across devices of different vendors.
Moreover, it is initially designed to utilize multiple GPGPUs to be scalable.
To sum up, the contribution of this work is the following.
\begin{itemize}
    \item Design of portable GPU GraphBLAS implementation proposed. The design involves the utilization of multipole GPUS. Additionally, the proposed design is aimed to simplify library tuning and wrappers for different high-level platforms and languages creation. 
    \item Subset of GraphBLAS API, including such operations as masking, matrix-matrix multiplication, matrix-matrix e-wise addition, is implemented. The current implementation is limited by COO and CSR matrix representation format and uses basic algorithms for some operations, but work in progress and more data formats will be supported and advanced algorithms will be implemented in the future.
    \item Preliminary evaluation on such algorithms as breadth-first search (BFS) and triangles counting (TC), and real-world graphs shows portability across different vendors and promising performance: for some problems Spla is comparable with GraphBLAST. Surprisingly, for some problems, the proposed solution on embedded Intel graphic card shows better performance than SuiteSparse:GraphBLAS on the same CPU. At the same time, the evaluation shows that further optimization is required.
\end{itemize} 
\section{Proposed Solution}

The goal of our research is to figure out can distillation~\cite{hamilton2021700} be a solution of the intermediate data structures problem in linear algebra based programs.
To answer this question we developed a library of matrix operations in POT language: simple functional language used by Geoff Hamilton in his distiller.  

We use a quad-tree representation~\cite{qtree} for matrices because it avoids indexing and natural for functional programming because can be defined as an algebraic data type.
Moreover it allows one to represent both sparse and dense matrices naturally, and basic operations over such a representation can be natively expressed as recursive functions which traverse this tree-like structure.
Additionally, this structure allows natively exploiting divide-and-conquer parallelism in matrices handing functions.

We selected two different target hardware platforms.
The first one is a Reduceron~\cite{naylorRunciman2012} --- general-purpose functional-language-specific processor.
The second one is program-specific hardware for arbitrary functional programs FHW~\cite{Edwards2019FHWP} which utilizes the flexibility of FPGA to create hardware for a particular program.
While the first case is more typical, the second one may provide higher performance for specific tasks.

At the current stage, we propose to use distillation as the first step of program optimization which, we hope, should reduce memory traffic, and then compile a distilled program to two different hardware platforms using the respective compiler with platform-specific optimizations.
For evaluation, we propose to create a library of linear algebraic operations, such as matrix-matrix, matrix-vector, and matrix-scalar operations.
Programs of interest are compositions of these basic functions.
\section{Preliminary Evaluation}

Compositions of such basic functions, matrix-matrix elementwise operations (\verb|mtxAdd|), matrix-scalar elementwise operation (\verb|map|), masking (\verb|mask|), Kronecker product (\verb|Kron|) were used for evaluation.
Namely, we use following compositions.
\begin{itemize}
     \item \verb|seqAdd m1 m2 m3 m4 = mtxAdd (mtxAdd (mtxAdd m1 m2) m3) m4|
     \item \verb|addMask m1 m2 m3 = mask (mtxAdd m1 m2) m3 |  
     \item \verb|kronMask m1 m2 m3 = mask (kron m1 m2) m3 |  
     \item \verb|addMap m1 m2 = map f (mtxAdd m1 m2)|  
     \item \verb|kronMap m1 m2 = map f (kron m1 m2)|  
\end{itemize}
We compare original versions of these functions and distilled ones in three ways: using interpreter of POT language to measure a number of reductions and memory allocation, using simulator of Reduceron, and using simulation of FHW to measure a number of clock ticks necessary to evaluate a program.
We use a set of randomly generated sparse matrices of appropreate size as a dynamic (not known statically) input for both versions.
Average results for !!! different inputs are presented in table~\ref{tbl:evaluationResults}. 

\begin{table}[ht]
    \centering    
    \begin{tabular}{|c|c|c|c|c||c|c|c|c|}
        \hline
        \multirow{2}{*}{Function} &  \multicolumn{4}{c||}{Matrix size}  & \multicolumn{2}{c|}{Interpreter}            & Reduceron & FHW \\
        \cline{2-7}
                                  &   m1 & m2 & m3 & m4                & Red-s & Allocs                               & Ticks     & Ticks\\
        \hline
        \hline
        seqAdd   & $64 \times 64$ & $64 \times 64$ & $64 \times 64$ & $64 \times 64$ & 2.7          & 1.9        & 1.8 & 10 \\ 
        addMask  & $64 \times 64$ & $64 \times 64$ & $64 \times 64$ & --             & 2.1          & 1.8        & 1.4 & 10 \\ 
        kronMask & $64 \times 64$ & $2 \times 2$   &$128 \times 128$& --             & 2.2          & 1.9        & 1.4 & 10 \\ 
        addMap   & $64 \times 64$ & $64 \times 64$ & --             & --             & 2.5          & 1.7        & 1.7 & 10 \\
        kronMap  & $64 \times 64$ & $2 \times 2$   & --             & --             & 2.9          & 2.2        & 1.8 & 10 \\ 
        \hline
        
    \end{tabular}
    \caption{Evaluation results: original program to distilled one ratio of measured metrics is presented}
    \label{tbl:evaluationResults}
\end{table}

We can see that !!!.
So, we can conclude !!!
\section{Future Work}

%We show that distillation is a promising way to optimize linear algebra-based programs, which makes it also applicable to optimize machine learning and graph processing procedures.

In the future, first, we should close a technical debt and make the distiller more stable to handle all the important cases: current implementation can not handle such important functions as matrix-matrix multiplication.
%Along with it, we should improve the input language to make it more user-friendly.
%The main challenge here is to find the balance between language expressivity and the practicality of distillation for it.
Having basic workflow implemented, we should explore how to utilize distillation in the best way for each particular platform. 
For example, which level of distillation is the best for our particular problem and set of functions?
Can we exploit more parallelism using distillation?
Can we efficiently exploit the tail-modulo-cons property of the distilled program?
What are the limitations of distillation: whether all important cases can be handled?

%When the language and the distiller are stable enough, we plan to implement a full-featured generic linear algebra library power enough to express basic graph analysis algorithms and to create and train neural networks.
%After that, a number of graph analysis algorithms and neural networks will be implemented and evaluated.

In addition to it, we plan to improve both FHW and Reduceron and compilers for them in order to make them mature enough to handle real-world examples.
The most relevant improvement here, for example, is the support for out-of-chip memory.

%Finally, a comprehensive evaluation of proposed solutions should be done using FPGA, not hardware simulators. The main challenge here is to utilize the resources of modern FPGA accelerators, such as HBM, to achieve the maximal performance of our solution.

\bibliographystyle{eptcs}
\bibliography{sparseLinearAlgebraDistillation}
\end{document}
