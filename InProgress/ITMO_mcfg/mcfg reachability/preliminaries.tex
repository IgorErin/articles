\section{Preliminaries}
\label{sec:preliminaries}

In this section, we introduce common definitions in graph theory and formal language theory which are used in this paper.
Also, we provide a brief description of the MCFL-reachability problem.

\subsection{Basic Definitions of Graph Theory}

In this paper, we use a labeled directed graph as a data model and define it as follows.
\begin{definition} \emph{Labeled directed graph} is a tuple $D = (V, E, \Sigma)$, where
\begin{itemize}
    \item $V$ is a finite set of vertices. For simplicity, we assume that the vertices are natural numbers ranging from $1$ to $|V|$,
    \item $\Sigma$ is a set of edge labels,
    \item $E \subseteq V \times \Sigma \times V$ is a set of labeled edges.
\end{itemize} 
\end{definition}

An example of the labeled directed graph $D_1$ is presented in Figure~\ref{fig:example_input_graph} where the set of vertices $V = \{1, 2, 3, 4, 5\}$, the set of edge labels $\Sigma = \{a, b, c, d\}$, and the set of labeled edges $$E = \{(1, a, 2), (2, d, 1), (2, b, 3), (3, c, 2), (3, c, 4), (4, b, 3), (4, d, 5), (5, a, 4)\}.$$

\begin{figure}[h]
    \centering
    \begin{tikzpicture}[shorten >=1pt,auto]
        \node[state] (q_0)                       {$1$};
        \node[state] (q_1) [right=of q_0]        {$2$};
        \node[state] (q_2) [right=of q_1]        {$3$};
        \node[state] (q_3) [right=of q_2]        {$4$};
        \node[state] (q_4) [right=of q_3]        {$5$};
        \path[->]
         (q_0) edge[bend left,above] node {$a$} (q_1)
         (q_1) edge[bend left,above] node {$b$} (q_2)
         (q_2) edge[bend left,above] node {$c$} (q_3)
         (q_3) edge[bend left,above] node {$d$} (q_4)
         (q_4) edge[bend left,below] node {$a$} (q_3)
         (q_3) edge[bend left,below] node {$b$} (q_2)
         (q_2) edge[bend left,below] node {$c$} (q_1)
         (q_1) edge[bend left,below] node {$d$} (q_0);
    \end{tikzpicture}
    \caption{The input graph $D_1$}
    \label{fig:example_input_graph}
\end{figure}

\begin{definition}
A \emph{path} $\pi$ in the graph $D = (V, E, \Sigma)$ is a finite sequence of labeled edges $(e_1, ..., e_{n})$, where $\forall i: 1 \leq i \leq n \ \ e_i=(v_{i - 1},l_i, v_{i}) \in E$.
\end{definition}

\begin{definition}
	The \emph{word} $l(\pi) \in \Sigma^*$ for the graph $D=(V, E, \Sigma)$ is the unique word $l_1 ... l_n$, obtained by concatenating the labels of the edges along the path $$\pi = (e_1 = (v_0, l_1, v_1), \ldots, e_n = (v_{n-1}, l_n, v_n)).$$
\end{definition}


\begin{definition}
    Let $D = (V, E, \Sigma)$ be an labeled directed graph. The \emph{adjacency matrix} $M_D$ of the graph $D$ is a matrix of size $|V| \times |V|$, such that 
    \begin{equation*}
        M_D[i,j] = \{l~|~(i, l, j) \in E\}
    \end{equation*}
\end{definition}

It is usual to decompose the adjacency matrix into a set of Boolean matrices for the implementation reasons. Thus, we introduce the following definitions.

\begin{definition}
    Let $M_D$ be the adjacency matrix of the graph $D = (V, E, \Sigma)$. An \emph{Boolean adjacency matrix for the label} $l \in \Sigma$ of the graph $D$ is a matrix $M_D^l$ of size $|V| \times |V|$, such that 
    \begin{equation*}
        M_D^l[i,j] = \begin{cases}
                        True &, l \in M_D[i,j]\\
                        False &, otherwise
                    \end{cases}
    \end{equation*}
\end{definition}

\begin{definition}
    A \emph{Boolean decomposition of adjacency matrix} $M_D$ of the graph $D$ is a set of Boolean matrices $\mathcal{M} = \{M_D^l~|~l \in \Sigma\}$.
\end{definition}

\subsection{Basic Definitions of Formal Languages}
We use multiple context-free languages (MCFLs) as path constraints, thus in this subsection, we define multiple context-free languages and grammars.

\begin{definition}A \emph{multiple context-free grammar} $G$ is a tuple $(N, \Sigma, P, S, d)$, where
\begin{itemize}
    \item $N$ is a finite set of nonterminals
    \item $S \in N$ is the start nonterminal 
    \item For each $A \in N$ a natural number $d(A)$, called the dimension of $A$, is defined. In particular, we assume $d(S) = 1$. Sometimes $A$ is written as $(A^1, \dots, A^{d(A)})$, where $\forall i: 1 \leq i \leq d(A) \ \ A^i$ are called component symbols of $A$. Let $N_c = \{A^i | A \in N, 1 \leq i \leq d(A)\}$  be the set of component symbols.
    \item $\Sigma$ is a finite set of terminals, $N \cap \Sigma = \varnothing$
    \item $P$ is a finite set of production rules. $p \in P$ has a form of $ p: (A^1, \dots, A^{d(A)}) \rightarrow (\gamma_1, \dots, \gamma_{d(A)}) $, where $A \in N$ and $\gamma_i \in (N_c \cup \Sigma)^*$. Each rule $p$ satisfies the following condition.
    \begin{itemize}
        \item \textbf{Right-linearity}: $\forall A^i \in N_c$ \ \ $A^i$ appears in the right-hand side (rhs) of $p$ at most once.
    \end{itemize}
\end{itemize} 
\end{definition}

\begin{definition}
    Let G be a multiple context-free grammar (MCFG). The dimension of G is $m =  \max\limits_{A \in N}(d(A))$.
\end{definition}
For example, $m = 1$ for the context-free grammars.

We use the conventional notation $A \xLongrightarrow[G]{*} (w_1, \dots, w_{d(A)})$, where $\forall i: w_i \in \Sigma^*$, to denote, that a tuple of
words can be derived from a nonterminal $A$ by some sequence of production rule applications from $P$ in grammar $G$ if $\forall i: w_i$ can be derived from $i-th$ component of a nonterminal A.

\begin{definition} A \emph{multiple context-free language} (MCFL) is a language generated by a multiple context-free grammar $G=(N, \Sigma, P, S, d)$:
\begin{align*}
    L(G) &= \{w \in \Sigma^* \mid S \xLongrightarrow[G]{*} w \}.
\end{align*}
\end{definition}

Also, we introduce the following normal form for the MCFGs that allow us to solve the MCFL-reachability problem using the LA operations.

\begin{definition} A multiple context-free grammar $G = (N, \Sigma, P, S, d)$ is in \emph{normal form} if every production rule $p: A \rightarrow (\gamma_1, \dots, \gamma_{d(A)}) \in P$ satisfies one of two forms:
    \begin{itemize}
        %\item $\forall i: 1 \leq i \leq d(A) \ \ \gamma_i \neq \varepsilon$
            \item $\forall i: |\gamma_i| = 1$ and $\gamma_i \in \Sigma \cup \{\varepsilon\}$. In this case, we call the rule p \textit{terminating},
            \item $\forall i: \gamma_i \in N_c^*$, i.e. no terminal symbol appears in the rhs of p. For simplicity of notation, we denote $p: A \rightarrow f(B_1, \dots, B_n)$, where $B_k \in N$. It is necessary that $n = 2$. To sum up, $p: A \rightarrow f(B_1, B_2)$, where $B_1, B_2 \in N$. In this case, we call the rule p \textit{nonterminating}. Any rule in this form must satisfy the following conditions.
            \begin{itemize}
                \item \textbf{Non-erasing condition}:$\forall i \in \{1,2\}$ $B_i^j \ \ 1 \leq j \leq d(B_i)$ appears in $\gamma_k$ for some k,
                \item no pair $B^j, B^k$ of component symbols of the same nonterminal appear adjacently in the rhs in one component, i.e. component symbols of $B_1, B_2$ appear alternately in one component,
                \item $\exists i: 1 \leq i \leq d(A) \ \ |\gamma_i| \geq 2$.
            \end{itemize}
        \end{itemize}
\end{definition}

According to~\cite{nakanishi1997efficient}, the following theorem holds.

\begin{mytheorem}\label{thm:formofmcfg}
Let $G$ be an MCFG. An MCFG $G'$ can be constructed from $G$ such that $L(G') = L(G)$ and $G'$ satisfies described normal form.
\end{mytheorem}

For example, for $m = 1$ the normal form defined above is the same as the weak Chomsky normal form for context-free grammars described in~\cite{azimov2021context}.

Let us consider the MCFG $G_1 = (N_1, \Sigma_1, P_1, S, d_1)$, where $N_1 = \{S_, A, B\}$ ($d_1(S) = 1$, $d_1(A) = d_1(B) = 2$), $\Sigma_1 = \{a,b,c,d\}$, $P_1 = \{$
\begin{align*}
    S^1 \rightarrow A^1 B^1 A^2 B^2 \\
    (A^1, A^2) \rightarrow (a,c) \\
    (B^1, B^2) \rightarrow (b,d) \\
    (B^1, B^2) \rightarrow (bB^1, dB^2) \\
    (A^1, A^2) \rightarrow (aA^1, cA^2)
\end{align*}
$\}.$

This grammar generates one of the classical MCFLs: $L(G_1) = \{a^nb^mc^nd^m \mid n,m \in \mathbb{N}\}$ that is known not to be context-free.

A grammar $G'_1$ in normal form can be constructed from $G$, where $L(G'_1) = L(G_1)$ and $N'_1 = \{S, S_1, S_2, A, B, C, D\}$, $P'_1 = \{$
\begin{align*}
    A^1 \rightarrow a \\
    B^1 \rightarrow b \\
    C^1 \rightarrow c \\
    D^1 \rightarrow d \\
    (S_1^1, S_1^2) \rightarrow (a,c) \\
    (S_2^1, S_2^2) \rightarrow (b,d) \\
    (S_1^1, S_1^2) \rightarrow (S_3^1, S_3^2 C^1) \\
    (S_3^1, S_3^2) \rightarrow (A^1 S_1^1, S_1^2) \\
    (S_2^1, S_2^2) \rightarrow (S_4^1, S_4^2 D^1) \\
    (S_4^1, S_4^2) \rightarrow (B^1 S_2^1, S_2^2) \\
    S^1 \rightarrow S_1^1 S_2^1 S_1^2 S_2^2
\end{align*}
$\}.$

\subsection{MCFL-reachability problem}

\begin{definition}
Let $D = (V, E, \Sigma)$ be a labeled graph, $L$ be an MCFL. Then a \emph{multiple context-free relation} with language $L$ on the labeled graph $D$ is the relation $R_{D, L} \subseteq V \times V$:
\begin{equation*} \label{eq1}
\begin{split}
R_{D, L} = \{ &(v_0, v_n) \in V \times V  \mid \\ &\ \exists \pi = ((v_0, l_1, v_1), \ldots, (v_{n-1}, l_n, v_n)) \in \pi(D): \\
      &\ l(\pi) \in L \}.
\end{split}
\end{equation*}
\end{definition}

Finally, we can define MCFL-reachability problem.
\begin{definition}
    \emph{MCFL-reachability} is the problem of finding multiple context-free relation $R_{D, L}$ for a given directed labeled graph $D$ and a multiple context-free language $L$.
\end{definition}

In other words, the result of MCFL-reachability evaluation is a set of vertex pairs such that there is a path between them that forms a word from the given MCFL.

Next, we build an algorithm to solve MCFL-reachability problem.