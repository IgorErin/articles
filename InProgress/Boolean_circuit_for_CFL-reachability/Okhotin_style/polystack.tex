\section{Languages with poly-slender store languages}
\label{sec:poly}
In the Section~\ref{sec:osc} restriction on PDA in terms of variability of stack height was described. This restriction does not hold for PDA for the language $D_1$; however this language has polynomial rational index. In this section, another kind of stack restriction is considered~--- poly-slenderness of a pushdown store language as a measure of how stack contents vary along accepting computations of PDA.


For a PDA $M$, its \textit{pushdown store language} $P(M)$ consists of all words
occurring on the stack in accepting computations of $M$. It is well-known that the store language of every PDA is regular. The language $D_1$ is a one-counter language, so its pushdown store language is $Z^*Z_0$, where $Z$ is a single pushdown symbol and $Z_0$ is the bottom symbol.


Afrati et al. \cite{ChainQ} define the notion of \textit{polynomial stack property} and show that if a PDA has the polynomial stack property, then the corresponding query has the polynomial fringe property (and hence, lies in NC). A PDA $M$ has the polynomial stack property if and only if the number of different strings of length $k$ occuring in the stack in any accepting computation of $M$ is bounded by $O(k^d)$,  for $d \ge 0$. For example, the usual PDA for $D_1$ has the polynomial stack property, because there is only one possible variant of contents for every stack height. 


Generalizing an example of the family of one-counter languages, we can define the family of languages whose PDAs have the polynomial stack property~--- languages with a \textit{poly-slender} pushdown store language (or storage language with the polynomial density). The density of a language is a function $f(n)$ that shows the number of words of length $n$ in the language. A language $L \subseteq \Sigma^*$ is called \textit{poly-slender language (or with the polynomial density)} if the function $f(n)$ is bounded by $O(n^k)$ for some $k \ge 0$. For example, the language $Z^*Z_0$ is of polynomial density (even of a constant density), whereas the language ${(Z_1 + Z_2)}^*Z_0$ is of exponential density.


Whereas the polynomial fringe property of a query is undecidable \cite{Ullman}, it is decidable in polynomial time whether a given PDA has a poly-slender storage language. At first, for a given NFA it is decidable whether its language has a polynomial or exponential density \cite*{sparseness, poldens}. Gawrychowski et al. \cite{Gawrychowski} give an algorithm for testing whether $L(M)$ is of polynomial or exponential density in $O(|Q| + |\delta|)$ time for an NFA $M = (Q,\Sigma,\delta ,q_{0},F)$. An NFA for pushdown store language of a given PDA $\mathcal{A} = (Q', \Sigma', \Gamma, \delta', q_0', Z_0, F')$ can be constructed directly in $O({|Q'|}^5{|\Gamma|}^2|\delta'|)$ time \cite{PSLDirect}. This construction uses the notion of meaningful triples, which form the states of NFA. A triple $[p, Z, q] \in Q' \times \Gamma \times Q'$ is \textit{meaningful} if there exists a computation of $\mathcal{A}$ starting from state $p$ with the sole symbol $Z$ in the pushdown, and ending in $q$ with the empty pushdown. By definition, there are at most $|\Gamma|{|Q'|}^2$ meaningful triples, and, hence, states of NFA. 