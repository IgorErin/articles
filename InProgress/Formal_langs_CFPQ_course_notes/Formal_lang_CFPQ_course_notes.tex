%\documentclass[a4paper,12pt]{article}  % standard LaTeX, 12 point type
\documentclass[12pt, a4paper, table]{book}

\usepackage{algpseudocode}
\usepackage{algorithm}
\usepackage{algorithmicx}

\usepackage{geometry}
\usepackage{amsfonts,latexsym}
\usepackage{amsthm}
\usepackage{amssymb}
\usepackage[utf8]{inputenc} % Кодировка
\usepackage[english,russian]{babel} % Многоязычность
\usepackage{mathtools}
\usepackage{hyperref}
\usepackage{tikz}
\usepackage{dsfont}
\usepackage{multicol}
\usepackage[bb=boondox]{mathalfa}

\usetikzlibrary{fit,calc,automata,positioning}

\theoremstyle{definition}
\newtheorem{definition}{Определение}[section]
\newtheorem{example}{Пример}[section]
\newtheorem{theorem}{Теорема}[section]
\newtheorem{proposition}[theorem]{Proposition}
\newtheorem{lemma}[theorem]{Лемма}
\newtheorem{corollary}[theorem]{Corollary}
\newtheorem{conjecture}[theorem]{Conjecture}
\newtheorem{note}[theorem]{Утверждение}


% unnumbered environments:

\theoremstyle{remark}
\newtheorem*{remark}{Remark}
%\newtheorem*{notation}{Notation}

\setlength{\parskip}{5pt plus 2pt minus 1pt}
%\setlength{\parindent}{0pt}


\algtext*{EndWhile}% Remove "end while" text
\algtext*{EndIf}% Remove "end if" text
\algtext*{EndFor}% Remove "end for" text
\algtext*{EndFunction}% Remove "end function" text


\usepackage{color}
\usepackage{listings}
\usepackage{caption}
\usepackage{graphicx}
\usepackage{ucs}

\graphicspath{{pics/}}

%\geometry{left=2cm}
%\geometry{right=1.5cm}
%\geometry{top=2cm}
%\geometry{bottom=2cm}




%\lstnewenvironment{algorithm}[1][]
%{
%    \lstset{
%        frame=tB,
%        numbers=left,
%        mathescape=true,
%        numberstyle=\small,
%        basicstyle=\small,
%        inputencoding=utf8,
%        extendedchars=\true,
%        keywordstyle=\color{black}\bfseries,
%        keywords={,function, procedure, return, datatype, function, in, if, else, for, foreach, while, denote, do, and, then, assert,}
%        numbers=left,
%        xleftmargin=.04\textwidth,
%        #1 % this is to add specific settings to an usage of this environment (for instnce, the caption and referable label)
%    }
%}
%{}

\newcommand{\tab}[1][0.3cm]{\ensuremath{\hspace*{#1}}}

\newcommand{\rvline}{\hspace*{-\arraycolsep}\vline\hspace*{-\arraycolsep}}

\newcommand{\derives}[1][*]{\xRightarrow[]{#1}}
\newcommand{\first}[1][1]{\textsc{first}_{#1}}
\newcommand{\follow}[1][1]{\textsc{follow}_{#1}}

\setcounter{MaxMatrixCols}{20}


\tikzset{
%->, % makes the edges directed
%>=stealth’, % makes the arrow heads bold
node distance=4cm, % specifies the minimum distance between two nodes. Change if necessary.
%every state/.style={thick, fill=gray!10}, % sets the properties for each ’state’ node
initial text=$ $, % sets the text that appears on the start arrow
}

\tikzstyle{symbol_node} = [shape=rectangle, rounded corners, draw, align=center]

\tikzstyle{r_state} = [shape=rectangle, draw, minimum size=0.2cm]

\tikzstyle{prod_node} = [shape=rectangle, draw, align=center]

\tikzset{
    between/.style args={#1 and #2}{
         at = ($(#1)!0.5!(#2)$)
    }
}

%every node/.style = {shape=rectangle, rounded corners,
%      draw, align=center,
%      top color=white, bottom color=blue!20}

\newcommand{\bfgray}[1]{\cellcolor{lightgray}\textbf{#1}}

\newenvironment{scaledalign}[4]
  {
    \begingroup
    #1
    \setlength\arraycolsep{#2}
    \renewcommand{\arraystretch}{#3}
    \begin{center}
    \begin{equation}
    \begin{aligned}
    #4
  }
  {
    \end{aligned}
    \end{equation}
    \end{center}
    \endgroup
  }

\title{Приложения теории формальных языков и синтаксического анализа}
\author{Семён Григорьев}
\date{\today}

\begin{document}
\maketitle
\newpage
\tableofcontents
\newpage

\input{List_of_contributors}
\chapter*{Введение}

Теория формальных языков находит применение не только для ставших уже классическими задач синтаксического анализа кода (языков программирования, искусственных языков) и естественных языков, но и в других областях, таких как статический анализ кода, графовые базы данных, биоинформатика, машинное обучение.

Например, в машинном обучении использование формальных грамматик позволяет передать искусственной нейронной сети, предназначенной для генерации цепочек с определёнными свойствами (генеративной нейронной сети), знания о синтаксической структуре этих цепочек, что позволяет существенно упростить процесс обучения и повысить качество результата~\cite{10.5555/3305381.3305582}.
Вместе с этим, развиваются подходы, позволяющие нейронным сетям наоборот извлекать синтаксическую структуру (строить дерево вывода) для входных цепочек~\cite{kasai-etal-2017-tag,kasai-etal-2018-end}.

В биоинформатике формальные грамматики нашли широкое применение для описания особенностей вторичной структуры геномных и белковых последовательностей~\cite{Dyrka2019,WJAnderson2012,zier2013rna}.
Соответствующие алгоритмы синтаксического анализа используются при создании инструментов обработки данных.

Таким образом, теория формальных языков выступает в качестве основы для многих прикладных областей, а алгоритмы синтаксического анлиза применимы не только для обработки естественных языков или языков программирования.
Нас же в данной работе будет интересовать применение теории формальных языков и алгоритмов синтаксического анализа для анализа графовых баз данных и для статического анализа кода.

Одна из классических задач, связанных с анализом графов --- это поиск путей в графе.
Возможны различные формулировки этой задачи.
В некоторых случаях необходимо выяснить, существует ли путь с определёнными свойствами между двумя выбранными вершинами.
В других же ситуациях необходимо найти все пути в графе, удовлетворяющие некоторым свойствам или ограничениям. 
Например, в качестве ограничений можно указать, что искомый путь должен быть простым, кратчайшим, гамильтоновым и так далее.

Один из способов задавать ограничения на пути в графе основан на использовании формальных языков.
Базовое определение языка говорит нам, что язык --- это множество слов над некоторым алфавитом.
Если рассмотреть граф, рёбра которого помечены символами из алфавита, то путь в таком графе будет задавать слово: достаточно соединить последовательно символы, лежащие на рёбрах пути.
Множество же таких путей будет задавать множество слов или язык.
Таким образом, если мы хотим найти некоторое множество путей в графе, то в качестве ограничения можно описать язык, который должно задавать это множество.
Иными словами, задача поиска путей может быть сформулирована следующим образом: необходимо найти такие пути в графе, что слова, получаемые конкатенацей меток их рёбер, принадлежат заданному языку.
Такой класс задач будем называть задачами поиска путей с ограничениям в терминах формальных языков.

Подобный класс задач часто возникает в областях, связанных с анализом граф-структурированных данных и активно исследуется~\cite{doi:10.1137/S0097539798337716,axelsson2011formal,10.1007/978-3-642-22321-1_24,Ward:2010:CRL:1710158.1710234,barrett2007label,doi:10.1137/S0097539798337716}.
Исследуются как классы языков, применяемых для задания ограничений, так и различные постановки задачи.

Граф-структурированные данные встречаются не только в графовых базах данных, но и при статическом анализе кода: по программе можно построить различные графы отображающие её свойства.
Скажем, граф вызовов, граф потока данных и так далее.
Оказывается, что поиск путей в специального вида графах с использованием ограничений в терминах формальных языков позволяет исследовать некоторые свойства программы.
Например проводить межпроцедурный анализ указателей или анализ алиасов~\cite{Zheng,10.1145/2001420.2001440,10.1145/2714064.2660213}, строить срезы программ~\cite{10.1145/193173.195287}, проводить анализ типов~\cite{10.1145/373243.360208}.

В данной работе представлен ряд алгоритмов для поиска путей с ограничениями в терминах формальных языков.
Основной акцент будет сделан на контекстно-свободных языках, однако будут затронуты и другие классы: регулярные, многокомпонентные контекстно-свободные (Multiple Context-Free Languages, MCFL~\cite{!!!}) и конъюнктивные языки.
Будет показано, что теория формальных языков и алгоритмы синтаксического анализа применимы не только для анализа языков программирования или естественных языков, а также для анализа графовых баз данных и статического анализа кода, что приводит к возникновению новых задач и переосмыслению старых.


Структура данной работы такова.
В первых двух частях (~\ref{chpt:LinAlIntro} и~\ref{chpt:GraphTheoryIntro}) мы рассмотрим основные понятия из алгебры и теории графов, необходимые в данной работе. Данные разделы являются подготовительными и не обязательны к прочтению, если такие понятия как \textit{полукольцо} и \textit{матрица смежности} вам известны. Более того, они лишь вводят определения, подазумевая, что более детальное изучение соответствующих разделов остаются за рамками этой работы и скорее всего уже проделано читателем.
Затем, в главе~\ref{chpt:FormalLanguageTheoryIntro} мы введём основные понятия из теории формальных языков.
Далее, в главе~\ref{chpt:CFPQ} рассмотрим различные варианты постановки задачи поиска путей с ограничениями в терминах формальнх языков, обсудим базовые свойства задач, её разрешимость в различных постановках и т.д..
И в итоге зафиксируем постановку, которую будем изучать далее.
После этого, в главах~\ref{chpt:CFPQ_CYK}--\ref{chpt:CFPQ_Derivatives} мы будем подробно рассматривать различные алгоритмы решения этой задачи, попутно вводя специфичные для рассматриваемого алгоритма структуры данных.
Большинство алгоритмов будут основаны на классических алгоритмах синтаксического анализа, таких как CYK или LR.
Все главы, начиная с~\ref{chpt:GraphTheoryIntro}, снабжены списком вопросов и задач для самостоятельного решения и закрепления материала.
\chapter[x]{Некоторые понятия линейной алгебры\footnote{Неообходимо понимать, что, с одной строны, в данном разделе рассматриваются самые базовыепонятия, которые даются практически в любом учебнике алгебры. С другой же стороны, определения данных понятий оказываются весьма вариативными и часто вызывают дискуссии. Напрмиер, интересный анализ тонкостей определения группы можно найти в первом и втором параграфах первого раздела книги Николая Александровича Вавилова ``Конкретная теория групп''~\cite{VavilovGroups}. Мы же дадим определения, удобные для дальнейшего изложения материала.}}\label{chpt:LinAlIntro}

При изложении ряда алгоритмов будут активно использоваться некоторые понятия и инструмены линейной алгебры, такие как моноид, полукольцо или матрица.
В данном разделе необходимые понятия будут определены и приведены некоторые примеры соответствующих конструкций. Для более глубокого изучения материала рекомендуются соответствующие разделы алгебры.

$$
\oplus
\otimes
\mathbb{1}
\mathbb{0}
$$

\section{Бинарные операции и их свойства}


Введём понятие \textit{бинарной операции} и рассмотрим некоторые её свойства, такие как \textit{коммутативность} и \textit{ассоциативность}.

\begin{definition}[Двухместная функция] Функцию, принимающую два аргумента, $f: S \times K \to Q$ будем называть двухместной или функцией арности два.
Для запси таких функций будем использовать типичную нотацию: $c = f(a,b)$.
\end{definition}


\begin{definition}[Бинарная операция] 
Бинарная операция --- это двухместная функция, от которой дополнительно требуется, чтобы оба аргумента и результат лежали в одном и том же множестве: $f: S \times S \to S$. В таком случае говорят, что бинарная операция определена на некотором множестве $S$. Для обозначения произвольной бинарной операции будем использовать символ $\circ$ и пользоваться инфиксной нотацией для записи: $c = a \circ b$.
\end{definition}




\begin{definition}[Внешняя бинарная операция]
Внешняя бинарная операция --- это бинарная операция, у которой аргументы лежат в разных множествах, при этом результат --- в одном из этих множеств. Иными словами $\circ: K \times S \to S$, где $K$ может быть не равно $S$  --- внешняя бинарная операция.
\end{definition}


Необходимо помнить, что как функции, так и бинарные операции, могут быть частично определёнными (частичные функции, частичные бинарные операции). Типичным примером частично определённой бинарной операции является деление на целых числах: она не определена, если второй аргумент равен нулю.


Бинарные операции могут обладать некоторыми дополнительными свойствами, такими как \textit{коммутативность} или \textit{ассоциативность}, позволяющими преобразовывать выражения, составленные с использованием этих операций.


\begin{definition}[Коммутативность]
Бинарная операция $\circ : S \times S \to S$ называется коммутативной, если для любых  $x_1 \in S, x_2 \in S$ верно, что  $x_1 \circ x_2 = x_2 \circ x_1$.
\end{definition}

\begin{example} Рассмотрим несколько примеров коммутативных и некоммутативных операций.
	\begin{itemize}
		\item Опреация сложения на целых числах $+$ является коммутативной: известный ещё со школы перестановочный закон сложения.
		\item Операция конкатенации на строках $+$ не является коммутативной: $$``ab" + ``c" \ = ``abc" \neq ``c" + ``ab" \ = ``cab".$$
		\item Операция умножения на целых числах является коммутативной: известный ещё со школы перестановочный закон умножения.
		\item Операция умножения матриц (над целыми числами) $\cdot$ не является коммутативной:
		$$\begin{pmatrix} 
		1 & 1 \\ 0 & 0
		\end{pmatrix}
		\cdot
		\begin{pmatrix} 
		0 & 0 \\ 1 & 1
		\end{pmatrix}
		=
		\begin{pmatrix} 
		1 & 1 \\ 0 & 0
		\end{pmatrix}
		\neq
		\begin{pmatrix} 
		0 & 0 \\ 1 & 1
		\end{pmatrix}
		\cdot
		\begin{pmatrix} 
		1 & 1 \\ 0 & 0
		\end{pmatrix}
		=
		\begin{pmatrix} 
		0 & 0 \\ 1 & 1
		\end{pmatrix}
		.$$
	\end{itemize}
\end{example}

\begin{definition}[Ассоциативность]
Бинарная операция $\circ : S \times S \to S$ называется ассоциативной, если для любых  $x_1 \in S, x_2 \in S, x_3 \in S$ верно, что  $(x_1 \circ x_2) \circ x_3 = x_1 \circ (x_2 \circ x_3)$. Иными словами, для ассоциативной операции результат вычислений не зависит от порядка применения операций.
\end{definition}

\begin{example} Рассмотрим несколько примеров ассоциативных и неассоциативных операций.
	\begin{itemize}
		\item Опреация сложения на целых числах $+$ является ассоциативной.
		\item Операция конкатенации на строках $+$ является ассоциативной: $$``ab" + ``c" \ = ``abc" \neq ``c" + ``ab" \ = ``cab".$$
		\item Операция умножения на целых числах является ассоциативной.
		\item Операция возведения в степень (над целыми числами) $\hat{\mkern6mu}$ не является ассоциативной:
		$$(2\hat{\mkern6mu}2)\hat{\mkern6mu}3 = 4 \hat{\mkern6mu} 3 = 64 \neq 2\hat{\mkern6mu}(2\hat{\mkern6mu}3) = 2 \hat{\mkern6mu} 8  = 256.$$
	\end{itemize}
\end{example}


\begin{definition}[дистрибутивность]
!!!
\end{definition}

\begin{definition}[идемпотентность]
!!!
\end{definition}

\begin{definition}[Нейтральный элемент]
Пусть есть коммутативная бинарная операция $\circ$ на множестве $S$. Говорят, что $x\in S$ является нейтарльным элементом по операции $\circ$, если для любого $y\in S$ верно, что $x \circ y = y \circ x = y$. Если бинарная операция не является коммутативной, то можно пределить \textit{нейтральный слева} и \textit{нейтральный справа} элементы по аналогии.
\end{definition}


\section{Полугруппа}


множество с заданной на нём ассоциативной бинарной операцией $(S,\cdot )$ 


Коммутативная полугруппа


The set of positive integers with addition. (With 0 included, this becomes a monoid.)
The set of integers with minimum or maximum. (With positive/negative infinity included, this becomes a monoid.)
Square nonnegative matrices of a given size with matrix multiplication.
Any ideal of a ring with the multiplication of the ring.
The set of all finite strings over a fixed alphabet $\Sigma$ with concatenation of strings as the semigroup operation — the so-called ``free semigroup over $\Sigma$''. With the empty string included, this semigroup becomes the free monoid over $\Sigma$.


\section{Моноид}


Полугруппа с нейтральным элементом.



\section{Группа}


Непустое множество $G$ с заданной на нём бинарной операцией $*$: $ \mathrm {G} \times \mathrm {G} \rightarrow \mathrm {G}$ называется группой $ (\mathrm {G} ,*)$, если выполнены следующие аксиомы:

ассоциативность: $\forall (a,b,c\in G)\colon (a*b)*c=a*(b*c)$;
наличие нейтрального элемента: $ \exists e\in G\quad \forall a\in G\colon (e*a=a*e=a)$;
наличие обратного элемента: $ \forall a\in G\quad \exists a^{-1}\in G\colon (a*a^{-1}=a^{-1}*a=e)$.

Иными словами, группа --- это моноид с дополнительным требованием наличия обратных элементов.

\begin{definition}[Абелева группа] --- операция коммутативна.
\end{definition}


\section{Полукольцо}

A semiring is a set R equipped with two binary operations + and $\otimes$, called addition and multiplication, such that:[3][4][5]

(R, +) is a commutative monoid with identity element 0:
(a + b) + c = a + (b + c)
0 + a = a + 0 = a
a + b = b + a
(R, $\otimes$) is a monoid with identity element 1:
(a$\otimes$b)$\otimes$c = a$\otimes$(b$\otimes$c)
1$\otimes$a = a$\otimes$1 = a
Multiplication left and right distributes over addition:
a$\otimes$(b + c) = (a$\otimes$b) + (a$\otimes$c)
(a + b)$\otimes$c = (a$\otimes$c) + (b$\otimes$c)
Multiplication by 0 annihilates R:
0$\otimes$a = a$\otimes$0 = 0


\section{Кольцо}


A ring is a set R equipped with two binary operations[a] + (addition) and $\otimes$ (multiplication) satisfying the following three sets of axioms, called the ring axioms[1][2][3]

R is an abelian group under addition, meaning that:
(a + b) + c = a + (b + c) for all a, b, c in R   (that is, + is associative).
a + b = b + a for all a, b in R   (that is, + is commutative).
There is an element 0 in R such that a + 0 = a for all a in R   (that is, 0 is the additive identity).
For each a in R there exists $-a$ in R such that $a + (-a) = 0$   (that is, $-a$ is the additive inverse of a).
R is a monoid under multiplication, meaning that:
(a $\otimes$ b) $\otimes$ c = a $\otimes$ (b $\otimes$ c) for all a, b, c in R   (that is, $\otimes$ is associative).
There is an element 1 in R such that a $\otimes$ 1 = a and 1 $\otimes$ a = a for all a in R   (that is, 1 is the multiplicative identity).[b]
Multiplication is distributive with respect to addition, meaning that:
a $\otimes$ (b + c) = (a $\otimes$ b) + (a $\otimes$ c) for all a, b, c in R   (left distributivity).
(b + c) $\otimes$ a = (b $\otimes$ a) + (c $\otimes$ a) for all a, b, c in R   (right distributivity).


\section{Поле}

\section{Матрицы и вектора}

Вектор

Матрица 

Про матричное произведение, тензорное произведение, ещё что-то.

\section{Вопросы и задачи}
\begin{enumerate}
	\item Привидите примеры некоммутативных операций.
	\item Привидите примеры ситуаций, когда наличие у бинарных операций каких-либо дополнитльных свойств (ассоциативности, коммутативности), позволяет строить более эффективные алгоритмы, чем в общем случае.
\end{enumerate}
\input{GraphTheoryIntro}
\chapter{Общие сведения теории формальных языков}\label{chpt:FormalLanguageTheoryIntro}

В данной главе мы рассмотрим основные понятия из теории формальных языков, которые пригодятся нам в дальнейшем изложении.

\begin{definition}
\textit{Алфавит} --- это конечное множество.
Элементы этого множества будем называть \textit{символами}.
\end{definition}

\begin{example}
  Примеры алфавитов

  \begin{itemize}
    \item Латинский алфавит $\Sigma = \{ a, b, c, \dots, z\}$
    \item Кириллический алфавит $\Sigma = \{ \text{а, б, в, \dots, я}\}$
    \item Алфавит чисел в шестнадцатеричной записи 
    $$\Sigma = \{0, 1, 2, 3, 4, 5, 6, 7 ,8,9, A, B, C, D, E, F \}$$
  \end{itemize}
\end{example}

Традиционное обозначение для алфавита --- $\Sigma$.
Также мы будем использовать различные прописные буквы латинского алфавита. Для обозначения символов алфавита будем использовать строчные буквы латинского алфавита: $a, b, \dots, x, y, z$.

Будем считать, что над алфавитом $\Sigma$ всегда определена операция конкатенации $(\cdot): \Sigma^* \times \Sigma^* \to \Sigma^*$.
При записи выражений символ точки (обозначение операции конкатенации) часто будем опускать: $a \cdot b = ab$.

\begin{definition}
\textit{Слово} над алфавитом $\Sigma$ --- это конечная конкатенация символов алфавита $\Sigma$: $\omega = a_0 \cdot a_1 \cdot \ldots \cdot a_m$, где $\omega$ --- слово, а для любого $i$ $a_i \in \Sigma$.
\end{definition}

\begin{definition}
Пусть $\omega = a_0 \cdot a_1 \cdot \ldots \cdot a_m$ --- слово над алфавитом $\Sigma$.
Будем называть $m + 1$ \textit{длиной слова} и обозначать как $|\omega|$.
\end{definition}

\begin{definition}
\textit{Язык} над алфавитом $\Sigma$ --- это множество слов над алфавитом $\Sigma$.
\end{definition}

\begin{example}

Примеры языков.

  \begin{itemize}
    \item Язык целых чисел в двоичной записи $\{0, 1, -1, 10, 11, -10, -11, \dots\}.$
    \item Язык всех правильных скобочных последовательностей $$\{(), (()), ()(), (())(), \dots\}.$$
  \end{itemize}
\end{example}

Любой язык над алфавитом $\Sigma$ является подмножеством $\Sigma^*$ --- множества всех слов над алфавитом $\Sigma$.

Заметим, что язык не обязан быть конечным множеством, в то время как алфавит всегда конечен и изучаем мы конечные слова.

%\begin{definition}
\textit{Способы задания языков}
\begin{itemize}
\item Перечислить все элементы. Такой способ работает только для конечных языков. Перечислить бесконечное множество не получится.
\item Задать генератор --- процедуру, которая возвращает очередное слово языка.
\item Задать распознователь --- процедуру, которая по данному слову может определить, принадлежит оно заданному языку или нет.
\end{itemize}


Теоретико-множественные задачи над языками и их применение. 
О том, что моногое --- про пересечение, проверку пустоты, вложенность.





\section{Вопросы и задачи}
\begin{enumerate}
  \item !!! 
  \item !!!
\end{enumerate}

\chapter{Регулярные языки}

Регулярные языки, конечные автоматы, взяимные конвертации, замкнутость.

Лемма о накачке

Линейная алгебра для работы с регулярными языками: пересечение, замыкание.



\section{Задача поиска путей с ограничениями в терминах регулярных языков}

Графовая база данных --- автомат.

Задача --- пересечение автоматов.

Линейная алгебра, производные, построение атомата, выяснение существования путей.

\section{Вопросы и задачи}

Построить базу.

Научиться выполнять запросы через линейку. 
\chapter{Контекстно-свободные грамматики и языки}\label{CFG}

Из всего многообразия нас будут интересовать прежде всего контекстно-свободные грамматики.

\begin{definition}
\textit{Контекстно-свободная грамматика} --- это четвёрка вида $\langle \Sigma, N, P, S \rangle$, где
\begin{itemize}
  \item $\Sigma$ --- это терминальный алфавит;
  \item $N$ --- это нетерминальный алфавит;
  \item $P$ --- это множество правил или продукций, таких что каждая продукция имеет вид $N_i \to \alpha$, где $N_i \in N$ и $\alpha \in \{\Sigma \cup N\}^* \cup {\varepsilon}$;
  \item $S$ --- стартовый нетерминал.
  Отметим, что $\Sigma \cap N = \varnothing$.
\end{itemize}
\end{definition}

\begin{example}
Грамматика, задающая язык целых чисел в двоичной записи без лидирующих нулей: $G = \langle \{0, 1, -\}, \{S, N, A\}, P, S \rangle$, где $P$ определено следующим образом:

\[
\begin{array}{rcl}
S& \rightarrow & 0 \mid N \mid - N  \\
N& \rightarrow & 1 A \\
A& \rightarrow & 0 A \mid 1 A  \mid \varepsilon\\
\end{array}
\]
\end{example}

При спецификации грамматики часто опускают множество терминалов и нетерминалов, оставляя только множество правил. При этом нетерминалы часто обозначаются прописными латинскими буквами, терминалы --- строчными, а стартовый нетерминал обозначается буквой~$S$. Мы будем следовать этим обозначениям, если не указано иное.


\begin{definition}\label{def derivability in CFG}
  \textit{Отношение непосредственной выводимости}. Мы говорим, что последовательность терминалов и нетерминалов $\gamma \alpha \delta$ \textit{непосредственно выводится из} $\gamma \beta \delta$ \textit{при помощи правила} $\alpha \rightarrow \beta$ ($\gamma \alpha \delta \Rightarrow \gamma \beta \delta$), если
  \begin{itemize}
    \item $\alpha \rightarrow \beta \in P$
    \item $\gamma, \delta \in \{\Sigma \cup N\}^* \cup {\varepsilon}$
  \end{itemize}
\end{definition}

\begin{definition}
  \textit{Рефлексивно-транзитивное замыкание отношения} --- это наименьшее рефлексивное и транзитивное отношение, содержащее исходное.
\end{definition}

\begin{definition}
\textit{Отношение выводимости} является рефлексивно-транзитивным замыканием отношения непосредственной выводимости
\begin{itemize}
  \item $\alpha \derives \beta$ означает $\exists \gamma_0, \dots \gamma_k: \ \alpha \derives[] \gamma_0 \derives[] \gamma_1 \derives[] \dots \derives[] \gamma_{k-1} \derives[] \gamma_{k} \derives[] \beta$
  \item Транзитивность: $\forall \alpha, \beta, \gamma \in \{\Sigma \cup N\}^* \cup {\varepsilon}: \ \alpha \derives \beta, \beta \derives \gamma \Rightarrow \alpha \derives \gamma$
  \item Рефлексивность: $\forall \alpha \in \{\Sigma \cup N\}^* \cup {\varepsilon}: \ \alpha \derives \alpha$
  \item $\alpha \derives \beta$ --- $\alpha$ выводится из $\beta$
  \item $\alpha \derives[k] \beta$ --- $\alpha$ выводится из $\beta$ за $k$ шагов
  \item $\alpha \derives[+] \beta$ --- при выводе использовалось хотя бы одно правило грамматики
\end{itemize}
\end{definition}


\begin{example}
Пример вывода цепочки $-1101$ в грамматике:

  \[
  \begin{array}{rcl}
  S& \rightarrow & 0 \mid N \mid - N  \\
  N& \rightarrow & 1 A \\
  A& \rightarrow & 0 A \mid 1 A  \mid \varepsilon\\
  \end{array}
  \]

  \[ S \Rightarrow - N \Rightarrow - 1 A \Rightarrow - 1 1 A \derives - 1 1 0 1 A \Rightarrow - 1 1 0 1 \]
\end{example}


\begin{definition}[Вывод слова в грамматике]
Слово $\omega \in \Sigma^*$ \textit{выводимо в грамматике} $\langle \Sigma, N, P, S \rangle$, если существует некоторый вывод этого слова из начального нетерминала $S \derives \omega$.

\end{definition}

\begin{definition}
\textit{Левосторонний вывод}. На каждом шаге вывода заменяется самый левый нетерминал.
\end{definition}

\begin{example}
Пример левостороннего вывода цепочки в грамматике

  \[
    \begin{array}{rcl}
    S& \rightarrow & A A \mid s  \\
    A& \rightarrow & A A \mid B b \mid a \\
    B& \rightarrow & c \mid d
    \end{array}
  \]

  \[ \boldsymbol{S} \derives[] \boldsymbol{A} A \derives[] \boldsymbol{B} b A \derives[] c b \boldsymbol{A} \derives[] c b \boldsymbol{A} A \derives[] c b a \boldsymbol{A} \derives[] c b a a \]
\end{example}

Аналогично можно определить правосторонний вывод.

\begin{definition}
\textit{Язык, задаваемый грамматикой} --- множество строк, выводимых в грамматике $L(G) = \{ \omega \in \Sigma^* \mid S \derives \omega \}$.
\end{definition}

\begin{definition}
  Грамматики $G_1$ и $G_2$ называются \textit{эквивалентными}, если они задают один и тот же язык: $L(G_1) = L(G_2)$
\end{definition}


\begin{example}  Пример эквивалентных грамматик для языка целых чисел в двоичной системе счисления.

  \begin{tabular}{p{0.4\textwidth} | p{0.5\textwidth}}

    \[
      \begin{array}{rcl}
      \Sigma &=& \{ 0, 1, - \} \\
      N &=& \{ S, N, A \} \\~\\
      S& \rightarrow & 0 \mid N \mid - N  \\
      N& \rightarrow & 1 A \\
      A& \rightarrow & 0 A \mid 1 A  \mid \varepsilon\\
      \end{array}
    \]

    &

    \[
      \begin{array}{rcl}
      \Sigma &=& \{ 0, 1, - \} \\
      N &=& \{ S, A \} \\~\\
      S& \rightarrow & 0 \mid 1 A  \mid - 1 A  \\
      A& \rightarrow &  0 A \mid 1 A  \mid \varepsilon\\
      \end{array}
    \]
    \end{tabular}

\end{example}


\begin{definition}
  \textit{Неоднозначная грамматика} --- грамматика, в которой существует 2 и более левосторонних (правосторонних) выводов для одного слова.
\end{definition}

\begin{example}
  Неоднозначная грамматика для правильных скобочных последовательностей

\[
    S \to (S) \mid S S \mid \varepsilon
\]
\end{example}

\begin{definition}
  \textit{Однозначная грамматика} --- грамматика, в которой существует не более одного левостороннего (правостороннего) вывода для каждого слова.
\end{definition}

\begin{example}
  Однозначная грамматика для правильных скобочных последовательностей

\[
    S \to (S)S \mid \varepsilon
\]
\end{example}

\begin{definition}
  \textit{Существенно неоднозначные языки} --- языки, для которых невозможно построить однозначную грамматику.
\end{definition}

\begin{example}
  Пример существенно неоднозначного языка

\[\{a^n b^n c^m \mid n, m \in \mathds{Z}\} \cup \{a^n b^m c^m \mid n,m \in \mathds{Z}\}\]
\end{example}

\section{Дерево вывода}\label{sect:DerivTree}
В некоторых случаях не достаточно знать порядок применения правил.
Необходимо структурное представление вывода цепочки в грамматике.
Таким представлением является \textit{дерево вывода}.
\begin{definition}
Деревом вывода цепочки $\omega$ в грамматике $G=\langle \Sigma, N, S, P \rangle$ называется дерево, удовлетворяющее следующим свойствам.

\begin{enumerate}
  \item Помеченное: метка каждого внутреннего узла --- нетерминал, метка каждого листа --- терминал или $\varepsilon$.
  \item Корневое: корень помечен стартовым нетерминалом.
  \item Упорядоченное.
  \item В дереве может существовать узел с меткой $N_i$ и сыновьями $M_j \dots M_k$ только тогда, когда в грамматике есть правило вида $N_i \to M_j \dots M_k$.
  \item Крона образует исходную цепочку $\omega$.
\end{enumerate}
\end{definition}

\begin{example}
  Построим дерево вывода цепочки $ababab$ в грамматике

  \[ G = \langle \{a,b\}, \{S\}, S, \{S \to a \ S \ b \ S, S \to \varepsilon\} \rangle \]

\begin{center}

    \begin{tikzpicture}[sibling distance=4em,
    every node/.style = {shape=rectangle, rounded corners,
      draw, align=center,
      top color=white, bottom color=blue!20}]]
    \node {S}
      child { node {a} }
      child { node {S}
        child { node {$\varepsilon$}}
      }
      child { node {b} }
      child { node {S}
        child {node {a}}
        child { node {S}
          child { node {$\varepsilon$}}
        }
        child { node {b} }
        child { node {S}
          child {node {a}}
          child {node {S}
            child {node {$\varepsilon$}}
          }
          child {node {b}}
          child {node {S}
            child {node {$\varepsilon$}}
          }
        }
      };
  \end{tikzpicture}
\end{center}

\end{example}

\begin{theorem}
  Пусть $G = \langle \Sigma, N, P, S \rangle$ --- КС-грамматика.
  Вывод $S \derives \alpha$, где $\alpha \in (N \cup \Sigma)^*, \alpha \neq \varepsilon$ существует $\Leftrightarrow$ существует дерево вывода в грамматике $G$ с кроной $\alpha$.
\end{theorem}

\section{Пустота КС-языка}

\begin{theorem}
  Существует алгоритм, определяющий, является ли язык, порождаемый КС грамматикой, пустым.
\end{theorem}

\begin{proof}
  Следующая лемма утверждает, что если в КС языке есть выводимое слово, то существует другое выводимое слово с деревом вывода не глубже количества нетерминалов грамматики.
  Для доказательства теоремы достаточно привести алгоритм, последовательно строящий все деревья глубины не больше количества нетерминалов грамматики, и проверяющий, являются ли такие деревья деревьями вывода.
  Если в результате работы алгоритма не удалось построить ни одного дерева, то грамматика порождает пустой язык.
\end{proof}

\begin{lemma}
  Если в данной грамматике выводится некоторая цепочка, то существует цепочка, дерево вывода которой не содержит ветвей длиннее m, где m --- количество нетерминалов грамматики.
\end{lemma}

\begin{proof}
  Рассмотрим дерево вывода цепочки $\omega$. Если в нем есть 2 узла, соответствующих одному нетерминалу A, обозначим их $n_1$ и $n_2$.

  Предположим, $n_1$ расположен ближе к корню дерева, чем $n_2$.

  $S \derives \alpha A_{n_1} \beta \derives \alpha \omega_1 \beta; S \derives \alpha \gamma A_{n_2} \delta \beta \derives \alpha \gamma \omega_2 \delta \beta$, при этом $\omega_2$ является подцепочкой $\omega_1$.

  Заменим в изначальном дереве узел $n_1$ на $n_2$. Полученное дерево является деревом вывода $\alpha \omega_2 \delta$.

  Повторяем процесс замены одинаковых нетерминалов до тех пор, пока в дереве не останутся только уникальные нетерминалы.

  В полученном дереве не может быть ветвей длины большей, чем m.

  По построению оно является деревом вывода.
\end{proof}


\section{Нормальная форма Хомского}
\label{section:CNF}

\begin{definition}
Контекстно-свободная грамматика $\langle \Sigma, N, P, S\rangle$ находится в \textit{Нормальной Форме Хомского}, если она содержит только правила следующего вида:

\begin{itemize}
  \item $A \to B C \text{, где } A, B, C \in N \text{, S не содержится в правой части правила }$
  \item $A \to a \text{, где } A \in N, a \in \Sigma$
  \item $S \to \varepsilon$
\end{itemize}
\end{definition}

\begin{theorem}
Любую КС грамматику можно преобразовать в НФХ.
\end{theorem}

\begin{proof}
  Алгоритм преобразования в НФХ состоит из следующих шагов:

  \begin{itemize}
    \item Замена неодиночных терминалов
    \item Удаление длинных правил
    \item Удаление $\varepsilon$-правил
    \item Удаление цепных правил
    \item Удаление бесполезных нетерминалов
  \end{itemize}

  То, что каждый из этих шагов преобразует грамматику к эквивалентной, при этом является алгоритмом, доказано в следующих леммах.
\end{proof}

\begin{lemma}
  Для любой КС-грамматики можно построить эквивалентную, которая не содержит правила с неодиночными терминалами.
\end{lemma}

\begin{proof}
  Каждое правило $A \to B_0 B_1 \dots B_k, k \geq 1$ заменить на множество правил:
  \begin{itemize}
    \item $A \to C_0 C_1 \dots C_k$
    \item $\{ C_i \to B_i \mid B_i \in \Sigma, C_i \text{ --- новый нетерминал} \}$
  \end{itemize}
\end{proof}

\begin{lemma}
  Для любой КС-грамматики можно построить эквивалентную, которая не содержит правил длины больше 2.
\end{lemma}

\begin{proof}
  Каждое правило $A \to B_0 B_1 \dots B_k, k \geq 2$ заменить на множество правил:
  \begin{itemize}
    \item $A \to B_0 C_0$
    \item $C_0 \to B_1 C_1$
    \item $\dots$
    \item $C_{k-3} \to B_{k-2} C_{k-2}$
    \item $C_{k-2} \to B_{k-1} B_k$
  \end{itemize}
\end{proof}


\begin{lemma}
  Для любой КС-грамматики можно построить эквивалентную, не содержащую $\varepsilon$-правил.
\end{lemma}

\begin{proof}
  Определим $\varepsilon$-правила:
  \begin{itemize}
    \item $A \to \varepsilon$
    \item $A \to B_0 \dots B_k, \forall i: \ B_i$ --- $\varepsilon$-правило.
  \end{itemize}

  Каждое правило $A \to B_0 B_1 \dots B_k$ заменяем на множество правил, где каждое $\varepsilon$-правило удалено во всех возможных комбинациях.
\end{proof}

\begin{lemma}
  Можно удалить все цепные правила
\end{lemma}

\begin{proof}
  \textit{Цепное правило} --- правило вида $A \to B\text{, где } A, B \in N\\$.
  \textit{Цепная пара} --- упорядоченная пара $(A,B)$, в которой $A\derives B$, используя только цепные правила.
  
  Алгоритм:
  \begin{enumerate}
  \item Найти все цепные пары в грамматике $G$.
  Найти все цепные пары можно по индукции:
  Базис: $(A,A)$ --- цепная пара для любого нетерминала, так как $A\derives A$ за ноль шагов.
  Индукция: Если пара $(A,B_0)$ --- цепная, и есть правило $B_0 \to B_1$, то $(A,B_1)$ --- цепная пара.
  \item Для каждой цепной пары $(A,B)$ добавить в грамматику $G'$ все правила вида $A \to a$, где $B \to a$ --- нецепное правило из $G$.
  \item Удалить все цепные правила
\end{enumerate}
Пусть $G$ --- контекстно-свободная грамматика. $G'$ --- грамматика, полученная в результате применения алгоритма к $G$. Тогда $L(G)=L(G')$.
\end{proof}

\begin{definition}
Нетерминал $A$ называется \textit{порождающим}, если из него может быть выведена конечная терминальная цепочка. Иначе он называется \textit{непорождающим}.
\end{definition}

\begin{lemma}
  Можно удалить все бесполезные (непорождающие) нетерминалы
\end{lemma}

\begin{proof}
  После удаления из грамматики правил, содержащих непорождающие нетерминалы, язык не изменится, так как непорождающие нетерминалы по определению не могли участвовать в выводе какого-либо слова.
  
  Алгоритм нахождения порождающих нетерминалов:
  \begin{enumerate}
  \item Множество порождающих нетерминалов пустое.
  \item Найти правила, не содержащие нетерминалов в правых частях и добавить нетерминалы, встречающихся в левых частях таких правил, в множество.
  \item Если найдено такое правило, что все нетерминалы, стоящие в его правой части, уже входят в множество, то добавить в множество нетерминалы, стоящие в его левой части.
  \item Повторить предыдущий шаг, если множество порождающих нетерминалов изменилось.
\end{enumerate}
В результате получаем множество всех порождающих нетерминалов грамматики, а все нетерминалы, не попавшие в него, являются непорождающими. Их можно удалить.
\end{proof}

\begin{example}
  Приведем в Нормальную Форму Хомского однозначную грамматику правильных скобочных последовательностей: $S \to a S b S \mid \varepsilon$

  Первым шагом добавим новый нетерминал и сделаем его стартовым: 

  \begin{align*}
    S_0 &\to S  \\ 
    S   &\to a S b S \mid \varepsilon
  \end{align*}

  Заменим все терминалы на новые нетерминалы: 

  \begin{align*}
    S_0 &\to S \\ 
    S   &\to L S R S \mid \varepsilon \\ 
    L   &\to a \\ 
    R   &\to b
  \end{align*}

  Избавимся от длинных правил: 

  \begin{align*}
    S_0 &\to S \\ 
    S   &\to L S' \mid \varepsilon \\ 
    S'  &\to S S'' \\ 
    S'' &\to R S \\
    L   &\to a \\ 
    R   &\to b
  \end{align*}

  Избавимся от $\varepsilon$-продукций: 

  \begin{align*}
    S_0 &\to S \mid \varepsilon \\ 
    S   &\to L S' \\ 
    S'  &\to S'' \mid S S'' \\ 
    S'' &\to R   \mid R S \\
    L   &\to a \\ 
    R   &\to b
  \end{align*}

  Избавимся от цепных правил: 

  \begin{align*}
    S_0 &\to L S' \mid \varepsilon \\ 
    S   &\to L S' \\ 
    S'  &\to b \mid R S \mid S S'' \\ 
    S'' &\to b \mid R S \\
    L   &\to a \\ 
    R   &\to b
  \end{align*}
\end{example}

\begin{definition}\label{defn:wCNF}
Контекстно-свободная грамматика $\langle \Sigma, N, P, S\rangle$ находится в \textit{ослабленной Нормальной Форме Хомского}, если она содержит только правила следующего вида:

\begin{itemize}
  \item $A \to B C \text{, где } A, B, C \in N$
  \item $A \to a \text{, где } A \in N, a \in \Sigma$
  \item $A \to \varepsilon \text{, где } A \in N$
\end{itemize}

То есть ослабленная НФХ отличается от НФХ тем, что:
\begin{enumerate}
  \item $\varepsilon$ может выводиться из любого нетерминала
  \item $S$ может появляться в правых частях правил
\end{enumerate}
\end{definition}

\section{Лемма о накачке}

\begin{lemma}
Пусть $L$ --- контекстно-свободный язык над алфавитом $\Sigma$, тогда существует такое $n$, что для любого слова $\omega \in L$, $|\omega| \geq n$ найдутся слова $u,v,x,y,z\in \Sigma^*$, для которых верно: $uvxyz = \omega, vy\neq \varepsilon,|vxy|\leq n$ и для любого $k \geq 0$  $uv^kxy^kz \in L$.
\end{lemma}

Идея доказательства леммы о накачке.

\begin{enumerate}
    \item Для любого КС языка можно найти грамматику в нормальной форме Хомского.
    \item Очевидно, что если брать достаточно длинные цепочки, то в дереве вывода этих цепочек, на пути от корня к какому-то листу обязательно будет нетерминал, встречающийся минимум два раза. Если $m$ --- количество нетерминалов в НФХ, то длины $2^{m+1}$ должно хватить. Это и будет $n$ из леммы.
    \item Возьмём путь, на котором есть хотя бы дважды повторяется некоторый нетерминал. Скажем, это нетерминал  $N_1$. Пойдём от листа по этому пути. Найдём первое появление $N_1$. Цепочка, задаваемая поддеревом для этого узла --- это $x$ из леммы.
    \item Пойдём дальше и найдём второе появление $N_1$. Цепочка, задаваемая поддеревом для этого узла --- это $vxy$ из леммы.
    \item Теперь мы можем копировать кусок дерева между этими повторениями $N_1$ и таким образом накачивать исходную цепочку.
\end{enumerate}

Надо только проверить выполение ограничений на длины.

Нахождение разбиения и пример накачки продемонстрированы на рисунках~\ref{fig:pumping1} и~\ref{fig:pumping2}.

\begin{figure}
\centering
\includegraphics[width=0.5\textwidth]{pics/pumping_tree_1.pdf}
\caption{Разбиение цепочки для леммы о накачке}
\label{fig:pumping1}
\end{figure}

\begin{figure}
\centering
\includegraphics[width=0.5\textwidth]{pics/pumping_tree_2.pdf}
\caption{Пример накачки цепочки с рисунка~\ref{fig:pumping1}}
\label{fig:pumping2}
\end{figure}


Для примера предлагается проверить неконтекстно-свободность языка $L=\{a^nb^nc^n \mid n>0\}$.


\section{Замкнутость КС языков относительно операций}


\begin{theorem}
Контекстно-свободные языки замкнуты относительно следующих операций:
\begin{enumerate}
  \item Объединение: если $L_1$ и $L_2$ --- контекстно-свободные языки, то и $L_3 = L_1 \cup L_2$ --- контекстно-свободный.
  \item Конкатенация: если $L_1$ и $L_2$ --- контекстно-свободные языки, то и $L_3 = L_1 \cdot L_2$ --- контекстно-свободный.
  \item Замыкание Клини: если $L_1$ --- контекстно-свободный, то и $L_2 = \bigcup\limits_{i=0}^{\infty} L_1^i $ --- контекстно-свободный.
  \item Разворот: если $L_1$ --- контекстно-свободный, то и $L_2 = {L_1}^r$ --- контекстно-свободный.
  \item Пересечение с регулярными языками: если $L_1$ --- контекстно-свободный, а $L_2$ --- регулярный, то  $L_3 = L_1 \cap L_2$ --- контекстно-свободный.
  \item Разность с регулярными языками: если $L_1$ --- контекстно-свободный, а $L_2$ --- регулярный, то  $L_3 = L_1 \setminus L_2$ --- контекстно-свободный.
\end{enumerate}
\end{theorem}
Для доказательства пунктов 1--4 можно построить КС граммтику нового языка имея грамматики для исходных. 
Будем предполагать, что множества нетерминальных символов различных граммтик для исходных языков не пересекаются.
\begin{enumerate}
\item $G_1=\langle\Sigma_1,N_1,P_1,S_1\rangle$ --- граммтика для $L_1$, $G_1=\langle\Sigma_2,N_2,P_2,S_2\rangle$ --- граммтика для $L_2$, тогда $G_3=\langle\Sigma_1 \cup \Sigma_2, N_1 \cup N_2 \cup \{S_3\}, P_1 \cup P_2 \cup \{S_3 \to S_1 \mid S_2\} ,S_3\rangle$ --- граммтика для $L_3$. 

\item $G_1=\langle\Sigma_1,N_1,P_1,S_1\rangle$ --- граммтика для $L_1$, $G_1=\langle\Sigma_2,N_2,P_2,S_2\rangle$ --- граммтика для $L_2$, тогда $G_3=\langle\Sigma_1 \cup \Sigma_2, N_1 \cup N_2 \cup \{S_3\}, P_1 \cup P_2 \cup \{S_3 \to S_1 S_2\} ,S_3\rangle$ --- граммтика для $L_3$. 

\item $G_1=\langle\Sigma_1,N_1,P_1,S_1\rangle$ --- граммтика для $L_1$, тогда $G_2=\langle\Sigma_1, N_1 \cup \{S_2\}, P_1 \cup \{S_2 \to S_1 S_2\ \mid \varepsilon\}, S_2\rangle$ --- граммтика для $L_2$. 

\item $G_1=\langle\Sigma_1,N_1,P_1,S_1\rangle$ --- граммтика для $L_1$, тогда $G_2=\langle\Sigma_1, N_1, \{N^i \to \omega^R \mid N^i \to \omega \in P_1 \}, S_1\rangle$ --- граммтика для $L_2$. 
\end{enumerate}

Чтобы доказать замкнутость относительно пересечения с регулярными языками, построим по КС грамматике рекурсивный автомат $R_1$, по регулярному выражению --- детерминированный конечный автомат $R_2$, и построим их прямое произведение $R_3$.
Переходы по терминальным символам в новом автомате возможны тогда и только тогда, когда они возможны одновременно и в исходном рекурсивном автомате и в исходном конечном. 
За рекурсивные вызовы отвечает исходныа рекурсивный автомат. 
Значит цепочка принимается $R_3$ тогда и только тогда, когда она принимается одновременно $R_1$ и $R_2$: так как состояния $R_3$ --- это пары из состояния $R_1$ и $R_2$, то по трассе вычислений $R_3$ мы всегда можем построить трассу для $R_1$ и $R_2$ и наоборот.

Чтобы доказать замкнутость относительно разности с регулятным языком, достаточно вспомнить, что регулярные языки замкнуты относительно дополнения, и выразить разность через пересечение с дополнением: 
$$
L_1 \setminus L_2 = L_1 \cap \overline{L_2}
$$

\qed

\begin{theorem}
Контекстно-свободные языки не замкнуты относительно следующих операций:
\begin{enumerate}
  \item Пересечение: если $L_1$ и $L_2$ --- контекстно-свободные языки, то и $L_3 = L_1 \cap L_2$ --- не контекстно-свободный.
  \item Разность: если $L_1$ и $L_2$ --- контекстно-свободные языки, то и $L_3 = L_1 \setminus L_2$ --- не контекстно-свободный.
\end{enumerate}
\end{theorem}

Чтобы доказать незамкнутость относительно пресечения, рассмотрим языки $L_1 = \{a^n b^n c^k \mid n \geq 0, k \geq 0\}$ и $L_2 = \{a^k b^n c^n \mid n \geq 0, k \geq 0\}$.
Очевидно, что $L_1$ и $L_2$ --- контекстно-свободные языки.
Рассмотрим $L_3 = L_1 \cap L_2 = \{a^n b^n c^n \mid n \geq 0\}$. 
$L_3$ не является контекстно-свободным по лемме о накачке для контекстно-свободных языков.

Чтобы доказать незамкнутость относительно разности проделаем следующее.
\begin{enumerate}
\item Рассмотрим языки $L_4 = \{a^m b^n c^k \mid m \neq n, k \geq 0\}$ и $L_5 = \{a^m b^n c^k \mid n \neq k, m \geq 0\}$. 
Эти языки являются контекстно-свободными.
Это легко заметить, если знать, что язык $L'_4 = \{a^m b^n c^k \mid 0 \leq m < n, k \geq 0\}$ задаётся следующей граммтикой:
\begin{align*}
S \to & S c & T \to & a T b \\
S \to & T &   T \to & T b \\
      &   &   T \to & b. 
\end{align*} 

\item Рассмотрим язык $L_6 = \overline{L'_6} = \overline{\{a^n b^m c^k \mid n \geq 0, m \geq 0, k \geq 0\}}$. Данный язык является регулярным.

\item Рассмотрим язык $L_7 = L_4 \cup L_5 \cup L_6$ --- контектсно свободный, так как является объединением контекстно-свободных.

\item Рассмотрим $\overline{L_7} = \{a^n b^n c^n \mid n \geq 0\} = L_3$: $L_4$ и $L_5$ задают языки с правильным порядком символов, но неравным их количеством, $L_6$ задаёт язык с неправильным порядком символов. 
Из пердыдущего пункта мы знаем, что $L_3$  не является контекстно-свободным.

\end{enumerate}

\qed

\section{Вопросы и задачи}
\begin{enumerate}
  \item Постройте дерево вывода цепочки $w=aababb$ в грамматике $G=\langle\{a,b\},\{S\},\{S\rightarrow \varepsilon \ | \ a \ S \ b \ S \}, S \rangle$.
  \item Постройте все левосторонние выводы цепочки $w=ababab$ в грамматике $G=\langle\{a,b\},\{S\},\{S\rightarrow \varepsilon \ | \ a \ S \ b \ | S \ S\}, S \rangle$.
  \item Постройте все правосторонние выводы цепочки $w=ababab$ в грамматике $G=\langle\{a,b\},\{S\},\{S\rightarrow \varepsilon \ | \ a \ S \ b \ | S \ S\}, S \rangle$.
  \item \label{t1}Постройте все деревья вывода цепочки $w=ababab$ в грамматике $G=\langle\{a,b\},\{S\},\{S\rightarrow \varepsilon \ | \ a \ S \ b \ | S \ S\}, S \rangle$, соответствующие левосторонним выводам.
  \item \label{t2}Постройте все деревья вывода цепочки $w=ababab$ в грамматике $G=\langle\{a,b\},\{S\},\{S\rightarrow \varepsilon \ | \ a \ S \ b \ | S \ S\}, S \rangle$, соответствующие правосторонним выводам.
\end{enumerate}

\input{CFPQ}
\input{CYK_for_CFPQ}
\input{Matrix-based_CFPQ}
\input{TensorProduct}
\input{SPPF}
\input{GLL-based_CFPQ}
\input{GLR-based_CFPQ}
\input{CombinatorsForCFPQ}
\input{DerivativesForCFPQ}
\input{CFPQ_to_Datalog}
\chapter{Многокомпонентные контекстно-свободные языки}

Общая теория. Лпределение, свойства, классы.

Про MIX и $O_n$

\section{Поиск путей с ограничениями в терминах многокомпонентных контекстно-свободных языков}

В статанализе --- ещё одна аппроксимация.

Алгоритм на матрицах
\chapter{Конъюнктивные и булевы грамматики}

Впервые конъюнктивные и булевы грамматики были предложены Александром Охотиным~\cite{DBLP:journals/jalc/Okhotin01,Okhotin:2003:BG:1758089.1758123}. Дадим определение конъюнктивной грамматики.

\begin{definition}
    \textit{Конъюнктивной грамматикой} называется $G = (\Sigma,N,P,S)$, где:
    \begin{itemize}
        \item $\Sigma$ и $N$ --- дизъюнктивные конечные непустые множества терминалов и нетерминалов.
        \item $P$ --- конечное множество продукций, каждая вида
        \[
        A\rightarrow \alpha_1\&...\&\alpha_n
        \]
        ,где $A \in N,n \geq 1$ и $\alpha_1,...,\alpha_n \in (\Sigma \cup N)^*$.
        \item $S \in N$  --- стартовый нетерминал.
    \end{itemize}
\end{definition}

Конъюнктивная грамматика генерирует строки, выводя их из начального символа, так же, как это происходит в контекстно-свободных грамматиках в параграфе~\ref{CFG}. Промежуточные строки, используемые в процессе вывода, являются формулами следующего вида:

\begin{definition}\label{Definition of conjunctive formula}
    Пусть $G = (\Sigma,N,P,S)$ --- конъюнктивная грамматика. Множество конъюнктивных формул $ \mathcal{F}$ определяется индуктивно:
    \begin{itemize}
        \item Пустая строка $\varepsilon$ --- конъюнктивная формула. 
        \item Любой символ из $(\Sigma \cup N)$ --- формула.
        \item Если $\mathcal{A}$ и $\mathcal{B}$ непустые формулы, тогда $\mathcal{AB}$ --- формула.
        \item Если $\mathcal{A}_1,\ldots,\mathcal{A}_n$ $(n \geqslant 1)$ --- формула, тогда $(\mathcal{A}_1\&\ldots\&\mathcal{A}_n)$ --- формула.
    \end{itemize}
\end{definition}

\begin{definition}
    Пусть $G = (\Sigma,N,P,S)$ --- конъюнктивная грамматика. Аналогично определению отношения непосредственной выводимости в контекстно-свободной грамматике~\ref{def derivability in CFG} определим $\xRightarrow[G]{}$ как отношение непосредственной выводимости на множестве конъюнктивных формул.
    \begin{itemize}
        \item Любой нетерминал в любой формуле может быть перезаписан телом любого правила для этого терминала заключенным в скобки. То есть для любых $s^{'},s^{''} \in (\Sigma \cup N \cup \{(, \&, )\})^*$ и $A\in N$, таких что $s^{'}As^{''}$ --- формула, и для всех правил вида $A \rightarrow \alpha_1\&\ldots\&\alpha_n \in P$, имеем $s^{'}As^{''}\xRightarrow[G]{}s^{'}(\alpha_1\&\ldots\&\alpha_n)s^{''}$. 
        \item Если формула содержит подформулу в виде конъюнкции одной или нескольких одинаковых терминальных строк, заключенных в скобки, тогда подформула может быть перезаписана терминальной строкой без скобок. То есть для любых $s^{'},s^{''} \in (\Sigma \cup N \cup \{(, \&, )\})^*$, $(n \geqslant 1)$ и $w \in \Sigma^*$, таких что $s^{'}(w\&\ldots\&w)s^{''}$ --- формула, имеем $s^{'}(w\&\ldots\&w)s^{''}\xRightarrow[G]{}s^{'}ws^{''}$.
    \end{itemize}
    Как и в случае контекстно-свободной грамматики обозначим $\xRightarrow[G]{}^*$ рефлексивное транзитивное замыкание отношения $\xRightarrow[G]{}$.
\end{definition}

\begin{definition}
    Пусть $G = (\Sigma,N,P,S)$ --- конъюнктивная грамматика. Язык, порождаемый формулой, это множество всех терминальных строк выводимых из этой формулы: $L_{G}(\mathcal{A}) = \{w\in\Sigma^* \mid \mathcal{A} \xRightarrow[G]{}^*w\}$. Очевидно, что язык порождаемый грамматикой, это язык порождаемый стартовым нетерминалом $S$ : $L(G) = L_{G}(S) = L(S)$.
\end{definition}

\begin{theorem}\label{Theorem language generated by a formula}
    Пусть $G = (\Sigma,N,P,S)$ --- конъюнктивная грамматика. Пусть $\mathcal{A}_1,\ldots,\mathcal{A}_n,\mathcal{B}$ --- формулы, $A \in N$, $a \in \Sigma$. Тогда,
    \begin{enumerate}
        \item $L(\varepsilon) = \{\varepsilon\}$.
        \item $L(a) = \{a\}$.
        \item $L(A) = \bigcup_{A \rightarrow \alpha_1\&\ldots\&\alpha_n \in P} L((\alpha_1\&\ldots\&\alpha_m))$.
        \item $L(\mathcal{AB}) = L(\mathcal{A})*L(\mathcal{B})$
        \item $L((\mathcal{A}_1\&\ldots\&\mathcal{A}_n)) = \bigcap_{i = 1}^{n}L(\mathcal{A}_i)$.
    \end{enumerate}
\end{theorem}

Теорема~\ref{Theorem language generated by a formula} уже подразумевает интерпретацию грамматики как системы уравнений. Используем математический подход, чтобы лучше охарактеризовать конъюнктивные языки с помощью систем уравнений.

\begin{definition}[Выражение]
    Пусть $\Sigma$ конечный непустой алфавит. Пусть $X = \{X_1,\ldots,X_N\}$ вектор переменных. Выражение над алфавитом $\Sigma$, зависящее от переменных $X$, определяется индуктивно:
    \begin{itemize}
       \item $\varepsilon$ --- выражение.
       \item Любой символ $a\in\Sigma$ --- выражение.
       \item Любая переменная $X_i\in X$ --- выражение.
       \item Если $\phi_1$ и $\phi_2$ выражения, то $\phi_1\phi_2, (\phi_1\mid\phi_2), (\phi_1\&\phi_2)$ также выражения.
    \end{itemize}
    Заметим, что любая формула, в терминах определения~\ref{Definition of conjunctive formula}, является выражением, где нетерминалы формулы это переменные выражения. С другой стороны, любое выражение, не содержащее дизъюнкции, формула.
\end{definition}

Предположим, что переменные $X_i$ приняли в качестве значений слова из языка над алфавитом $\Sigma$. Определим значение всего выражения.

\begin{definition}[Значение выражения]\label{Value of conjunctive expression}
    Пусть $L = (L_1,\ldots,L_n) (L_i \subseteq \Sigma^*)$ вектор из $n$ языков над $\Sigma$, где $n \geqslant 1$. Пусть $\phi$ выражение над $\Sigma$, зависящее от переменных $X_1,\ldots,X_n$. Значение выражения $\phi$ на векторе $L$ --- это язык над тем же алфавитом $\Sigma$. Обозначим его $\phi(L)$ и определим индуктивно на структуре выражения:
    \begin{itemize}
       \item $\varepsilon(L) = \{\varepsilon\}$.
       \item $a(L) = \{a\}$ для любого $a\in\Sigma$.
       \item $X_i(L) = L_i$ для любого $X_i \in X$.
       \item $\phi_1\phi_2 = \phi_1(L) * \phi_2(L), (\phi_1\mid\phi_2)(L) = \phi_1(L) \cup \phi_2(L), (\phi_1\&\phi_2)(L) = \phi_1(L) \cap \phi_2(L)$ для любых выражений $\phi_1$ и $\phi_2$.
    \end{itemize}
\end{definition}

Обобщим определение~\ref{Value of conjunctive expression} на случай вектора выражений.

\begin{definition}[Значение вектора выражений]
    Пусть $L = (L_1,\ldots,L_n) (L_i \subseteq \Sigma^*)$ вектор из $n$ языков над $\Sigma$, где $n \geqslant 1$. Пусть $\phi_1,\ldots,\phi_m$ выражения над $\Sigma$, зависящее от переменных $X_1,\ldots,X_n$. Значение вектора выражений $P = (\phi_1,\ldots,\phi_m)$ на векторе $L$ --- это вектор языков $P(L) = (\phi_1(L),\ldots,\phi_m(L))$ над тем же алфавитом $\Sigma$. 
\end{definition}

Зададим частичный порядок относительно включения $``\preccurlyeq"$ на множестве языков и расширим его на вектора языков длины $n$: $(L_1^{'},\ldots,L_n^{'})\preccurlyeq(L_1^{''},\ldots,L_n^{''})$ если и только если $L_i^{'} \subseteq L_i^{''}$ для любого $1\leqslant i \leqslant n$

\begin{definition}\label{Definition a conjuctive system of equations}
   $X = P(X)$ система уравнений над алфавитом $\Sigma$ и $X = \{X_1,\ldots,X_n\}$, где $P = (\phi_1,\ldots,\phi_n)$ вектор выражений над алфавитом $\Sigma$, зависящий от $X$.
   
   Вектор языков $L = (L_1,\ldots,L_n)$ является решением системы уравнений если $L = P(L)$.
   
   Наименьшее решение $L$ это вектор языков, такой что для любого другого сравнимого вектора языков $L^{'}$ выполняется $L \preccurlyeq L^{'}$.
\end{definition}

Заметим, что оператор $P$ на множестве $2^{\Sigma}\times\ldots\times2^{\Sigma}$, что решение системы~\ref{Definition a conjuctive system of equations} это неподвижная точка $P$ и что наименьшее решение системы это наименьшая неподвижная точка оператора $P$.

\begin{theorem}\label{Theorem of a least fixed point solution}
    Для любой системы из определения~\ref{Definition a conjuctive system of equations} с переменными $X_1,\ldots,X_n$, оператор $P = (\phi_1,\ldots,\phi_n)$ имеет наименьшую неподвижную точку $L = (L_1,\ldots,L_n) = \lim_{i\to\infty}P^{i}\underbrace{(\varnothing,\ldots,\varnothing)}_n$.
\end{theorem}

Приведем пример конъюнктивной грамматики.

\begin{example}[Пример конъюнктивной грамматики]
    Следующая конъюнктивная грамматика $G$ порождает язык $\{a^nb^nc^n\mid n \geq 0\}$:
    
    \begin{align*}
    1.\ S   &\to A B \& D C \\
    2.\ A  &\to a A \mid \varepsilon \\ 
    3.\ B &\to b B c \mid \varepsilon \\
    4.\ C   &\to c C \mid \varepsilon \\ 
    5.\ D   &\to aDb \mid \varepsilon
    \end{align*}
    
    Легко видеть, что $L(AB) = \{a^ib^jc^k\mid j = k\}$ и $L(DC) = \{a^ib^jc^k\mid i = j\}$. Тогда $L(S) = L(AB) \cap L(DC) = \{a^nb^nc^n\mid n \geq 0\}$. 
    
    В этой грамматике строка $abc$ может быть получена следующим образом. Для начала представим грамматику в виде системы уравнений:
    \begin{align*}
    S &= A B \cap D C \\ 
    A &= \{a\}A \cup \varepsilon \\ 
    B &= \{b\}B\{c\} \cup \varepsilon \\
    C &= \{c\}C \cup \varepsilon \\ 
    D &= \{a\}D\{b\} \cup \varepsilon
    \end{align*}
    Используя теорему~\ref{Theorem of a least fixed point solution}, будем итеративно вычислять $P^{i}\underbrace{(\varnothing,\ldots,\varnothing)}_5$. На каждом шаге будем подставлять все терминальные строки из языков, порожденных нетерминалами на предыдущем шаге, в соответствующие нетерминалы правой части каждого уравнения и записывать получившиеся терминальные строки в языки нетерминалов текущего шага. Продолжаем до тех пор пока язык, порождаемый нетерминалом $S$, не будет содержать терминальную строку $``abc''$.
    \begin{enumerate}
        \item На начальном этапе имеем $P^{0}(\varnothing,\ldots,\varnothing) = (S: \varnothing, A: \varnothing, B: \varnothing, C: \varnothing, D: \varnothing)$ 
        \item Подставляем в первое уравнение терминальные строки из шага 1 в соответствующие нетерминалы, т.е. 
        \begin{align*}
            S:  \varnothing &= \varnothing\varnothing \cap \varnothing\varnothing \\ 
            A: \{\varepsilon\} &= \{a\}\varnothing \cup \{\varepsilon\} \\ 
            B: \{\varepsilon\} &= \{b\}\varnothing\{c\} \cup \{\varepsilon\} \\
            C: \{\varepsilon\} &= \{c\}\varnothing \cup \{\varepsilon\} \\ 
            D: \{\varepsilon\} &= \{a\}\varnothing\{b\} \cup \{\varepsilon\}
        \end{align*}
        В конце итерации получаем $P^{1}(\varnothing,\ldots,\varnothing) = (S: \varnothing, A: \{\varepsilon\}, B: \{\varepsilon\}, C: \{\varepsilon\}, D: \{\varepsilon\})$
        \item Делаем еще одну итерацию,
        \begin{align*}
            S:  \{\varepsilon\} &= \{\varepsilon\}\{\varepsilon\} \cap \{\varepsilon\}\{\varepsilon\} \\ 
            A: \{a, \varepsilon\} &= \{a\}\{\varepsilon\} \cup \{\varepsilon\} \\ 
            B: \{bc, \varepsilon\} &= \{b\}\{\varepsilon\}\{c\} \cup \{\varepsilon\} \\
            C: \{c, \varepsilon\} &= \{c\}\{\varepsilon\} \cup \{\varepsilon\} \\ 
            D: \{ab, \varepsilon\} &= \{a\}\{\varepsilon\}\{b\} \cup \{\varepsilon\}
        \end{align*}
        В конце итерации получаем $P^{2}(\varnothing,\ldots,\varnothing) = (S: \{\varepsilon\}, A: \{a, \varepsilon\}, B: \{bc, \varepsilon\}, C: \{c, \varepsilon\}, D: \{ab, \varepsilon\})$
        \item Еще одна итерация,
        \begin{align*}
            S:  \{\fbox{abc}, \varepsilon\} &= \{a, \varepsilon\}\{bc, \varepsilon\} \cap \{ab, \varepsilon\}\{c, \varepsilon\} \\ 
            A: \{a, aa, \varepsilon\} &= \{a\}\{a, \varepsilon\} \cup \{\varepsilon\} \\ 
            B: \{bc, bbcc, \varepsilon\} &= \{b\}\{bc, \varepsilon\}\{c\} \cup \{\varepsilon\} \\
            C: \{c, cc, \varepsilon\} &= \{c\}\{c, \varepsilon\} \cup \{\varepsilon\} \\ 
            D: \{ab, aabb, \varepsilon\} &= \{a\}\{ab, \varepsilon\}\{b\} \cup \{\varepsilon\}
        \end{align*}
        В конце итерации получили $P^{3}(\varnothing,\ldots,\varnothing) = (S: \{\fbox{abc}, \varepsilon\}, A: \{a, aa, \varepsilon\}, B: \{bc, bbcc, \varepsilon\}, C: \{c, cc, \varepsilon\}, D: \{ab, aabb, \varepsilon\})$. Заметим, что терминальная строка $``abc"$ появилась в языке, который порождает стартовый нетерминал $S$. Т.е. терминальная строка $``abc"$ выводима в грамматике $G$, что и требовалось показать.
    \end{enumerate}
    
    Заметим, что строку $``abc"$ также можно получить применением правил вывода. здесь цифра над стрелкой соответствует номеру примененного правила. 
    \begin{align*}
        S &\xRightarrow{1}(AB\&DC) \\
        &\xRightarrow{2}(aAB\&DC) \xRightarrow{2} (a\varepsilon B\&DC) \\
        &\xRightarrow{3}(abBc\&DC) \xRightarrow{3}(ab\varepsilon c\&DC) \\
        &\xRightarrow{5}(abc\&aDbC) \xRightarrow{5}(abc\&a\varepsilon bC) \\
        &\xRightarrow{4}(abc\&abcC) \xRightarrow{4}(abc\&abc\varepsilon) \\
        &\Rightarrow(abc\&abc) \Rightarrow abc
    \end{align*}
\end{example}

\begin{example}
    Конъюнктивная грамматика $G$ для языка $L = \{wcw \mid w \in \{a, b\}^*\}$:
    \begin{align*}
    S &\to C \& D \\ 
    C &\to aCa \mid aCb \mid bCa \mid bCb \mid c \\ 
    D &\to aA\&aD \mid bB\&bD \mid cE \\
    A &\to aAa \mid aAb \mid bAa \mid bAb \mid cEa \\
    B &\to aBa \mid aBb \mid bBa \mid bBb \mid cEb \\
    E &\to aE \mid bE \mid \varepsilon
    \end{align*}
\end{example}

Подробнее о конъюнктивных грамматиках можно прочитать в статьях~\cite{DBLP:journals/jalc/Okhotin01, Okhotin2002, DBLP:journals/tcs/Okhotin03a, f60a33d409364914be560cac0e54b12c}.

Дадим определение булевой грамматики.

\begin{definition}
    \textit{Булевой грамматикой} называется $G = (\Sigma,N,P,S)$, где:
    \begin{itemize}
        \item $\Sigma$ и $N$ --- дизъюнктивные конечные непустые множества терминалов и нетерминалов.
        \item $P$ --- конечное множество продукций, каждая вида
        \[
        A\rightarrow \alpha_1\&...\&\alpha_m\&\neg\beta_1\&...\&\neg\beta_n
        \]
        ,где $A \in N, m, n >=0, m+n \geq 1$ и $\alpha_i,\beta_j \in (\Sigma \cup N)^*$.
        \item $S \in N$  --- стартовый нетерминал.
    \end{itemize}
\end{definition}

Приведем пример булевой грамматики.

\begin{example}
    Следующая булева грамматика порождает язык  $\{a^mb^nc^n\mid m,n \geq 0, m \neq n \}$:
    
    \begin{align*}
    S   &\to A B \& \neg D C \\ 
    A  &\to a A \mid \varepsilon \\ 
    B &\to b B c \mid \varepsilon \\
    C   &\to c C \mid \varepsilon \\ 
    D   &\to aDb \mid \varepsilon
    \end{align*}
    
    Очевидно, что $L(AB) = \{a^mb^nc^n\mid m,n \in \mathbb{N}\}$ и $L(DC) = \{a^nb^nc^m\mid m,n \in \mathbb{N}\}$. Тогда $L(AB)\cap\overline{L(DC)} = \{a^mb^nc^n\mid m,n \geq 0, m \neq n \}$.
\end{example}

Подробнее о булевых грамматиках можно прочитать в статьях~\cite{Okhotin:2003:BG:1758089.1758123,Okhotin:2014:PMM:2565359.2565379}.

Определим бинарную нормальную форму конъюнктивной грамматики.
\begin{definition}[Бинарная нормальная форма]
    Конъюнктивная грамматика $G = (\Sigma, N, P, S)$ находится в бинарной нормальной форме, если каждое правило из P имеет вид,
    \begin{itemize}
        \item $A \rightarrow B_1 C_1 \& \ldots\& B_m C_m$, где $m \geqslant 1; A,B_i,C_i \in N$.
        \item $A \rightarrow a$, где $A \in N, a \in \Sigma$.
        \item $S \rightarrow \varepsilon$, если только $S$ не содержится в правой части всех правил.
    \end{itemize}
\end{definition}

\begin{theorem}\label{Binary normal form conjunctive grammar theorem}
    Для каждой конъюнктивной грамматики $G$ можно построить конъюнктивную грамматику в бинарной нормальной форме $G^{'}$, такую что $L(G) = L(G^{'})$.
\end{theorem}
Доказательство теоремы~\ref{Binary normal form conjunctive grammar theorem} описано в статье~\cite{DBLP:journals/jalc/Okhotin01}.



\section{Вопросы и задачи}
\begin{enumerate}
  \item !!! 
  \item !!!
\end{enumerate}

%\input{Conclusion}

\bibliographystyle{abbrv}
\bibliography{Formal_lang_CFPQ_course_notes}


\end{document}
