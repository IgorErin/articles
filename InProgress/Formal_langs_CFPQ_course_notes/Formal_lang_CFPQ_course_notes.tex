%\documentclass[a4paper,12pt]{article}  % standard LaTeX, 12 point type
\documentclass[12pt, a4paper, table]{book}

\usepackage{algpseudocode}
\usepackage{algorithm}
\usepackage{algorithmicx}

\usepackage{geometry}
\usepackage{amsfonts,latexsym}
\usepackage{amsthm}
\usepackage{amssymb}
\usepackage[utf8]{inputenc} % Кодировка
\usepackage[english,russian]{babel} % Многоязычность
\usepackage{mathtools}
\usepackage{hyperref}
\usepackage{tikz}
\usepackage{dsfont}
\usepackage{multicol}
\usetikzlibrary{fit,calc,automata,positioning}

\theoremstyle{definition}
\newtheorem{definition}{Определение}[section]
\newtheorem{example}{Пример}[section]
\newtheorem{theorem}{Теорема}[section]
\newtheorem{proposition}[theorem]{Proposition}
\newtheorem{lemma}[theorem]{Лемма}
\newtheorem{corollary}[theorem]{Corollary}
\newtheorem{conjecture}[theorem]{Conjecture}
\newtheorem{note}[theorem]{Утверждение}


% unnumbered environments:

\theoremstyle{remark}
\newtheorem*{remark}{Remark}
%\newtheorem*{notation}{Notation}

\setlength{\parskip}{5pt plus 2pt minus 1pt}
%\setlength{\parindent}{0pt}


\algtext*{EndWhile}% Remove "end while" text
\algtext*{EndIf}% Remove "end if" text
\algtext*{EndFor}% Remove "end for" text
\algtext*{EndFunction}% Remove "end function" text


\usepackage{color}
\usepackage{listings}
\usepackage{caption}
\usepackage{graphicx}
\usepackage{ucs}

\graphicspath{{pics/}}

%\geometry{left=2cm}
%\geometry{right=1.5cm}
%\geometry{top=2cm}
%\geometry{bottom=2cm}




%\lstnewenvironment{algorithm}[1][]
%{
%    \lstset{
%        frame=tB,
%        numbers=left,
%        mathescape=true,
%        numberstyle=\small,
%        basicstyle=\small,
%        inputencoding=utf8,
%        extendedchars=\true,
%        keywordstyle=\color{black}\bfseries,
%        keywords={,function, procedure, return, datatype, function, in, if, else, for, foreach, while, denote, do, and, then, assert,}
%        numbers=left,
%        xleftmargin=.04\textwidth,
%        #1 % this is to add specific settings to an usage of this environment (for instnce, the caption and referable label)
%    }
%}
%{}

\newcommand{\tab}[1][0.3cm]{\ensuremath{\hspace*{#1}}}

\newcommand{\rvline}{\hspace*{-\arraycolsep}\vline\hspace*{-\arraycolsep}}

\newcommand{\derives}[1][*]{\xRightarrow[]{#1}}
\newcommand{\first}[1][1]{\textsc{first}_{#1}}
\newcommand{\follow}[1][1]{\textsc{follow}_{#1}}

\setcounter{MaxMatrixCols}{20}


\tikzset{
%->, % makes the edges directed
%>=stealth’, % makes the arrow heads bold
node distance=4cm, % specifies the minimum distance between two nodes. Change if necessary.
%every state/.style={thick, fill=gray!10}, % sets the properties for each ’state’ node
initial text=$ $, % sets the text that appears on the start arrow
}

\tikzstyle{symbol_node} = [shape=rectangle, rounded corners, draw, align=center]

\tikzstyle{r_state} = [shape=rectangle, draw, minimum size=0.2cm]

\tikzstyle{prod_node} = [shape=rectangle, draw, align=center]

\tikzset{
    between/.style args={#1 and #2}{
         at = ($(#1)!0.5!(#2)$)
    }
}

%every node/.style = {shape=rectangle, rounded corners,
%      draw, align=center,
%      top color=white, bottom color=blue!20}

\newcommand{\bfgray}[1]{\cellcolor{lightgray}\textbf{#1}}

\newenvironment{scaledalign}[4]
  {
    \begingroup
    #1
    \setlength\arraycolsep{#2}
    \renewcommand{\arraystretch}{#3}
    \begin{center}
    \begin{equation}
    \begin{aligned}
    #4
  }
  {
    \end{aligned}
    \end{equation}
    \end{center}
    \endgroup
  }

\title{Приложения теории формальных языков и синтаксического анализа}
\author{Семён Григорьев}
\date{\today}

\begin{document}
\maketitle
\newpage
\tableofcontents
\newpage

\input{List_of_contributors}
\chapter*{Введение}

Теория формальных языков находит применение не только для ставших уже классическими задач синтаксического анализа кода (языков программирования, искусственных языков) и естественных языков, но и в других областях, таких как статический анализ кода, графовые базы данных, биоинформатика, машинное обучение.

Например, в машинном обучении использование формальных грамматик позволяет передать искусственной нейронной сети, предназначенной для генерации цепочек с определёнными свойствами (генеративной нейронной сети), знания о синтаксической структуре этих цепочек, что позволяет существенно упростить процесс обучения и повысить качество результата~\cite{10.5555/3305381.3305582}.
Вместе с этим, развиваются подходы, позволяющие нейронным сетям наоборот извлекать синтаксическую структуру (строить дерево вывода) для входных цепочек~\cite{kasai-etal-2017-tag,kasai-etal-2018-end}.

В биоинформатике формальные грамматики нашли широкое применение для описания особенностей вторичной структуры геномных и белковых последовательностей~\cite{Dyrka2019,WJAnderson2012,zier2013rna}.
Соответствующие алгоритмы синтаксического анализа используются при создании инструментов обработки данных.

Таким образом, теория формальных языков выступает в качестве основы для многих прикладных областей, а алгоритмы синтаксического анлиза применимы не только для обработки естественных языков или языков программирования.
Нас же в данной работе будет интересовать применение теории формальных языков и алгоритмов синтаксического анализа для анализа графовых баз данных и для статического анализа кода.

Одна из классических задач, связанных с анализом графов --- это поиск путей в графе.
Возможны различные формулировки этой задачи.
В некоторых случаях необходимо выяснить, существует ли путь с определёнными свойствами между двумя выбранными вершинами.
В других же ситуациях необходимо найти все пути в графе, удовлетворяющие некоторым свойствам или ограничениям. 
Например, в качестве ограничений можно указать, что искомый путь должен быть простым, кратчайшим, гамильтоновым и так далее.

Один из способов задавать ограничения на пути в графе основан на использовании формальных языков.
Базовое определение языка говорит нам, что язык --- это множество слов над некоторым алфавитом.
Если рассмотреть граф, рёбра которого помечены символами из алфавита, то путь в таком графе будет задавать слово: достаточно соединить последовательно символы, лежащие на рёбрах пути.
Множество же таких путей будет задавать множество слов или язык.
Таким образом, если мы хотим найти некоторое множество путей в графе, то в качестве ограничения можно описать язык, который должно задавать это множество.
Иными словами, задача поиска путей может быть сформулирована следующим образом: необходимо найти такие пути в графе, что слова, получаемые конкатенацей меток их рёбер, принадлежат заданному языку.
Такой класс задач будем называть задачами поиска путей с ограничениям в терминах формальных языков.

Подобный класс задач часто возникает в областях, связанных с анализом граф-структурированных данных и активно исследуется~\cite{doi:10.1137/S0097539798337716,axelsson2011formal,10.1007/978-3-642-22321-1_24,Ward:2010:CRL:1710158.1710234,barrett2007label,doi:10.1137/S0097539798337716}.
Исследуются как классы языков, применяемых для задания ограничений, так и различные постановки задачи.

Граф-структурированные данные встречаются не только в графовых базах данных, но и при статическом анализе кода: по программе можно построить различные графы отображающие её свойства.
Скажем, граф вызовов, граф потока данных и так далее.
Оказывается, что поиск путей в специального вида графах с использованием ограничений в терминах формальных языков позволяет исследовать некоторые свойства программы.
Например проводить межпроцедурный анализ указателей или анализ алиасов~\cite{Zheng,10.1145/2001420.2001440,10.1145/2714064.2660213}, строить срезы программ~\cite{10.1145/193173.195287}, проводить анализ типов~\cite{10.1145/373243.360208}.

В данной работе представлен ряд алгоритмов для поиска путей с ограничениями в терминах формальных языков.
Основной акцент будет сделан на контекстно-свободных языках, однако будут затронуты и другие классы: регулярные, многокомпонентные контекстно-свободные (Multiple Context-Free Languages, MCFL~\cite{!!!}) и конъюнктивные языки.
Будет показано, что теория формальных языков и алгоритмы синтаксического анализа применимы не только для анализа языков программирования или естественных языков, а также для анализа графовых баз данных и статического анализа кода, что приводит к возникновению новых задач и переосмыслению старых.


Структура данной работы такова.
В первых двух частях (~\ref{chpt:LinAlIntro} и~\ref{chpt:GraphTheoryIntro}) мы рассмотрим основные понятия из алгебры и теории графов, необходимые в данной работе. Данные разделы являются подготовительными и не обязательны к прочтению, если такие понятия как \textit{полукольцо} и \textit{матрица смежности} вам известны. Более того, они лишь вводят определения, подазумевая, что более детальное изучение соответствующих разделов остаются за рамками этой работы и скорее всего уже проделано читателем.
Затем, в главе~\ref{chpt:FormalLanguageTheoryIntro} мы введём основные понятия из теории формальных языков.
Далее, в главе~\ref{chpt:CFPQ} рассмотрим различные варианты постановки задачи поиска путей с ограничениями в терминах формальнх языков, обсудим базовые свойства задач, её разрешимость в различных постановках и т.д..
И в итоге зафиксируем постановку, которую будем изучать далее.
После этого, в главах~\ref{chpt:CFPQ_CYK}--\ref{chpt:CFPQ_Derivatives} мы будем подробно рассматривать различные алгоритмы решения этой задачи, попутно вводя специфичные для рассматриваемого алгоритма структуры данных.
Большинство алгоритмов будут основаны на классических алгоритмах синтаксического анализа, таких как CYK или LR.
Все главы, начиная с~\ref{chpt:GraphTheoryIntro}, снабжены списком вопросов и задач для самостоятельного решения и закрепления материала.
\chapter[x]{Некоторые понятия линейной алгебры\footnote{Неообходимо понимать, что, с одной строны, в данном разделе рассматриваются самые базовыепонятия, которые даются практически в любом учебнике алгебры. С другой же стороны, определения данных понятий оказываются весьма вариативными и часто вызывают дискуссии. Напрмиер, интересный анализ тонкостей определения группы можно найти в первом и втором параграфах первого раздела книги Николая Александровича Вавилова ``Конкретная теория групп''~\cite{VavilovGroups}. Мы же дадим определения, удобные для дальнейшего изложения материала.}}\label{chpt:LinAlIntro}

При изложении ряда алгоритмов будут активно использоваться некоторые понятия и инструмены линейной алгебры, такие как моноид, полукольцо или матрица.
В данном разделе необходимые понятия будут определены и приведены некоторые примеры соответствующих конструкций. Для более глубокого изучения материала рекомендуются соответствующие разделы алгебры.

$$
\oplus
\otimes
\mathbb{1}
\mathbb{0}
$$

\section{Бинарные операции и их свойства}


Введём понятие \textit{бинарной операции} и рассмотрим некоторые её свойства, такие как \textit{коммутативность} и \textit{ассоциативность}.

\begin{definition}[Двухместная функция] Функцию, принимающую два аргумента, $f: S \times K \to Q$ будем называть двухместной или функцией арности два.
Для запси таких функций будем использовать типичную нотацию: $c = f(a,b)$.
\end{definition}


\begin{definition}[Бинарная операция] 
Бинарная операция --- это двухместная функция, от которой дополнительно требуется, чтобы оба аргумента и результат лежали в одном и том же множестве: $f: S \times S \to S$. В таком случае говорят, что бинарная операция определена на некотором множестве $S$. Для обозначения произвольной бинарной операции будем использовать символ $\circ$ и пользоваться инфиксной нотацией для записи: $c = a \circ b$.
\end{definition}




\begin{definition}[Внешняя бинарная операция]
Внешняя бинарная операция --- это бинарная операция, у которой аргументы лежат в разных множествах, при этом результат --- в одном из этих множеств. Иными словами $\circ: K \times S \to S$, где $K$ может быть не равно $S$  --- внешняя бинарная операция.
\end{definition}


Необходимо помнить, что как функции, так и бинарные операции, могут быть частично определёнными (частичные функции, частичные бинарные операции). Типичным примером частично определённой бинарной операции является деление на целых числах: она не определена, если второй аргумент равен нулю.


Бинарные операции могут обладать некоторыми дополнительными свойствами, такими как \textit{коммутативность} или \textit{ассоциативность}, позволяющими преобразовывать выражения, составленные с использованием этих операций.


\begin{definition}[Коммутативность]
Бинарная операция $\circ : S \times S \to S$ называется коммутативной, если для любых  $x_1 \in S, x_2 \in S$ верно, что  $x_1 \circ x_2 = x_2 \circ x_1$.
\end{definition}

\begin{example} Рассмотрим несколько примеров коммутативных и некоммутативных операций.
	\begin{itemize}
		\item Опреация сложения на целых числах $+$ является коммутативной: известный ещё со школы перестановочный закон сложения.
		\item Операция конкатенации на строках $+$ не является коммутативной: $$``ab" + ``c" \ = ``abc" \neq ``c" + ``ab" \ = ``cab".$$
		\item Операция умножения на целых числах является коммутативной: известный ещё со школы перестановочный закон умножения.
		\item Операция умножения матриц (над целыми числами) $\cdot$ не является коммутативной:
		$$\begin{pmatrix} 
		1 & 1 \\ 0 & 0
		\end{pmatrix}
		\cdot
		\begin{pmatrix} 
		0 & 0 \\ 1 & 1
		\end{pmatrix}
		=
		\begin{pmatrix} 
		1 & 1 \\ 0 & 0
		\end{pmatrix}
		\neq
		\begin{pmatrix} 
		0 & 0 \\ 1 & 1
		\end{pmatrix}
		\cdot
		\begin{pmatrix} 
		1 & 1 \\ 0 & 0
		\end{pmatrix}
		=
		\begin{pmatrix} 
		0 & 0 \\ 1 & 1
		\end{pmatrix}
		.$$
	\end{itemize}
\end{example}

\begin{definition}[Ассоциативность]
Бинарная операция $\circ : S \times S \to S$ называется ассоциативной, если для любых  $x_1 \in S, x_2 \in S, x_3 \in S$ верно, что  $(x_1 \circ x_2) \circ x_3 = x_1 \circ (x_2 \circ x_3)$. Иными словами, для ассоциативной операции результат вычислений не зависит от порядка применения операций.
\end{definition}

\begin{example} Рассмотрим несколько примеров ассоциативных и неассоциативных операций.
	\begin{itemize}
		\item Опреация сложения на целых числах $+$ является ассоциативной.
		\item Операция конкатенации на строках $+$ является ассоциативной: $$``ab" + ``c" \ = ``abc" \neq ``c" + ``ab" \ = ``cab".$$
		\item Операция умножения на целых числах является ассоциативной.
		\item Операция возведения в степень (над целыми числами) $\hat{\mkern6mu}$ не является ассоциативной:
		$$(2\hat{\mkern6mu}2)\hat{\mkern6mu}3 = 4 \hat{\mkern6mu} 3 = 64 \neq 2\hat{\mkern6mu}(2\hat{\mkern6mu}3) = 2 \hat{\mkern6mu} 8  = 256.$$
	\end{itemize}
\end{example}


\begin{definition}[дистрибутивность]
!!!
\end{definition}

\begin{definition}[идемпотентность]
!!!
\end{definition}

\begin{definition}[Нейтральный элемент]
Пусть есть коммутативная бинарная операция $\circ$ на множестве $S$. Говорят, что $x\in S$ является нейтарльным элементом по операции $\circ$, если для любого $y\in S$ верно, что $x \circ y = y \circ x = y$. Если бинарная операция не является коммутативной, то можно пределить \textit{нейтральный слева} и \textit{нейтральный справа} элементы по аналогии.
\end{definition}


\section{Полугруппа}


множество с заданной на нём ассоциативной бинарной операцией $(S,\cdot )$ 


Коммутативная полугруппа


The set of positive integers with addition. (With 0 included, this becomes a monoid.)
The set of integers with minimum or maximum. (With positive/negative infinity included, this becomes a monoid.)
Square nonnegative matrices of a given size with matrix multiplication.
Any ideal of a ring with the multiplication of the ring.
The set of all finite strings over a fixed alphabet $\Sigma$ with concatenation of strings as the semigroup operation — the so-called ``free semigroup over $\Sigma$''. With the empty string included, this semigroup becomes the free monoid over $\Sigma$.


\section{Моноид}


Полугруппа с нейтральным элементом.



\section{Группа}


Непустое множество $G$ с заданной на нём бинарной операцией $*$: $ \mathrm {G} \times \mathrm {G} \rightarrow \mathrm {G}$ называется группой $ (\mathrm {G} ,*)$, если выполнены следующие аксиомы:

ассоциативность: $\forall (a,b,c\in G)\colon (a*b)*c=a*(b*c)$;
наличие нейтрального элемента: $ \exists e\in G\quad \forall a\in G\colon (e*a=a*e=a)$;
наличие обратного элемента: $ \forall a\in G\quad \exists a^{-1}\in G\colon (a*a^{-1}=a^{-1}*a=e)$.

Иными словами, группа --- это моноид с дополнительным требованием наличия обратных элементов.

\begin{definition}[Абелева группа] --- операция коммутативна.
\end{definition}


\section{Полукольцо}

A semiring is a set R equipped with two binary operations + and $\otimes$, called addition and multiplication, such that:[3][4][5]

(R, +) is a commutative monoid with identity element 0:
(a + b) + c = a + (b + c)
0 + a = a + 0 = a
a + b = b + a
(R, $\otimes$) is a monoid with identity element 1:
(a$\otimes$b)$\otimes$c = a$\otimes$(b$\otimes$c)
1$\otimes$a = a$\otimes$1 = a
Multiplication left and right distributes over addition:
a$\otimes$(b + c) = (a$\otimes$b) + (a$\otimes$c)
(a + b)$\otimes$c = (a$\otimes$c) + (b$\otimes$c)
Multiplication by 0 annihilates R:
0$\otimes$a = a$\otimes$0 = 0


\section{Кольцо}


A ring is a set R equipped with two binary operations[a] + (addition) and $\otimes$ (multiplication) satisfying the following three sets of axioms, called the ring axioms[1][2][3]

R is an abelian group under addition, meaning that:
(a + b) + c = a + (b + c) for all a, b, c in R   (that is, + is associative).
a + b = b + a for all a, b in R   (that is, + is commutative).
There is an element 0 in R such that a + 0 = a for all a in R   (that is, 0 is the additive identity).
For each a in R there exists $-a$ in R such that $a + (-a) = 0$   (that is, $-a$ is the additive inverse of a).
R is a monoid under multiplication, meaning that:
(a $\otimes$ b) $\otimes$ c = a $\otimes$ (b $\otimes$ c) for all a, b, c in R   (that is, $\otimes$ is associative).
There is an element 1 in R such that a $\otimes$ 1 = a and 1 $\otimes$ a = a for all a in R   (that is, 1 is the multiplicative identity).[b]
Multiplication is distributive with respect to addition, meaning that:
a $\otimes$ (b + c) = (a $\otimes$ b) + (a $\otimes$ c) for all a, b, c in R   (left distributivity).
(b + c) $\otimes$ a = (b $\otimes$ a) + (c $\otimes$ a) for all a, b, c in R   (right distributivity).


\section{Поле}

\section{Матрицы и вектора}

Вектор

Матрица 

Про матричное произведение, тензорное произведение, ещё что-то.

\section{Вопросы и задачи}
\begin{enumerate}
	\item Привидите примеры некоммутативных операций.
	\item Привидите примеры ситуаций, когда наличие у бинарных операций каких-либо дополнитльных свойств (ассоциативности, коммутативности), позволяет строить более эффективные алгоритмы, чем в общем случае.
\end{enumerate}
\input{GraphTheoryIntro}
\chapter{Общие сведения теории формальных языков}\label{chpt:FormalLanguageTheoryIntro}

В данной главе мы рассмотрим основные понятия из теории формальных языков, которые пригодятся нам в дальнейшем изложении.

\begin{definition}
\textit{Алфавит} --- это конечное множество.
Элементы этого множества будем называть \textit{символами}.
\end{definition}

\begin{example}
  Примеры алфавитов

  \begin{itemize}
    \item Латинский алфавит $\Sigma = \{ a, b, c, \dots, z\}$
    \item Кириллический алфавит $\Sigma = \{ \text{а, б, в, \dots, я}\}$
    \item Алфавит чисел в шестнадцатеричной записи 
    $$\Sigma = \{0, 1, 2, 3, 4, 5, 6, 7 ,8,9, A, B, C, D, E, F \}$$
  \end{itemize}
\end{example}

Традиционное обозначение для алфавита --- $\Sigma$.
Также мы будем использовать различные прописные буквы латинского алфавита. Для обозначения символов алфавита будем использовать строчные буквы латинского алфавита: $a, b, \dots, x, y, z$.

Будем считать, что над алфавитом $\Sigma$ всегда определена операция конкатенации $(\cdot): \Sigma^* \times \Sigma^* \to \Sigma^*$.
При записи выражений символ точки (обозначение операции конкатенации) часто будем опускать: $a \cdot b = ab$.

\begin{definition}
\textit{Слово} над алфавитом $\Sigma$ --- это конечная конкатенация символов алфавита $\Sigma$: $\omega = a_0 \cdot a_1 \cdot \ldots \cdot a_m$, где $\omega$ --- слово, а для любого $i$ $a_i \in \Sigma$.
\end{definition}

\begin{definition}
Пусть $\omega = a_0 \cdot a_1 \cdot \ldots \cdot a_m$ --- слово над алфавитом $\Sigma$.
Будем называть $m + 1$ \textit{длиной слова} и обозначать как $|\omega|$.
\end{definition}

\begin{definition}
\textit{Язык} над алфавитом $\Sigma$ --- это множество слов над алфавитом $\Sigma$.
\end{definition}

\begin{example}

Примеры языков.

  \begin{itemize}
    \item Язык целых чисел в двоичной записи $\{0, 1, -1, 10, 11, -10, -11, \dots\}.$
    \item Язык всех правильных скобочных последовательностей $$\{(), (()), ()(), (())(), \dots\}.$$
  \end{itemize}
\end{example}

Любой язык над алфавитом $\Sigma$ является подмножеством $\Sigma^*$ --- множества всех слов над алфавитом $\Sigma$.

Заметим, что язык не обязан быть конечным множеством, в то время как алфавит всегда конечен и изучаем мы конечные слова.

%\begin{definition}
\textit{Способы задания языков}
\begin{itemize}
\item Перечислить все элементы. Такой способ работает только для конечных языков. Перечислить бесконечное множество не получится.
\item Задать генератор --- процедуру, которая возвращает очередное слово языка.
\item Задать распознователь --- процедуру, которая по данному слову может определить, принадлежит оно заданному языку или нет.
\end{itemize}


Теоретико-множественные задачи над языками и их применение. 
О том, что моногое --- про пересечение, проверку пустоты, вложенность.





\section{Вопросы и задачи}
\begin{enumerate}
  \item !!! 
  \item !!!
\end{enumerate}

\input{CFPQ}
\input{CYK_for_CFPQ}
\input{Matrix-based_CFPQ}
\input{TensorProduct}
\input{SPPF}
\input{GLL-based_CFPQ}
\input{GLR-based_CFPQ}
\input{CombinatorsForCFPQ}
\input{DerivativesForCFPQ}
\input{CFPQ_to_Datalog}
%\input{Conclusion}

\bibliographystyle{abbrv}
\bibliography{Formal_lang_CFPQ_course_notes}


\end{document}
