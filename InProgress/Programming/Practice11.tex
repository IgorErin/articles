\section{Практика 11}

Мини-IDE для языка из предыдущей работы. Проект оформляется вотдельном репозитории по всем правилам. Процесс сдачи задач прежний.

\begin{enumerate}
    \item \textbf{[2 балла]} Расширить синтаксис соответствующего языка логическими выражениями (с переменными) и условными операторами.
    \item \textbf{[9 баллов]} Разработать среду разработки для полученного языка. Среда должна предоставлять следующие возможности:
    \begin{itemize}
        \item Редактировать код.
        \item Работать с файлами: создать новый, открыть существующий, сохранить изменения. 
        \item Выводить сообщения о (синтаксических) ошибках.
        \item Запустить программу на исполнение.
        \item Видеть результат исполнения в «консоли».
    \end{itemize}
    \item \textbf{[3 балла]} Расширить IDE возможностью подсветки синтаксиса.
    \item \textbf{[8 баллов]} Расширить IDE возможностью устанавливать точки останова. В момент остановки должна быть возможность просмотреть значения всех ``живых'' переменных.
  
\end{enumerate}