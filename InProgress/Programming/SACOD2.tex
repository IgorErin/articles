\section{Лекция 2: Структуры данных и алгоритмы линейной алгебры}


Основы линейной алгебры: примитивы (матрицы,  вектора, поля, кольца, полукольца) и их свойства (конечность и идемпотентность, коммутативность и т.д.), операции над матрицами и векторами: поэлементные, умножение матриц, умножение матрицы на вектор, тензорное произведение, транспонирование.


Разреженное представление матриц и векторов. Основные форматы разреженного представления матриц: покоординатный, CSR, Quad-tree. Специализированные форматы: диагональные матрицы, HiCOO, и др. Их преимущества и недостатки.

\section{Лекция 2: Структуры данных и алгоритмы линейной алгебры}

Параллельная обработка разреженных матриц и векторов. Особенности соответствующих алгоритмов для GPGPU.

Прикладные задачи, сводимые к линейной алгебре. Обработка графов, GraphBLAS API. BFS, транзитивное замыкание, кратчайшие пути, подсчёт треугольников, минимальное остовное дерево. Пересечение автоматов, объединение автоматов.

\url{https://github.com/GraphBLAS/GraphBLAS-Pointers}

\url{https://archive.fosdem.org/2020/schedule/event/graphblas/}

\subsection{Домшняя работа 2}

Данная работа посвящена реализации алгоритмов анализа графов с использованием операций линейной алгебры. Необходимо выбрать минимум 3 различных алгоритма (требующих различных операций). Неполный список: подсчёт треугольников, BFS, минимальное остовное дерево, поиск кратчайших путей. Можно предложить свой. Выбранные алгоритмы необходимо реализовать с использованием различных библиотек и сравнить их производительность.

\begin{enumerate}
   \item \textbf{3 балла.} Реализовать выбранные алгоритмы на (py)graphblas.
   \item \textbf{3 балла.} Реализовать выбранные алгоритмы на sciPy.
   \item \textbf{3 балла.} Реализовать выбранные на стандартной библиотеке для анализа графов (можно выбрать в зависимости от языка).
   \item \textbf{10 баллов.} Сравнить производительность полученных реализаций, составить отчёт.
\end{enumerate}


\section{Лекция 3: Структуры данных и алгоритмы линейной алгебры}

Раздел 3: Основы анализа сложности алгоритмов.
    1. Введение. Классическая теория сложности, анализ сложности алгоритмов в теории и что это значит для практики. 
    2. Основы fine-grained complexity.
    3. Основы анализа сложности параллельных алгоритмов.
		Домашняя работа 5. Выбрать любой алгоритм на графах из реализованных и провести его анализ сложности.
    4. Проверочная работа


\subsection{Домшняя работа 3}

\begin{enumerate}
   \item \textbf{7 баллов.} Бибилиотека плотных операций на ГПУ.
   \item \textbf{5 баллов.} Задачи на графах.
   \item \textbf{10 баллов.} Производительность.
   \item \textbf{13 баллов.} Попробовать что-то реализовать на разреженной алгебре и сравнить производиительность.
\end{enumerate}
