Домашняя работа 4
    4. Actor-ориентированное программирование как реализация асинхронного (concurrent) программирования. Коммуникация на сообщениях. Mailbox processor и Hopac как реализации. Примеры использования Mailbox processor и Hopac.
Домашняя работа 5
    5. Контрольная работа.

Период обучения (модуль): семестр 3.

Раздел 1: Особенности исследовательских проектов.
    1. Жизненный цикл исследовательских проектов и возможные пути развития. Отличие от прикладных/промышленных проектов и сходство с ними. Цели и задачи исследовательских проектов. 
    2. Эксперименты: воспроизводимость, анализ результатов, оформление результатов. Примеры соответствующих инструментальных средств. Цели и задачи экспериментов. Экспериментальное исследование, сравнение, проверка гипотезы.
Домашняя работа 1
    3. Оформление результатов в виде текста: статья, технический отчёт и другие типы текстов. Особенности процесса их написания. Типичная структура и особенности. Курсовая/диплом как научная работа и отчёт по ним как научный текст.
    4. Представление результатов в виде презентации, доклада. Различные виды докладов, презентаций. Типичные структуры. Пример презентации для курсовой.
Домашняя работа 2
Раздел 2: Продвинутые техники программирования 
    1. Метапрограммирование вообще и в F# в частности. Понятие метапрограммирования. Подходы к реализации техник метапрограммирования.  Программирование времени выполнения и времени компиляции. Примеры: шаблоны в С++ (тьюринг-полнота), системы макросов, вычислимые выражения. Встраивание языков.
    2. Рефлексия, вообще и в .NET. Сборки в .NET, сильные и слабые имена сборок. Получение информации о сборках, типах, полях, методах и т.д., создание экземпляров объектов, вызов методов. Компиляция F#-кода во время выполнения.
Домашняя работа 3
    3. Разбор домашней работы 2   
    4. Поставщики типов: их типы, решаемые с их помощью задачи. Особенности использования. Особенности создания. Примеры готовых поставщиков и их использования.
    5. F# quotations. Трансформация кода во время выполнения. Возможности и ограничения. Типизированный и нетипизированный варианты. Примеры использования.
    6. Событийно-ориентированное программирование, реактивное программирование RxExtension.
Домашняя работа 4
    7. Проверочная работа
Раздел 3: Программирование на GPGPU
    1. Основы. Архитектура, логическая модель вычислителя, плюсы/минусы, сферы применения.
    2. OpenCL, логическая и физическая модели, переносимость, ядра, атомарные операции, сравнение с CUDA. Основы языка OpenCL C. Примеры простых ядер.
    3. Программирование на GPGPU с использованием высокоуровневых средств. Программирование GPGPU на F# как пример метапрограммирования. Особенности и проблемы использования высокоуровневых средств.
Домашняя работа 5
    4. Проверочная работа.

