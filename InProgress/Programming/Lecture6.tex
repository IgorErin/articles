\section{Лекция 6}
 
11. Контрольная работа.

\begin{enumerate}
	\item \textbf{[3 балла]} Что такое хвостовая рекурсия? Чем отна отличается от ``обычной'' рекурсии? Привелите пример кода на F\# для следующих функций, реализованных с использованием хвостовой рекурсии
	\begin{enumerate}
	    \item сумма элементов списка, 
	    \item минимальный элемент списка, 
	    \item максимальный элемент списка, 
	    \item факториал, 
	    \item длина списка
    \end{enumerate}
	\item \textbf{[3 балла]}
		\begin{enumerate}
		   \item Расскажите про этапы жизни программнго продукта (проекта). Опишите каждый этап.
           \item Для каких целей может проводиться экспериментальное исследование реализации алгоритма. В чём отличие между теоретическим и экспериментальным исследованием?
           \item Что такое отсортированная коллекция? Каким условиям должны удовлетворять элементы коллекции, чтобы коллекцию из этих элементов можно было отсортировать? Какие алгоритмы сортировки вы знаете? Чем они отличаются?
           \item Перечислите компоненты инфраструктуры проекта. Какую роль они выполняют? Приведите примеры конкретных реализаций  компонент.
           \item Что такое отладка? Что такое тестирование? Чем отладка отличается от тестирования? Какие средства отладки вы знаете? Какие средства тестирования вы знаете?
       \end{enumerate}
    \item \textbf{[4 балла]}
    	\begin{enumerate}
    	   \item Сколько операций сложения и умножения чисел требуется для перемножения двух матриц? Почему?
           \item Сколько операций сравнения потребуется для сотрировки списка длины n сортировкой Хоара? Почему?      
           \item Сколько операций сравнения потребуется для сотрировки списка длины n сортировкой пузырьком? Почему?
           \item Сколько операций сложения потребуется для вычисления n-ого числа Фибоначчи наивным рекурсивным методом? Почему?
           \item Сколько операций сложения и умножения чисел потребуется для вычисления n-ого числа Фибоначчи через умнжение матриц ``умным'' (не наивным) способом? Почему?
        \end{enumerate} 
\end{enumerate}

12. Про анализ результатов экспериментов. Бокс-плот. Придумать что-то про JIT? 




















