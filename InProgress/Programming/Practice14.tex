\section{Практика 14}

Семестр 3.

Домашняя работа 1
    1. (5 баллов) Сформулировать гипотезы относительно зависимости времени работы алгоритмов  сортировок относительно размера входных данных. Проверить их. Обосновать полученные результаты. Оформить соответствующий отчёт. Должны быть проанализированы следующие алгоритмы.
        1. Стандартная сортировка для List
        2. Различные варианты реализации быстрой сортировки для MyList.
        3. Сортировка пузырьком для List.
        4. Сортировка пузырьком для Array.
        5. Можно включить сортировки для списка с конкатенацией за константу, а также другие реализации быстрой сортировки (на массиве, например).
    2. (7 баллов) Сформулировать гипотезы относительно зависимости времени работы матричных алгоритмов относительно размера входных матриц. Проверить их. Обосновать полученные результаты. Оформить соответствующий отчёт. Должны быть проанализированы следующие алгоритмы.
        1. Последовательный для дерева квадрантов
        2. Параллельный для дерева квадрантов
        3. Последовательный для плотных матриц
        4. Параллельный для плотных матриц
Проверяемые компетенции: ОПК-2, ОПК-4, ОПК-5, ПКП-3, ПКП-4, УКБ-3
Критерии оценивания: решения задачи 1 оцениваются по шкале от 0 (нет решения или решение имеет существенные недостатки) до 5 (решение работоспособно, аккуратно реализовано), решения задачи 2 оцениваются по тем же критериям, но по шкале от 0 до 7.

Домашняя работа 2
    1. (6 баллов) Подготовить презентацию «Анализ времени работы алгоритмов сортировок» или «Анализ времени работы алгоритмов умножения матриц». Презентация готовится в TeX, по установленному шаблону.
    2. (8 баллов) Подготовить отчёт на тему «Анализ времени работы алгоритмов сортировок» или «Анализ времени работы алгоритмов умножения матриц». Отчёт готовится в TeX, по установленному шаблону.
Проверяемые компетенции: ОПК-2, ОПК-4, ОПК-5, ПКП-3, ПКП-4, УКБ-3
Критерии оценивания: решения задачи 1 оцениваются по шкале от 0 (нет презентации или презентация имеет существенные недостатки) до 6 (презентация полностью раскрывает тему и аккуратно оформлена), решения задачи 2 оцениваются по тем же критериям, но по шкале от 0 до 8.

Домашняя работа 3
    1. (3 балла) Реализовать «логгер» с использованием workflow builder. Необходимо предоставить возможность логгировать входные аргументы некоторых функций. Предусмотреть возможность указывать, куда печатать вывод: в консоль, файл, куда-то ещё.
    2. (5 баллов) Реализовать list builder для типа MyList или списка с конкатенацией за константу. Ориентироваться на билдер seq для лучшего понимания ожидаемой функциональности.
Проверяемые компетенции: ОПК-2, ОПК-5, ПКП-3
Критерии оценивания: решения задачи 1 оцениваются по шкале от 0 (нет решения или решение имеет существенные недостатки) до 3 (решение работоспособно, аккуратно реализовано), решения задачи 2 оцениваются по тем же критериям, но по шкале от 0 до 5.

Домашняя работа 4
    1. (10 баллов) Встроить на основе F# quotations разработанный ранее язык арифметики или регулярных выражений. Обеспечить вычисление задаваемых выражений и возможность работы с полученными значениями на стороне F#. Необходимо максимальное переиспользование готовых компонент.
    2. (4 баллов) Внедрить RxExtensions для работы с событиями в разработанной ранее IDE. 
Проверяемые компетенции: ОПК-2, ОПК-5, ПКП-3, ПКП-4
Критерии оценивания: решения задачи 1 оцениваются по шкале от 0 (нет решения или решение имеет существенные недостатки) до 10 (решение работоспособно, аккуратно реализовано), решения задачи 2 оцениваются по тем же критериям, но по шкале от 0 до 4.

Домашняя работа 5
    1. (4 баллов) Реализовать функцию перемножения двух плотных матриц с использованием Brahma.FSharp. 
    2. (9 баллов) Расширить решение из предыдущего семестра возможностью конкурентно умножать матрицы на GPGPU (использовать предыдущее решение).
    3. (9 баллов) Проанализировать полученное в предыдущем пункте решение и оформить соответствующий отчёт.
        1. В каких случаях лучше использовать CPU, а в каких GPGPU?
        2. Какая конфигурация конкурентно выполняющихся задач оптимальна? Имеет ли смысл поддерживать больше одного агента, работающего с GPGPU?
        3. Имеет ли смысл использовать Brahma.FSharp для выполнения кода на CPU?
Проверяемые компетенции: ОПК-2, ОПК-4, ОПК-5, ПКП-3, ПКП-4, УКБ-3
Критерии оценивания: решения задачи 1 оцениваются по шкале от 0 (нет решения или решение имеет существенные недостатки) до 4 (решение работоспособно, аккуратно реализовано), решения задач 2 и 3 оцениваются по тем же критериям, но по шкале от 0 до 9.

