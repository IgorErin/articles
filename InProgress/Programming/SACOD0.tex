\section{Введение}

Баллы можно получать за домашние задачи и доклады.

Домашние задачи сдаются через оформленный на GitHub репозиторий. Кроме кода, репозиторий должен содержать readme с информацией о проекте, инструкциями по запуску и установке, тесты, подключённый и настроенный CI, build script для автоматизации установки и сборки. Разрабатываемые решения должны быть кроссплатформенными (или хотя бы безболезненно запускаться на Ubuntu).

Сдача домашней работы --- запрос ревью соответствующего реквеста. Реквест должен содержать только изменения, имеющие непосредственное отношение к сдаваемой работе. Работа состоит из несколькоих задач и сдавать можно отдельные задачи. У каждой работы есть дедлайн, после которого любая сданная задача оценивается в четверть от полного балла.

Все отчёты и презентации готовятся в TeX и снабжаются соответствующим скриптом сборки. Публикуются на GitHub как исходники, так и результирующий PDF-файл.

Темы докладов:
    \begin{enumerate}
    	\item Инфраструктура проекта на примере F\#: FAKE, Scaffold, FsCheck    
    	\item Веб-приложения в комбинаторном стиле: Suave
        \item Программирование, ориентированное на обработку данных: Type Providers, F\# Data
        \item Распределённое программирование на F\#: MBrace  
        \item Программирование GPGPU на F\#: Alea CUDA
    	\item Мапрограммирование на F\#: F\# code quotations
    \end{enumerate}

\begin{itemize}
   \item 91--100 : A (отл)
   \item 81--90 : B (хор)
   \item 71--80 : C (хор)
   \item 61--70 : D (удвл)
   \item 51--60 : E (удвл)
   \item 0--50 : F (неуд)
\end{itemize}