\section{Лекция 9}

Ещё раз про то, как условный оператор форматировать.
 

Граф как формальный объект и как структура данных. Понятие о бинарном отношении и его свойствах: транзитивность, рефлексивность, симметричность. (Не)Ориентированные, (не)помеченные графы. Способы представления графов: список рёбер, список смежности, матрица смежности.

\href{https://graphviz.org/}{GraphViz}

Базовые алгоритмы на графах. Обходы в глубину и ширину, построение транзитивного замыкания, поиск кратчайшего пути.

Линейная алгебра. Основы: матрица, вектор, полукольцо, кольцо, монид, полугруппа. Сведение некоторых задач к операциям линейной алгебры (транзитивное замыкание, кратчайшие пути, пересечение автоматов). Особенности практического использования такого подхода: разреженные структуры данных, абстрактность, композициональность.

Множество $S$, с заданными на нем бинарными операциями $+$ и $\cdot$, называется полукольцом, если для любых элементов $a,b,c$ верно следующее:
\begin{enumerate}
\item $\langle S,+\rangle$ --- коммутативный моноид. То есть имеют место свойства:
\begin{enumerate}
	\item Коммутативности: $a+b=b+a$
    \item Ассоциативности: $(a+b)+c=a+(b+c)$
   	\item Существования нейтрального элемента (нуля): $a+0=0+a=a$
\end{enumerate}
\item $\langle S,\cdot \rangle$ --- полугруппа. Необходимо свойство ассоциативности: $(a\cdot b)\cdot c=a\cdot (b\cdot c)$
\item Умножение дистрибутивно относительно сложения:
\begin{enumerate}
	\item Левая дистрибутивность: $a\cdot (b+c)=a\cdot b+a\cdot c$
    \item Правая дистрибутивность: $(a+b)\cdot c=a\cdot c+b\cdot c$
\end{enumerate}
\item Мультипликативное свойство нуля: $a\cdot 0=0\cdot a=0$
\end{enumerate}

Моноид --- это полугруппа с нейтральным элементом.
Кольцо, в отличие от полукольца, по сложению образуеи комутативную группу (содержит обратные по сложению).

В отечественной культуре в полукольце нет нейтрального по умножению, зато есть полукольцо с единицей --- полукольцо с нейтральным по умножению. Фактически, умножение начинает задавать моноид. Однако часто определение полукольца включает требование наличие нейтрального по умножению.

Полукольцо называют коммутативным, если операция умножения в нём коммутативна.

Полукольцо называют идемпотентным, если для любого $s \in S, s + s = s$

    
    
Дерево квадрантов.  
    

Примеры матриц смежности и тензорное произведение можно посмотреть в \href{https://github.com/YaccConstructor/articles/blob/master/InProgress/Formal_langs_CFPQ_course_notes/Formal_lang_CFPQ_course_notes.pdf}{этом документе}. Соответственно, разделы 2.1 ``Основные определения'' и 7.2 ``Тензорное произведение''.
