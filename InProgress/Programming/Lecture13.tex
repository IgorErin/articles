\section{Лекция 13}
 
И тут начался второй секместр.

Программный продукт, проект.

\begin{enumerate}
    \item Программа, проект, продукт – что есть что, различия. Жизненный цикл продукта.
    \item Открытый исходный код: окружение, инструменты, лицензии. Экосистема проектов с открытым исходным кодом. Непрерывная интеграция: задачи, облачный сервис AppVeyor, настройка сборки, матрица сборки. Облачный сервис Travis. GitHub Actions. Инструменты анализа качества, линтеры, покрытие тестами. Инструменты планирования и управления проектом: Trello, Pivotal Tracker. Средства коммуникации: Slack, Gitter. Багтрекер GitHub Issues. Другие средства управления проектом GitHub. Авторское право и лицензии.
    \item Документация, комментирование, автоматическая генерация документации по комментариям. Публикация документации на gh-pages.
    \item Визуальное моделирование, UML. Метафора моделирования, цель моделирования. Диаграммы UML. Диаграмма классов: синтаксис, синтаксис свойств, агрегация и композиция. Диаграмма компонентов. Диаграмма случаев использования. Диаграммы активностей, последовательностей, конечных автоматов. Генерация кода по диаграммам конечных автоматов. Диаграммы развёртывания. Примеры CASE-инструментов. Предметно-ориентированные визуальные языки.

\end{enumerate}


FSharpLint: \url{https://fsprojects.github.io/FSharpLint/}. 

Пример матрицы и установки FSharpLint: \url{https://github.com/YaccConstructor/Brahma.FSharp/blob/master/.github/workflows/build.yml}

Покрытие тестами уже есть в build.fsx. 

Лицензия: если просто выложили на гитхаб, то, вообще говоря, ПО не открытое и не свободное. Скорее всего им никто не сможет воспользоваться.

Документация

ReadTheDocs (\url{https://readthedocs.org/}) --- хостинг для документации.

Автоматическая генерация документации: Doxigen(\url{https://www.doxygen.nl/index.html}),
JavaDoc(\url{https://docs.oracle.com/javase/8/docs/technotes/guides/javadoc/index.html})

BuildDocs

ReleaseDocs

Release:\url{https://www.jimmybyrd.me/MiniScaffold/Tutorials/Getting_Started_With_Libraries.html#Making-a-Release}

UML

\url{draw.io}

\url{https://www.lucidchart.com/}

\url{https://www.lucidchart.com/pages/uml-component-diagram}

\url{https://www.lucidchart.com/pages/uml-class-diagram#section_2}

\url{https://www.lucidchart.com/pages/uml-use-case-diagram}

\url{https://www.lucidchart.com/pages/uml-sequence-diagram}


Примеры
\url{https://github.com/vdshk/graph-database}

\url{https://github.com/SergeyKuz1001/formal_languages_autumn_2020}

\url{https://github.com/AnzhelaSukhanova/Minimal_GDB}