\section{Лекция 8}
 
    Про то, что теперь не указываем типы там, где это не нужно. А точнее, указываем только там, где это необходимо.
    Про думать головой над постановками задачи, а не просто кодить.


    Ещё раз про полиморфизм, Ad-hoc полиморфизм и бинарные операции. 
    Типовые параметры и ограничения на них в F\# (\url{https://docs.microsoft.com/en-us/dotnet/fsharp/language-reference/generics/constraints}). 
    %Структурный полиморфизм в Ocaml.
    
    Списки, деревья: как формальные объекты, структуры данных и как примеры алгебраических обобщённых типов. 
    Реализация списка и дерева. 
    Обходы списков и деревьев. 
    
    Про fold, map, iter.

    Длинная арифметика. Практика работы со списками. Ещё раз о проблеме переполнения. Целочисленная арифметика на списках. 
    
