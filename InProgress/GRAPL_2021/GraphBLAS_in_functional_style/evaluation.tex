\section{Evaluation}

While our implementation is on very early stage, we cannot evaluate it on well-known linear algebra based algorithms. 
But in order to demonstrate applicability of proposed solutions we evaluate element-wise matrix-matrix operations. 

We perform our experiments on the PC with Ubuntu 18.04 installed and with the following hardware configuration: !!! CPU, !!! RAM, !!!GPGPU with !!!!.

our solution on CPU and GPGPU.
For comparison we choose the following libraries.

\begin{itemize}
\item SuiteSparse:GraphBLAS as a reference CPU implementation of GraphBLAS API.
\item GraphBLAST as a most stable GPGPU implementation of GraphBLAS API.
\end{itemize}


\subsection {Dataset}

For evaluation we select a set of matrices from SuiteSparse matrix collection collection\footnote{SuiteSparse matrix collection: \url{https://sparse.tamu.edu/}.}
To simplify evaluation of element-wise operations over matrices with different structure we precompute square of each matrix. 
Characteristics of selected matrices are presented in table~\ref{matrices}.

\begin{table}[h]
    \centering
    \caption{Matrices for evaluation}
    \label{matrices}  
    \begin{tabular}{ | c || c | c | c | }
    \hline
    Matrix & Size & NNZ & Squared matrix NNZ \\ \hline
    \hline
    wing & 62 032 & 243 088 & 714,200 \\ \hline
    luxembourg\_osm & 114 599 & 119 666 & 4 582 \\ \hline
    amazon0312 & 400 727 & 3 200 440 & 14 390 544 \\ \hline
    amazon-2008 & 735 323 & 5 158 388 & 25 366 745 \\ \hline
    web-Google & 916 428 & 5 105 039 & 30 811 855 \\ \hline
    webbase-1M & 1 000 005 & 3 105 536 & 51 111 996 \\ \hline
    cit-Patents & 3 774 768 & 16 518 948 & 469 \\ \hline
    \end{tabular}
\end{table}

We evaluate our generic element-wise function parameterized by two operations: \texttt{op\_int\_add} (listing~\ref{lst:opIntAdd}) to get element-wise addition in terms of GraphBLAS API, and \texttt{op\_int\_mult} (listing~\ref{lst:opIntMult}) to get element-wise multiplication.


\subsection{Evaluation Results}

To benchmark .NET-based implementation we use \textit{BenchmarkDotNet}\footnote{\textit{BenchmarkDotNet}: \url{https://benchmarkdotnet.org/}. Access date: 12.06.2022.} which allows one to automate benchmarking process for .NET platform.
We run each function XXX times, !!!
Time is measured in milliseconds.

Results of performance evaluation are presented in tables~\ref{add-comparison} and~\ref{mult-comparison}.


\begin{table}[h]
    \centering    
    \caption{Evaluation results for element-wise addition, time in ms}    
    \label{add-comparison}
    \begin{tabular}{|c||c|c|c|}
    \hline
    Matrix            & GraphBLAS-sharp & SuiteSparse & CUSP        \\
    \hline
    \hline
    wing            & $1,8 \pm 0,1$      & $1,9\pm 0,1$   & $0,5\pm 0,2$   \\
    \hline
    luxembourg\_osm & $2,9 \pm 0,3$      & $1.9\pm 0,5$   & $0,5\pm 0,1$   \\
    \hline
    amazon0312      & $17,0 \pm 0,8$      & $28,9\pm 0,2$  & $2,8\pm 0,1$   \\
    \hline
    amazon-2008     & $12,2 \pm 0,8$     & $50,1\pm 2,4$  & $3,5\pm 0,1$   \\
    \hline
    web-Google      & $18,4 \pm 0,6$     & $58.8\pm 0,7$  & $3,6\pm 0,1$   \\
    \hline
    webbase-1M      & $70,7 \pm 1,0$      & $72,9\pm 0,4$  & $24,6\pm 2,1$  \\
    \hline
    cit-Patents     & $54,6 \pm 1,2$      & $157,4\pm 1,2$ & $8,5\pm 1,2$   \\     
    \hline
    \end{tabular}    
\end{table}

\begin{table}[h]

    \centering
    \caption{Evaluation results for element-wise multiplication, time in ms}
    \label{mult-comparison}
    
    \begin{tabular}{|c||c|c|}
    \hline
    Matrix            & GraphBLAS-sharp & SuiteSparse    \\
    \hline
    \hline
    wing            & $2,5 \pm 0,4$      & $1,0 \pm 0,1$ \\
    \hline
    luxembourg\_osm & $2,6 \pm 0,3$       & $1,4 \pm 0,3$ \\
    \hline
    amazon0312      & $13,0 \pm 1,0$     & $23,0 \pm 0,9$ \\
    \hline
    amazon-2008     & $9,1 \pm 0,8$    & $35,2 \pm 4,0$ \\
    \hline
    web-Google      & $14,7 \pm 0,8$      & $43,9 \pm 0,2$  \\
    \hline
    webbase-1M      & $55,4 \pm 1,2$      & $31,0 \pm 1,6$ \\
    \hline
    cit-Patents     & $47,9 \pm 0,9$      & $107,9 \pm 0,4$  \\     
    \hline
    \end{tabular}
\end{table}

\begin{table}[h]

    \centering
    \caption{Performance comparison of base version and version for \texttt{AtLeastOne}: element-wise addition, time in ms}
    \label{option-noOption-comparison}
    
    \begin{tabular}{|c||c|c|}
    \hline
    Matrix             & CSR No Option & CSR with Option  \\
        \hline
        \hline
        wing            & $1,6 \pm 0,3$      & $1,5 \pm 0,2$ \\
        \hline
        luxembourg\_osm & $2,0 \pm 0,3$       & $2,0 \pm 0,3$ \\
        \hline
        amazon0312      & $9,1 \pm 0,8$     & $9,2 \pm 0,8$ \\
        \hline
        amazon-2008     & $7,2 \pm 0,7$    & $7,0 \pm 0,6$ \\
        \hline
        web-Google      & $9,8 \pm 0,9$      & $9,3 \pm 0,6$  \\
        \hline
        webbase-1M      & $58,2 \pm 1,1$      & $58,0 \pm 1,0$ \\
        \hline
        cit-Patents     & $26,1 \pm 1,0$      & $25,5 \pm 0,9$  \\     
        \hline
    \end{tabular}
\end{table}

We can see, that for element-wise addition our implementation slower than SuiteSparse:GraphBLAS for small matrices (\textbf{luxembourg\_osm}) and up to 4 times faster for big matrices.
For element-wise multiplication results are almost similar except matrix \textbf{webbase-1M} for which our implementation slower than SuiteSparse:GraphBLAS while this matrix contains big number of non-zero values.

Comparison between original element-wise addition over primitive types, without \texttt{AtLeasOne} and generalized version which uses \texttt{AtLeastOne} type is presented in table~\ref{option-noOption-comparison}.
We can see, that more complex data types and element-wise operations do not poor performance of matrix-matrix operations. We think that the reason of such behavior is that data transfer dominates arithmetic computations.


