\documentclass[conference]{IEEEtran}
\IEEEoverridecommandlockouts
% The preceding line is only needed to identify funding in the first footnote. If that is unneeded, please comment it out.
\usepackage{hyperref}
\usepackage{minted}
\usepackage[table]{xcolor}
\usepackage{multirow}

\usepackage{cite}
\usepackage{amsmath,amssymb,amsfonts}
\usepackage{algorithmic}
\usepackage{graphicx}
\usepackage{textcomp}
\usepackage{xcolor}
\def\BibTeX{{\rm B\kern-.05em{\sc i\kern-.025em b}\kern-.08em
    T\kern-.1667em\lower.7ex\hbox{E}\kern-.125emX}}
\begin{document}

\makeatletter % changes the catcode of @ to 11
\newcommand{\linebreakand}{%
  \end{@IEEEauthorhalign}
  \hfill\mbox{}\par
  \mbox{}\hfill\begin{@IEEEauthorhalign}
}
\makeatother % changes the catcode of @ back to 12


\title{GraphBLAS-like API Design in Functional Style\\
%\thanks{Identify applicable funding agency here. If none, delete this.}
}

\author{\IEEEauthorblockN{Dmitriy Panfilyonok}
\IEEEauthorblockA{\textit{St Petersburg University, }\\                          
                  {St. Petersburg, Russia}\\
                  dmitriy.panfilyonok@gmail.com}
\and
\IEEEauthorblockN{Kirill Garbar}
\IEEEauthorblockA{\textit{St Petersburg University, }\\                          
                  {St. Petersburg, Russia}\\
email@address}
\and
\IEEEauthorblockN{Artyom Chernikov}
\IEEEauthorblockA{\textit{St Petersburg University, }\\                          
                  {St. Petersburg, Russia}\\
email@address}
\and
\IEEEauthorblockN{Arseniy Terekhov}
\IEEEauthorblockA{\textit{St Petersburg University, }\\
                  {St. Petersburg, Russia}\\
email@address}
\and
\linebreakand
\IEEEauthorblockN{Igor Erin}
\IEEEauthorblockA{\textit{St Petersburg University, }\\                          
                  {St. Petersburg, Russia}\\
email@address}
\and
\IEEEauthorblockN{Semyon Grigorev}
\IEEEauthorblockA{\textit{St Petersburg University, }\\                          
                  {St. Petersburg, Russia}\\
s.v.grigoriev@spbu.ru}
}

\maketitle

\begin{abstract}
    GraphBLAS API standard describes linear algebra based blocks to build parallel graph analysis algorithms.
    While it is a promising way to high-performance graph analysis, there are a number of drawbacks such as complicated API, hardness of implementation for GPGPU, and explicit zeroes problem.
    We show that the utilization of techniques from functional programming can help to solve some GraphBLAS design problems.
\end{abstract}

\begin{IEEEkeywords}
graph analysis, sparse linear algebra, GraphBLAS API, GPGPU, parallel programming, functional programming, .NET, OpenCL, FSharp
\end{IEEEkeywords}


\section{Introduction}

Scalable high-performance graph analysis is an actual challenge.
There is a big number of ways to attack this challenge~\cite{Coimbra2021} and the first promising idea is to utilize general-purpose graphic processing units (GPGPU-s).
Such existing solutions, as CuSha~\cite{10.1145/2600212.2600227} and Gunrock~\cite{7967137} show that utilization of GPUs can improve the performance of graph analysis, moreover it is shown that solutions may be scaled to multi-GPU systems.
But low flexibility and high complexity of API are problems of these solutions.

The second promising thing which provides a user-friendly API for high-performance graph analysis algorithms creation is a GraphBLAS API~\cite{7761646} which provides linear algebra based building blocks to create graph analysis algorithms.
The idea of GraphBLAS is based on is a well-known fact that linear algebra operations can be efficiently implemented on parallel hardware.
Along with this, a graph can be natively represented using matrices: adjacency matrix, incidence matrix, etc.
While reference CPU-based implementation of GraphBLAS, SuiteSparse:GraphBLAS~\cite{10.1145/3322125}, demonstrates good performance in real-world tasks, GPU-based implementation is challenging.

One of the challenges in this way is that real data are often sparse, thus underlying matrices and vectors are also sparse, and, as a result, classical dense data structures and respective algorithms are inefficient. 
So, it is necessary to use advanced data structures and procedures to implement sparse linear algebra, but the efficient implementation of them on GPU is hard due to the irregularity of workload and data access patterns.
Though such well-known libraries as cuSparse show that sparse linear algebra operations can be efficiently implemented for GPGPU-s, it is not so trivial to implement GraphBLAS on GPGPU. 
First of all, it requires \textit{generic} sparse linear algebra, thus it is impossible just to reuse existing libraries which are almost all specified for operations over floats.
The second problem is specific optimizations, such as maskings fusion, which can not be natively implemented on top of existing kernels.
Nevertheless, there is a number of implementations of GraphBLAS on GPGPU, such as GraphBLAST:~\cite{yang2019graphblast}, GBTL~\cite{7529957}, which show that GPGPUs utilization can improve the performance of GraphBLAS-based graph analysis solutions.
But these solutions are not portable because they are based on Nvidia Cuda stack.
Moreover, the scalability problem is not solved: all these solutions support only single-GPU, not multi-GPU computations.

To provide portable GPU implementation of GraphBLAS API we developed a \textit{SPLA} library (sources are published on GitHub: \url{https://github.com/JetBrains-Research/spla}).
This library utilizes OpenCL for GPGPU computing to be portable across devices of different vendors.
Moreover, it is initially designed to utilize multiple GPGPUs to be scalable.
To sum up, the contribution of this work is the following.
\begin{itemize}
    \item Design of portable GPU GraphBLAS implementation proposed. The design involves the utilization of multipole GPUS. Additionally, the proposed design is aimed to simplify library tuning and wrappers for different high-level platforms and languages creation. 
    \item Subset of GraphBLAS API, including such operations as masking, matrix-matrix multiplication, matrix-matrix e-wise addition, is implemented. The current implementation is limited by COO and CSR matrix representation format and uses basic algorithms for some operations, but work in progress and more data formats will be supported and advanced algorithms will be implemented in the future.
    \item Preliminary evaluation on such algorithms as breadth-first search (BFS) and triangles counting (TC), and real-world graphs shows portability across different vendors and promising performance: for some problems Spla is comparable with GraphBLAST. Surprisingly, for some problems, the proposed solution on embedded Intel graphic card shows better performance than SuiteSparse:GraphBLAS on the same CPU. At the same time, the evaluation shows that further optimization is required.
\end{itemize} 
%\section{Brahma.FSharp}

Runtime code generation: generics, functions.... 
In comparison to templates (GraphBLAST): compiled library, not header-only library. 
Easy compilation, no additional compile-time dependencies, compilers and so on. 
Cold start problem. 
More flexible, more information can be captured by translator.
Kernels can be cashed. 

Basic OpenCL C: basic control flow, local memory, atomic operations, barriers, memory flags,

Supported F\#-related features: DU, pattern matching, structures, ....

Message-based asynchronious API F\#-native \textbf{Mailbox processor}\footnote{Mailbox processor is a standard primitive to organize message-based asynchronious computations. Official documentation: \url{https://fsharp.github.io/fsharp-core-docs/reference/fsharp-control-fsharpmailboxprocessor-1.html}} primitive.


\section{Design Principles}

In this work we are focused on making development process easier and safer by using !!!. 
Automate optimization.
Accurate type-level encoding of domain: monoids, semirings.

Monoids and semirings are closed under operations. 
Thus, in contrast with GraphBLAS API, all operations in semirings and monoids have the following type: $t \to t \to t$ (instead of $t_1 \to t_2 \to t_3$ as proposed in the GraphBLAS specification).  
It makes our definition less flexible, but allows one to generalize some operations, such as closure of relation.
We realize, that in some cases such restrictive constrains are not required.
Namely, definition of matrix multiplication does not requires a semiring, it just requires two operations $\oplus$ and $\otimes$ with following types: $\otimes: t_1 \to t_2 \to t_3$, $\oplus: t_3 \to t_3 \to t_3 $. 
But formally, a set with such operations is not a semiring.
We think that such case should be investigated separately from semirings, because additional guaranties provided by semirings may be used for code simplification and optimization.
For example, it may help to solve a problem with explicit zeros\footnote{Discussion on zeros removing \url{!!!}. Access date: 12.01.2021. } because we should explicitly specify conversion from one semiring to another if required.  

Matrices and vectors are equipped with monoid or semiring.
Explicit type conversions. 
Can be automatically removed in some cases during translation time.

We propose to generate OpenCL c code in running time as a way to solve problems with generics: with strong typing all type information become known and can be used to generate kernels for specific types.
Moreover, running time code generation is a way to apply advances optimization techniques, such as partial evaluation (or code specialization), which can improve performance of generated code when part of input parameter of kernel becomes known prior its generation~\cite{10.1145/3332466.3374507}.

The example of \texttt{Min-Plus} semiring definition is provided in listing~\ref{lst_example}.
Type is defined using discriminated unions (line 1): new set can contains both floats, marked with  \texttt{R} and a special value \texttt{Infinity}.
Thus floats are extended with infinity as required for accurate definition of \texttt{Min-Plus} semiring.
Semiring definition (lines 3--21) includes definition of zero (idenity), operations $\oplus$ (lines 8--13) and $\otimes$ (lines 14--19), !!!!


\begin{listing}
\begin{minted}
[
frame=lines,
fontsize=\footnotesize,
framesep=2mm,
linenos,
autogobble,
numbersep=4pt
]
{fsharp}
type RInfinity = R of float | Infinity

[<Struct>]
type MinPlusSemiring = 
   MinPlusSemiring of RInfinity
with
   static member Zero = MinPlusSemiring Infinity
   static member (+) 
      (MinPlusSemiring x, MinPlusSemiring y) = 
          match x, y with
          | R x, R y -> System.Math.Min(x,y) |> R
          | _        -> Infinity 
          |> MinPlusSemiring
   static member (*) 
      (MinPlusSemiring x, MinPlusSemiring y) = 
          match x, y with
          | R x, R y -> x + y |> R
          | _        -> Infinity 
          |> MinPlusSemiring
   static member op_Implicit (MinPlusSemiring src) = 
      src
\end{minted}
\caption{Example om \texttt{Min-Plus} semiring definition}
\label{lst_example}
\end{listing}
%\section{Implementation Details}

Details on implementation. 

Architecture.
\section{Evaluation}

For performance analysis of proposed solution we evaluated some most common graph algorithms using real-world sparse matrix data. 
As a baseline for comparison we chose LAGraph~\cite{szarnyas2021lagraph} in connection with SuiteSparse~\cite{10.1145/3322125} as a CPU tool, Gunrock~\cite{7967137} and GraphBLAST~\cite{yang2019graphblast} as a Nvidia GPU tools. 
Also, we tested algorithms on several devices with distinct OpenCL vendors in order to validate portability of the proposed solution. 
In general, these evaluation intentions are summarized in the following research questions. 

\vspace{0.2cm}
\begin{itemize}
    \item[\textbf{RQ1}] What is the performance of the proposed solution relative to existing tools for both CPU and GPU analysis?
    
    \item[\textbf{RQ2}] What is the portability of the proposed solution with respect to various device vendors and OpenCL runtimes?
\end{itemize}

\subsection{Evaluation Setup}

For evaluation, we use a PC with Ubuntu 20.04 installed, which has 3.40Hz Intel Core i7-6700 4-core CPU, DDR4 64Gb RAM, and Nvidia GeForce GTX 1070 GPU with 8Gb VRAM. 
Host programs were compiled with GCC 9.3.0 compiler. Programs using CUDA were compiled with GCC 8.4.0 and Nvidia NVCC 10.1.243 compiler.
Release mode and maximum optimization level was enabled for all tested programs. 
Data loading time, preparation, format transformations and host-device initial communications are excluded from time measurements. 
All tests are averaged across 10 runs.
Additional warm-up run for each test execution is excluded from measurements.

\subsection{Graph Algorithms}

For preliminary study \textit{breadth-first search} (bfs) and \textit{triangles counting} (tc) algorithms were chosen, since they allows analyse the performance of \textit{vxm} and \textit{mxm} operations, rely heavily on \textit{masking}, and utilize \textit{reduction} or \textit{assignment}. 
BFS implementation utilizes automated vector storage from sparse to dense switch and only \textit{}{push optimization}. 
TC implementation uses masked \textit{mxm} of source lower-triangular matrix with second transposed argument.

\subsection{Dataset}

Nine graph matrices were selected from the Sparse Matrix Collection at University of Florida~\cite{dataset:10.1145/2049662.2049663}. 
Information about graphs is summarized in Table~\ref{dataset:info}. 
All datasets are converted to undirected graphs. 
Self-loops and duplicated edges are removed.

\begin{table}[htbp]
\caption{Dataset description.} 
\begin{center}
    \rowcolors{2}{black!2}{black!10}
    \begin{tabular}{|l|r|r|r|}
    \hline
    Dataset & Vertices  & Edges & Max Degree \\
    \hline
    \hline
    coAuthorsCiteseer & 227.3K &   1.6M &    1372 \\
    coPapersDBLP      & 540.4K &  30.4M &    3299 \\
    hollywood-2009    &   1.1M & 113.8M &  11,467 \\
    roadNet-CA        &   1.9M &   5.5M &      12 \\
    com-Orkut         &     3M &   234M &   33313 \\
    cit-Patents       &   3.7M &  16.5M &     793 \\
    rgg\_n\_2\_22\_s0 &   4.1M &  60.7M &      36 \\
    soc-LiveJournal   &   4.8M &  68.9M &  20,333 \\
    indochina-2004    &   7.5M & 194.1M & 256,425 \\
    \hline
    \end{tabular}
    \label{dataset:info}
\end{center}
\end{table}

\subsection{Results}

Table~\ref{results} presents results of the evaluation and compares performance of Spla against other tool on different execution platforms.
Tools are grouped by the type of the device for the execution, where either Nvidia GPU or Intel CPU are used. 
Cell left empty if tested tool failed to analyse graph due to \textit{out of memory} exception.

In general, Spla BFS shows acceptable performance, especially on graphs with large vertex degree, such as soc-LiveJournal and com-Orkut.
On graphs roadNet-CA and rgg it has a significant performance drop due to the nature of underlying algorithms and data structures. 
Firstly, library utilizes immutable data buffers. Thus, iteratively updated dense vector of reached vertices must be copied for each modification, what dominates the performance of the library on a graph with large search depth. 
Secondly, Spla BFS does not utilise \textit{pull optimization}, what is critical in a graph with relatively small search frontier. 

Spla TC has a good performance on GPU, which is better in all cases that reference SuiteSparse solution. 
But in most tests GPU competitors, especially Gunrock, show smaller processing times. 
GraphBLAST shows better performance as well. 
Library utilises masked SpGEMM algorithm, the same as in GraphBLAST, but without \textit{identity} element to fill gaps. 
Library explicitly stores all non-zero elements, and uses mask to reduce only non-zero while evaluating dot products of rows and columns. 
What causes extra divergence inside work groups. 
On Intel device Spla shows better performance compared to SuiteSparse on com-Orkut, cit-Patents and soc-LiveJournal. 
A possible reason is the large lengths of processed rows and columns in the product of matrices.

Gunrock shows nearly best average performance due to its specialized and optimized algorithms.
Also, it has good time characteristics on a mentioned earlier roadNet-CA and rgg in BFS algortihm. 
GraphBLAST follows Gunrock and show good performance as well. 
But it runs out of memory on a two significantly large graphs con-Orkut and indochina-2004. 
Spla does not rut out of memory on any test due to simplified storage scheme.

\begin{table}[htbp]
\caption{Graph algorithms evaluation results.\\Time in milliseconds (lower is better).} 
\begin{center}
    \begin{tabular}{|l|r|r|r|r|r|}
    \hline
    \multirow{2}{*}{Dataset} & \multicolumn{3}{c|}{Nvidia} & \multicolumn{2}{c|}{Intel} \\
    \cline{2-6}
    & GR & GB & SP & SS & SP \\
    \hline
    \hline
    \multicolumn{6}{|c|}{BFS} \\
    \hline
    \rowcolor{black!10} hollywood-2009    &  20.3 &  82.3 &   36.9 &   23.7 &   303.4 \\
    \rowcolor{black!2 } roadNet-CA        &  33.4 & 130.8 & 1456.4 &  168.2 &   965.6 \\
    \rowcolor{black!10} soc-LiveJournal   &  60.9 &  80.6 &   90.6 &   75.2 &  1206.3 \\
    \rowcolor{black!2 } rgg\_n\_2\_22\_s0 &  98.7 & 414.9 & 4504.3 & 1215.7 & 15630.1 \\
    \rowcolor{black!10} com-Orkut         & 205.2 & -- -- &  117.9 &   43.2 &   903.6 \\
    \rowcolor{black!2 } indochina-2004    &  32.7 & -- -- &  199.6 &  227.1 &  2704.6 \\
    \hline
    \hline
    \multicolumn{6}{|c|}{TC} \\
    \hline
    \rowcolor{black!10} coAuthorsCiteseer &   2.1 &    2.0 &    9.5 &    17.5 &    64.9 \\
    \rowcolor{black!2 } coPapersDBLP      &   5.7 &   94.4 &  201.9 &   543.1 &  1537.8 \\
    \rowcolor{black!10} roadNet-CA        &  34.3 &    5.8 &   16.1 &    47.1 &   357.6 \\
    \rowcolor{black!2 } com-Orkut         & 218.1 & 1583.8 & 2407.4 & 23731.4 & 15049.5 \\
    \rowcolor{black!10} cit-Patents       &  49.7 &   52.9 &   90.6 &   698.3 &   684.1 \\
    \rowcolor{black!2 } soc-LiveJournal   &  69.1 &  449.6 &  673.9 &  4002.6 &  3823.9 \\
    \hline
    \hline
    \multicolumn{6}{l}{Tools: Gunrock (GR), GraphBLAST (GB), SuiteSparse (SS), Spla (SP).} \\
    \end{tabular}
    \label{results}
\end{center}
\end{table}
 
% Two GPU

% \begin{table}[htbp]
%     \caption{Table Type Styles}
%     \begin{center}
%     \begin{tabular}{|c|c|c|c|}
%     \hline
%     \textbf{Table}&\multicolumn{3}{|c|}{\textbf{Table Column Head}} \\
%     \cline{2-4} 
%     \textbf{Head} & \textbf{\textit{Table column subhead}}& \textbf{\textit{Subhead}}& \textbf{\textit{Subhead}} \\
%     \hline
%     copy& More table copy$^{\mathrm{a}}$& &  \\
%     \hline
%     \multicolumn{4}{l}{$^{\mathrm{a}}$Sample of a Table footnote.}
%     \end{tabular}
%     \label{tab2}
%     \end{center}
% \end{table}

\section{Conclusion}

In this paper we present a library for sparse Boolean linear algebra which implements such basic operations as matrix-matrix multiplication and element-wise matrix-matrix addition in both Cuda and OpenCL.
Evaluation shows that our Boolean-specific implementations faster and require less memory than generic, not the Boolean optimized, operations from state-of-the-art libraries. 
Thus, the specialization of operations for this data type makes sense. 

The first direction of the future work is to integrate all parts (OpenCL and Cuda backends) into a single library and improve its documentation and prepare to publish.
Moreover, it is necessary to extend the library with other operations, including matrix-vector operations, masking, and so on.
As a result a Python package should be published.

Another important step is to evaluate the library on different algorithms and devices.
Namely, algorithms for RPQ and CFPQ should be implemented and evaluated on related data sets.
Also, it is necessary to evaluate OpenCL version on FPGA which may require additional technical effort and code changes.

Finally, we plan to discuss with GraphBLAS community possible ways to use our library as a backend for GraphBLAST or SuiteSparse in case of Boolean computations.
Moreover, it may be possible to use implemented algorithms as a foundation for generalization to arbitrary semirings.


%\section*{Acknowledgment}
%
%The preferred spelling of the word ``acknowledgment'' in America is without 
%an ``e'' after the ``g''. Avoid the stilted expression ``one of us (R. B. 
%G.) thanks $\ldots$''. Instead, try ``R. B. G. thanks$\ldots$''. Put sponsor 
%acknowledgments in the unnumbered footnote on the first page.


\bibliographystyle{./IEEEtran}
\bibliography{./GraphBLAS_in_functional_style}

\end{document}
