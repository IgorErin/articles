\section{Experimental evaluation}
\textbf{Наша цель здесь это представить некоторый стандарт того, как можно проводить эксперименты.}
Стоит описать, стандартный путь проведения эксперимента и классические базовые показатели извлекаемые из работы алгоритма.

\subsection{Datasets}
\textbf{Здесь нужно описать какие графы мы взяли для проведения СВОЕГО эксперимента.}
Мы взяли RDF потому что это классика. 
Взяли MemoryAliases потому что они большие. 
Взяли WorstCase потому что это важный случай в теории.

\subsection{Algorithms}
\textbf{Здесь нужно описать какие мы алгоритмы использовали в СВОЕМ эксперименте.}
Думаю достаточно взять пару матричных алгоритмов и тензорный.

\subsection{Results and discussion}
\textbf{Здесь мы суммируем полученные результаты.}
Как-то описываем полученные результаты.
Возможно сравниваем их с полученными в других статьях.
Можно написать насколько это было просто провести эксперимент в имеющейся инфраструктуре.
Если код эксперимента маленький, то можно его вставить.

\begin{table}
  \caption{Таблица с результатами эксперимента}
  \label{tab:results}
  \begin{tabular}{cccc}
    \toprule
    Graph & Grammar & Algorithm & Time\\
    \midrule
    RDF&$G_1$&Tensor&$0.001$\\
    RDF&$G_1$&Tensor&$0.001$\\
    RDF&$G_1$&Tensor&$0.001$\\
    RDF&$G_1$&Tensor&$0.001$\\
  \bottomrule
\end{tabular}
\end{table}
