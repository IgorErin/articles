%%
%% This is file `sample-sigconf.tex',
%% generated with the docstrip utility.
%%
%% The original source files were:
%%
%% samples.dtx  (with options: `sigconf')
%% 
%% IMPORTANT NOTICE:
%% 
%% For the copyright see the source file.
%% 
%% Any modified versions of this file must be renamed
%% with new filenames distinct from sample-sigconf.tex.
%% 
%% For distribution of the original source see the terms
%% for copying and modification in the file samples.dtx.
%% 
%% This generated file may be distributed as long as the
%% original source files, as listed above, are part of the
%% same distribution. (The sources need not necessarily be
%% in the same archive or directory.)
%%
%% The first command in your LaTeX source must be the \documentclass command.
\documentclass[sigconf]{acmart}

%Russian-specific packages
%--------------------------------------
\usepackage[T2A]{fontenc}
\usepackage[utf8]{inputenc}
\usepackage[russian]{babel}
%--------------------------------------
 
%Hyphenation rules
%--------------------------------------
\usepackage{hyphenat}
\hyphenation{ма-те-ма-ти-ка вос-ста-нав-ли-вать}
%--------------------------------------

%% NOTE that a single column version may be required for 
%% submission and peer review. This can be done by changing
%% the \doucmentclass[...]{acmart} in this template to 
%% \documentclass[manuscript,screen]{acmart}
%% 
%% To ensure 100% compatibility, please check the white list of
%% approved LaTeX packages to be used with the Master Article Template at
%% https://www.acm.org/publications/taps/whitelist-of-latex-packages 
%% before creating your document. The white list page provides 
%% information on how to submit additional LaTeX packages for 
%% review and adoption.
%% Fonts used in the template cannot be substituted; margin 
%% adjustments are not allowed.
%%
%%
%% \BibTeX command to typeset BibTeX logo in the docs
\AtBeginDocument{%
  \providecommand\BibTeX{{%
    \normalfont B\kern-0.5em{\scshape i\kern-0.25em b}\kern-0.8em\TeX}}}

%% Rights management information.  This information is sent to you
%% when you complete the rights form.  These commands have SAMPLE
%% values in them; it is your responsibility as an author to replace
%% the commands and values with those provided to you when you
%% complete the rights form.
\setcopyright{acmcopyright}
\copyrightyear{2018}
\acmYear{2018}
\acmDOI{10.1145/1122445.1122456}

%% These commands are for a PROCEEDINGS abstract or paper.
% \acmConference[Woodstock '18]{Woodstock '18: ACM Symposium on Neural
%   Gaze Detection}{June 03--05, 2018}{Woodstock, NY}
% \acmBooktitle{Woodstock '18: ACM Symposium on Neural Gaze Detection,
%   June 03--05, 2018, Woodstock, NY}
% \acmPrice{15.00}
% \acmISBN{978-1-4503-XXXX-X/18/06}


%%
%% Submission ID.
%% Use this when submitting an article to a sponsored event. You'll
%% receive a unique submission ID from the organizers
%% of the event, and this ID should be used as the parameter to this command.
%%\acmSubmissionID{123-A56-BU3}

%%
%% The majority of ACM publications use numbered citations and
%% references.  The command \citestyle{authoryear} switches to the
%% "author year" style.
%%
%% If you are preparing content for an event
%% sponsored by ACM SIGGRAPH, you must use the "author year" style of
%% citations and references.
%% Uncommenting
%% the next command will enable that style.
%%\citestyle{acmauthoryear}

%%
%% end of the preamble, start of the body of the document source.
\begin{document}

%%
%% The "title" command has an optional parameter,
%% allowing the author to define a "short title" to be used in page headers.
\title{Super Duper Dataset for Experimental Analysis of CFPQ Algorithms}

%%
%% The "author" command and its associated commands are used to define
%% the authors and their affiliations.
%% Of note is the shared affiliation of the first two authors, and the
%% "authornote" and "authornotemark" commands
%% used to denote shared contribution to the research.
\author{Author 1}
\authornote{Both authors contributed equally to this research.}
\email{author_1@samplemail.com}
\orcid{1234-5678-9012}
\author{Author 2}
\authornotemark[1]
\email{author_2@samplemail.com}
\affiliation{%
  \institution{institution}
  \streetaddress{streetaddress}
  \city{city}
  \state{state}
  \country{country}
  \postcode{43017-6221}
}

\author{Author 3}
\email{author_3@samplemail.com}
\orcid{1234-5678-9012}
\affiliation{%
  \institution{institution}
  \streetaddress{streetaddress}
  \city{city}
  \state{state}
  \country{country}
  \postcode{43017-6221}
}

%%
%% By default, the full list of authors will be used in the page
%% headers. Often, this list is too long, and will overlap
%% other information printed in the page headers. This command allows
%% the author to define a more concise list
%% of authors' names for this purpose.
% \renewcommand{\shortauthors}{Trovato and Tobin, et al.}
%%
%% The abstract is a short summary of the work to be presented in the
%% article.
\begin{abstract}
Recently, there has been an increasing interest in solving problems related to context-free path queries (CFPQ) on graphs.
However, the development of meaningful benchmark datasets and standardized evaluation procedures is lagging, consequently hindering advancements in this area.
To solve this, we introduce the CFPQ\_Data dataset, which contains the most popular graphs for experimental analysis of CFPQ algorithms.
The collection consists of over 40 graphs of varying sizes.
We provide Python-based data loaders and implementations of the well-known CFPQ algorithms.
Here, we give an overview of the CFPQ\_Data dataset, standardized evaluation procedures, and provide baseline experiments.
All datasets are available at \href{https://github.com/JetBrains-Research/CFPQ_Data}{https://github.com/JetBrains-Research/CFPQ\_Data}.
The experiments are fully reproducible from the code available at \href{https://github.com/JetBrains-Research/CFPQ_PyAlgo}{https://github.com/JetBrains-Research/CFPQ\_PyAlgo}.
\end{abstract}

%%
%% The code below is generated by the tool at http://dl.acm.org/ccs.cfm.
%% Please copy and paste the code instead of the example below.
%%
% \begin{CCSXML}
% <ccs2012>
%  <concept>
%   <concept_id>10010520.10010553.10010562</concept_id>
%   <concept_desc>Computer systems organization~Embedded systems</concept_desc>
%   <concept_significance>500</concept_significance>
%  </concept>
%  <concept>
%   <concept_id>10010520.10010575.10010755</concept_id>
%   <concept_desc>Computer systems organization~Redundancy</concept_desc>
%   <concept_significance>300</concept_significance>
%  </concept>
%  <concept>
%   <concept_id>10010520.10010553.10010554</concept_id>
%   <concept_desc>Computer systems organization~Robotics</concept_desc>
%   <concept_significance>100</concept_significance>
%  </concept>
%  <concept>
%   <concept_id>10003033.10003083.10003095</concept_id>
%   <concept_desc>Networks~Network reliability</concept_desc>
%   <concept_significance>100</concept_significance>
%  </concept>
% </ccs2012>
% \end{CCSXML}

% \ccsdesc[500]{Computer systems organization~Embedded systems}
% \ccsdesc[300]{Computer systems organization~Redundancy}
% \ccsdesc{Computer systems organization~Robotics}
% \ccsdesc[100]{Networks~Network reliability}

%%
%% Keywords. The author(s) should pick words that accurately describe
%% the work being presented. Separate the keywords with commas.
% \keywords{datasets, neural networks, gaze detection, text tagging}

%% A "teaser" image appears between the author and affiliation
%% information and the body of the document, and typically spans the
%% page.
% \begin{teaserfigure}
%   \includegraphics[width=\textwidth]{sampleteaser}
%   \caption{Seattle Mariners at Spring Training, 2010.}
%   \Description{Enjoying the baseball game from the third-base
%   seats. Ichiro Suzuki preparing to bat.}
%   \label{fig:teaser}
% \end{teaserfigure}

%%
%% This command processes the author and affiliation and title
%% information and builds the first part of the formatted document.
\maketitle

\section{Introduction}

Scalable high-performance graph analysis is an actual challenge.
There is a big number of ways to attack this challenge~\cite{Coimbra2021} and the first promising idea is to utilize general-purpose graphic processing units (GPGPU-s).
Such existing solutions, as CuSha~\cite{10.1145/2600212.2600227} and Gunrock~\cite{7967137} show that utilization of GPUs can improve the performance of graph analysis, moreover it is shown that solutions may be scaled to multi-GPU systems.
But low flexibility and high complexity of API are problems of these solutions.

The second promising thing which provides a user-friendly API for high-performance graph analysis algorithms creation is a GraphBLAS API~\cite{7761646} which provides linear algebra based building blocks to create graph analysis algorithms.
The idea of GraphBLAS is based on is a well-known fact that linear algebra operations can be efficiently implemented on parallel hardware.
Along with this, a graph can be natively represented using matrices: adjacency matrix, incidence matrix, etc.
While reference CPU-based implementation of GraphBLAS, SuiteSparse:GraphBLAS~\cite{10.1145/3322125}, demonstrates good performance in real-world tasks, GPU-based implementation is challenging.

One of the challenges in this way is that real data are often sparse, thus underlying matrices and vectors are also sparse, and, as a result, classical dense data structures and respective algorithms are inefficient. 
So, it is necessary to use advanced data structures and procedures to implement sparse linear algebra, but the efficient implementation of them on GPU is hard due to the irregularity of workload and data access patterns.
Though such well-known libraries as cuSparse show that sparse linear algebra operations can be efficiently implemented for GPGPU-s, it is not so trivial to implement GraphBLAS on GPGPU. 
First of all, it requires \textit{generic} sparse linear algebra, thus it is impossible just to reuse existing libraries which are almost all specified for operations over floats.
The second problem is specific optimizations, such as maskings fusion, which can not be natively implemented on top of existing kernels.
Nevertheless, there is a number of implementations of GraphBLAS on GPGPU, such as GraphBLAST:~\cite{yang2019graphblast}, GBTL~\cite{7529957}, which show that GPGPUs utilization can improve the performance of GraphBLAS-based graph analysis solutions.
But these solutions are not portable because they are based on Nvidia Cuda stack.
Moreover, the scalability problem is not solved: all these solutions support only single-GPU, not multi-GPU computations.

To provide portable GPU implementation of GraphBLAS API we developed a \textit{SPLA} library (sources are published on GitHub: \url{https://github.com/JetBrains-Research/spla}).
This library utilizes OpenCL for GPGPU computing to be portable across devices of different vendors.
Moreover, it is initially designed to utilize multiple GPGPUs to be scalable.
To sum up, the contribution of this work is the following.
\begin{itemize}
    \item Design of portable GPU GraphBLAS implementation proposed. The design involves the utilization of multipole GPUS. Additionally, the proposed design is aimed to simplify library tuning and wrappers for different high-level platforms and languages creation. 
    \item Subset of GraphBLAS API, including such operations as masking, matrix-matrix multiplication, matrix-matrix e-wise addition, is implemented. The current implementation is limited by COO and CSR matrix representation format and uses basic algorithms for some operations, but work in progress and more data formats will be supported and advanced algorithms will be implemented in the future.
    \item Preliminary evaluation on such algorithms as breadth-first search (BFS) and triangles counting (TC), and real-world graphs shows portability across different vendors and promising performance: for some problems Spla is comparable with GraphBLAST. Surprisingly, for some problems, the proposed solution on embedded Intel graphic card shows better performance than SuiteSparse:GraphBLAS on the same CPU. At the same time, the evaluation shows that further optimization is required.
\end{itemize} 
\section{The CFPQ\_Data collection}
\textbf{Краткое описание того, что собрано.}
Коллекция состоит из более чем 40 графов разного размера.
Также мы предоставляем загрузчики данных и реализации наиболее популярных алгоритмов на основе Python, а также стандарт проведения экспериментов и базовых показателей работы алгоритма.
\textbf{Возможно сказать про листинг с примером кода.}
Например выложить на сайт питоновский ноутбук с примером применения.
Подробное описание каждого графа и другую документацию вы сможете найти на сайте коллекции.

\subsection{Graphs and grammars}
\textbf{Описываем коллекцию.}
Наша коллекция наборов данных охватывает графы из разных областей, предоставленные разными авторами. 
Здесь мы даем обзор некоторых репрезентативных областей и моделей графов.

\textbf{RDF.}
\textbf{Рассказать откуда взялись RDF.}
Small graphs is a set of popular semantic web ontologies. This set is introduced by Xiaowang Zhang in "Context-Free Path Queries on RDF Graphs".

\textbf{MemoryAliases.}
\textbf{Рассказать откуда взялись MemoryAliases.}
Потом еще про MemoryAliases что-то вроде MemoryAliases — real-world data for points-to analysis of C code.
First part is a dataset form Graspan tool. The original data is placed here. This part is placed in Graspan folder.
Second part is a part of dataset form "Demand-driven alias analysis for C". This part is placed in small folder.

\textbf{LUBM.}
\textbf{Рассказать почему отдельно выделили LUBM.}
LUBM - the Lehigh University Benchmark graphs.

\textbf{Synthetic.}
\textbf{Рассказать про важность синтетических графов.}
WorstCase — graphs with two cylces; the query Brackets is a grammar for the language of correct bracket sequences.
SparseGraph — graphs generated with NetworkX to emulate sparse data. The grammar provided is a variant of the same-generation query.
ScaleFree — graphs generated by using the Barab'asi-Albert model of scale-free networks. Use with grammar ** an\_bm\_cm\_dn**, which is a query for AnBmCmDn language.
FullGraph — cycle graphs, all edges are labeled with the same token. Use with A\_star queries, which produce full graph on that dataset.

\subsection{Baselines methods (GraphBuilders, GraphLoaders & GraphSavers)}
\textbf{Здесь описываем как реализовали загрузку и работу с графами и почему так.}
Любой граф их датасета можно либо построить (если он синтетический) либо загрузить с сайта коллекции (или по пути к имеющемуся в системе графу).
А с помощью GraphSaver'а можно любой граф сохранить в нужном исследователю виде.
\textbf{Возможно стоит упомянуть про грамматики.}

\subsection{Evaluation methods (Evaluators)}
\textbf{Тут надо описать каким образом происходит запуск алгоритма на графе и грамматике из коллекции.}
Возможно описать как это выглядит технически.
А также как можно написать свой алгоритм в имеющейся инфраструктуре.

\section{Experimental evaluation}
\textbf{Наша цель здесь это представить некоторый стандарт того, как можно проводить эксперименты.}
Стоит описать, стандартный путь проведения эксперимента и классические базовые показатели извлекаемые из работы алгоритма.

\subsection{Datasets}
\textbf{Здесь нужно описать какие графы мы взяли для проведения СВОЕГО эксперимента.}
Мы взяли RDF потому что это классика. 
Взяли MemoryAliases потому что они большие. 
Взяли WorstCase потому что это важный случай в теории.

\subsection{Algorithms}
\textbf{Здесь нужно описать какие мы алгоритмы использовали в СВОЕМ эксперименте.}
Думаю достаточно взять пару матричных алгоритмов и тензорный.

\subsection{Results and discussion}
\textbf{Здесь мы суммируем полученные результаты.}
Как-то описываем полученные результаты.
Возможно сравниваем их с полученными в других статьях.
Можно написать насколько это было просто провести эксперимент в имеющейся инфраструктуре.
Если код эксперимента маленький, то можно его вставить.

\begin{table}
  \caption{Таблица с результатами эксперимента}
  \label{tab:results}
  \begin{tabular}{cccc}
    \toprule
    Graph & Grammar & Algorithm & Time\\
    \midrule
    RDF&$G_1$&Tensor&$0.001$\\
    RDF&$G_1$&Tensor&$0.001$\\
    RDF&$G_1$&Tensor&$0.001$\\
    RDF&$G_1$&Tensor&$0.001$\\
  \bottomrule
\end{tabular}
\end{table}

\section{Conclusion}

In this paper we present a library for sparse Boolean linear algebra which implements such basic operations as matrix-matrix multiplication and element-wise matrix-matrix addition in both Cuda and OpenCL.
Evaluation shows that our Boolean-specific implementations faster and require less memory than generic, not the Boolean optimized, operations from state-of-the-art libraries. 
Thus, the specialization of operations for this data type makes sense. 

The first direction of the future work is to integrate all parts (OpenCL and Cuda backends) into a single library and improve its documentation and prepare to publish.
Moreover, it is necessary to extend the library with other operations, including matrix-vector operations, masking, and so on.
As a result a Python package should be published.

Another important step is to evaluate the library on different algorithms and devices.
Namely, algorithms for RPQ and CFPQ should be implemented and evaluated on related data sets.
Also, it is necessary to evaluate OpenCL version on FPGA which may require additional technical effort and code changes.

Finally, we plan to discuss with GraphBLAS community possible ways to use our library as a backend for GraphBLAST or SuiteSparse in case of Boolean computations.
Moreover, it may be possible to use implemented algorithms as a foundation for generalization to arbitrary semirings.

\section{Acknowledgments}
\textbf{Мы очень благодарны всем, кто принимал участие в этой нелегкой работе.}

%%
%% The acknowledgments section is defined using the "acks" environment
%% (and NOT an unnumbered section). This ensures the proper
%% identification of the section in the article metadata, and the
%% consistent spelling of the heading.
% \begin{acks}
% 1 We thank everybody who provided datasets for the CFPQ\_Data collection.
% 2 We thank everybody who provided datasets for the CFPQ\_Data collection.
% 3 We thank everybody who provided datasets for the CFPQ\_Data collection.
% 4 We thank everybody who provided datasets for the CFPQ\_Data collection.
% \end{acks}

%%
%% If your work has an appendix, this is the place to put it.
% \appendix


%%
%% The next two lines define the bibliography style to be used, and
%% the bibliography file.
\bibliographystyle{ACM-Reference-Format}
\bibliography{paper}

\end{document}
\endinput
%%
%% End of file `sample-sigconf.tex'.
