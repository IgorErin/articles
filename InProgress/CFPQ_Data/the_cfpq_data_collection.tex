\section{The CFPQ\_Data collection}
\textbf{Краткое описание того, что собрано.}
Коллекция состоит из более чем 40 графов разного размера.
Также мы предоставляем загрузчики данных и реализации наиболее популярных алгоритмов на основе Python, а также стандарт проведения экспериментов и базовых показателей работы алгоритма.
\textbf{Возможно сказать про листинг с примером кода.}
Например выложить на сайт питоновский ноутбук с примером применения.
Подробное описание каждого графа и другую документацию вы сможете найти на сайте коллекции.

\subsection{Graphs and grammars}
\textbf{Описываем коллекцию.}
Наша коллекция наборов данных охватывает графы из разных областей, предоставленные разными авторами. 
Здесь мы даем обзор некоторых репрезентативных областей и моделей графов.

\textbf{RDF.}
\textbf{Рассказать откуда взялись RDF.}
Small graphs is a set of popular semantic web ontologies. This set is introduced by Xiaowang Zhang in "Context-Free Path Queries on RDF Graphs".

\textbf{MemoryAliases.}
\textbf{Рассказать откуда взялись MemoryAliases.}
Потом еще про MemoryAliases что-то вроде MemoryAliases — real-world data for points-to analysis of C code.
First part is a dataset form Graspan tool. The original data is placed here. This part is placed in Graspan folder.
Second part is a part of dataset form "Demand-driven alias analysis for C". This part is placed in small folder.

\textbf{LUBM.}
\textbf{Рассказать почему отдельно выделили LUBM.}
LUBM - the Lehigh University Benchmark graphs.

\textbf{Synthetic.}
\textbf{Рассказать про важность синтетических графов.}
WorstCase — graphs with two cylces; the query Brackets is a grammar for the language of correct bracket sequences.
SparseGraph — graphs generated with NetworkX to emulate sparse data. The grammar provided is a variant of the same-generation query.
ScaleFree — graphs generated by using the Barab'asi-Albert model of scale-free networks. Use with grammar ** an\_bm\_cm\_dn**, which is a query for AnBmCmDn language.
FullGraph — cycle graphs, all edges are labeled with the same token. Use with A\_star queries, which produce full graph on that dataset.

\subsection{Baselines methods (GraphBuilders, GraphLoaders & GraphSavers)}
\textbf{Здесь описываем как реализовали загрузку и работу с графами и почему так.}
Любой граф их датасета можно либо построить (если он синтетический) либо загрузить с сайта коллекции (или по пути к имеющемуся в системе графу).
А с помощью GraphSaver'а можно любой граф сохранить в нужном исследователю виде.
\textbf{Возможно стоит упомянуть про грамматики.}

\subsection{Evaluation methods (Evaluators)}
\textbf{Тут надо описать каким образом происходит запуск алгоритма на графе и грамматике из коллекции.}
Возможно описать как это выглядит технически.
А также как можно написать свой алгоритм в имеющейся инфраструктуре.
