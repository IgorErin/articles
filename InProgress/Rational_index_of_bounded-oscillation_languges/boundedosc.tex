\section{Rational index of bounded-oscillation languages}
\label{sec:osc}
\subsection{Upper bounds on the rational index of bounded oscillation languages}
Before we consider the value of the rational index for $k$-bounded-oscillation languages, we need to prove the following.
\begin{lemma}
\label{lem:treeheight}
Let  $G = (\Sigma, N, P, S)$ be a context-free grammar in Chomsky normal form,  $D=(V, E, \Sigma)$ be a directed labeled graph with $n$ nodes. Let $w$ be the shortest string in $L(G)\cap L(D)$. Then the height of every parse tree for $w$ in $G$ does not exceed $|N|n^2$.
\end{lemma}

\begin{proof}
Consider grammar $G'$ for $L(G)\cap L(D)$. The grammar $G = (\Sigma, N', P', S')$ can be constructed from $G$ using the classical Bar-Hillel et al.~\cite{BarHillel} construction: $N' \subseteq N \times V \times V $  contains all tiples $(A, i, j)$ such that $A \in N, i, j \in V$ ; $P'$ contains production rules in one of the following forms:
\begin{enumerate}
\item $(A, i, j) \rightarrow (B, i, k), (C, k, j)$ for all $(i, k, j)$ in $V$  if $A \rightarrow BC \in P$
\item $(A, i, j) \rightarrow a$ for all $(i, j)$ in $V$ if $A \rightarrow a$.
\end{enumerate}
A triple $(A, i, j)$ is \textit{realizable} if and only if there is a path $i\pi j$ such that $A \stackrel {*}{\Rightarrow } l(\pi)$ for some nontermimal $A \in N$. Then the parse tree $t_G$ for $w$ in $G$ can be converted into parse tree $t_{G'}$ in $G'$. Notice that every node of $t_{G'}$ is realizable triple. Also it is easy to see that the height of $t_G$ is equal to the height of $t_{G'}$. Assume that $t_{G'}$ for $w$ has a height of more than $|N|n^2$. Consider a path from the root of the parse tree to a leaf, which has length greater than $|N|n^2$. There are $|N|n^2$ unique labels $(A, i, j)$ for nodes of the parse tree, so according to the pigeonhole principle, this path has at least two nodes with the same label. This means that the parse tree for $w$ contains at least one subtree $t$ with label $(A, i, j)$ at the root, which has a subtree $t'$ with the same label. Then we can change $t$ with $t'$ and get a new string $w'$ which is shorter than $w$, because the grammar is in Chomsky normal form. But $w$ is the shortest, then we have a contradiction.

\end{proof}
From Lemma \ref{lem:treeheight} one can deduce an alternative proof of the fact that the rational index of linear languages is in $O(n^2)$~\cite{RatBasic}: the number of leaves in a parse tree in linear grammar in Chomsky normal form is proportional to its height, and thus it is in $O(n^2)$.
\begin{lemma}
\label{oscbnddim}
Let $G$ be a grammar $G = (\Sigma, N, P, S)$ in Chomsky normal form, such that every parse tree $t$ has $dim(t) \le d$, where $d$ is some constant. Let $D=(V, E, \Sigma)$ be a directed labeled graph with $n$ nodes. Then $\rho_{L(G)}$ is in $O(h^d)$ in the worst case.
\end{lemma}
\begin{proof}
Proof by induction on dimension $dim(t)$.
\begin{enumerate} 
\item \textbf{Basis.} $dim(t) = 1$. Consider a tree $t$ with the dimension $dim(t) = 1$. The root of the tree has the same dimension and has two children (because the grammar is in Chomsky normal form). There are two cases:  first, when both of child nodes have dimension equal to 0, then the tree has only two leaves, and second, when one of the children has dimension 1, and the second child has dimension 0. For the second case we can recursively construct a tree with the maximum number of leaves in the following way. Every internal node of such a tree has two children, one of which has dimension equal to 0 and therefore has only one leaf. This means that the number of leaves (and, hence, $\rho_{L(G)}$) in such a tree is bounded by its height and is in $O(h)$. 

\item \textbf{Inductive step.} $dim(t) = d + 1$. Assume that $\rho_{L(G)}$ is at most $O(h^{d})$ for every $d$ in the worst case, where $h$ is the height of the tree. We have two cases for the root node with dimension equal to $d+1$: 1) both of children have a dimension equal to $d$, then by proposition the tree of height $h$ has no more than $O(h^{d})$ leaves; 2) one of the children has a dimension $d + 1$, and the second child $v$ has a dimension $dim(v) \le d$. Again, a tree with the maximum number of leaves can be constructed recursively:  each node of such tree has two children $u$ and $v$ with dimensions $d+1$ and $d$ respectively (the greater the dimension of the node, the more leaves are in the corresponding tree in the worst case). By the induction assumption there are no more than $(h-1)^d + (h-2)^d + (h-3)^d + ... + 1 = O(h^{d+1})$ leaves, so the claim holds for $dim = d+1$.
\end{enumerate}
\end{proof}
Combining Lemma \ref{lem:treeheight} and Lemma \ref{oscbnddim}, we can deduce the following.
\begin{corollary}
\label{finaldim}
Let $G$ be a grammar $G = (\Sigma, N, P, S)$ in Chomsky normal form, such that every parse tree $t$ has $dim(t) \le d$, where $d$ is some constant. Let $D=(V, E, \Sigma)$ be a directed labeled graph with $n$ nodes. Then $\rho_{L(G)}$ is in $O({(|N|n^2)}^d)$ in the worst case.
\end{corollary}
\begin{theorem}
\label{oscbndosc}
Let $L$ be a $k$-bounded-oscillation language with grammar $G = (\Sigma, N, P, S)$ in Chomsky normal form and $D=(V, E, \Sigma)$ be a directed labeled graph with $n$ nodes. Then $\rho_{L(G)}$ is in $O({|N|}^{2k}n^{4k})$ in the worst case.
\end{theorem}
\begin{proof}
By Lemma~\ref{boscdim}, every parse tree of bounded-oscillation language has also bounded dimension. Then the maximum value of the dimension of every parse tree of $k$-bounded-oscillation language is $2k$. By Corollary~\ref{finaldim}, $\rho_{L(G)}$ is in $O({(|N|n^2)}^d)$ and, thus, $\rho_{L(G)}$ does not exceed $O({(|N|n^2)}^{2k}) = O({|N|}^{2k}n^{4k})$.
\end{proof}

As we can see from the proof of Lemma~\ref{oscbnddim}, the family of linear languages is included in the family of bounded-oscillation languages. The reason is that the family of bounded-oscillation languages generalizes the family of languages accepted by finite-turn pushdown automata~\cite{BoundOsc}. It is interesting that for general PDA, particularly for $D_2$, the value of oscillation is not constant-bounded: it depends on the length of input and does not exceed $O(\log n)$ for the input of length $n$~\cite*{Gundermann, Wechsung}. However, for some previously studied subclasses of context-free languages,  oscillation is bounded by a constant.

\begin{subsection}{The rational indices of some subclasses of bounded-oscillation languages} 

\paragraph{Superlinear languages.} 
A context-free grammar $G = (\Sigma, N, P, S)$ is \textit{superlinear}~\cite{superlinear} if all productions of $P$ satisfy these conditions:
\begin{enumerate}
\item there is a subset $N_L \subseteq N$ such that every $A \in N_L$ has only linear productions $A\rightarrow aB$ or $A\rightarrow Ba$, where $B \in N_L$ and $a \in \Sigma$.
\item if $A \in N \setminus N_L$, then $A$ can have non-linear productions of the form $A \rightarrow BC$ where $B\in N_L$ and $C \in N$, or linear productions of the form $A\rightarrow \alpha B$ $\vert$ $B \alpha$ $\vert$ $\alpha$ for $B \in N_L$, $\alpha \in \Sigma^*$.
\end{enumerate}
A language is \textit{superlinear} if it is generated by some superlinear grammar. 
\begin{theorem} Let $G$ be a superlinear grammar. Then $\rho_{L(G)}$ is in $O(n^4)$.
\end{theorem}
\begin{proof}
From the definition of superlinear grammar $G$ it is observable that its parse trees have dimension at most 2. From 
Corollary~\ref{finaldim}, if dimensions of all parse trees are bounded by some $k$ then the rational index $\rho_{L(G)}$ of such language is in $O(n^4)$.
\end{proof}
\end{subsection}
\paragraph{Ultralinear languages.} A context-free grammar $G = (\Sigma, N, P, S)$ is \textit{ultrealinear} if there exists a partition $\{N_0, N_1, ..., N_k\}$ of $N$ such that $S \in N_k$ and if $A \in N_i$, where $0 \le i \le k$, then $(A \rightarrow w) \in P$ implies $w \in \Sigma^*N_i\Sigma^*$ or $w \in {(\Sigma \cup N_0 \cup ... \cup N_i-1)}^*$. Such a partition is called an \textit{ultralinear decomposition}. A language is \textit{ultralinear} if it is generated by some ultralinear grammar. 


The ultralinear languages were originally defined by Ginsburg and Spanier~\cite{Ginsburg1966FiniteTurnPA} as languages recognizable by finite-turn pushdown PDAs (a finite-turn PDA is a PDA with a fixed constant bound on the number of switches between push and pop operations in accepting computation paths). 

Every ultralinear language is generated by an ultralinear grammar in \textit{reduced form}~\cite{WORKMAN1976188}.
\begin{definition}[The reduced form of ultralinear grammar.]
An ultralinear grammar $G = (\Sigma, N, P, S)$ is in \textit{reduced form} if its ultralinear decomposition $\{N_0, N_1, ..., N_k\}$ is in the following form:
\begin{enumerate}
\item $N_k=\{S\}$ and $S$ does not appear in the right part of any production rule
\item if $(A \rightarrow w) \in P \setminus \{S \rightarrow \varepsilon\}$ and $A \in N_i$, $0 \le i \le k$, then $w \in (\Sigma \cup N_i\Sigma \cup \Sigma N_i \cup N_jN_j')$, where $j, j' < i$.
\end{enumerate}
\end{definition}
\begin {theorem}
Let $G = (\Sigma, N, P, S)$ be an ultralinear grammar with the ultralinear decomposition $\{N_0, N_1, ..., N_k\}$. Then $\rho_{L(G)}$ is in $O(n^{2k})$.
\end{theorem}
\begin{proof}
Recall that by definition dimension of a parse tree is the height of its largest perfect subtree. Consider the maximum possible size of a perfect subtree which occurs in the parse tree in ultralinear grammar in reduced form. It is easy to see that the rules of the form $A \rightarrow BC$, where $A \in N_i, B, C \in N_{i-1}$ should be used as often as possible to construct the largest binary subtree. Therefore, if grammar has the subset of rules of the form $\{S \rightarrow AB, A \rightarrow A_1A_2, B \rightarrow B_1B_2, A_1\rightarrow A_3A_4, ..., A_i \rightarrow A_{i+2}, A_{i+3}, ...\}$, where $A, B \in N_{k-1}, A_1, A_2, B_1, B_2 \in N_{k-2}, A_3, A_4, ... \in N_{k-3}, ... , A_{i+2}, A_{i+3}, ... \in N_0$, the perfect binary subtree obtained with these rules will be of height not greater than $k$, so the maximum dimension of the parse tree in a ultralinear grammar in reduced form is $k$. By Corollary~\ref{finaldim}  $\rho_{L(G)}$ is in $O(n^{2k})$.
\end{proof}