\section{Conclusion}

In this paper, we present a portable generic sparse linear algebra library for GPU computations, which provides sparse primitives and supports a set of common operations. Evaluation of proposed solution for real-world graph analysis shows, that initial OpenCL-based operations implementation with a limited set of optimizations has promising performance compared to other tools and can be easily on devices of multiple vendors, which gives significant flexibility in choose of HPC hardware. Since the project is still in active development, the following major and important development tasks must be highlighted.

\begin{itemize}
    \item \textit{Operations.} New mathematical operations must be implemented, as well as required and most used data manipulation, row/column management, filtering, selection, and transformation subroutines must be supported.
    
    \item \textit{Performance tuning.} State-of-the-art high-performance algorithms must be implemented for such operations, as SpMSpV, SpMV, SpGEMM, etc. in order to increase library efficiency. Also, different techniques, such as automated storage format transitions, pull-push direction optimization, etc. must be employed for better performance on a large range of data.
    
    \item \textit{Graph algorithms.} Algorithms such as \textit{single-source shortest paths}, \textit{page rank}, \textit{connected components}, etc. must be implemented using library API and their performance must be analyzed as well. 
    
    \item \textit{API exporting.} The library interface must be exported to other programming languages, such as C or Python. Finally, a Python package for applied graph analysis must be published.
    
    \item \textit{Multi-GPU scheduling.} Currently, the library supports scheduling only on a single GPU. But, its internal hybrid storage format and tasking allow multi-node execution. Thus, multi-GPU scheduling and precise data sharing must be implemented and performance analysis of the solution must be carried on.
\end{itemize}

Finally, we plan to study the characteristics of the implemented library with all algorithms and improvements, and compare its performance to existing tools using uniform benchmarking platform such as Graphalytics~\cite{Graphalytics:iosup2021ldbc}.