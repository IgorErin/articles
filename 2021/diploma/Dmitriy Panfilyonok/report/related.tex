\label{sec:relatedworks}
\paragraph{} 
В таблице~\ref{tab:libs_comparison} представлено сравнение текущих реализаций спецификации GraphBLAS, а также некоторых других популярных библиотек для анализа графов на GPU, основанных на иных подходах к построению алгоритмов.

Одним из критериев оценки существующих решений стало наличие возможности использования графических процессоров видеокарты в качестве платформы для исполнения вычислений. Такая возможность может существенно повысить производительность решений, основанных на линейной алгебре, так как операции с разреженными матрицами поддаются массовому параллелизму~\cite{blast}. Однако реализация GraphBLAS на GPU является открытой проблемой --- как показано в таблице~\ref{tab:libs_comparison}, на данный момент нет решений, которые бы полностью поддерживали вычисления на графическом процессоре. Это связано, в том числе с тем, что библиотеки, использующее для этого языки C и С++, сталкиваются с рядом проблем при попытке реализовать обобщенные операции на GPU. Кроме того, все существующие решения, которые разрабатываются для исполнения на GPU, используют программно-аппаратную архитектуру CUDA. Это оз\-на\-ча\-ет, что они могут исполняться только на графических устройствах от NVIDIA, что негативно сказывается на переносимости таких библиотек. Решение, описываемое в данной работе, предлагает использовать открытую платформу OpenCL, которая позволяет осуществлять вычисления на любом устройстве с поддержкой данного стандарта, будь то GPU, CPU или FPGA. Такой подход не только позволит использовать всю вычислительную силу современных графических систем, но и облегчит разработку и тестирование алгоритмов широкому кругу исследовательских групп и независимых разработчиков.

Другим критерием оценки существующих библиотек стал язык программирования, на котором она написаны. Как показано в таблице~\ref{tab:libs_comparison}, на текущий момент большинство реализаций написаны для языков C и C++. В то же время использование языков более высокого уровня имеет свои преимущества. Среди них можно выделить возможность дополнительных проверок во время компиляции, богатую экосистему некоторых платформ, абстрактную систему типов, благодаря которой можно более полно описывать абстракции линейной алгебры. Для функциональных языков программирования могут быть доступны некоторые техники суперкомпиляции и оптимизации, например kernel fusion\cite{fusion} и deforestation\cite{deforset}.

\begin{table}
    % \centering
    \begin{minipage}{\linewidth}
    \begin{tabular}{l|l|l}
        Реализация & Язык & Поддержка GPU \\
        \hline \hline
        SuiteSparse~\cite{suitesparse} & C & Нет (поддержка CUDA в разработке) \\
        IBM GraphBLAS\footnote{Репозиторий библиотеки IBM GraphBLAS: \url{https://github.com/IBM/ibmgraphblas}. Дата посещения: 13.12.2020} & C & Нет \\
        GBTL~\cite{gbtl} & C++ & Нет (версия с CUDA устарела) \\
        GraphBLAST~\cite{blast} & C++ & CUDA (в разработке) \\
        CombBLAS~\cite{combblas} & C++ & Частичная, CUDA \\
        Graphulo\footnote{Репозиторий библиотеки Graphulo: \url{https://github.com/Accla/graphulo}. Дата посещения: 13.12.2020} & Java & Нет \\ 
        GraphMat\footnote{Репозиторий библиотеки GraphMat: \url{https://github.com/narayanan2004/GraphMat}. Дата посещения: 13.12.2020} & С++ & Нет \\
        \hline
        Gunrock~\cite{gunrock} & C++ & CUDA \\
        CuSha\footnote{Репозиторий библиотеки CuSha: \url{https://github.com/farkhor/CuSha}. Дата посещения: 13.12.2020} & C++ & CUDA \\
        \hline
    \end{tabular}
    \end{minipage}
    \caption{Сравнение текущих реализаций GraphBLAS (верхний сегмент) и некоторых популярных библиотек для анализа графов на GPU, основанных на иных подходах к построению алгоритмов (нижний сегмент).}
    \label{tab:libs_comparison}
\end{table}

В таблице~\ref{tab:opencl_comparison} приведены инструменты для работы с OpenCL на платформах .NET и Java. Эти платформы достаточно популярны, чтобы предлагаемое решение было более востребовано. Библиотеки \\ OpenCL.NET, Cloo, JavaCL и JOCL предоставляют высокоуровневый интерфейс для работы с ядрами, написанными на C. Эти библиотеки не позволяют естественным образом работать со сложными типами высокоуровневого языка, поэтому не подходят для данной работы. Среди оставшихся инструментов была выбрана библиотека Brahma.FSharp для языка программирования F\#. На данный момент она активную разрабатывается на кафедре системного программирования СПбГУ, поэтому в случае возникновения ошибок можно ожидать их своевременного исправления.

\begin{table}
    % \centering
    \begin{minipage}{\linewidth}
    \begin{tabularx}{\textwidth}{X|X}
    .NET & Java \\
    \hline
    OpenCL.NET\footnote{Репозиторий библиотеки OpenCL.NET: \url{https://github.com/dgsantana/OpenCL.NET}. Дата посещения: 01.06.2021} & JavaCL\footnote{Репозиторий библиотеки JavaCL: \url{https://github.com/nativelibs4java/JavaCL}. Дата посещения: 01.06.2021} \\
    Cloo\footnote{Репозиторий библиотеки Cloo: \url{https://github.com/clSharp/Cloo}. Дата посещения: 01.06.2021} & JOCL\footnote{Репозиторий библиотеки JOCL: \url{https://github.com/gpu/JOCL}. Дата посещения: 01.06.2021} \\
    FSCL\footnote{Репозиторий библиотеки FSCL: \url{https://github.com/FSCL/FSCL.Compiler}. Дата посещения: 01.06.2021} & Aparapi\footnote{Репозиторий библиотеки Aparapi: \url{https://github.com/aparapi/aparapi}. Дата посещения: 01.06.2021} \\
    Brahma.FSharp\footnote{Репозиторий библиотеки Brahma.FSharp: \url{https://github.com/YaccConstructor/Brahma.FSharp}. Дата посещения: 01.06.2021} & ScalaCL\footnote{Репозиторий библиотеки ScalaCL: \url{https://github.com/nativelibs4java/ScalaCL}. Дата посещения: 01.06.2021} \\
    \hline
  \end{tabularx}
  \end{minipage}
  \caption{Инструменты для работы с OpenCL на платформах .NET и Java}
  \label{tab:opencl_comparison}
\end{table}
