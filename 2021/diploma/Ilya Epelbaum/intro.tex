
Одной из фундаментальных областей программной инженерии является разработка СУБД. На данный момент создано множество разных моделей построения таковых. Например, одной из них является графовая модель, в которой данные представлены в виде вершин --- сущности и дуг --- связи между сущностями. Примерами графовых СУБД являются RedisGraph\footnote{Репозиторий проекта RedisGraph: https://github.com/RedisGraph/RedisGraph. Дата посещения: 20.04.2021.}, Neo4j\footnote{Сайт проекта Neo4j: https://neo4j.com. Дата посещения: 20.04.2021.} и TigerGraph\footnote{Сайт проекта TigerGraph: https://www.tigergraph.com. Дата посещения: 20.04.2021.}.

К графовой базе данных необходимо выполнять запросы. В ходе их рассмотрения появляется задача поиска путей в графе с ограничениями, которые выразимы в терминах формальной грамматики. В таком случае каждому пути соответствует слово, полученное конкатенацией меток, находящихся на его дугах в порядке обхода. Именно это слово будет проверяться на принадлежность заданной грамматике.

В этой связи особый интерес представляют контекстно-свободные (КС) грамматики, так как обладают более выразительной способностью в отличии, например, от регулярных. То есть данный тип грамматики позволяет задавать более сложные ограничения на пути. Поэтому запросы с КС-ограничениями используют и в других областях, например, биоинформатика~\cite{Bio} или статический анализ кода~\cite{Stat}.

Существует множество алгоритмов, решающих задачу поиска путей основываясь на теории формальных языков, например, алгоритм Элле Хеллингса~\cite{Hellings2015PathRF}, восходящий алгоритм Фреди Сантоса~\cite{inbook} или алгоритм Петтери Севона~\cite{Bio}. Однако, в статье Йохима Куйперса и др.~\cite{oper_matrix} продемонстрировано, что существующие алгоритмы могут быть успешно применены лишь при определенных условиях на входные данные, а в общем случае их реализации требуют большой объем вычислительных мощностей в процессе работы, что не оправдывает их применение в промышленных продуктах. В то же время Никита Мишин и др. в статье~\cite{mishin} и Егор Орачев и др. в исследовании~\cite{10.1007/978-3-030-54832-2_6}, взяв собранный набор данных \textit{CFPQ\_Data}\footnote{Репозиторий набора данных \textit{CFPQ\_Data}, наполненный различными реальными и синтетическими графами: https://github.com/JetBrains-Research/CFPQ\_Data. Дата посещения: 20.04.2021}, показали, что алгоритмы, основанные на линейной алгебре, а именно алгоритм, основанный на матричном умножении, и алгоритм, основанный на произведении Кронекера, демонстрируют высокую производительность на реальных входных данных. 

Однако в исследовании~\cite{10.1007/978-3-030-54832-2_6} авторы замечают неоптимальность предложенного алгоритма, основанного на произведении Кронекера, что влечет потребность в улучшении текущего результата.