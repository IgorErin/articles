\begin{table}[h!]
\begin{adjustbox}{max width=\textwidth}
\begin{tabular}{|c|c|c|}
\hline
$p \in L_P$                                                                                  & $\llbracket p \rrbracket_{G}$                                                                                                                                                                                   & Описание шаблона пути                                                                                                         \\ \hline
\hline
()                                                                                            & $\{(v, v): v \in V\}$                                                                                                                                                                                           & \begin{tabular}[c]{@{}c@{}}Пустой путь, состоящий из \\ одной произвольной вершины\end{tabular}                               \\ \hline
:a                                                                                            & $\{e=(v,to): e \in E, type(e)=a\}$                                                                                                                                                                              & \begin{tabular}[c]{@{}c@{}}Путь единичной длины,\\  состоящий из ребра с типом $a$\end{tabular}                               \\ \hline
(:b)                                                                                          & $\{(v, v): v \in V, label(v)=b\}$                                                                                                                                                                               & \begin{tabular}[c]{@{}c@{}}Пустой путь, состоящий из одной\\  вершины, помеченной меткой $b$\end{tabular}                     \\ \hline
$\alpha~\beta$                                                                                & $\llbracket \alpha \rrbracket_{G}\circ \llbracket \beta \rrbracket_{G}$                                                                                                                                         & Конкатенация путей $\alpha$ и $\beta$                                                                                         \\ \hline
$\alpha~|~\beta$                                                                              & $\llbracket \alpha \rrbracket_{G}\cup \llbracket \beta \rrbracket_{G}$                                                                                                                                          & Альтренатива между путями $\alpha$ и $\beta$                                                                                  \\ \hline
$[\alpha]$                                                                                    & $\llbracket \alpha \rrbracket_{G}$                                                                                                                                                                              & \begin{tabular}[c]{@{}c@{}}Квадратные скобки позволяют \\ группировать  выражения \\ для задания ассоциативности\end{tabular} \\ \hline
\textless{}$\alpha$                                                                           & $\{(to, v): (v, to) \in \llbracket \alpha \rrbracket_{G}\}$                                                                                                                                                     & Путь, обратный к пути $\alpha$                                                                                                \\ \hline
\textless{}$\alpha$\textgreater{}                                                             & $\llbracket \alpha~|~$\textless{}$\alpha \rrbracket_{G} $                                                                                                                                                       & \begin{tabular}[c]{@{}c@{}}Альтернатива между путём $\alpha$ и\\ обратным к нему\end{tabular}                                 \\ \hline
$\alpha^*$                                                                                    & $\llbracket \alpha \rrbracket_{G}^{*}$                                                                                                                                                                          & \begin{tabular}[c]{@{}c@{}}Путь, состоящий из\\ конкатенации 0 или более путей $\alpha$\end{tabular}                          \\ \hline
\begin{tabular}[c]{@{}c@{}}$\{S_i = p_i\}_{i=1}^{n}$\\ -- named\\  path patterns\end{tabular} & \begin{tabular}[c]{@{}c@{}}$P = \{S_i \rightarrow p_i\}_{i=1}^n$\\ $Gram_j = (\Sigma, \{S_i\}_{i=1}^n, P, S_j)$\\ $\llbracket S_j \rrbracket_{G} = \bigcup\limits_{p \in L(G)}{\llbracket p \rrbracket_{G}}$\end{tabular} & Именнованые шалоны путей                                                                                                      \\ \hline
$\sim$$S$                                                                                     & $\llbracket S \rrbracket_{G}$                                                                                                                                                                                   & \begin{tabular}[c]{@{}c@{}}Ссылка на именнованный\\  шаблон пути\end{tabular}                                                 \\ \hline
\end{tabular}
\end{adjustbox}
\caption{Семантика языка шаблонов путей}
\label{tab:cypher_sematic}
\end{table}