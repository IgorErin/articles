% У введения нет номера главы
\section*{Введение}
Представление данных с помощью помеченных графов находит своё применение в биоинформатике~\cite{QVCRNASSP}, в статическом анализе кода и многих других областях.
Всё более популярными становятся графовые базы данных~\cite{FRSPGD}.
При работе с такими данными зачастую возникают запросы навигации и поиска путей, удовлетворяющих заданным ограничениям.
Результат обработки такого рода запросов, как правило, представляет собой набор отношений между вершинами графа.
Один из естественных способов определить подобные отношения над помеченным графом --- указать соответствующие пути, используя формальные грамматики над алфавитом меток рёбер.
При этом соответствующие запросы навигации естественным образом могут быть выражены с помощью кон\-текст\-но-свобод\-ных грамматик~\cite{GTMIDT}.
Таким образом встает вопрос о необходимости разработки и реализации алгоритмов поиска путей с кон\-текст\-но-свобод\-ными ограничениям (англ. <<Context-Free Path Querying>>, кратко CFPQ). 

Ввиду широкой применимости кон\-текст\-но-свобод\-ных запросов в перечисленных выше практических областях, критически важной становится потребность в измерении производительности алгоритмов реализующих эти запросы.
Для того чтобы показать применимость алгоритма на практике, возникает необходимость проведения экспериментального исследования на помеченных графах, отвечающих реальным данным.
Однако поиск и подготовка таких графов весьма сложны и могут занять достаточно продолжительное время.

Одним из решений подобных проблем во многих областях исследований является использование единого стандартизированного набора данных.
Например, в биоинформатике очень важно иметь набор данных для проверки производительности алгоритмов кластеризации и проекции данных~\cite{FCPS}.
А в области машинного обучения необходимо иметь стандартный набор данных, позволяющий исследователям выбирать какой метод лучше подходит для решения конкретной задачи~\cite{PMLBDBLP}.
В области алгоритмов, реализующих кон\-текстно-свобод\-ные запросы к помеченным графам, на данный момент самым перспективным является набор данных \textsc{CFPQ\_Data}\footnote{GitHub репозиторий \textsc{CFPQ\_Data}: \url{https://github.com/JetBrains-Research/CFPQ_Data}, дата последнего доступа --- 04.06.2021}.
Но он имеет ряд проблем, не позволяющих использовать его, как полноценное решение задачи подготовки экспериментального исследования CFPQ алгоритмов.
Именно об устранении этих проблем и пойдёт речь в данной работе.
