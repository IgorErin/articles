\documentclass[sigconf,edbt,table]{acmart-edbt2021}

\def\BibTeX{{\rm B\kern-.05em{\sc i\kern-.025em b}\kern-.08em
    T\kern-.1667em\lower.7ex\hbox{E}\kern-.125emX}}

\usepackage{booktabs} % For formal tables
\usepackage{tikz}
\usepackage[noend]{algpseudocode}
\usepackage{algorithm}
\usepackage{algorithmicx}
\usepackage{xcolor}
\usepackage{balance}

\usepackage{multirow}

%\usepackage{colortbl}
\usepackage{subcaption}
\usepackage{listings}
\usepackage{extarrows}
\usetikzlibrary{shapes,shapes.geometric,fit,automata,positioning}
\usetikzlibrary{decorations.pathmorphing}
\tikzset{snake it/.style={decorate, decoration=snake}}

%\definecolor{lightgray}{gray}{0.9}

\newcommand{\cho}[1]{{\color{red} #1}}

\newtheorem{mytheorem}{Theorem}
\newtheorem{myproposition}{Proposition}

% Copyright
\setcopyright{rightsretained}

% DOI
\acmDOI{}

% ISBN
\acmISBN{978-3-89318-084-4}

%Conference
\acmConference[EDBT 2021]{24th International Conference on Extending Database Technology (EDBT)}{March 23-26, 2021}{Nicosia, Cyprus} 
\acmYear{2021}

\settopmatter{printacmref=false, printccs=false, printfolios=false}

\pagestyle{empty} % removes running headers

% Math square brackets
\DeclareMathOperator{\rank}{rank}
\makeatletter
\newenvironment{sqcases}{%
  \matrix@check\sqcases\env@sqcases
}{%
  \endarray\right.%
}
\def\env@sqcases{%
  \let\@ifnextchar\new@ifnextchar
  \left\lbrack
  \def\arraystretch{1.2}%
  \array{@{}l@{\quad}l@{}}%
}
\makeatother

% Footnote reference
\makeatletter
\newcommand\footnoteref[1]{\protected@xdef\@thefnmark{\ref{#1}}\@footnotemark}
\makeatother

\newcommand\gsv[2]{{\color{red}{#1}( {#2} $^{\text{gsv}}$)}}
\newcommand\simpleton[1]{{\color{blue}{#1}}}

\tikzset{elliptic state/.style={draw,ellipse}}

\begin{document}
\title{Context-Free Path Querying with All-Path Semantics by Matrix Multiplication}

\author{Rustam Azimov}
\email{st013567@student.spbu.ru}
\email{rustam.azimov@jetbrains.com}
%\authornote{}
\affiliation{
	\institution{Saint Petersburg State University}
	\streetaddress{7/9 Universitetskaya nab.}
	%\city{St. Petersburg}
	%\country{Russia}
	\postcode{199034}
}
\affiliation{
	\institution{JetBrains Research}
	\streetaddress{Primorskiy prospekt 68-70, Building 1}
	\city{St. Petersburg}
	\country{Russia}
	\postcode{199034}
}

\author{Ilya Epelbaum}
\email{iliyepelbaun@gmail.com}
%\authornote{}
\affiliation{
	\institution{Saint Petersburg State University}
	\streetaddress{7/9 Universitetskaya nab.}
	%\city{St. Petersburg}
	%\country{Russia}
	\postcode{199034}
}
\affiliation{
	\institution{JetBrains Research}
	\streetaddress{Primorskiy prospekt 68-70, Building 1}
	\city{St. Petersburg}
	\country{Russia}
	\postcode{199034}
}

\author{Semyon Grigorev}
	\email{s.v.grigoriev@spbu.ru}
	\email{semyon.grigorev@jetbrains.com}
	\orcid{0000-0002-7966-0698}
	\affiliation{
		\institution{Saint Petersburg State University}
		\streetaddress{7/9 Universitetskaya nab.}
		% \city{St. Petersburg}
		% \country{Russia}
		\postcode{199034}
	}
	\affiliation{
		\institution{JetBrains Research}
		\streetaddress{Primorskiy prospekt 68-70, Building 1}
		\city{St. Petersburg}
		\country{Russia}
		\postcode{199034}
	}

% The default list of authors is too long for headers}
% \renewcommand{\shortauthors}{B. Trovato et al.}
\renewcommand{\shortauthors}{Rustam Azimov et al.}


\begin{abstract}
	Context-Free Path Querying (CFPQ) allows one to use context-free grammars as path constraints in navigational graph queries. Many algorithms for CPFQ were proposed, but recently showed that the state-of-the-art CFPQ algorithms are still not performant enough for practical use. One promising way to achieve high-performance solutions for graph querying problems is to reduce them to linear algebra operations. Recently, there are two CFPQ solutions formulated in terms of linear algebra: the Azimov's matrix-based CFPQ algorithm (2018) and the Kronecker product-based CFPQ algorithm proposed by Orachev et al. (2020). However, the Azimov's algorithm still not support the most expressive all-path query semantics and cannot be truly compared with Kronecker product-based CFPQ algorithm. In this work, we introduce a new matrix-based CFPQ algorithm with all-path query semantics that allows us to extract all found paths for each pair of vertices. Also, we implement our algorithm by using appropriate high-performance libraries for linear algebra. Finally, we compare our algorithm with other most performant CFPQ algorithms.
\end{abstract}

%
% % The code below should be generated by the tool at
% % http://dl.acm.org/ccs.cfm
% % Please copy and paste the code instead of the example below.
% %
\begin{CCSXML}
		<ccs2012>
		<concept>
			<concept_id>10002951.10002952.10003197.10010825</concept_id>
			<concept_desc>Information systems~Query languages for non-relational engines</concept_desc>
			<concept_significance>500</concept_significance>
		</concept>
		<concept>
			<concept_id>10003752.10003766.10003771</concept_id>
			<concept_desc>Theory of computation~Grammars and context-free languages</concept_desc>
			<concept_significance>500</concept_significance>
		</concept>
		<concept>
			<concept_id>10002950.10003624.10003633.10003640</concept_id>
			<concept_desc>Mathematics of computing~Paths and connectivity problems</concept_desc>
			<concept_significance>300</concept_significance>
		</concept>
		<concept>
			<concept_id>10002951.10002952.10002953.10010146</concept_id>
			<concept_desc>Information systems~Graph-based database models</concept_desc>
			<concept_significance>500</concept_significance>
		</concept>
		</ccs2012>
\end{CCSXML}

    \ccsdesc[500]{Information systems~Graph-based database models}
	\ccsdesc[500]{Information systems~Query languages for non-relational engines}
	\ccsdesc[500]{Theory of computation~Grammars and context-free languages}
    \ccsdesc[300]{Mathematics of computing~Paths and connectivity problems}


% \keywords{ACM proceedings, \LaTeX, text tagging}

%% A "teaser" image appears between the author and affiliation
%% information and the body of the document, and typically spans the
%% page.
%\begin{teaserfigure}
%  \includegraphics[width=\textwidth]{new_hope.png}
%  \caption{Episode IV: A New Hope}
%  \label{fig:teaser}
%\end{teaserfigure}

\maketitle

\section{Introduction}

Scalable high-performance graph analysis is an actual challenge.
There is a big number of ways to attack this challenge~\cite{Coimbra2021} and the first promising idea is to utilize general-purpose graphic processing units (GPGPU-s).
Such existing solutions, as CuSha~\cite{10.1145/2600212.2600227} and Gunrock~\cite{7967137} show that utilization of GPUs can improve the performance of graph analysis, moreover it is shown that solutions may be scaled to multi-GPU systems.
But low flexibility and high complexity of API are problems of these solutions.

The second promising thing which provides a user-friendly API for high-performance graph analysis algorithms creation is a GraphBLAS API~\cite{7761646} which provides linear algebra based building blocks to create graph analysis algorithms.
The idea of GraphBLAS is based on is a well-known fact that linear algebra operations can be efficiently implemented on parallel hardware.
Along with this, a graph can be natively represented using matrices: adjacency matrix, incidence matrix, etc.
While reference CPU-based implementation of GraphBLAS, SuiteSparse:GraphBLAS~\cite{10.1145/3322125}, demonstrates good performance in real-world tasks, GPU-based implementation is challenging.

One of the challenges in this way is that real data are often sparse, thus underlying matrices and vectors are also sparse, and, as a result, classical dense data structures and respective algorithms are inefficient. 
So, it is necessary to use advanced data structures and procedures to implement sparse linear algebra, but the efficient implementation of them on GPU is hard due to the irregularity of workload and data access patterns.
Though such well-known libraries as cuSparse show that sparse linear algebra operations can be efficiently implemented for GPGPU-s, it is not so trivial to implement GraphBLAS on GPGPU. 
First of all, it requires \textit{generic} sparse linear algebra, thus it is impossible just to reuse existing libraries which are almost all specified for operations over floats.
The second problem is specific optimizations, such as maskings fusion, which can not be natively implemented on top of existing kernels.
Nevertheless, there is a number of implementations of GraphBLAS on GPGPU, such as GraphBLAST:~\cite{yang2019graphblast}, GBTL~\cite{7529957}, which show that GPGPUs utilization can improve the performance of GraphBLAS-based graph analysis solutions.
But these solutions are not portable because they are based on Nvidia Cuda stack.
Moreover, the scalability problem is not solved: all these solutions support only single-GPU, not multi-GPU computations.

To provide portable GPU implementation of GraphBLAS API we developed a \textit{SPLA} library (sources are published on GitHub: \url{https://github.com/JetBrains-Research/spla}).
This library utilizes OpenCL for GPGPU computing to be portable across devices of different vendors.
Moreover, it is initially designed to utilize multiple GPGPUs to be scalable.
To sum up, the contribution of this work is the following.
\begin{itemize}
    \item Design of portable GPU GraphBLAS implementation proposed. The design involves the utilization of multipole GPUS. Additionally, the proposed design is aimed to simplify library tuning and wrappers for different high-level platforms and languages creation. 
    \item Subset of GraphBLAS API, including such operations as masking, matrix-matrix multiplication, matrix-matrix e-wise addition, is implemented. The current implementation is limited by COO and CSR matrix representation format and uses basic algorithms for some operations, but work in progress and more data formats will be supported and advanced algorithms will be implemented in the future.
    \item Preliminary evaluation on such algorithms as breadth-first search (BFS) and triangles counting (TC), and real-world graphs shows portability across different vendors and promising performance: for some problems Spla is comparable with GraphBLAST. Surprisingly, for some problems, the proposed solution on embedded Intel graphic card shows better performance than SuiteSparse:GraphBLAS on the same CPU. At the same time, the evaluation shows that further optimization is required.
\end{itemize} 
\section{Preliminaries}

We introduce !!!!

\subsection{Context-Free Path Querying}

Graph, grammar, etc.

Let $i\pi j$ denote a unique path between nodes $i$ and $j$ of the graph and $l(\pi)$ denotes a unique string which is obtained from the concatenation of edge labels along the path $\pi$.
For a context-free grammar $G = (\Sigma, N, P, S)$ and directed labelled graph $D = (Q, \Sigma, \delta)$, a triple $(A, i, j)$ is \textit{realizable} iff there is a path $i\pi j$ such that nonterminal $A \in N$ derives $l(\pi)$.

\subsection{Tensor-Based algorithm for CFPQ}

\begin{algorithm}[H]
\begin{algorithmic}[1]
\caption{Kronecker product context-free recognizer for graphs}
\label{alg:Kronecker}
\Function{contextFreePathQuerying}{D, G}
\EndFunction
\end{algorithmic}
\end{algorithm}

\subsection{Planar Graphs}

A planar graph $G = (V, E)$ is a graph that can be embedded in the plane.

Outer face - unbounded face in specific embedding.

Directed graph (\textit{digraph})

...

\subsection{Dynamic reachability algorithms}

We consider algorithms that solve the problem of reachability in planar directed graphs. In the \textit{dynamic reachability problem} we are given a graph $G$ subject to edge updates (insertions or deletions) and the goal is to design a data structure that would allow answering queries about the existence of a path.

We need to answer the queries of type: "Is there a directed path from $u$ to $v$ in $G$?". If vertex $u$ in all the queries is fixed we say that algorithm is \textit{single-source}. It is said to be \textit{all-pairs} if vertices $u, v$ can be any vertices of planar digraph $G$, in this case it can be also called \textit{dynamic transitive closure}.

We say that the algorithm is \textit{fully dynamic} if it supports both additions and deletions of edges. It is said to be \textit{semi dynamic} if it supports only one of these updates. If semi dynamic algorithm supports additions only it is called \textit{incremental}, if deletions only - \textit{decremental}.






\section{Matrix-based CFPQ algorithm for all-path query semantics}
\label{sec:all-path-algo}
In this section, we introduce the AllPathIndex structure which is used as a base of our solution for all-path query semantics. Also, we propose the matrix-based algorithm for CFPQ w.r.t. the all-path query semantics.

\subsection{AllPathIndex}
Our algorithm is based on Azimov's CFPQ algorithm~\cite{Azimov:2018:CPQ:3210259.3210264} which is based on matrix operations.
This algorithm reduces CFPQ to operations over Boolean matrices and as a result, allows one to use high-performance linear algebra libraries and utilize modern parallel hardware for CFPQ.

Note, that the algorithm computes not only the context-free relation $R_{G,D}$ but also a set of context-free relations $R_{G_A,D} \subseteq V \times V$ for every $A \in N$ where $G_A = (N, \Sigma, P, A)$.
Thus it provides information about paths that form words derivable from any nonterminal $A \in N$.

We use an idea similar to one that was used for the CFPQ with single-path query semantics in~\cite{10.1145/3398682.3399163}. We store additional information in matrices to be able to restore all paths which form words derivable from any nonterminal in the given grammar.

In order to do this, we introduce the 
$$\textit{AllPathIndex} = (\textit{left},\textit{right},\textit{middles})$$ 
--- the elements of matrices that describe the found paths as concatenations of two smaller paths and help to restore each path after the index creation. Here \textit{left} and \textit{right} stand for the indexes of starting and ending vertices in the founded path, \textit{middles} --- the set of indexes of intermediate vertices used in the concatenation of two smaller paths. When we do not find the path for some vertex pair $i,j$, we use the $\textit{AllPathIndex} = \bot = (0,0,\emptyset)$.

Additionally, we will use the notation of \textit{proper matrix} which means that for every element of the matrix with indexes $i,j$ it either $\textit{AllPathIndex} = (i,j,\_)$ or $\bot$.

We use a binary operation $\otimes$ defined for AllPathIndexes \mbox{$AP_1, AP_2$} which are not equal to $\bot$ as
\begin{align*}
	AP_1 \otimes AP_2 = (&AP_1.left, AP_2.right, \{AP_1.right\}).
\end{align*}

And if at least one operand is equal to $\bot$ then $AP_1 \otimes AP_2 = \bot$. Note that we will use this operation only for multiplication of proper matrices where $AP_1 \neq \bot \neq AP_2$ only when $AP_1.\textit{right} = AP_2.\textit{left}$.

We also use a binary operation $\oplus$ defined for AllPathIndexes \mbox{$AP_1, AP_2$} which are not equal to $\bot$ as $AP_1 \oplus AP_2$ equal to
	$$(AP_1.left, AP_1.right, AP_1.middles \cup AP_2.middles).$$
	
If only one operand is equal to $\bot$ then $AP_1 \oplus AP_2$ equal to another operand. If both operands are equal to $\bot$ then $AP_1 \oplus AP_2 = \bot$. Note that we will use this operation only for multiplication and element-wise addition of proper matrices where $AP_1 \neq \bot \neq AP_2$ only when $AP_1.\textit{left} = AP_2.\textit{left}$ and $AP_1.\textit{right} = AP_2.\textit{right}$

Using $\otimes$ as multiplication of AllPathIndexes, and $\oplus$ as an addition, we can define a \emph{matrix multiplication}, \mbox{$a \odot b = c$}, where $a$ and $b$ are matrices of a suitable size, that have AllPathIndexes as elements, as $c_{i,j} = \bigoplus^{n}_{k=1}{a_{i,k} \otimes b_{k,j}}.$

Also, we use the element-wise $+$ operation on matrices $a$ and $b$ with the same size: \mbox{$a + b = c$}, where $c_{i,j} = a_{i,j} \oplus b_{i,j}.$


\subsection{The matrix-based algorithm}
We introduce the matrix-based algorithm for CFPQ w.r.t. the all-path query semantics (see Listing~\ref{lst:algo1}). This algorithm is a modification of Azimov's matrix-based algorithm for CFPQ and it constructs the set of matrices $T$ with AllPathIndexes as elements.
Let $G = (N, \Sigma, P, S)$ be the input context-free grammar, $D = (V, E, \Sigma)$ be the input graph.
The result of the algorithm is a set of matrices $T$ which stores information about all paths in the graph $D$ that form a word derivable from some nonterminal of the context-free grammar $G$. Note that in line 4 we add the special value $n$ (it can be any number that is not equal to any of the vertex indices) to the $T^{A}_{i,j}.middles$ to specify that this path is a single-edge path or an empty path $\pi_{\varepsilon}$.

\begin{algorithm}
	\small
	\begin{algorithmic}[1]
		\floatname{algorithm}{Listing}
		\caption{CFPQ algorithm for all-path query semantics}
		\label{lst:algo1}
		\Function{AllPathCFPQ}{\par
			\hskip\algorithmicindent $D = (V, E, \Sigma)$, \par
			\hskip\algorithmicindent $G=(N,\Sigma,P,S)$} \Comment{Grammar in WCNF}\par
		\State{$n \gets$ |V|}
		\State{$T \gets \{T^{A} \mid A \in N, T^{A}$ is a matrix $n \times n$, $T^{A}_{i,j} \gets \bot$ \} }
		\ForAll{$(i,x,j) \in E$, $A \mid A \to x \in P$}
		%\Comment{Matrices initialization}
		%\For{$A_k \mid A_k \to x \in P$}
		{$T^{A}_{i,j} \gets (i,j,\{n\})$}
		%\EndFor
		\EndFor
		\ForAll{$A \mid A \to \varepsilon \in P$}
		{$T^{A}_{i,i} \gets (i,i,\{n\})$}
		\EndFor
		
		\While{any matrix in $T$ is changing}
		%\Comment{Transitive closure calculation}
		\ForAll{$A \to B C \in P$ where $T^{B}$ or $T^{C}$ are changed}
		\State{ $T^{A} \gets T^{A} + (T^{B} \odot T^{C})$ } 
		\EndFor
		\EndWhile
		\State \Return $T$
		\EndFunction
		
	\end{algorithmic}
\end{algorithm}

Also, we can show that the proposed algorithm terminates in finite number of steps. It is sufficient to show, that the operations in the lines 7 and 8 change the matrices in $T$ only finite number of times. These operations can change the matrices in $T$ no more than $|V|^3|N|$ times since they can only add some vertices to the $middles$ component of some element, and there are $|N|$ matrices in $T$ with $|V|^2$ elements and for each element the maximum size of the set $middles$ is equal to $|V|$.



After constructing a set of matrices $T$ or so-called \textit{index}, we can construct a set of all paths $\pi$ between specified vertex pair $(i, j)$ and a non-terminal $A$ such that $A \xLongrightarrow[G]{*} l(\pi)$. The index $T$ already stores data about all paths derivable from each nonterminal. However, the
set of such paths can be infinite. From a practical perspective, it is necessary
to use lazy evaluation or limit the resulting set of paths in some other way.
For example, one can try to query some fixed number of paths or query paths
of fixed maximum length.

We propose the algorithm (see Listing~\ref{lst:algo2}) for extracting these paths. Our algorithm returns a set with the empty path $\pi_{\varepsilon}$ only if $i = j$ and $A \to \varepsilon \in P$. If the AllPathIndex for the given $i,j,A$ is equal to $\bot$ then our algorithm returns the empty set since such paths do not exist. Note that in line 19 we use
the operation $\cdot$ which naturally generalizes the path concatenation operation
by constructing all possible concatenations of path pairs from the given two
sets. It is assumed that the sets are computed lazily, to ensure the termination in case of an infinite number of paths.

\begin{algorithm}
	\small
	\begin{algorithmic}[1]
		\floatname{algorithm}{Listing}
		\caption{All paths extraction algorithm}
		\label{lst:algo2}		
		\Function{extractAllPaths}{$i, j, A, T=\{T^{A_k} \mid A_k \in N\}, G=(N,\Sigma,P,S)$}
		\State{$index \gets T^{A}_{i,j}$ }
		
		\If{$index = \bot$}
		\State \Return $\emptyset$
		\Comment{Such paths do not exist}
		\EndIf
		
		\State{$n \gets $ size of the square matrix $T^{A}$}
		\State{$resultPaths \gets \emptyset$}
		
		\ForAll{$m \in index.middles$}		
		\If{$m = n$}  \Comment{Add single-edge or empty paths}
		\ForAll{$x \mid A \to x \in P$}
		\If{$(i,x,j) \in E$}
		\State{$resultPaths \gets resultPaths \cup \{((i,x,j))\}$}
		\EndIf
		\EndFor
		\If{$(i = j) \wedge (A \to \varepsilon \in P)$}
		\State{$resultPaths \gets resultPaths \cup \{\pi_{\varepsilon}\}$}
		\EndIf
		\Else \Comment{Add to result the concatenated paths from $i$ to $m$ and from $m$ to $j$}
		\ForAll{$A \to B C \in P$}
		\State{$index_B \gets T^{B}_{i,m}$ }
		\State{$index_C \gets T^{C}_{m,j}$ }
		\If{$(index_B \neq \bot) \wedge (index_C \neq \bot)$}
		\State{$lPaths \gets$ \Call{extractAllPaths}{$i, m, B, T, G$}}
		\State{$rPaths \gets$ \Call{extractAllPaths}{$m, j, C, T, G$}}
		\State{$resultPaths \gets resultPaths \cup lPaths \cdot rPaths$}
		\EndIf
		\EndFor
		\EndIf
		\EndFor
		\State \Return $resultPaths$
		\EndFunction
	\end{algorithmic}
\end{algorithm}

\subsection{Correctness}

The following correctness theorem holds.

\begin{mytheorem}\label{thm:correct}
Let $G = (N, \Sigma, P, S)$ be the input context-free grammar, $D = (V, E, \Sigma)$ be the input graph, and $T$ be a set of matrices returned by the algorithm in Listing~\ref{lst:algo1}. Then for any $i, j$ and for any non-terminal $A \in N$, $index = T^A_{i,j}$ and $index = (i,j,middles) \neq \bot$ iff $(i,j) \in R_{G_A, D}$ and there is a path $\pi$ from vertex $i$ to $j$ such that $l(\pi) \in G_A = (N,\Sigma,P,A)$.
\end{mytheorem}
\begin{proof}[Proof sketch]
	At each iteration of the main cycle in lines 6-8 of the algorithm, the new paths corresponding to nonterminals $A \in N$ are considered using the rules $A \to B C \in P$. These new paths are obtained by the concatenation of two smaller paths corresponding to the nonterminals $B$ and $C$. At the initialization step of the algorithm in lines 3-5, we consider all single-edge or empty paths corresponding to the derivation tree of height 1. Thus, it can be shown that at iteration $l$ of the main cycle we consider all paths $\pi$ such that there is a derivation tree of the height $h \leq l + 1$ for the string $l(\pi)$ and a context-free grammar $G_A$. Therefore, the theorem can be proved using the induction on the height of such derivation trees.	
\end{proof}

Now, using the theorem~\ref{thm:correct} and induction on the length of the path, it can be easily shown that the following theorem holds.

\begin{mytheorem}\label{thm:correct_extraction}
Let $G = (N, \Sigma, P, S)$ be the input context-free grammar, $D = (V, E, \Sigma)$ be the input graph, and $T$ be a set of matrices returned by the algorithm in Listing~\ref{lst:algo1}. Then for any $i, j$ and for any non-terminal $A \in N$ such that $index = T^A_{i,j}$ and $index = (i,j,middles) \neq \bot$, the algorithm in Listing~\ref{lst:algo2} for these parameters will return a set of all paths $\pi$ from vertex $i$ to $j$ such that $l(\pi) \in G_A = (N,\Sigma,P,A)$.
\end{mytheorem}

We can, therefore, determine whether $(i,j) \in R_{G, D}$ by asking whether $T^S_{i,j} = \bot$. Also, we can extract all paths which form a word from the context-free language $L(G)$ by using our algorithm in Listing~\ref{lst:algo2}. Thus, we show how the context-free path query evaluation w.r.t. the all-path query semantics can be solved in terms of matrix operations.

%\subsection{Complexity}

%Denote the number of elementary operations executed by the algorithm of multiplying two $n \times n$ matrices with PathIndexes as $MM(n)$. Also, denote the number of elementary operations, executed by the matrix element-wise + operation of two $n \times n$ matrices with PathIndexes as $MA(n)$. Since the line \textbf{7} of the algorithm in listing~\ref{lst:algo2} is executed no more than $|V|^2|N|$ times (for the same reasons as in the original paper~\cite{Azimov:2018:CPQ:3210259.3210264} of the matrix-based CFPQ algorithm), the following theorem holds.

%\begin{myproposition}\label{thm:time}
%	Let $D = (V,E)$ be a graph and let $G =(N,\Sigma,P)$ be a grammar. The algorithm in listing~\ref{lst:algo2} calculates the set of matrices $T$ in $O(|V|^2|N|^3(MM(|V|) + MA(|V|)))$.
%\end{myproposition}

%Also, denote the time complexity of the access to the PathIndex in the $n \times n$ matrix as $Access(n)$. Then the following theorem on the time complexity of the path extraction algorithm holds.

%\begin{myproposition}\label{thm:time_extraction}
%	Let $D = (V,E)$ be a graph, let $G =(N,\Sigma,P)$ be a grammar and $T$ be a set of matrices returned by the algorithm in listing~\ref{lst:algo2}. Then for any $i, j$ and for any non-terminal $A \in N$ such that $index = T^A_{i,j}$ and $index = (i,j,k,h,l) \neq \bot$, the algorithm in listing~\ref{lst:algo3} for these parameters calculates a path $i \pi j$ in $O(l \times N \times Access(|V|))$.
%\end{myproposition}

\subsection{An Example}
In this section, we provide a step-by-step demonstration of the proposed algorithms. %For this, we consider the example with the worst-case time complexity.

We run the query on a graph $D_1$, presented in Figure~\ref{fig:example_input_graph}. We provide a step-by-step demonstration of the work of algorithm in Listing~\ref{lst:algo1} with the given graph $D$ and grammar $G_1^{\text{wcnf}}$ from section~\ref{sec:preliminaries}. After the matrix initialization in lines \textbf{3-5} of this algorithm, we have a set of matrices $T^{(1)}$, presented in Figure~\ref{ExampleQueryInitMatrix}.

{\small
	\begin{figure}[h]
		\[
		T^{(1),A} = \begin{pmatrix}
			\bot & (0,1,\{4\})       & \bot & \bot       \\
			\bot & \bot & (1,2,\{4\})       & \bot \\
			(2,0,\{4\})       & \bot & \bot & \bot \\
			\bot       & \bot & \bot & \bot \\
		\end{pmatrix}
		\]
		\[
		T^{(1),B} = \begin{pmatrix}
			\bot & \bot       & \bot & (0,3,\{4\})       \\
			\bot & \bot & \bot       & \bot \\
			\bot       & \bot & \bot & \bot \\
			(3,0,\{4\})      & \bot & \bot & \bot \\
		\end{pmatrix}
		\]
		\caption{The initial matrices for the example query. The PathIndexes $T^{(1),S_1}_{i,j}$ and $T^{(1),S}_{i,j}$ are equal to $\bot$ for every $i,j$}
		\label{ExampleQueryInitMatrix}
	\end{figure}
}

After the initialization, the only matrices which will be updated are $T^{S_1}$ and $T^{S}$. These matrices obtained after the first loop iteration is shown in Figure~\ref{ExampleQueryFirstIteration}.

{\small
	\begin{figure}[h]
		\[
		T^{(2),S} = \begin{pmatrix}
			\bot & \bot       & \bot & \bot       \\
			\bot & \bot & \bot       & \bot \\
			\bot       & \bot & \bot & (2,3,\{0\}) \\
			\bot       & \bot & \bot & \bot \\
		\end{pmatrix}
		\]
		\caption{The first iteration of computing the transitive closure for the example query. The PathIndexes $T^{(1),S_1}_{i,j}$ are equal to $\bot$ for every $i,j$}
		\label{ExampleQueryFirstIteration}
	\end{figure}
}

When the algorithm at some iteration finds new paths for some non-terminal in the graph $D_1$, then it adds corresponding AllPathIndexes to the matrix for this non-terminal. For example, after the first loop iteration, AllPathIndex $(2,3,\{0\})$ is added to the matrix $T^{S}$. This AllPathIndex is added to the element with a row index $i = 2$ and a column index $j = 3$. This means, that there is a path $\pi$ from the vertex 2 to the vertex 3, such that $S \xLongrightarrow[G_1^{\text{wcnf}}]{*} l(\pi)$ and this path obtained by concatenation of two smaller paths via vertex 0.

The calculation of the index $T$ is completed after $k$ iterations, when a fixpoint is reached: $T^{(k)} = T^{(k-1)}$. For the example query, $k = 14$ since $T_{14} = T_{13}$. The resulted matrix for non-terminal $S$ is presented in Figure~\ref{ExampleQueryFinalMatrices}.

{\small
	\begin{figure}[h]
		\[
		T^{(14),S} = \begin{pmatrix}
			(0,0,\{1\}) & \bot       & \bot & (0,3,\{1\})       \\
			(1,0,\{2\}) & \bot & \bot       & (1,3,\{2\}) \\
			(2,0,\{0\})       & \bot & \bot & (2,3,\{0\}) \\
			\bot       & \bot & \bot & \bot \\
		\end{pmatrix}
		\]
		\caption{The final matrix for non-terminal $S$ after computing the index}
		\label{ExampleQueryFinalMatrices}
	\end{figure}
}

Now, after constructing the index, we can construct the context-free relation $$R_{G_1^{\text{wcnf}}, D_1}=\{(0,0),(0,3),(1,0),(1,3),(2,0),(2,3)\}.$$

\begin{table*}[t]
	{
		\caption{Index creation time in seconds and memory in megabytes for the same-generation query}
		\label{tbl:index_creation}
		\small
		\rowcolors{4}{black!2}{black!10}
		\begin{tabular}{|l|l|l|l|l|l|l|l|l|l|l|}
			\hline
			\multicolumn{1}{|c|}{\multirow{2}{*}{Graph}} & \multicolumn{1}{c|}{\multirow{2}{*}{\#V}} & \multicolumn{1}{c|}{\multirow{2}{*}{\#E}} &  \multicolumn{2}{c|}{MtxRel} & \multicolumn{2}{c|}{MtxSingle} & \multicolumn{2}{c|}{MtxAll} & \multicolumn{2}{c|}{Tns} \\ \cline{4-11} 
			\multicolumn{1}{|c|}{}                       & \multicolumn{1}{c|}{}                     & \multicolumn{1}{c|}{}                     & Time         & Mem          & Time        & Mem        & Time           & Mem           & Time         & Mem    \\ \hline
			pathways                                     & 6 238                                     & 18 598 & 0.01         & 140  & 0.01           & 671 & 0.01         & 49           & 0.01        & 122               \\ \hline
			go-hierarchy                                 & 45 007                                    & 980 218                                   & 0.09         & 255 & 0.84           & 671 & 0.35        & 195        & 0.24        & 252                             \\ \hline
			enzyme                                       & 48 815                                    & 109 695                                   & 0.01         & 181 & 0.01           & 217 & 0.02         & 61           & 0.02        & 132                            \\ \hline
			eclass\_514en                                & 239 111                                   & 523 727                                   & 0.06         & 181 & 0.16           & 216    & 0.22         & 126          & 0.27        & 193                         \\ \hline
			go                                           & 272 770                                   & 534 311                                   & 0.94         & 246  & 0.93           & 217  & 1.13         & 990          & 1.27        & 243                          \\ \hline
			%geospecies                                   & 450 609                                   & 2 311 461                   & 0.01         & 248 & 0.01           & 2251            & 0.34         & 156          & 0.01        & 196              \\ \hline
			geospecies & 450 609                                   & 2 311 461 & 7.48         & 7645     & 15.54          & 22941 & 32.06        & 44235        & 26.32       & 19537 \\ \hline
			taxonomy                                     & 5 728 398                                 & 14 922 125                            & 0.72         & 1175      & 1.15           & 2250    & 3.84        & 1507        & 3.56        & 1776                   \\ \hline
		\end{tabular}
	}
\end{table*}

In the relation $R_{G_1^{\text{wcnf}}, D_1}$, we have all vertex pairs corresponding to paths, whose labeling is in the language $\{a^n b^n \mid n \geq 1\}$. Using the algorithm in Listing~\ref{lst:algo2} we can restore paths for each vertex pair from the context-free relation. For example, given $i=j=0$, non-terminal $S$, set of resulted matrices $T$, and context-free grammar $G_1^{\text{wcnf}}$, the algorithm in Listing~\ref{lst:algo2} returns an infinite set of all paths from vertex 0 to vertex 0 whose labeling form words from the following set $\{a^6 b^6, a^{12} b^{12}, a^{18} b^{18}, \ldots \}$. Following the path corresponding to the word $a^{6m} b^{6m}$, we will go through the cycle with $a$ labels $2m$ times and through the cycle with $b$ labels $3m$ times for all $m \geq 1$.
\section{Evaluation}

For performance analysis of proposed solution we evaluated some most common graph algorithms using real-world sparse matrix data. 
As a baseline for comparison we chose LAGraph~\cite{szarnyas2021lagraph} in connection with SuiteSparse~\cite{10.1145/3322125} as a CPU tool, Gunrock~\cite{7967137} and GraphBLAST~\cite{yang2019graphblast} as a Nvidia GPU tools. 
Also, we tested algorithms on several devices with distinct OpenCL vendors in order to validate portability of the proposed solution. 
In general, these evaluation intentions are summarized in the following research questions. 

\vspace{0.2cm}
\begin{itemize}
    \item[\textbf{RQ1}] What is the performance of the proposed solution relative to existing tools for both CPU and GPU analysis?
    
    \item[\textbf{RQ2}] What is the portability of the proposed solution with respect to various device vendors and OpenCL runtimes?
\end{itemize}

\subsection{Evaluation Setup}

For evaluation, we use a PC with Ubuntu 20.04 installed, which has 3.40Hz Intel Core i7-6700 4-core CPU, DDR4 64Gb RAM, and Nvidia GeForce GTX 1070 GPU with 8Gb VRAM. 
Host programs were compiled with GCC 9.3.0 compiler. Programs using CUDA were compiled with GCC 8.4.0 and Nvidia NVCC 10.1.243 compiler.
Release mode and maximum optimization level was enabled for all tested programs. 
Data loading time, preparation, format transformations and host-device initial communications are excluded from time measurements. 
All tests are averaged across 10 runs.
Additional warm-up run for each test execution is excluded from measurements.

\subsection{Graph Algorithms}

For preliminary study \textit{breadth-first search} (bfs) and \textit{triangles counting} (tc) algorithms were chosen, since they allows analyse the performance of \textit{vxm} and \textit{mxm} operations, rely heavily on \textit{masking}, and utilize \textit{reduction} or \textit{assignment}. 
BFS implementation utilizes automated vector storage from sparse to dense switch and only \textit{}{push optimization}. 
TC implementation uses masked \textit{mxm} of source lower-triangular matrix with second transposed argument.

\subsection{Dataset}

Nine graph matrices were selected from the Sparse Matrix Collection at University of Florida~\cite{dataset:10.1145/2049662.2049663}. 
Information about graphs is summarized in Table~\ref{dataset:info}. 
All datasets are converted to undirected graphs. 
Self-loops and duplicated edges are removed.

\begin{table}[htbp]
\caption{Dataset description.} 
\begin{center}
    \rowcolors{2}{black!2}{black!10}
    \begin{tabular}{|l|r|r|r|}
    \hline
    Dataset & Vertices  & Edges & Max Degree \\
    \hline
    \hline
    coAuthorsCiteseer & 227.3K &   1.6M &    1372 \\
    coPapersDBLP      & 540.4K &  30.4M &    3299 \\
    hollywood-2009    &   1.1M & 113.8M &  11,467 \\
    roadNet-CA        &   1.9M &   5.5M &      12 \\
    com-Orkut         &     3M &   234M &   33313 \\
    cit-Patents       &   3.7M &  16.5M &     793 \\
    rgg\_n\_2\_22\_s0 &   4.1M &  60.7M &      36 \\
    soc-LiveJournal   &   4.8M &  68.9M &  20,333 \\
    indochina-2004    &   7.5M & 194.1M & 256,425 \\
    \hline
    \end{tabular}
    \label{dataset:info}
\end{center}
\end{table}

\subsection{Results}

Table~\ref{results} presents results of the evaluation and compares performance of Spla against other tool on different execution platforms.
Tools are grouped by the type of the device for the execution, where either Nvidia GPU or Intel CPU are used. 
Cell left empty if tested tool failed to analyse graph due to \textit{out of memory} exception.

In general, Spla BFS shows acceptable performance, especially on graphs with large vertex degree, such as soc-LiveJournal and com-Orkut.
On graphs roadNet-CA and rgg it has a significant performance drop due to the nature of underlying algorithms and data structures. 
Firstly, library utilizes immutable data buffers. Thus, iteratively updated dense vector of reached vertices must be copied for each modification, what dominates the performance of the library on a graph with large search depth. 
Secondly, Spla BFS does not utilise \textit{pull optimization}, what is critical in a graph with relatively small search frontier. 

Spla TC has a good performance on GPU, which is better in all cases that reference SuiteSparse solution. 
But in most tests GPU competitors, especially Gunrock, show smaller processing times. 
GraphBLAST shows better performance as well. 
Library utilises masked SpGEMM algorithm, the same as in GraphBLAST, but without \textit{identity} element to fill gaps. 
Library explicitly stores all non-zero elements, and uses mask to reduce only non-zero while evaluating dot products of rows and columns. 
What causes extra divergence inside work groups. 
On Intel device Spla shows better performance compared to SuiteSparse on com-Orkut, cit-Patents and soc-LiveJournal. 
A possible reason is the large lengths of processed rows and columns in the product of matrices.

Gunrock shows nearly best average performance due to its specialized and optimized algorithms.
Also, it has good time characteristics on a mentioned earlier roadNet-CA and rgg in BFS algortihm. 
GraphBLAST follows Gunrock and show good performance as well. 
But it runs out of memory on a two significantly large graphs con-Orkut and indochina-2004. 
Spla does not rut out of memory on any test due to simplified storage scheme.

\begin{table}[htbp]
\caption{Graph algorithms evaluation results.\\Time in milliseconds (lower is better).} 
\begin{center}
    \begin{tabular}{|l|r|r|r|r|r|}
    \hline
    \multirow{2}{*}{Dataset} & \multicolumn{3}{c|}{Nvidia} & \multicolumn{2}{c|}{Intel} \\
    \cline{2-6}
    & GR & GB & SP & SS & SP \\
    \hline
    \hline
    \multicolumn{6}{|c|}{BFS} \\
    \hline
    \rowcolor{black!10} hollywood-2009    &  20.3 &  82.3 &   36.9 &   23.7 &   303.4 \\
    \rowcolor{black!2 } roadNet-CA        &  33.4 & 130.8 & 1456.4 &  168.2 &   965.6 \\
    \rowcolor{black!10} soc-LiveJournal   &  60.9 &  80.6 &   90.6 &   75.2 &  1206.3 \\
    \rowcolor{black!2 } rgg\_n\_2\_22\_s0 &  98.7 & 414.9 & 4504.3 & 1215.7 & 15630.1 \\
    \rowcolor{black!10} com-Orkut         & 205.2 & -- -- &  117.9 &   43.2 &   903.6 \\
    \rowcolor{black!2 } indochina-2004    &  32.7 & -- -- &  199.6 &  227.1 &  2704.6 \\
    \hline
    \hline
    \multicolumn{6}{|c|}{TC} \\
    \hline
    \rowcolor{black!10} coAuthorsCiteseer &   2.1 &    2.0 &    9.5 &    17.5 &    64.9 \\
    \rowcolor{black!2 } coPapersDBLP      &   5.7 &   94.4 &  201.9 &   543.1 &  1537.8 \\
    \rowcolor{black!10} roadNet-CA        &  34.3 &    5.8 &   16.1 &    47.1 &   357.6 \\
    \rowcolor{black!2 } com-Orkut         & 218.1 & 1583.8 & 2407.4 & 23731.4 & 15049.5 \\
    \rowcolor{black!10} cit-Patents       &  49.7 &   52.9 &   90.6 &   698.3 &   684.1 \\
    \rowcolor{black!2 } soc-LiveJournal   &  69.1 &  449.6 &  673.9 &  4002.6 &  3823.9 \\
    \hline
    \hline
    \multicolumn{6}{l}{Tools: Gunrock (GR), GraphBLAST (GB), SuiteSparse (SS), Spla (SP).} \\
    \end{tabular}
    \label{results}
\end{center}
\end{table}
 
% Two GPU

% \begin{table}[htbp]
%     \caption{Table Type Styles}
%     \begin{center}
%     \begin{tabular}{|c|c|c|c|}
%     \hline
%     \textbf{Table}&\multicolumn{3}{|c|}{\textbf{Table Column Head}} \\
%     \cline{2-4} 
%     \textbf{Head} & \textbf{\textit{Table column subhead}}& \textbf{\textit{Subhead}}& \textbf{\textit{Subhead}} \\
%     \hline
%     copy& More table copy$^{\mathrm{a}}$& &  \\
%     \hline
%     \multicolumn{4}{l}{$^{\mathrm{a}}$Sample of a Table footnote.}
%     \end{tabular}
%     \label{tab2}
%     \end{center}
% \end{table}

\section{Conclusion}

In this paper we present a library for sparse Boolean linear algebra which implements such basic operations as matrix-matrix multiplication and element-wise matrix-matrix addition in both Cuda and OpenCL.
Evaluation shows that our Boolean-specific implementations faster and require less memory than generic, not the Boolean optimized, operations from state-of-the-art libraries. 
Thus, the specialization of operations for this data type makes sense. 

The first direction of the future work is to integrate all parts (OpenCL and Cuda backends) into a single library and improve its documentation and prepare to publish.
Moreover, it is necessary to extend the library with other operations, including matrix-vector operations, masking, and so on.
As a result a Python package should be published.

Another important step is to evaluate the library on different algorithms and devices.
Namely, algorithms for RPQ and CFPQ should be implemented and evaluated on related data sets.
Also, it is necessary to evaluate OpenCL version on FPGA which may require additional technical effort and code changes.

Finally, we plan to discuss with GraphBLAS community possible ways to use our library as a backend for GraphBLAST or SuiteSparse in case of Boolean computations.
Moreover, it may be possible to use implemented algorithms as a foundation for generalization to arbitrary semirings.


\begin{acks}
	The reported study was funded by RFBR, project number 19-37-90101, and grant from JetBrains Research.
\end{acks}

%\section*{Acknowledgements}
%The research was supported by the Russian Science Foundation, grant \textnumero 18-11-00100.

%We thank Roi Lipman for his help with investigation of the RedisGraph internals and pointing out the impractical memory consumption of the original Azimov's algorithm which gave us the motivation to develop the presented solution.

%We thank Ekaterina Verbitskaia for the fruitful discussion and feedback which helped us to improve the paper.

%Anonimus reviewers !!!

\balance

%%
%% The next two lines define the bibliography style to be used, and
%% the bibliography file.
\bibliographystyle{ACM-Reference-Format}
\bibliography{all-path-matrix-cfpq}

%%
%% If your work has an appendix, this is the place to put it.
%% Please note that all the content must fit within the page limits, including any appendices.
%\appendix
%
%\section{Research Methods}
% ...

\end{document}
\endinput




