\section{Introduction}

Language-constrained path querying~\cite{doi:10.1137/S0097539798337716} is a way to search for paths in edge-labeled graphs where constraints are formulated in terms of a formal language.
The language restricts the set of accepted paths: the sentence formed by the labels of a path should be in the language.
Regular languages are the most popular class of constraints used as navigational queries in graph databases.
In some cases, regular languages are not expressive enough and context-free languages are used instead.
Context-free path querying (CFPQ), can be used for RDF analysis~\cite{10.1007/978-3-319-46523-4_38}, biological data analysis~\cite{SubgraphQueriesbyContextfreeGrammars}, static code analysis~\cite{Zheng,10.1145/373243.360208}, and in other areas.

% The way authors are mentioned is not the same
CFPQ have been studied a lot since the problem was first stated by Mihalis Yannakakis in 1990~\cite{Yannakakis}.
Jelle Hellings investigates various aspects of CFPQ in~\cite{hellingsPathQuerying,hellingsRelational,DBLP:journals/corr/Hellings15}.
A number of CFPQ algorithms were proposed: (G)LL and (G)LR based algorithms by Ciro M. Medeiros et al.~\cite{Medeiros:2018:EEC:3167132.3167265}, Fred C. Santos et al.~\cite{10.1007/978-3-319-91662-0_17}, Semyon Grigorev et al.~\cite{Grigorev:2017:CPQ:3166094.3166104}, and Ekaterina Verbitskaia et al.~\cite{10.1007/978-3-319-41579-6_22}; CYK-based algorithm by Xiaowang Zhang et al.~\cite{10.1007/978-3-319-46523-4_38}; combinators-based approach to CFPQ by Ekaterina Verbitskaia et al.~\cite{Verbitskaia:2018:PCC:3241653.3241655}.
Nevertheless, the application of context-free constraints for real-world data analysis still faces many problems.
The first problem is bad performance of the proposed algorithms on real-world data, as shown by Jochem Kuijpers et al.~\cite{Kuijpers:2019:ESC:3335783.3335791}.
The second problem is that no graph database provides full-stack support of CFPQ, since most effort was made in developping algorithms and researching their theoretical properties.
This fact hinders research of problems which can be reduced to CFPQ, thus it hinders the development of new solutions for them.
For example, graph segmentation in data provenance analysis was recently reduced to CFPQ~\cite{8731467}, but evaluation of the proposed approach was complicated because no graph database supported CFPQ.

Rustam Azimov proposed a matrix-based algorithm for CFPQ in~\cite{Azimov:2018:CPQ:3210259.3210264}.
This algorithm provides a solution performant enough for real-world data analysis, as shown by Nikita Mishim et al. in~\cite{Mishin:2019:ECP:3327964.3328503} and Arseniy Terekhov et al. in~\cite{10.1145/3398682.3399163}.
This algorithm computes reachability or provides a single path which satisfies constraints for \emph{every} vertex pair in the graph.
Namely it solves \emph{all-pairs} context-free path querying problem.
In many real-world scenarios it is redundant to handle all possible pairs, instead one can provide one or a relatively small set of start vertices.

While all-pairs context-free path querying is a problem well studied, there is no, best to our knowledge, solutions for the single-source and multiple-source CFPQ.
In this work we propose a matrix-based \textit{multiple-source} (and \textit{single-source} as a partial case) CFPQ algorithm.

We also provide full-stack support of CFPQ for the RedisGraph\footnote{RedisGraph graph database Web-page: \url{https://redislabs.com/redis-enterprise/redis-graph/}. Access date: 19.07.2020.}~\cite{8778293} graph database.
We implement a Cypher query language extension\footnote{Proposal which describes path patterns specification syntax for Cypher query language: \url{https://github.com/thobe/openCypher/blob/rpq/cip/1.accepted/CIP2017-02-06-Path-Patterns.adoc}.
The proposed syntax allows one to specify context-free constraints. Access date: 19.07.2020.} that makes it possible to use context-free constraints, and extend the RedisGraph to support this extension.
As far as we know, it is the first full-stack implementation of CFPQ.

To sum up, we make the following contributions in this paper.
\begin{enumerate}
	\item We modify Azimov's matrix-based CFPQ algorithm and provide a multiple-source matrix-based CFPQ algorithm.
	As a partial case, it is possible to use our algorithm in a single-source scenario.
	Our modification is still based on linear algebra, hence it is simple to implement and allows one to use high-performance libraries and utilize modern parallel hardware for queries evaluation.
	\item We evaluate two versions of the proposed algorithm: with caching of results and without caching (naive).
	Caching is aimed to reduce repeated calculation of the same data.
	Our evaluation shows that the naive version is more performant and memory-efficient than the version with results caching in almost the all cases.
	We believe, it is a good choice for implementation in real-world graph database.
	\item We provide full-stack support of CFPQ by extending the RedisGraph graph database.
	To do it, we extended Cypher with syntax for context-free constraints, implemented the proposed algorithm in a RedisGraph backend, and supported the new syntax in the RedisGraph query execution engine.
	Finally, we evaluate the proposed solution and show that it is performant and memory-efficient enough to be applicable for real-world graph querying.
\end{enumerate}